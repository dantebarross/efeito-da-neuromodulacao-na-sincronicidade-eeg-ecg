% ---
% RESUMOS
% ---

% RESUMO em português
\setlength{\absparsep}{18pt} % ajusta o espaçamento dos parágrafos do resumo
\begin{resumo}
A sincronização entre oscilações neurais e ritmos fisiológicos, conceito central do \textit{Body-Brain Dynamic System} (BBDS), desempenha um papel fundamental na regulação e integração neural. Este estudo investigou os efeitos da neuromodulação catódica por estimulação transcraniana por corrente contínua de alta definição (\textit{High-Definition transcranial Direct Current Stimulation}, HD-tDCS), aplicada sobre o córtex pré-frontal dorsolateral (DLPFC) esquerdo, na sincronização cerebral em atletas de elite de basquetebol feminino em repouso (\textit{resting-state}). Utilizando um delineamento experimental cruzado e duplo-cego, com sessões ativa (catódica, 2 mA por 20 min) e simulada (\textit{sham}), analisou-se o acoplamento fásico intrafrequencial entre pares de canais de eletroencefalografia (EEG), bem como o acoplamento fásico \textit{cross-frequency} entre EEG e eletrocardiograma (ECG). Os índices utilizados foram o \textit{Phase Lag Index} (PLI) para conexões EEG-EEG na mesma banda de frequência e o \textit{Cross-Frequency Phase Linearity Measurement} (CF-PLM) para conexões EEG-ECG entre frequências distintas. Os resultados revelaram alterações robustas nos pares EEG-EEG e EEG-ECG, com efeitos tanto positivos quanto negativos da estimulação catódica, globais e localizados, por banda de frequência, destacando o potencial da HD-tDCS para modular a conectividade funcional de forma seletiva.

\vspace{\onelineskip}

\textbf{Palavras-chave}: Neuromodulação catódica. HD-tDCS. EEG. ECG. Conectividade funcional. Sincronização de fase. Atletas.
\end{resumo}

% ABSTRACT in english
\begin{resumo}[Abstract]
\begin{otherlanguage*}{english}
The synchronization between neuronal oscillations and physiological rhythms, a central concept of the Body-Brain Dynamic System (BBDS), plays a fundamental role in neural regulation and integration. This study investigated the effects of cathodal neuromodulation via High-Definition transcranial Direct Current Stimulation (HD-tDCS), applied over the left dorsolateral prefrontal cortex (DLPFC), on brain synchronization in elite female basketball athletes at rest (resting-state). Using a randomized, double-blind, crossover design with active (cathodal, 2 mA for 20 min) and sham sessions, we analyzed intrafrequency phase coupling between electroencephalography (EEG) channel pairs, as well as cross-frequency phase coupling between EEG and electrocardiogram (ECG). The metrics used were the Phase Lag Index (PLI) for same-band EEG-EEG connections and the Cross-Frequency Phase Linearity Measurement (CF-PLM) for EEG-ECG connections across distinct frequencies. Results revealed robust alterations in both EEG-EEG and EEG-ECG pairs, with positive and negative effects of cathodal stimulation, both global and localized, depending on frequency band, highlighting the potential of HD-tDCS to selectively modulate functional connectivity.

\vspace{\onelineskip}
 
\noindent 
\textbf{Keywords}: Cathodal neuromodulation. HD-tDCS. EEG. ECG. Functional connectivity. Phase synchronization. Athletes.
\end{otherlanguage*}
\end{resumo}