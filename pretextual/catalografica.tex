% ---
% Ficha Catalográfica
% ---
% Exemplo de Ficha Catalográfica (Dados Internacionais de Catalogação).
% Se a biblioteca enviar um PDF oficial da ficha, basta:
%
% \begin{fichacatalografica}
%     \includepdf{fig_ficha_catalografica.pdf}
% \end{fichacatalografica}
%
% Caso contrário, use o modelo abaixo.
% ---
% Ficha Catalográfica
% ---

\begin{fichacatalografica}
    \vspace*{\fill}
    \hrule
    \begin{center}
    \begin{minipage}[c]{12cm}

    \imprimirautor

    \hspace{0.5cm} \imprimirtitulo /\\
    \hspace{0.5cm} \imprimirautor. -- \imprimirlocal, \imprimirdata.

    \hspace{0.5cm} \pageref{LastPage} p. : il. color. ; 30 cm.

    \hspace{0.5cm} \imprimirorientadorRotulo~\imprimirorientador

    \hspace{0.5cm}
    \parbox[t]{0.9\textwidth}{%
      \imprimirtipotrabalho~--~\imprimirinstituicao,
      \imprimirdata.
    }

    \hspace{0.5cm} Desenvolvido com apoio da Coordenação de Aperfeiçoamento de Pessoal de Nível Superior - Brasil (CAPES) - Código de Financiamento 001.

    \hspace{0.5cm}
    1. Neuromodulação. 2. Sincronização Cérebro-Corpo. 3. Basquetebol.\\
    I. Orientador. II. Universidade Federal do ABC. III. Programa de Pós-Graduação em Neurociência e Cognição. IV. Título.

    \hspace{8.75cm} CDU 02:141:005.7

    \end{minipage}
    \end{center}
    \hrule
\end{fichacatalografica}