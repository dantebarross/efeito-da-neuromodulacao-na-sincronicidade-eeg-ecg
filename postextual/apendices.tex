% ----------------------------------------------------------
% Apêndices
% ----------------------------------------------------------

\begin{apendicesenv}

    \partapendices

    %%%%%%%%%%%%%%%%%%%%%%%%%%%%%%%%%%%%%%%%%%%%%%%
    \chapter{Análise de Redes de Conectividade para PLI (EEG-EEG)}
    \label{apendice:pli_eeg_eeg}
    Nesta seção, apresentamos as redes de conexões significativas obtidas pela métrica PLI para pares EEG-EEG, tanto para o cenário sem outliers quanto para o cenário com outliers. Cada figura corresponde a uma banda de frequência e complementa a análise apresentada no corpo do texto.
    
    \section{Cenário Sem Outliers}
    
    \subsection{Banda Delta (0.5--4\,Hz)}
    \begin{figure}[H]
      \centering
      \includegraphics[width=0.8\textwidth]{figs/7_bootstrap_results_analysis/2_network_graphs/PLI_EEG-EEG_Sem_Outliers/Banda_Delta_(0.5_a_4_Hz)_-_Análise_de_Rede_-_PLI_EEG-EEG_Sem_Outliers.png}
      \caption{Rede de conexões significativas na banda Delta (0.5--4\,Hz) para PLI (EEG-EEG) sem outliers.}
      \label{fig:apend_pli_delta_sem}
    \end{figure}
    
    \subsection{Banda Theta (4--8\,Hz)}
    \begin{figure}[H]
      \centering
      \includegraphics[width=0.8\textwidth]{figs/7_bootstrap_results_analysis/2_network_graphs/PLI_EEG-EEG_Sem_Outliers/Banda_Theta_(4_Hz_a_8_Hz)_-_Análise_de_Rede_-_PLI_EEG-EEG_Sem_Outliers.png}
      \caption{Rede de conexões significativas na banda Theta (4--8\,Hz) para PLI (EEG-EEG) sem outliers.}
      \label{fig:apend_pli_theta_sem}
    \end{figure}
    
    \subsection{Banda Alpha (8--13\,Hz)}
    \begin{figure}[H]
      \centering
      \includegraphics[width=0.8\textwidth]{figs/7_bootstrap_results_analysis/2_network_graphs/PLI_EEG-EEG_Sem_Outliers/Banda_Alpha_(8_Hz_a_13_Hz)_-_Análise_de_Rede_-_PLI_EEG-EEG_Sem_Outliers.png}
      \caption{Rede de conexões significativas na banda Alpha (8--13\,Hz) para PLI (EEG-EEG) sem outliers.}
      \label{fig:apend_pli_alpha_sem}
    \end{figure}
    
    \subsection{Banda Beta (13--30\,Hz)}
    \begin{figure}[H]
      \centering
      \includegraphics[width=0.8\textwidth]{figs/7_bootstrap_results_analysis/2_network_graphs/PLI_EEG-EEG_Sem_Outliers/Banda_Beta_(13_Hz_a_30_Hz)_-_Análise_de_Rede_-_PLI_EEG-EEG_Sem_Outliers.png}
      \caption{Rede de conexões significativas na banda Beta (13--30\,Hz) para PLI (EEG-EEG) sem outliers.}
      \label{fig:apend_pli_beta_sem}
    \end{figure}
    
    \subsection{Banda Gamma (30--60\,Hz)}
    \begin{figure}[H]
      \centering
      \includegraphics[width=0.8\textwidth]{figs/7_bootstrap_results_analysis/2_network_graphs/PLI_EEG-EEG_Sem_Outliers/Banda_Gamma_(30_Hz_a_60_Hz)_-_Análise_de_Rede_-_PLI_EEG-EEG_Sem_Outliers.png}
      \caption{Rede de conexões significativas na banda Gamma (30--60\,Hz) para PLI (EEG-EEG) sem outliers.}
      \label{fig:apend_pli_gamma_sem}
    \end{figure}
    
    \section{Cenário Com Outliers}
    
    \subsection{Banda Delta (0.5--4\,Hz)}
    \begin{figure}[H]
      \centering
      \includegraphics[width=0.8\textwidth]{figs/7_bootstrap_results_analysis/2_network_graphs/PLI_EEG-EEG_Com_Outliers/Banda_Delta_(0.5_a_4_Hz)_-_Análise_de_Rede_-_PLI_EEG-EEG_Com_Outliers.png}
      \caption{Rede de conexões significativas na banda Delta (0.5--4\,Hz) para PLI (EEG-EEG) com outliers.}
      \label{fig:apend_pli_delta_com}
    \end{figure}
    
    \subsection{Banda Theta (4--8\,Hz)}
    \begin{figure}[H]
      \centering
      \includegraphics[width=0.8\textwidth]{figs/7_bootstrap_results_analysis/2_network_graphs/PLI_EEG-EEG_Com_Outliers/Banda_Theta_(4_Hz_a_8_Hz)_-_Análise_de_Rede_-_PLI_EEG-EEG_Com_Outliers.png}
      \caption{Rede de conexões significativas na banda Theta (4--8\,Hz) para PLI (EEG-EEG) com outliers.}
      \label{fig:apend_pli_theta_com}
    \end{figure}
    
    \subsection{Banda Alpha (8--13\,Hz)}
    \begin{figure}[H]
      \centering
      \includegraphics[width=0.8\textwidth]{figs/7_bootstrap_results_analysis/2_network_graphs/PLI_EEG-EEG_Com_Outliers/Banda_Alpha_(8_Hz_a_13_Hz)_-_Análise_de_Rede_-_PLI_EEG-EEG_Com_Outliers.png}
      \caption{Rede de conexões significativas na banda Alpha (8--13\,Hz) para PLI (EEG-EEG) com outliers.}
      \label{fig:apend_pli_alpha_com}
    \end{figure}
    
    \subsection{Banda Beta (13--30\,Hz)}
    \begin{figure}[H]
      \centering
      \includegraphics[width=0.8\textwidth]{figs/7_bootstrap_results_analysis/2_network_graphs/PLI_EEG-EEG_Com_Outliers/Banda_Beta_(13_Hz_a_30_Hz)_-_Análise_de_Rede_-_PLI_EEG-EEG_Com_Outliers.png}
      \caption{Rede de conexões significativas na banda Beta (13--30\,Hz) para PLI (EEG-EEG) com outliers.}
      \label{fig:apend_pli_beta_com}
    \end{figure}
    
    \subsection{Banda Gamma (30--60\,Hz)}
    \begin{figure}[H]
      \centering
      \includegraphics[width=0.8\textwidth]{figs/7_bootstrap_results_analysis/2_network_graphs/PLI_EEG-EEG_Com_Outliers/Banda_Gamma_(30_Hz_a_60_Hz)_-_Análise_de_Rede_-_PLI_EEG-EEG_Com_Outliers.png}
      \caption{Rede de conexões significativas na banda Gamma (30--60\,Hz) para PLI (EEG-EEG) com outliers.}
      \label{fig:apend_pli_gamma_com}
    \end{figure}
    
    \chapter{Análise de Redes de Conectividade para CF-PLM (EEG-ECG, Cross-frequency)}
    \label{apendice:cfplm_eeg_ecg}
    Nesta seção, apresentamos as redes de conectividade obtidas pela métrica CF-PLM para a interação entre EEG e ECG, demonstrando os efeitos de sincronização entre frequências distintas. Observa-se que o número total de pares significativos é constante (11 casos) independentemente da remoção de outliers.
    
    \section{Cenário Sem Outliers}
    
    \subsection{Banda Delta (0.5--4\,Hz)}
    \begin{figure}[H]
      \centering
      \includegraphics[width=0.8\textwidth]{figs/7_bootstrap_results_analysis/2_network_graphs/CF-PLM_EEG-ECG_Sem_Outliers/Banda_Delta_(0.5_a_4_Hz)_-_Análise_de_Rede_-_CF-PLM_EEG-ECG_Sem_Outliers.png}
      \caption{Rede CF-PLM na banda Delta (0.5--4\,Hz) para EEG-ECG sem outliers.}
      \label{fig:apend_cfplm_delta_sem}
    \end{figure}
    
    \subsection{Banda Theta (4--8\,Hz)}
    \begin{figure}[H]
      \centering
      \includegraphics[width=0.8\textwidth]{figs/7_bootstrap_results_analysis/2_network_graphs/CF-PLM_EEG-ECG_Sem_Outliers/Banda_Theta_(4_Hz_a_8_Hz)_-_Análise_de_Rede_-_CF-PLM_EEG-ECG_Sem_Outliers.png}
      \caption{Rede CF-PLM na banda Theta (4--8\,Hz) para EEG-ECG sem outliers.}
      \label{fig:apend_cfplm_theta_sem}
    \end{figure}
    
    \subsection{Banda Alpha (8--13\,Hz)}
    \begin{figure}[H]
      \centering
      \includegraphics[width=0.8\textwidth]{figs/7_bootstrap_results_analysis/2_network_graphs/CF-PLM_EEG-ECG_Sem_Outliers/Banda_Alpha_(8_Hz_a_13_Hz)_-_Análise_de_Rede_-_CF-PLM_EEG-ECG_Sem_Outliers.png}
      \caption{Rede CF-PLM na banda Alpha (8--13\,Hz) para EEG-ECG sem outliers.}
      \label{fig:apend_cfplm_alpha_sem}
    \end{figure}
    
    \subsection{Banda Beta (13--30\,Hz)}
    \begin{figure}[H]
      \centering
      \includegraphics[width=0.8\textwidth]{figs/7_bootstrap_results_analysis/2_network_graphs/CF-PLM_EEG-ECG_Sem_Outliers/Banda_Beta_(13_Hz_a_30_Hz)_-_Análise_de_Rede_-_CF-PLM_EEG-ECG_Sem_Outliers.png}
      \caption{Rede CF-PLM na banda Beta (13--30\,Hz) para EEG-ECG sem outliers.}
      \label{fig:apend_cfplm_beta_sem}
    \end{figure}
    
    \subsection{Banda Gamma (30--60\,Hz)}
    \begin{figure}[H]
      \centering
      \includegraphics[width=0.8\textwidth]{figs/7_bootstrap_results_analysis/2_network_graphs/CF-PLM_EEG-ECG_Sem_Outliers/Banda_Gamma_(30_Hz_a_60_Hz)_-_Análise_de_Rede_-_CF-PLM_EEG-ECG_Sem_Outliers.png}
      \caption{Rede CF-PLM na banda Gamma (30--60\,Hz) para EEG-ECG sem outliers.}
      \label{fig:apend_cfplm_gamma_sem}
    \end{figure}
    
    \section{Cenário Com Outliers}
    
    \subsection{Banda Delta (0.5--4\,Hz)}
    \begin{figure}[H]
      \centering
      \includegraphics[width=0.8\textwidth]{figs/7_bootstrap_results_analysis/2_network_graphs/CF-PLM_EEG-ECG_Com_Outliers/Banda_Delta_(0.5_a_4_Hz)_-_Análise_de_Rede_-_CF-PLM_EEG-ECG_Com_Outliers.png}
      \caption{Rede CF-PLM na banda Delta (0.5--4\,Hz) para EEG-ECG com outliers.}
      \label{fig:apend_cfplm_delta_com}
    \end{figure}
    
    \subsection{Banda Theta (4--8\,Hz)}
    \begin{figure}[H]
      \centering
      \includegraphics[width=0.8\textwidth]{figs/7_bootstrap_results_analysis/2_network_graphs/CF-PLM_EEG-ECG_Com_Outliers/Banda_Theta_(4_Hz_a_8_Hz)_-_Análise_de_Rede_-_CF-PLM_EEG-ECG_Com_Outliers.png}
      \caption{Rede CF-PLM na banda Theta (4--8\,Hz) para EEG-ECG com outliers.}
      \label{fig:apend_cfplm_theta_com}
    \end{figure}
    
    \subsection{Banda Alpha (8--13\,Hz)}
    \begin{figure}[H]
      \centering
      \includegraphics[width=0.8\textwidth]{figs/7_bootstrap_results_analysis/2_network_graphs/CF-PLM_EEG-ECG_Com_Outliers/Banda_Alpha_(8_Hz_a_13_Hz)_-_Análise_de_Rede_-_CF-PLM_EEG-ECG_Com_Outliers.png}
      \caption{Rede CF-PLM na banda Alpha (8--13\,Hz) para EEG-ECG com outliers.}
      \label{fig:apend_cfplm_alpha_com}
    \end{figure}
    
    \subsection{Banda Beta (13--30\,Hz)}
    \begin{figure}[H]
      \centering
      \includegraphics[width=0.8\textwidth]{figs/7_bootstrap_results_analysis/2_network_graphs/CF-PLM_EEG-ECG_Com_Outliers/Banda_Beta_(13_Hz_a_30_Hz)_-_Análise_de_Rede_-_CF-PLM_EEG-ECG_Com_Outliers.png}
      \caption{Rede CF-PLM na banda Beta (13--30\,Hz) para EEG-ECG com outliers.}
      \label{fig:apend_cfplm_beta_com}
    \end{figure}
    
    \subsection{Banda Gamma (30--60\,Hz)}
    \begin{figure}[H]
      \centering
      \includegraphics[width=0.8\textwidth]{figs/7_bootstrap_results_analysis/2_network_graphs/CF-PLM_EEG-ECG_Com_Outliers/Banda_Gamma_(30_Hz_a_60_Hz)_-_Análise_de_Rede_-_CF-PLM_EEG-ECG_Com_Outliers.png}
      \caption{Rede CF-PLM na banda Gamma (30--60\,Hz) para EEG-ECG com outliers.}
      \label{fig:apend_cfplm_gamma_com}
    \end{figure}
    
    \section{Comparação Geral entre Cenários para CF-PLM}
    Ao comparar os resultados dos cenários com e sem outliers para CF-PLM (EEG-ECG), observa-se que:
    \begin{itemize}
        \item A estrutura das redes, ou seja, a topologia e a contagem de conexões, permanece praticamente inalterada entre os cenários.
        \item O número total de pares significativos (11 casos) é o mesmo, evidenciando a robustez dos efeitos cross-frequency.
        \item Os pares com \emph{Wilcoxon RBC} de +1 persistem, reforçando a consistência dos resultados.
    \end{itemize}
    
\end{apendicesenv}