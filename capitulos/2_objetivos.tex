\chapter{Objetivos}
\label{chap:objetivos}

Este estudo tem como objetivo investigar os efeitos da estimulação transcraniana por corrente contínua de alta definição (HD-tDCS), aplicada de forma catódica sobre o córtex pré-frontal dorsolateral esquerdo (DLPFC esquerdo), na conectividade funcional em atletas de elite do basquetebol feminino durante o repouso (\textit{resting-state}). A conectividade será analisada entre sinais de EEG dentro da mesma banda de frequência (delta, theta, alpha, beta e gamma), bem como entre sinais EEG (nas mesmas bandas citadas) e ECG, em uma abordagem \textit{cross-frequency}, já que o ECG possui uma frequência intrínseca única correspondente ao ciclo cardíaco. Para tal, serão empregadas técnicas de análise de sincronicidade, especificamente o \textit{Phase Lag Index} (PLI), para conectividade EEG-EEG intrafrequencial, e o \textit{Cross-Frequency Phase Linearity Measurement} (CF-PLM), para conectividade EEG-ECG. As análises serão realizadas comparando os períodos pré e pós-estimulação, nas condições sham (controle) e catódica.