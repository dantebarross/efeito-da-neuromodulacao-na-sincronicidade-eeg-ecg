\chapter{Objetivos}
\label{chap:objetivos}

Este estudo tem como objetivo investigar os efeitos da estimulação transcraniana por corrente contínua de alta definição (HD-tDCS) sobre a conectividade funcional entre o cérebro e o coração em atletas de elite em repouso. Para isso, serão aplicadas técnicas avançadas de análise de sincronicidade, utilizando o Phase Lag Index (PLI) para avaliar o acoplamento iso-frequencial entre canais EEG e o Cross-Frequency Phase Linearity Measurement (CF-PLM) para explorar o acoplamento cross-frequency entre sinais de EEG e ECG. O projeto busca identificar alterações na intensidade e direção da sincronicidade cérebro-coração induzidas pela neuromodulação, proporcionando insights sobre os mecanismos neurofisiológicos subjacentes a esses fenômenos.
