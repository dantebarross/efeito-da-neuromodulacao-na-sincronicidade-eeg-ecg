\chapter{Análise de Centralidade de Grafos}
\label{chap:analise_centralidade_de_grafos}


Este capítulo examina a \textbf{topologia} das mudanças de sincronia intra-frequencial (PLI) induzidas pela HD-tDCS catódica com \textit{sham} em referência, empregando métricas de centralidade em grafos para identificar \textit{hubs} de modulação. Cada aresta representa um par de canais cujo efeito (Wilcoxon RBC) permaneceu significativo após \textit{bootstrap} BCa (10.000 iterações) e correção de Bonferroni. O sinal do RBC indica direção (positivo = efeito positivo na condição catódica; negativo = efeito negativo na condição catódica), e seu módulo reflete a magnitude da alteração, todos os casos significativos com magnitude máxima.

Os resultados desta fase utilizam os mesmos dados de resultados do Capítulo \ref{chap:analise_de_rede} (análise de rede dos resultados dos testes \textit{bootstrap} não-paramétricos par a par, que revelou os pares responsáveis por esses efeitos). A hierarquia de \textit{hubs} obtida aqui, estável mesmo após remover outliers, aponta regiões candidatas a intervenções futuras.

\bigskip
\noindent\textbf{Métricas de centralidade:}\footnote{Definições formais e aplicações em neurociência em \citeonline{rubinov2010complex,bullmore2009complex,newman2018networks,freeman1977centrality}.}
\begin{itemize}
  \item \textbf{Degree} (\(\mathcal{D}\)) - número de conexões diretas de um nó, normalizado em \([0,1]\) por \texttt{nx.degree\_centrality} \cite{networkx_doc}.
  \item \textbf{Betweenness} (\(\mathcal{B}\)) - fração de caminhos mínimos nos quais o nó atua como ``ponte'' \cite{freeman1977centrality}.
  \item \textbf{Eigenvector} (\(\mathcal{E}\)) - influência recursiva de um nó: ele é central se estiver conectado a outros nós centrais \cite{lohmann2010eigenvector}.
\end{itemize}

\noindent\textbf{Implementação em \texttt{Python/NetworkX}:}
\begin{itemize}
    \item \textbf{Grafo binário}: para \textbf{Degree} e \textbf{Betweenness} definimos \texttt{weight=None}; todas as arestas significativas contam igualmente (\(w=1\)). Isso evita interpretar o RBC (que não é distância) como ``custo'' nos caminhos mínimos \cite{networkx_doc}.
    \item \textbf{Grafo ponderado}: para \textbf{Eigenvector} passamos \texttt{weight='weight'}, atribuindo a cada aresta o próprio valor RBC. A rotina soma \(I\) à matriz de adjacência, garantindo convergência mesmo com arestas negativas \cite{networkx_doc}.
\end{itemize}

\noindent\textbf{Layout gráfico:} As figuras usam o algoritmo \textbf{Kamada-Kawai} da biblioteca Python \texttt{NetworkX}, que posiciona os nós \textbf{exclusivamente} segundo a estrutura do grafo; a disposição visual não corresponde às coordenadas reais dos eletrodos.

\noindent\textbf{Cenários analisados:}
\begin{itemize}
  \item \textbf{Com outliers} - conjunto original.
  \item \textbf{Sem outliers} - após remover \approx 5\% das observações pelo ECOD.
\end{itemize}

\noindent\textbf{Notas interpretativas:}
\begin{itemize}
  \item As métricas quantificam \textbf{alterações funcionais} em nível de sensor, não conexões anatômicas. Um canal com alta \(\mathcal{D}\) ou \(\mathcal{E}\) indica que a região cortical subjacente participou intensamente das mudanças de conectividade.
  \item Conexões com RBC positivo sinalizam aumentos de sincronia; RBC negativo, reduções. Assim, um ``\textit{hub}'' pode refletir tanto acréscimos quanto quebras de conectividade, nuance discutida na Seção~\ref{sec:fase3_centralidade}.
\end{itemize}

A seguir apresentamos, para cada banda (alpha, beta, delta, gamma e theta), duas figuras por métrica (com / sem outliers), para cada métrica de centralidade, totalizando trinta figuras. As hierarquias indicadas nas legendas correspondem ao ranking dos valores de centralidade, onde são apresentadas apenas as mais destacadas.

%\clearpage
% \section{Centralidade por Banda de Frequência}
% \subsection{Banda Alpha (8-13 Hz)}
% \subsubsection{\textit{Betweenness Centrality}}
\ultrawidefigure
{figs/7_bootstrap_results_analysis/3_centrality_graphs/Com_Outliers/Betweenness_Centrality__alpha__PLI_EEGEGG_Com_Outliers.png}
{\textit{Betweenness Centrality} - Banda Alpha (8-13 Hz), \textbf{com outliers}: hierarquia de canais (1. Fp2; 2. PO8; 3. TP10, Fp1, AF3; 4. F2, C5, F4).}
{betweenness_alpha_com}

\ultrawidefigure
{figs/7_bootstrap_results_analysis/3_centrality_graphs/Sem_Outliers/Betweenness_Centrality__alpha__PLI_EEGEGG_Sem_Outliers.png}
{\textit{Betweenness Centrality} - Banda Alpha (8-13 Hz), \textbf{sem outliers}: hierarquia de canais (1. Fp2; 2. PO8, C6, F4, O1; 3. Fp1, TP9; 4. AF3, FC3, F2, TP10).}
{betweenness_alpha_sem}

% \subsubsection{\textit{Degree Centrality}}

\ultrawidefigure
{figs/7_bootstrap_results_analysis/3_centrality_graphs/Com_Outliers/Degree_Centrality__alpha__PLI_EEGEGG_Com_Outliers.png}
{\textit{Degree Centrality} - Banda Alpha (8-13 Hz), \textbf{com outliers}: hierarquia de canais (1. Fp2; 2. FC3, Fp1; 3. CP4, TP10, PO8, F2; 4. F4, C6, P4, C5).}
{degree_alpha_com}

\ultrawidefigure
{figs/7_bootstrap_results_analysis/3_centrality_graphs/Sem_Outliers/Degree_Centrality__alpha__PLI_EEGEGG_Sem_Outliers.png}
{\textit{Degree Centrality} - Banda Alpha (8-13 Hz), \textbf{sem outliers}: hierarquia de canais (1. Fp2; 2. PO8; 3. FC3, F2, F4; 4. C6, Fp1, P4, C4, CP4).}
{degree_alpha_sem}

% \subsubsection{\textit{Eigenvector Centrality}}

\ultrawidefigure
{figs/7_bootstrap_results_analysis/3_centrality_graphs/Com_Outliers/Eigenvector_Centrality__alpha__PLI_EEGEGG_Com_Outliers.png}
{\textit{Eigenvector Centrality} - Banda Alpha (8-13 Hz), \textbf{com outliers}: hierarquia de canais (1. Fp2; 2. FC3; 3. CP4, F4, TP10, Fp1; 4. PO8).}
{eigenvector_alpha_com}

\ultrawidefigure
{figs/7_bootstrap_results_analysis/3_centrality_graphs/Sem_Outliers/Eigenvector_Centrality__alpha__PLI_EEGEGG_Sem_Outliers.png}
{\textit{Eigenvector Centrality} - Banda Alpha (8-13 Hz), \textbf{sem outliers}: hierarquia de canais (1. Fp2; 2. F4; 3. FC3, TP10; 4. Cpz, C4; 5. PO8, P4, FC1, F2, P4).}
{eigenvector_alpha_sem}


% \subsection{Banda Beta (13-30 Hz)}
% \subsubsection{\textit{Betweenness Centrality}}

\ultrawidefigure
{figs/7_bootstrap_results_analysis/3_centrality_graphs/Com_Outliers/Betweenness_Centrality__beta__PLI_EEGEGG_Com_Outliers.png}
{\textit{Betweenness Centrality} - Banda Beta (13-30 Hz), \textbf{com outliers}: hierarquia de canais (1. CP5, F5; 2. CP1; 3. F2, POz; 4. TPP7h).}
{betweenness_beta_com}

\ultrawidefigure
{figs/7_bootstrap_results_analysis/3_centrality_graphs/Sem_Outliers/Betweenness_Centrality__beta__PLI_EEGEGG_Sem_Outliers.png}
{\textit{Betweenness Centrality} - Banda Beta (13-30 Hz), \textbf{sem outliers}: hierarquia de canais (1. TPP7h; 2. CP5, F5, CP2; 3. FFT7h; 4. TTP8h; 5. CPz; 6. P6).}
{betweenness_beta_sem}

% \subsubsection{\textit{Degree Centrality}}

\ultrawidefigure
{figs/7_bootstrap_results_analysis/3_centrality_graphs/Com_Outliers/Degree_Centrality__beta__PLI_EEGEGG_Com_Outliers.png}
{\textit{Degree Centrality} - Banda Beta (13-30 Hz), \textbf{com outliers}: hierarquia de canais (1. TPP7h; 2. F5, CPz; 3. CP1, CP5, F8, FTT7h, P6, C5, Fz).}
{degree_beta_com}

\ultrawidefigure
{figs/7_bootstrap_results_analysis/3_centrality_graphs/Sem_Outliers/Degree_Centrality__beta__PLI_EEGEGG_Sem_Outliers.png}
{\textit{Degree Centrality} - Banda Beta (13-30 Hz), \textbf{sem outliers}: hierarquia de canais (1. TPP7h; 2. FTT7h, CPz; 3. CP5, F5, CP2, FTT7h).}
{degree_beta_sem}

% \subsubsection{\textit{Eigenvector Centrality}}

\ultrawidefigure
{figs/7_bootstrap_results_analysis/3_centrality_graphs/Com_Outliers/Eigenvector_Centrality__beta__PLI_EEGEGG_Com_Outliers.png}
{\textit{Eigenvector Centrality} - Banda Beta (13-30 Hz), \textbf{com outliers}: hierarquia de canais (1. F5; 2. C5, CPz; 3. P6, Fz; 4. CP5, O1, FFT7h, FC3).}
{eigenvector_beta_com}

\ultrawidefigure
{figs/7_bootstrap_results_analysis/3_centrality_graphs/Sem_Outliers/Eigenvector_Centrality__beta__PLI_EEGEGG_Sem_Outliers.png}
{\textit{Eigenvector Centrality} - Banda Beta (13-30 Hz), \textbf{sem outliers}: hierarquia de canais (1. TPP7h; 2. FTT7h; 3. CP5).}
{eigenvector_beta_sem}

% \subsection{Banda Delta (0{,}5-4 Hz)}
% \subsubsection{\textit{Betweenness Centrality}}

\ultrawidefigure
{figs/7_bootstrap_results_analysis/3_centrality_graphs/Com_Outliers/Betweenness_Centrality__delta__PLI_EEGEGG_Com_Outliers.png}
{\textit{Betweenness Centrality} - Banda Delta (0{,}5-4 Hz), \textbf{com outliers}: único nodo destacado - Fp2.}
{betweenness_delta_com}

\ultrawidefigure
{figs/7_bootstrap_results_analysis/3_centrality_graphs/Sem_Outliers/Betweenness_Centrality__delta__PLI_EEGEGG_Sem_Outliers.png}
{\textit{Betweenness Centrality} - Banda Delta (0{,}5-4 Hz), \textbf{sem outliers}: hierarquia de canais (1. CP5, CP4; 2. FC3; 3. AF4; 4. FFT8h).}
{betweenness_delta_sem}

% \subsubsection{\textit{Degree Centrality}}

\ultrawidefigure
{figs/7_bootstrap_results_analysis/3_centrality_graphs/Com_Outliers/Degree_Centrality__delta__PLI_EEGEGG_Com_Outliers.png}
{\textit{Degree Centrality} - Banda Delta (0{,}5-4 Hz), \textbf{com outliers}: hierarquia de canais (1. Fp2; 2. TTP8h; 3. AF4).}
{degree_delta_com}

\ultrawidefigure
{figs/7_bootstrap_results_analysis/3_centrality_graphs/Sem_Outliers/Degree_Centrality__delta__PLI_EEGEGG_Sem_Outliers.png}
{\textit{Degree Centrality} - Banda Delta (0{,}5-4 Hz), \textbf{sem outliers}: hierarquia de canais (1. CP5, CP4; 2. FC6, F4, Fz).}
{degree_delta_sem}

% \subsubsection{\textit{Eigenvector Centrality}}

\ultrawidefigure
{figs/7_bootstrap_results_analysis/3_centrality_graphs/Com_Outliers/Eigenvector_Centrality__delta__PLI_EEGEGG_Com_Outliers.png}
{\textit{Eigenvector Centrality} - Banda Delta (0{,}5-4 Hz), \textbf{com outliers}: hierarquia de canais (1. Fp2; 2. TTP8h, FC6, CP5, TPP7h).}
{eigenvector_delta_com}

\ultrawidefigure
{figs/7_bootstrap_results_analysis/3_centrality_graphs/Sem_Outliers/Eigenvector_Centrality__delta__PLI_EEGEGG_Sem_Outliers.png}
{\textit{Eigenvector Centrality} - Banda Delta (0{,}5-4 Hz), \textbf{sem outliers}: hierarquia de canais (1. CP5; 2. CP4; 3. FC6; 4. Fp1, CP2, FC3).}
{eigenvector_delta_sem}

% \subsection{Banda Gamma (30-60 Hz)}
% \subsubsection{\textit{Betweenness Centrality}}

\ultrawidefigure
{figs/7_bootstrap_results_analysis/3_centrality_graphs/Com_Outliers/Betweenness_Centrality__gamma__PLI_EEGEGG_Com_Outliers.png}
{\textit{Betweenness Centrality} - Banda Gamma (30-60 Hz), \textbf{com outliers}: hierarquia de canais (1. C5; 2. FFT8h, F3; 3. FT10, C1, CP4, P6; 4. Cz, C4, F1, Fp2, P5).}
{betweenness_gamma_com}

\ultrawidefigure
{figs/7_bootstrap_results_analysis/3_centrality_graphs/Sem_Outliers/Betweenness_Centrality__gamma__PLI_EEGEGG_Sem_Outliers.png}
{\textit{Betweenness Centrality} - Banda Gamma (30-60 Hz), \textbf{sem outliers}: hierarquia de canais (1. F3; 2. C5; 3. FFT8h, FT10; 4. P5).}
{betweenness_gamma_sem}

% \subsubsection{\textit{Degree Centrality}}

\ultrawidefigure
{figs/7_bootstrap_results_analysis/3_centrality_graphs/Com_Outliers/Degree_Centrality__gamma__PLI_EEGEGG_Com_Outliers.png}
{\textit{Degree Centrality} - Banda Gamma (30-60 Hz), \textbf{com outliers}: hierarquia de canais (1. FT10; 2. FTT8h; 3. F3; 4. P5, P6, C4, F1, C5, C1, TPP8h).}
{degree_gamma_com}

\ultrawidefigure
{figs/7_bootstrap_results_analysis/3_centrality_graphs/Sem_Outliers/Degree_Centrality__gamma__PLI_EEGEGG_Sem_Outliers.png}
{\textit{Degree Centrality} - Banda Gamma (30-60 Hz), \textbf{sem outliers}: hierarquia de canais (1. F3, FT10; 2. FTT8h; 3. C5; 4. P5, P6).}
{degree_gamma_sem}

% \subsubsection{\textit{Eigenvector Centrality}}

\ultrawidefigure
{figs/7_bootstrap_results_analysis/3_centrality_graphs/Com_Outliers/Eigenvector_Centrality__gamma__PLI_EEGEGG_Com_Outliers.png}
{\textit{Eigenvector Centrality} - Banda Gamma (30-60 Hz), \textbf{com outliers}: hierarquia de canais (1. F3, FT10; 2. FTT8h; 3. F6; 4. Fz, P5, C2, C5).}
{eigenvector_gamma_com}

\ultrawidefigure
{figs/7_bootstrap_results_analysis/3_centrality_graphs/Sem_Outliers/Eigenvector_Centrality__gamma__PLI_EEGEGG_Sem_Outliers.png}
{\textit{Eigenvector Centrality} - Banda Gamma (30-60 Hz), \textbf{sem outliers}: hierarquia de canais (1. FTT8h, C5, FT10; 2. F3; 3. P5; 4. AF7; 5. C2).}
{eigenvector_gamma_sem}

% \subsection{Banda Theta (4-8 Hz)}
% \subsubsection{\textit{Betweenness Centrality}}

\ultrawidefigure
{figs/7_bootstrap_results_analysis/3_centrality_graphs/Com_Outliers/Betweenness_Centrality__theta__PLI_EEGEGG_Com_Outliers.png}
{\textit{Betweenness Centrality} - Banda Theta (4-8 Hz), \textbf{com outliers}: hierarquia de canais (1. FC1, PO4; 2. F8; 3. Fp2; 4. CP3, C1).}
{betweenness_theta_com}

\ultrawidefigure
{figs/7_bootstrap_results_analysis/3_centrality_graphs/Sem_Outliers/Betweenness_Centrality__theta__PLI_EEGEGG_Sem_Outliers.png}
{\textit{Betweenness Centrality} - Banda Theta (4-8 Hz), \textbf{sem outliers}: hierarquia de canais (1. PO4; 2. FC1; 3. C5, PO7, P7; 4. F8, AF3, Fp2).}
{betweenness_theta_sem}

% \subsubsection{\textit{Degree Centrality}}

\ultrawidefigure
{figs/7_bootstrap_results_analysis/3_centrality_graphs/Com_Outliers/Degree_Centrality__theta__PLI_EEGEGG_Com_Outliers.png}
{\textit{Degree Centrality} - Banda Theta (4-8 Hz), \textbf{com outliers}: hierarquia de canais (1. C1, C3, TP9, F8, AF4; 2. Pz, P6).}
{degree_theta_com}

\ultrawidefigure
{figs/7_bootstrap_results_analysis/3_centrality_graphs/Sem_Outliers/Degree_Centrality__theta__PLI_EEGEGG_Sem_Outliers.png}
{\textit{Degree Centrality} - Banda Theta (4-8 Hz), \textbf{sem outliers}: hierarquia de canais (1. C1, C5, TP9, PO7, PO4, F8, AF4; 2. POz, P7, FC1, P6, C3).}
{degree_theta_sem}

% \subsubsection{\textit{Eigenvector Centrality}}

\ultrawidefigure
{figs/7_bootstrap_results_analysis/3_centrality_graphs/Com_Outliers/Eigenvector_Centrality__theta__PLI_EEGEGG_Com_Outliers.png}
{\textit{Eigenvector Centrality} - Banda Theta (4-8 Hz), \textbf{com outliers}: hierarquia de canais (1. F8; 2. AF4; 3. FCz; 4. O1; 5. PO4, TP9).}
{eigenvector_theta_com}

\ultrawidefigure
{figs/7_bootstrap_results_analysis/3_centrality_graphs/Sem_Outliers/Eigenvector_Centrality__theta__PLI_EEGEGG_Sem_Outliers.png}
{\textit{Eigenvector Centrality} - Banda Theta (4-8 Hz), \textbf{sem outliers}: hierarquia de canais (1. TP9; 2. C5; 3. Fp2, C1, P3, F3, F8, FC5, FC1, C3, AFz, AF4, PO4, FC1).}
{eigenvector_theta_sem}