\chapter{Resultados e Discussões}
\label{chap:resultados_e_discussoes}

A compreensão dos mecanismos neurobiológicos subjacentes aos efeitos da neuromodulação não-invasiva é uma das fronteiras atuais da neurociência aplicada ao esporte de alto rendimento. Neste capítulo, integramos os resultados obtidos com a estimulação transcraniana por corrente contínua de alta definição (HD-tDCS) catódica aplicada ao córtex pré-frontal dorsolateral (DLPFC) esquerdo, discutindo-os em quatro eixos complementares:

\begin{enumerate}
  \item \textbf{Sincronização de fase intrafrequencial} (\emph{EEG-EEG}) quantificada pelo \textit{Phase Lag Index} (PLI);
  \item \textbf{Acoplamento \textit{cross-frequency}} (\emph{EEG-ECG}) mensurado pelo \textit{Cross-Frequency Phase Linearity Measurement} (CF-PLM);
  \item \textbf{Significância estatística e magnitude dos efeitos} - primeiro por testes não-paramétricos globais (Mann-Whitney U, Wilcoxon signed-rank e Kruskal-Wallis) e, em seguida, por estimativas \emph{bootstrap} par-a-par (Hedges' g e \textit{Rank-Biserial Correlation});
  \item \textbf{Organização topológica do efeito da neuromodulação} avaliada por métricas de teoria de grafos e centralidade.
\end{enumerate}

Esses domínios metodológicos são detalhados nos Capítulos~\ref{chap:6_metodos_de_analise_de_sincronizacao_de_fase}-\ref{chap:analise_centralidade_de_grafos}. A seguir apresentamos os achados, distinguindo a fase \emph{macro} (testes globais sobre o conjunto completo de amostras) da fase \emph{micro} (\emph{bootstrap} par-a-par, cujas análises gráficas estão nos Capítulos \ref{chap:analise_de_rede} e \ref{chap:analise_centralidade_de_grafos}, e os dados completos estão no repositório público\footnote{\url{https://github.com/dantebarross/efeito-da-neuromodulacao-na-sincronicidade-eeg-ecg}}).

%-------------------------------------------------------------------
\section{Análise de Sincronização de Fase}
%-------------------------------------------------------------------

\subsection{Conectividade intrafrequencial (\emph{EEG-EEG})}

Os testes globais (Mann-Whitney U/Kruskal-Wallis) apontaram:
\begin{itemize}
  \item alpha, delta, theta e gamma: $p<\alpha_{\mathrm{corr}}=\mathbf{0{,}005}$;
  \item beta: efeito marginalmente não-significativo ($p=0{,}052$).
\end{itemize}

\inputtable{tabelas/nonparametric_tests_results.tex}
{Resultados dos testes não-paramétricos (Mann-Whitney U, Wilcoxon signed-rank e Kruskal-Wallis) por faixa de frequência e grupo de canais}
{tab:nonparametric_results}
{Elaborado pelo autor (2025). Nota: * indica significância estatística ($p < \alpha_{\mathrm{corr}}$).}

Em todas as faixas em que houve significância global (alpha, delta, theta, gamma), a análise \emph{bootstrap} confirmou robustez do sinal e permitiu estimar:
\begin{itemize}
  \item \textbf{Alpha:} aumento da diferença Pós–Pré sob catódica (RBC médio $\approx+1$);
  \item \textbf{Delta \& Gamma:} diminuição da sincronia sob catódica (RBC médio $\approx-1$).
\end{itemize}

%-------------------------------------------------------------------
\subsection{Acoplamento \emph{cross‐frequency} (\emph{EEG–ECG})}
%-------------------------------------------------------------------

Nos testes globais, apenas beta e delta atingiram $p<0{,}005$ (gamma marginalmente não-significativo, $p\approx0{,}00505$); alpha e theta ficaram fora do limiar.

A fase \emph{micro} (\emph{bootstrap}) confirmou:
\begin{itemize}
  \item \textbf{Beta e Delta:} redução consistente do acoplamento catódica \textit{versus} sham (RBC médio $\approx+0.9$);
  \item \textbf{Gamma:} redução moderada (RBC $\approx+1$).
\end{itemize}

%-------------------------------------------------------------------
\subsection{Pipeline Estatístico}
%-------------------------------------------------------------------

\subsubsection{Fase 1 - Testes Globais} 
Aplicamos Mann-Whitney U e Kruskal-Wallis para amostras independentes e Wilcoxon signed-rank para amostras pareadas, com correção de Bonferroni em função dos 10 testes totais (2 grupos × 5 bandas) ($\alpha_{\mathrm{corr}}=0{,}005$). Resultados completos na Tabela \ref{tab:nonparametric_results}. Aqui está a síntese dos resultados globais:
\begin{itemize}
    \item A HD-tDCS catódica \emph{potenciou} significativamente a sincronia alpha em EEG-EEG e \emph{reduziu} sincronia em delta/gamma.
    \item Em EEG-ECG, houve \emph{redução} de acoplamento em beta, delta e gamma.
    \item Os testes globais e o \textit{bootstrap} convergiram nas direções e magnitudes de efeito (RBC próximos a \(\pm1\)).
  \end{itemize}

\subsubsection{Fase 2 - \textit{Bootstrap} Par-a-Par} 
Nessa etapa, para cada par de canais realizamos 10.000 reamostragens (\textit{bootstrap}) BCa aceleradas por GPU, a partir das quais extraímos a diferença média Pós–Pré (\(\bar{\Delta}\)), o \textit{bias} da estimativa, o erro-padrão (\(\mathrm{SE}\)) e o intervalo de confiança BCa a 95\%. Além disso, calculamos a estatística de Wilcoxon (\(W\)) e sua correlação bisserial de postos (RBC), o tamanho de efeito de Hedges' \(g\) corrigido para amostras pequenas, o p-valor empírico em duas faces gerado pelo bootstrap. Utilizamos Bonferroni, Holm e FDR-BH como métodos de correção; para as análises de redes e medidas de centralidade, adotamos Bonferroni. Os resultados par-a-par completos (54.900 pares EEG-EEG e 1.830 pares EEG-ECG) estão no repositório.

\section{Discussão Integrada}
% --- síntese entre intrafreq e cross-freq ---
Em ambas as análises, intrafrequencial (PLI) e \textit{cross-frequency} (CF-PLM), o ritmo alpha emergiu como o mais sensível à HD-tDCS catódica, tanto estatisticamente quanto em tamanho de efeito. Já a banda beta mostrou um padrão heterogêneo, com significância marginal nos testes globais mas efeitos moderados em pares selecionados. Essa dissociação reforça a ideia de que diferentes circuitos e escalas de acoplamento (local \textit{versus} corpo-coração) respondem de maneira distinta à neuromodulação, possivelmente por diferenças em sua arquitetura anatômica e fisiologia de geração de oscilação.

% --- implicações de rede e hubs ---
Na construção da rede, cada aresta reflete o quanto a estimulação catódica alterou a sincronia em relação ao sham (diferença cathodic-sham) após 10.000 reamostragens BCa e correção de Bonferroni. Ou seja, não se trata de uma rede da sincronia basal em repouso, mas de uma rede das mudanças induzidas pela HD-tDCS.

Nas medidas de centralidade, os canais de maior centralidade correspondem, portanto, àqueles que participaram de um maior número de pares em que o efeito da HD-tDCS sobre a sincronia foi mais pronunciado. Na banda \textbf{alpha} (8-13 Hz), o canal \textbf{Fp2} emergiu como hub de maior centralidade em todas as métricas (\textit{Betweenness}, \textit{Degree} e \textit{Eigenvector}), tanto com outliers quanto sem outliers, indicando que é justamente nessa região que as ocorrências de pares com maiores diferenças cathodic-sham foram mais frequentes.

Em outras faixas de frequência, observamos reorganizações na hierarquia de \textit{hubs}: por exemplo, \textbf{TPP7h} na banda \textbf{beta} e, no cenário sem remoção de outliers, \textbf{CP5} e \textbf{CP4} na banda \textbf{delta}.

% --- link com a literatura ---
Esses achados corroboram o conceito de que a HD-tDCS induz mudanças agudas e persistentes na sincronização cortical (ou seja, uma forma de plasticidade funcional induzida pela corrente contínua) conforme demonstrado por \citeonline{kunze2014high}, e o conceito de ``integração corpo-cérebro'' de \citeonline{criscuolo2022cognition}, ampliando-os ao contexto de atletas de elite em \emph{resting-state}.

% --- brainstorming quanto ao futuro ---
Futuros estudos poderão investigar se esses \emph{hubs} de modulação de sincronia, especialmente o reforço da sincronização alpha em Fp2 e o aumento do acoplamento cérebro-coração, traduzem-se em ganhos funcionais. Por exemplo, seria interessante testar se atletas que exibem maiores correlações bisseriais de postos (RBC) nesses canais também apresentam melhor desempenho em tarefas.

Além disso, protocolos de neuromodulação poderiam ser otimizados de forma individualizada, aplicando estimulação focalizada nesses \textit{hubs} e em faixas de frequência específicas, para potencializar processos neurofisiológicos associados à alta performance esportiva. Essa abordagem de \textit{stimulation-to-performance} alinha-se ao modelo Body-Brain Dynamic System (BBDS) de \citeonline{criscuolo2022cognition}, que postula que oscilações corticais lentas modulam a excitabilidade neural em sincronia com o ciclo cardíaco. Assim, ao intensificar seletivamente o acoplamento alfa e corpo-cérebro, a HD-tDCS personalizada poderia melhorar a alocação de recursos autonômicos em situações de alta demanda competitiva, abrindo caminho para intervenções neuromodulatórias de precisão no esporte de elite.

\subsection{Limitações e Considerações Metodológicas}
As sessões de estimulação (\emph{sham} e \emph{cathodic}) foram realizadas em dias distintos, o que pode introduzir variáveis confundidoras (estado fisiológico, cansaço, fatores ambientais). Embora tivéssemos padronizado horários e instruções, recomenda-se um desenho \emph{crossover} contrabalanceado em futuros estudos para minimizar esse viés.

O recrutamento de atletas de elite limitou o número de participantes, reduzindo a potência estatística e a possibilidade de generalizar os achados a populações não atléticas. Estudos posteriores devem incluir amostras maiores e grupos-controle não atletas.

Apesar do uso de ICA e filtros avançados, artefatos residuais (especialmente na banda gamma) não podem ser totalmente descartados. A análise paralela com e sem remoção de outliers mostrou estabilidade geral, mas sugere a necessidade de métodos ainda mais robustos de pré-processamento.

Observou-se que as bandas delta e gamma apresentaram leve sensibilidade à remoção de outliers (\(d\approx5\)\%) dos pares, indicando maior variabilidade intrínseca ou suscetibilidade a artefatos. Cautela é recomendada ao interpretar efeitos isolados nessas faixas.

Embora proporcione focalidade superior, o custo e a complexidade do HD-tDCS podem limitar sua adoção em laboratórios com recursos restritos. Comparações diretas com tDCS de esponja são necessárias para avaliar \textit{trade-offs} entre eficácia e viabilidade.