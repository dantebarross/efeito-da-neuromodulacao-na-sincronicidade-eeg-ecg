\chapter{Resultados e Discussões}
\label{chap:resultados_e_discussoes}
Este capítulo reúne os resultados obtidos ao longo deste estudo e promove uma discussão integrada sobre os efeitos da neuromodulação catódica, via HD-tDCS, na conectividade funcional de atletas de elite. A análise abrange a sincronização de fase (medida pelo PLI e pelo CF-PLM), os testes estatísticos não paramétricos, a análise de rede e a avaliação de centralidade dos grafos, conforme descrito nos Capítulos \ref{chap:metodos_de_analise_de_sincronizacao_de_fase}, \ref{chap:analise_distribuicao_normalidade}, \ref{chap:analise_estatistica_np}, \ref{chap:analise_de_rede} e \ref{chap:analise_centralidade_de_grafos}. A seguir, apresenta-se uma síntese dos principais achados, acompanhada de uma possível interpretação dos resultados.

\section{Análise de Sincronização de Fase}
A aplicação de métodos baseados na análise da fase (conforme detalhado no Capítulo \ref{chap:metodos_de_analise_de_sincronizacao_de_fase}) demonstrou que:
\begin{itemize}
    \item A estimulação catódica sobre o DLPFC produz alterações significativas na conectividade intrafrequencial (EEG--EEG), medida pelo PLI, com efeitos marcantes nas bandas \emph{alpha}, \emph{delta}, \emph{theta} e \emph{gamma}. A banda \emph{beta} apresentou resultados marginalmente significativos.
    \item A análise do acoplamento \emph{cross-frequency} entre EEG e ECG, obtida via CF-PLM, evidenciou diferenças notáveis, sobretudo nas bandas \emph{alpha}, \emph{beta} e \emph{delta}. Isso sugere que a neuromodulação pode modular a interação entre os ritmos corticais e o ciclo cardíaco.
\end{itemize}
Esses resultados corroboram a hipótese de que a intervenção neuromodulatória altera a dinâmica dos sistemas neural e cardiovascular, promovendo uma reorganização dos ritmos cerebrais e suas interações com processos fisiológicos.

\section{Testes Estatísticos e Estimativas de Efeito}
Como demonstrado pelos testes de normalidade (Tabela~\ref{tab:normality_tests}), as distribuições das métricas de conectividade apresentaram desvios significativos da normalidade, justificando a escolha de métodos não paramétricos para as análises subsequentes.

Como exposto no Capítulo \ref{chap:analise_estatistica_np}, a aplicação dos testes não paramétricos (\textit{Mann-Whitney U}, \textit{Wilcoxon signed-rank} e \textit{Kruskal-Wallis}) demonstrou que:
\begin{itemize}
    \item Nos pares EEG--EEG, as diferenças (pós--pré) foram estatisticamente significativas nas bandas \emph{alpha}, \emph{delta}, \emph{gamma} e \emph{theta}, com a banda \emph{beta} apresentando uma tendência marginal.
    \item Nos pares EEG--ECG, os testes indicaram diferenças significativas em bandas como \emph{alpha}, \emph{beta} e \emph{delta}. Embora o teste global (\textit{Kruskal-Wallis}) tenha mostrado variações na sensibilidade, o teste de \textit{Wilcoxon} reforçou a consistência das mudanças intraindividuais.
\end{itemize}
Adicionalmente, as estimativas de tamanho de efeito (como o \textit{Wilcoxon Rank-Biserial Correlation}, \textit{Cohen's d} e \textit{Hedges' g}) sugerem efeitos de magnitude moderada a alta em determinadas faixas, reforçando que as alterações observadas são não somente estatisticamente significativas, mas também relevantes do ponto de vista funcional.

\section{Análise de Rede e Conectividade Funcional}
A visualização dos grafos de conectividade (conforme ilustrado no Capítulo \ref{chap:analise_de_rede}) permitiu identificar os padrões espaciais das conexões significativas:
\begin{itemize}
    \item Os grafos baseados no PLI (EEG--EEG) revelaram que, na condição \emph{cathodic}, as conexões positivas (indicadas por valores de RBC $+1$) se distribuem de maneira contínua, especialmente na banda \emph{alpha}, com uma extensão que vai da região frontal até a occipital. Em contrapartida, na banda \emph{delta} predominam conexões negativas (RBC $-1$), sugerindo uma redução da sincronização em determinadas áreas.
    \item Nos grafos obtidos via CF-PLM (EEG--ECG), as conexões significativas foram invariavelmente positivas, indicando um aumento do acoplamento \emph{cross-frequency} sob a estimulação catódica.
\end{itemize}
A comparação entre as análises com e sem remoção de \textit{outliers} demonstrou que os padrões gerais se mantêm, confirmando a robustez dos achados.

\section{Análise de Centralidade de Grafos}
A análise de centralidade (Capítulo \ref{chap:analise_centralidade_de_grafos}) permitiu identificar os hubs e a hierarquia dos canais na rede:
\begin{itemize}
    \item Na banda \emph{alpha}, o canal \textbf{Fp2} destacou-se consistentemente como o nodo mais central, aparecendo com os maiores valores nas métricas de \textit{Betweenness}, \textit{Degree} e \textit{Eigenvector Centrality}. Tal resultado sugere que \textbf{Fp2} desempenha um papel crucial na mediação da comunicação entre as regiões cerebrais.
    \item Na banda \emph{beta}, canais como \textbf{TPP7h} e \textbf{F5} emergiram como hubs principais, refletindo uma reorganização da conectividade em frequências mais altas.
    \item As bandas \emph{delta} e \emph{gamma} evidenciaram maior sensibilidade à remoção de \textit{outliers}, embora a análise tenha apontado a relevância de regiões parietais e frontais, respectivamente.
\end{itemize}
Estes resultados enfatizam que a neuromodulação não só altera os níveis de sincronia, mas também modifica a topologia da rede, favorecendo a emergência de pontos de integração que podem ser fundamentais para a transmissão de informação.

\section{Discussão Integrada}
A partir dos achados apresentados, conclui-se que a intervenção neuromodulatória catódica sobre o DLPFC altera significativamente tanto a sincronização de fase quanto a organização da conectividade funcional. Essas alterações foram observadas nas conexões intrafrequenciais (EEG--EEG) e no acoplamento \textit{cross-frequency} (EEG--ECG), com efeitos mais marcantes nas bandas \emph{alpha}, \emph{delta} e \emph{theta} – enquanto a banda \emph{beta} apresenta alterações menos pronunciadas. Tais resultados sugerem uma reconfiguração dos caminhos de comunicação neural, evidenciada pela emergência de hubs (por exemplo, o canal \textbf{Fp2} na banda \emph{alpha}), corroborando a hipótese de que a neuromodulação modula a integração entre os sistemas neural e cardiovascular.


\section{Limitações e Considerações Metodológicas}
\label{sec:limitacoes}
Embora os resultados apresentados indiquem efeitos significativos da HD-tDCS catódica no DLPFC esquerdo sobre a conectividade funcional, este estudo possui algumas limitações que merecem consideração:

\begin{itemize}
    \item \textbf{Procedimentos de Estimulação em Dias Diferentes:} As sessões de estimulação \emph{sham} e catódica foram realizadas em dias distintos, o que pode introduzir variáveis confundidoras (como variações no estado fisiológico, psicológico ou condições ambientais) que influenciam os resultados. Apesar dessa limitação, a robustez de um desenho experimental bem controlado e da análise estatística empregada (com testes não paramétricos e métodos de correção para comparações múltiplas) contribui para mitigar parte da variabilidade natural dos dados.
    
    \item \textbf{Análise em Cenários com e sem \textit{Outliers}:} Para reforçar a confiabilidade dos achados, as análises foram conduzidas em dois cenários: considerando e excluindo \textit{outliers}. O pré-processamento cuidadoso dos dados de EEG – com filtragem rigorosa e avaliação dos componentes individuais via técnicas de ICA – buscou identificar e remover artefatos indesejáveis.
    
    \item \textbf{Acessibilidade da HD-tDCS:} Embora a \emph{HD-tDCS} proporcione uma estimulação mais focalizada, sua implementação requer equipamentos específicos e consideravelmente mais custoso, o que pode representar uma limitação para a reprodutibilidade do estudo em outros laboratórios. Em comparação com a tDCS clássica, a \emph{HD-tDCS} é menos acessível.
    
    \item \textbf{Amostra de Sujeitos:} A realização do experimento com atletas de elite envolve desafios inerentes à disponibilidade e ao recrutamento desses indivíduos, o que pode limitar o tamanho amostral e a generalização dos resultados. A especificidade desse grupo, embora conforte a relevância do estudo para contextos de alta performance, também implica que as conclusões possam não ser diretamente extrapoláveis para populações mais amplas.
\end{itemize}

Esses pontos devem ser considerados na interpretação dos resultados, e futuras pesquisas poderão buscar estratégias para minimizar ainda mais tais viéses, por exemplo, ampliando a amostra com a inclusão de um maior número de sujeitos.