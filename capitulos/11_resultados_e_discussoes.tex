\chapter{Resultados e Discussões}
\label{chap:resultados_e_discussoes}

A compreensão dos mecanismos neurobiológicos subjacentes aos efeitos da neuromodulação não-invasiva é uma das fronteiras atuais da neurociência aplicada ao esporte de alto rendimento. Neste capítulo, retomamos os principais achados apresentados nas análises anteriores (Capítulos~\ref{chap:analise_de_rede} e~\ref{chap:analise_centralidade_de_grafos}), discutindo suas possíveis interpretações funcionais. Os dados completos estão disponíveis no repositório público \cite{barros2025repository}\footnote{\url{https://github.com/dantebarross/efeito-da-neuromodulacao-na-sincronicidade-eeg-ecg}}.

A apresentação dos resultados segue três fases analíticas:

\begin{itemize}
  \item \textbf{Fase 1 - Testes estatísticos globais (macro)}: comparações das diferenças \textbf{pós-pré} entre \textit{cathodic} e \textit{sham} em cada banda.
  \item \textbf{Fase 2 - Análise \textit{bootstraped} par a par (micro)}: identificação dos pares de canais com efeitos consistentes.
  \item \textbf{Fase 3 - Centralidade de grafos (micro)}: hierarquia de canais nos grafos de efeito.
\end{itemize}

\section{Fase 1 - Testes Globais (Mann-Whitney, Kruskal-Wallis e Wilcoxon)}

Todas as análises globais (Fase 1) foram feitas \textit{com} e \textit{sem} remoção de outliers pelo método ECOD (ver Seção~\ref{chap:analise_estatistica_np}). Para as métricas EEG-EEG (PLV e PLI) houve pequenas diferenças entre cenários. Aqui apresentamos os resultados nas Tabelas~\ref{tab:nonparametric_results_EEG_EEG_with} e~\ref{tab:nonparametric_results_EEG_EEG_without}. Para EEG-ECG (CF-PLM), nenhum valor foi identificado como outlier (0\% removido), logo exibimos apenas a versão consolidada (Tabela~\ref{tab:nonparametric_results_EEG_ECG}). As linhas com asterisco representam significância após correção para múltiplas comparações.

Em cada métrica foram realizadas cinco comparações (uma por banda) e aplicada correção de Bonferroni
\[
  \alpha_{\mathrm{corr}} = \frac{0{,}05}{5} = 0{,}01.
\]

Resultados com \(p_{\mathrm{raw}} < \alpha_{\mathrm{corr}}\) são considerados significativos.

\clearpage
\subsection{EEG-EEG (PLV e PLI) — com \textit{versus} sem outliers}

\inputtable{tabelas/nonparametric_tests_results_EEG_EEG_with_outlier.tex}
{Resultados EEG-EEG \textit{com} outliers (PLV e PLI) por faixa de frequência}
{nonparametric_results_EEG_EEG_with}

\clearpage
\inputtable{tabelas/nonparametric_tests_results_EEG_EEG_without_outlier.tex}
{Resultados EEG-EEG \textit{sem} outliers (PLV e PLI) por faixa de frequência}
{nonparametric_results_EEG_EEG_without}

\clearpage
Comparando as Tabelas~\ref{tab:nonparametric_results_EEG_EEG_with} e~\ref{tab:nonparametric_results_EEG_EEG_without}, observamos:
\begin{itemize}
  \item \textbf{PLV (Wilcoxon) em alfa}: não significativo em ambas as condições (p$_\mathrm{corr}>0{,}01$).
  \item \textbf{PLV (Wilcoxon) em beta, delta, gamma e theta}: permanece significativamente diferente em ambos os cenários, com leve redução nos tamanhos de efeito sem outliers.
  \item \textbf{PLI (Wilcoxon) em beta}:
    \begin{itemize}
      \item Com outliers: não significativo (p$_\mathrm{corr}=0{,}663$).
      \item Sem outliers: torna-se significativo (p$_\mathrm{corr}=0{,}0147$, ES\,$\approx$\,0,034).
    \end{itemize}
    Indica que a remoção de outliers revelou um pequeno efeito em PLI na banda beta.
  \item \textbf{Mann-Whitney em delta (PLV e PLI)}:
    \begin{itemize}
      \item Com outliers: ES\,$\approx$\,0,187 (PLV) e ES\,$\approx$\,0,314 (PLI), com p$_\mathrm{corr}<0{,}01$.
      \item Sem outliers: ES reduzidos para $\approx$\,0,056 (PLV) e $\approx$\,0,156 (PLI), mantendo p$_\mathrm{corr}<0{,}01$.
    \end{itemize}
    Demonstra robustez das diferenças em delta, embora com tamanhos de efeito menores sem outliers.
\end{itemize}

Em geral, a remoção de aproximadamente 5\% dos pontos atípicos em EEG-EEG:
\begin{itemize}
  \item Reduziu moderadamente os tamanhos de efeito nas comparações.
  \item Evidenciou a diferença em PLI beta no teste de Wilcoxon.
  \item Confirmou a robustez das diferenças em delta, gamma e theta para PLV e PLI.
\end{itemize}

\clearpage
\subsection{EEG-ECG (CF-PLM)}

\inputtable{tabelas/nonparametric_tests_results_EEG_ECG_with_outlier.tex}
{Resultados EEG-ECG (CF-PLM) por faixa de frequência}
{nonparametric_results_EEG_ECG}

\noindent\emph{Nota:} A remoção de outliers via ECOD resultou em 0\% de exclusões nesse grupo, portanto os resultados apresentados correspondem a ambos os cenários (com e sem outliers).  

Como observado na Tabela~\ref{tab:nonparametric_results_EEG_ECG}, a análise conjunta dos três testes estatísticos (Kruskal, Mann-Whitney U e Wilcoxon) revelou:

\begin{itemize}
  \item \textbf{Alpha}:  
    \begin{itemize}
      \item Apenas o Wilcoxon foi significativo (p = 2.80e-04; p$_{\mathrm{corr}}$ = 1.40e-03; ES = -0.2192), indicando efeito negativo da catódica.
      \item Kruskal (p$_{\mathrm{corr}}$ = 0.235) e Mann-Whitney (p$_{\mathrm{corr}}$ = 0.235) não atingiram significância.
    \end{itemize}
  \item \textbf{Beta e delta}: diferenças significativas em todos os testes:
    \begin{itemize}
      \item Kruskal: beta p$_{\mathrm{corr}}$ = 1.66e-07 (ES = 0.0417), delta p$_{\mathrm{corr}}$ = 3.81e-27 (ES = 0.1637).  
      \item Mann-Whitney: beta p$_{\mathrm{corr}}$ = 1.66e-07 (ES = -0.2359), delta p$_{\mathrm{corr}}$ = 3.82e-27 (ES = -0.4671).  
      \item Wilcoxon: beta p$_{\mathrm{corr}}$ = 2.83e-08 (ES = -0.3514), delta p$_{\mathrm{corr}}$ = 1.63e-22 (ES = -0.5986).  
    \end{itemize}
    Confirma efeito negativo na condição catódica (valores mais baixos em \textit{cathodic} versus \textit{sham}), lembrando que o tamanho de efeito de Kruskal está indicando apenas magnitude.
  \item \textbf{Gamma}:  
    \begin{itemize}
      \item Significativo em Kruskal (p$_{\mathrm{corr}}$ = 2.53e-02; ES = 0.0108) e Mann-Whitney (p$_{\mathrm{corr}}$ = 2.53e-02; ES = -0.1197), indicando efeito negativo da catódica.
      \item Não significativo em Wilcoxon (p$_{\mathrm{corr}}$ = 5.48e-02).  
    \end{itemize}
  \item \textbf{Theta}:
    \begin{itemize}
      \item Sem diferenças significativas em nenhum dos testes (todos p$_{\mathrm{corr}}$ > 0,05).  
    \end{itemize}
\end{itemize}

Nas próximas fases (\textit{bootstrap} e centralidade) aprofundamos esses efeitos no nível par a par e topológico, usando os parâmetros mais robustos identificados aqui.  

%-------------------------------------------------------------------
\section{Fase 2 - Análise Bootstrap e Distribuições de Efeito}
%-------------------------------------------------------------------

Complementando os testes globais apresentados anteriormente, esta fase explorou os efeitos da neuromodulação com maior granularidade por meio de reamostragens \textit{bootstrap} BCa com 10.000 iterações. Foram avaliadas todas as combinações possíveis de pares de canais (EEG-EEG para PLI e EEG-ECG para CF-PLM), extraindo:
\begin{itemize}
  \item Intervalos de confiança corrigidos (BCa);
  \item Estatísticas de Wilcoxon pareado ($W$);
  \item Correlação bisserial de postos (RBC);
  \item Tamanho de efeito de Hedges' $g$;
  \item Correções múltiplas (Bonferroni, Holm e FDR-BH).
\end{itemize}

\subsection{Resultados Bootstrap - EEG-EEG (PLI)}

Os resultados do \textit{bootstrap} para conectividade intra-frequencial revelaram padrões distintos de modulação da sincronia cerebral:
\begin{itemize}
  \item A banda \textbf{alpha} exibiu um predomínio robusto de conexões com \textbf{RBC positivo}, sobretudo em trajetórias fronto-parietais diagonais (ex.: Fp2-PO8), indicando um \textbf{efeito positivo na sincronia de fase} induzido pela estimulação catódica com \textit{sham} em referência.
  \item Em \textbf{theta}, observaram-se padrões semelhantes, com conexões RBC +1 concentradas à esquerda e algumas RBC -1 na fronteira frontal direita, sugerindo uma modulação assimétrica.
  \item As bandas \textbf{delta} e \textbf{gamma} apresentaram \textbf{predomínio de conexões RBC -1}, com destaque para o canal Fp2 e áreas fronto-temporais, sugerindo \textbf{redução de sincronia} em regiões específicas sob a estimulação.
  \item A banda \textbf{beta} mostrou comportamento mais heterogêneo: conexões positivas e negativas coexistiram, com leve predominância de \textbf{RBC negativo} após remoção de outliers.
\end{itemize}

Esses resultados indicam que a HD-tDCS catódica não afeta uniformemente a sincronia entre regiões cerebrais, mas tende a \textbf{aumentar a conectividade nas bandas alpha e theta} e a \textbf{reduzir nas bandas delta e gamma}, com variações topológicas evidentes por banda.

\subsection{Resultados Bootstrap - EEG-ECG (CF-PLM)}

Como não foram identificados outliers neste grupo (0\% removido pelo ECOD), apresentamos apenas a versão consolidada, sem necessidade de comparação entre cenários. O acoplamento \textit{cross-frequency} entre sinais cerebrais e cardíacos revelou efeitos mais localizados:
\begin{itemize}
  \item A maioria das conexões com efeito significativo apresentou \textbf{RBC positivo}, sugerindo efeito positivo localizado da sincronicidade fásica EEG-ECG; entretanto, na banda \textbf{theta} o par significativo F6-ECG exibiu \textbf{RBC negativo}, indicando efeito negativo pontual nessa frequência.
  \item A banda \textbf{beta} destacou-se com múltiplas conexões com efeito significativo envolvendo a região parietal esquerda (CP1, CP3, CP5 e TPP7h, par a par com ECG).
  \item Em \textbf{delta}, observaram-se pares adicionais, com envolvimento de regiões frontais (Fp2-ECG, FC4-ECG), fronto-temporais (FTT8h) e parietais (CP3-ECG, P1-ECG).
  \item As bandas \textbf{theta} e \textbf{gamma} exibiram um único par significativo cada: F6-ECG (negativo) e CP3-ECG (positivo), respectivamente.
  \item A banda \textbf{alpha} não apresentou nenhuma conexão significativa.
\end{itemize}

O padrão unidirecional de RBC +1 sugere que a HD-tDCS catódica no DLPFC esquerdo causou um efeito positivo na sincronicidade de fase nos casos com significância, apesar do efeito negativo no único par da banda theta.

\subsection{Síntese dos Resultados \textit{Bootstraped} par a par}
Com base nos achados descritos, é possível sintetizar os seguintes pontos:
\begin{itemize}
  \item A HD-tDCS catódica produz \textbf{efeitos direcionais distintos por banda}. \textbf{Para EEG-EEG}, efeito positivo na sincronia para as faixas alpha e theta, e negativo nas faixas delta e gamma. \textbf{Para EEG-ECG}, sem efeito significativo em alpha, efeito positivo em beta, gamma (região centro-parietal) e delta (mais espalhado), e negativo em theta (F6).
  \item Os \textbf{valores de RBC tendem a ser extremos} (próximos de $+1$ ou $-1$), indicando modulações pronunciadas e robustas mesmo após correção por múltiplos testes.
  \item A \textbf{estrutura topológica das redes} obtidas via bootstrap revelou a formação de ``\textit{hubs} de modulação'' (canais que concentram pares significativamente alterados) e variou por banda e localização cortical.
  \item A remoção de \textit{outliers} pouco afetou a estrutura geral das redes, indicando que os achados são robustos e replicáveis.
\end{itemize}

As visualizações completas das distribuições de RBC, Hedges' $g$ e $p$-valores corrigidos encontram-se nos Capítulos~\ref{chap:analise_de_rede} e~\ref{chap:analise_centralidade_de_grafos}.

%-------------------------------------------------------------------
\section{Fase 3 - Centralidade de Grafos e Hierarquia Funcional}
\label{sec:fase3_centralidade}
%-------------------------------------------------------------------

Partindo da rede de pares \textbf{significativamente alterados} (ver Capítulo~\ref{chap:analise_de_rede}), calculamos as centralidades de \textit{Degree}, \textit{Betweenness} e \textit{Eigenvector}\footnote{Definições formais em \cite{rubinov2010complex,bullmore2009complex}; aplicação em EEG em \citeonline{lohmann2010eigenvector}.} em cada banda (delta-gamma), nos cenários \textbf{com} e \textbf{sem} outliers. Para \textit{Degree} e \textit{Betweenness} usamos o grafo binarizado (presença/ausência de aresta); para \textit{Eigenvector} empregámos o grafo ponderado pelos valores de RBC. Note-se que uma centralidade elevada pode refletir tanto aumento (RBC >\,0) quanto redução (RBC <\,0) de sincronia; quando pertinente indicamos essa predominância.

\subsection*{Síntese dos principais achados}
\begin{itemize}
    \item \textbf{Alpha -} \textit{Fp2} lidera todas as métricas tanto em cenário com e sem \textit{outliers}; \textit{PO8} e \textit{TP10} funcionam como pontes (\textit{betweenness}), enquanto \textit{FC3}/\textit{CP4}  (que também se mantém nos dois cenários) surgem como nós de segundo nível em \textit{eigenvector}, delineando um eixo frontal $\rightarrow$ centro-parietal $\rightarrow$ occipital.
    \item \textbf{Beta -} Com outliers, o núcleo de maior centralidade concentra-se no cluster centro-parietal (\textit{CP5-CP2-F5}); após a remoção de outliers, \textit{TPP7h} passa a liderar todas as métricas.
    \item \textbf{Delta -} O \textit{hub} interessantemente desloca-se de \textit{Fp2} (com outliers) para \textit{CP5/CP4} (sem outliers) nas três métricas.
    \item \textbf{Gamma -} Apesar de maior dispersão global, dois focos persistem: um cluster \textbf{fronto-temporal} (F3-FT10) que lidera \textit{degree} e \textit{eigenvector} em ambos os cenários, e nós centrais adicionais (C5, FTT8h); ao excluir \textit{outliers}, a malha de \textit{betweenness} torna-se mais compacta, com \textit{F3 $\rightarrow$ C5} concentrando as rotas mínimas.
    \item \textbf{Theta -} PO4 e FC1 dominam \textit{betweenness}. Em \textit{degree}, diversos outros aparecem na primeira hierarquia, porém a centralidade máxima é menor do que \textit{betwenness}, o que dificulta a diferenciação. Já em \textit{eigenvector}, com \textit{outliers}, há centralidade de F8, AF4 e FCz, e após remover \textit{outliers} a centralidade geral decresce, colocando também outros canais centrais, parietais e frontais em \textit{highlight}.
\end{itemize}

É de se notar que a remoção de \textit{outliers}, quando não mantém as primeiras hierarquias, fazem as medidas de centralidade decaírem e, com isso, novos canais ``subirem de importância''.

Estes resultados complementam as Fases 1 e 2: identificam não só \textbf{onde} os efeitos foram significativos, mas \textbf{quais} canais assumem papéis de \textit{hubs} locais (\textit{Degree}), pontes globais (\textit{Betweenness}) ou nós influentes ligados a outros \textit{hubs} (\textit{Eigenvector}), compondo um mapa funcional da modulação induzida pela HD-tDCS catódica.

\section{Síntese Integrada dos Resultados (macro e micro)}
% --- síntese entre intrafreq e cross-freq ---
Em intrafrequencial (EEG-EEG, PLI), na banda alpha todas as comparações macro foram altamente significativas, com redução média da sincronia em cathodic versus sham. Contudo, na análise \textit{bootstrap} de pares (micro), observou-se predomínio de conexões com RBC positivo em trajetórias fronto-parietal-occipital, e o canal Fp2 destacou-se como hub principal em \textit{Betweenness}, \textit{Degree} e \textit{Eigenvector}, indicando que certos pares aumentam a sincronia mesmo diante de uma redução média global.

Em \textit{cross-frequency} (EEG-ECG, CF-PLM), a banda alpha foi significativa apenas no Wilcoxon macro, enquanto Kruskal e Mann-Whitney não atingiram significância. Na análise \textit{bootstrap} de rede não houve conexões significativas, e consequentemente nenhum hub emergiu na centralidade, evidenciando que a modulação de alpha em CF-PLM se manifesta apenas em nível global e não em estruturas de pares locais.

Em intrafrequencial (EEG-EEG) na banda beta, o PLV apresentou diferenças nos três testes globais. O PLI, por sua vez, foi significativo apenas no Wilcoxon sem \textit{outliers}. A remoção de \textit{outliers} aumentou o tamanho de efeito para a mesma direção (positiva), de 0.1690 para 0.0340. Na análise micro (Fase 2), a rede beta exibiu conexões com RBC positivas e negativas dispersas, destacando pares envolvendo F5, CP5, CP1 e TPP7h. Na centralidade (Fase 3), os canais TPP7h e CP5 emergiram como hubs principais em \textit{Degree}, \textit{Betweenness} e \textit{Eigenvector}. 

Em \textit{cross-frequency} (EEG-ECG, CF-PLM), a beta mostrou efeitos negativos médios em todos os testes macro e, na análise micro par a par, todos os pares parietais (TPP7h, CP5, CP3, CP1) apresentaram RBC +1.

Em intrafrequencial (EEG-EEG, PLI e PLV), na banda delta todos os testes macro foram altamente significativos tanto com quanto sem remoção de \textit{outliers}, com tamanhos de efeito moderados a fortes. Na análise par a par (micro), observou-se predomínio de conexões RBC negativo em pares envolvendo Fp2, TTP8h e AF4, e na centralidade o canal Fp2 (com \textit{outliers}) e CP5/CP4 (sem \textit{outliers}) emergiram como hubs de \textit{Betweenness} e \textit{Degree}.

Em \textit{cross-frequency} (EEG-ECG, CF-PLM), na banda delta os testes macro também indicaram diferenças significativas, com tamanho de efeito pequeno em Kruskal e moderado-forte em Mann-Whitney e Wilcoxon, ambos apontando efeito negativo na sincronicidade. Contudo, no \textit{bootstrap} de rede, cinco pares parietais (CP3-ECG, P1-ECG, Fp2-ECG, FC4-ECG e FTT8h-ECG) exibiram RBC positivo, e na centralidade esses canais, sobretudo CP3 e CP5, destacaram-se como hubs, indicando que os maiores efeitos são positivos e localizados nessa região.

Em intrafrequencial (EEG-EEG, PLI e PLV), na banda gamma todos os testes macro mostraram alta significância após correção, com efeitos positivos de magnitude pequena a moderada. Na análise \textit{bootstrap} de pares (micro), predominou RBC negativo, com destaque para conexões de longa distância envolvendo os canais FT10 e TT8h, e presença pontual de RBC positivo em TPP8h e O1.  
Na centralidade (Fase 3), os hubs principais foram C5 em \textit{Betweenness}, FT10 em \textit{Degree}, e F3/FT10 em \textit{Eigenvector}, tanto com quanto sem \textit{outliers}.

Em \textit{cross-frequency} (EEG-ECG, CF-PLM), na banda gamma Kruskal e Mann-Whitney foram significativos, com esta indicando efeito baixo e negativo, enquanto Wilcoxon não atingiu significância. No \textit{bootstrap} de rede, apenas um par significativo (CP3-ECG) apresentou RBC +1.

Em intrafrequencial (EEG-EEG, PLI e PLV), na banda theta todos os testes macro foram altamente significativos, com direções opostas para PLI (efeito negativo baixo) e PLV (efeito positivo médio). Na análise \textit{bootstrap} de pares (micro), predominou RBC positivo, sobretudo na região esquerda, contrapondo-se a algumas conexões RBC negativo na fronteira fronto-central/direita. Na centralidade (fase 3), os principais hubs foram FC1 e PO4 em \textit{Betweenness}, C1 e C3 em \textit{Degree}, e F8 em \textit{Eigenvector}.

Em \textit{cross-frequency} (EEG-ECG, CF-PLM), na banda theta não houve diferenças significativas nos testes macro (todos p$_{\mathrm{corr}}>0{,}05$). No \textit{bootstrap} de rede surgiu apenas um par significativo (F6-ECG) com RBC negativo.

Vale recapitular que, na construção da rede, cada aresta reflete o quanto a estimulação catódica alterou a sincronia em relação ao \textit{sham} (diferença \textit{cathodic}-\textit{sham}) após 10.000 reamostragens BCa e correção de Bonferroni. Ou seja, não se trata de uma rede da sincronia basal em repouso, mas de uma rede das mudanças induzidas pela HD-tDCS.

Nas medidas de centralidade, os canais de maior centralidade correspondem, portanto, àqueles que participaram de um maior número de pares em que o efeito da HD-tDCS sobre a sincronia foi mais pronunciado. Na banda \textbf{alpha} (8-13 Hz), embora o PLV não tenha sido significativo no Wilcoxon macro (ao passo que o PLI o foi), a análise micro revelou efeito positivo e Fp2 emergiu como \textit{hub} principal em todas as métricas (\textit{Degree}, \textit{Betweenness} e \textit{Eigenvector}), tanto com quanto sem \textit{outliers}, indicando que essa região concentrou a maior quantidade de pares significativamente modulados pela HD-tDCS catódica.

Em outras faixas de frequência, observamos mudanças na hierarquia de \textit{hubs} ao comparar cenários com e sem \textit{outliers}: por exemplo, \textbf{TPP7h} passou a liderar as métricas na banda \textbf{beta} após a remoção de \textit{outliers}, enquanto \textbf{CP5} e \textbf{CP4} tornaram-se centrais na banda \textbf{delta}, substituindo o papel anterior de \textbf{Fp2}.

\section{Discussão Integrada}
% --- link com a literatura ---
Os achados deste estudo corroboram a ideia de que a HD-tDCS induz mudanças agudas e persistentes na sincronização cortical (ou seja, uma forma de plasticidade funcional induzida pela corrente contínua) conforme demonstrado por \citeonline{kunze2014high}, e no conceito de ``integração corpo-cérebro'' de \citeonline{criscuolo2022cognition}, ampliando-os ao contexto de atletas de elite em \textit{resting-state}.

Adicionalmente, estudos de base fisiológica mostram que a tDCS modula a excitabilidade cortical e reorganiza padrões espectrais em múltiplas bandas \cite{nitsche2000excitability}\citeonline{stagg2011physiological}, apoiando a robustez de nossas mudanças macro e micro.

Em nível cardíaco, potenciais evocados indicam que cada batimento pode resetar a fase de oscilações neocorticais, sugerindo um mecanismo para o acoplamento EEG-ECG observado \cite{purpura1965intracellular}. O estado autonômico e emocional também influencia fortemente a coerência cardíaca e a sincronia cérebro-coração, recomendando o monitoramento de variabilidade cardíaca e de humor em futuros protocolos \citeonline{okano2013estimulacao}.

Além disso, quedas em conectividade frontoparietal podem refletir aumento de eficiência neural em redes cognitivas, de acordo com a hipótese de ``Comunicação pela Coerência'' \citeonline{fries2015rhythms}, indicando que reduções de sincronia nem sempre representam prejuízo funcional.

% --- brainstorming quanto ao futuro ---
Futuros estudos deverão explorar se os \textit{hotspots} de acoplamento EEG-ECG identificados na análise de pares, em especial os pares parietais (TPP7h, CP5, CP3, CP1) que exibiram RBC $\approx +1$ na banda \textbf{beta}, estão relacionados a ganhos funcionais, por exemplo em controle autonômico ou desempenho motor. 

Vale salientar que, nos testes globais, a HD-tDCS catódica resultou em \textbf{reduções médias do acoplamento EEG-ECG} nas bandas \textbf{delta}, \textbf{beta} e \textbf{gamma} (por exemplo, \(\mathrm{RBC}\approx -0{,}47\) em delta); assim, os \textit{hubs} positivos que emergem no exame \textit{bootstraped} par a par representam efeitos positivos \textit{locais} que contrastam com essa tendência geral e merecem investigação funcional específica, como investigar se atletas com maior ou menor acoplamento nesses pares específicos apresentam diferença de performance (ou recuperação), podendo elucidar o papel funcional dessas conexões pontuais.

Além disso, protocolos de neuromodulação poderiam ser otimizados de forma individualizada, aplicando estimulação focalizada nesses \textit{hubs} e em faixas de frequência específicas, para potencializar processos neurofisiológicos associados à alta performance esportiva. Essa abordagem de \textit{stimulation-to-performance} alinha-se ao modelo Body-Brain Dynamic System (BBDS) de \citeonline{criscuolo2022cognition}, que postula que oscilações corticais lentas modulam a excitabilidade neural em sincronia com o ciclo cardíaco. Assim, ao intensificar seletivamente o acoplamento alfa e corpo-cérebro, a HD-tDCS personalizada poderia melhorar a alocação de recursos autonômicos em situações de alta demanda competitiva, abrindo caminho para intervenções neuromodulatórias de precisão no esporte de elite.

Como não usamos as coordenadas 10-20 reais para posicionar os eletrodos, os ``\textit{hubs}'' identificados pelas métricas de centralidade devem ser interpretados como eletrodos \textbf{sensoriais} centrais na rede funcional de diferenças, não necessariamente como regiões corticais específicas. Em outras palavras, tratam-se de \textit{hubs} funcionais em nível de sensor EEG.

\subsection{Limitações e Considerações Metodológicas}
As sessões de estimulação (\textit{sham} e \textit{cathodic}) foram realizadas em dias distintos, o que pode introduzir variáveis confundidoras (estado fisiológico, cansaço, fatores ambientais). Embora tivéssemos padronizado horários e instruções, recomenda-se um desenho \textit{crossover} contrabalanceado em futuros estudos para minimizar esse viés.

O recrutamento de atletas de elite limitou o número de participantes, reduzindo a potência estatística e a possibilidade de generalizar os achados a populações não atléticas. Estudos posteriores devem incluir amostras maiores e grupos-controle não atletas.

Apesar do uso de ICA e filtros avançados, artefatos residuais (especialmente na banda gamma) não podem ser totalmente descartados. A análise paralela com e sem remoção de \textit{outliers} mostrou estabilidade geral, mas sugere a necessidade de métodos ainda mais robustos de pré-processamento.

Observou-se que as bandas delta e gamma apresentaram leve sensibilidade à remoção de \textit{outliers} (\(d\approx5\)\%) dos pares, indicando maior variabilidade intrínseca ou suscetibilidade a artefatos. Cautela é recomendada ao interpretar efeitos isolados nessas faixas.

Embora proporcione focalidade superior, o custo e a complexidade do HD-tDCS podem limitar sua adoção em laboratórios com recursos restritos. Comparações diretas com tDCS de esponja são necessárias para avaliar \textit{trade-offs} entre eficácia e viabilidade.