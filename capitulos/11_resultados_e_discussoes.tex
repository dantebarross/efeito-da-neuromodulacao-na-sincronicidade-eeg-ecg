\chapter{Resultados e Discussões}
\label{chap:resultados_e_discussoes}

A compreensão dos mecanismos neurobiológicos subjacentes aos efeitos da neuromodulação não-invasiva é uma das fronteiras atuais da neurociência aplicada ao esporte de alto rendimento. Neste capítulo, retomamos os principais achados apresentados nas análises anteriores (Capítulos~\ref{chap:analise_de_rede} e~\ref{chap:analise_centralidade_de_grafos}), discutindo suas implicações funcionais e possíveis interpretações neurobiológicas. Os dados completos estão disponíveis no repositório público\footnote{\url{https://github.com/dantebarross/efeito-da-neuromodulacao-na-sincronicidade-eeg-ecg}}.

A apresentação dos resultados está organizada em \textbf{três fases analíticas complementares}, com crescente complexidade interpretativa:

\begin{itemize}
    \item \textbf{Fase 1 — Testes Estatísticos Globais:} comparação geral dos efeitos da catódica em relação à sham sobre as métricas de sincronização em diferentes bandas de frequência.
    \item \textbf{Fase 2 — Análise Bootstrap Par-a-Par:} identificação dos pares de canais com diferenças significativas na catódica em relação à sham..
    \item \textbf{Fase 3 — Análise de Centralidade:} construção de redes de centralidade baseadas nas relações par-a-par para os casos de efeitos significativos. Através dos resultados plotados, foi possível analisar a hierarquia de importância na centralidade de canais.
\end{itemize}

Importante destacar que as análises aqui não visam descrever o estado absoluto de sincronização. Em vez disso, investigamos o \textbf{efeito da estimulação HD-tDCS catódica em comparação à condição sham}, por meio de \textit{contrastes de diferença pré-pós}: para cada condição, calculou-se \emph{pós - pré}, e as análises estatísticas comparam essas variações entre as condições, globalmente na fase 1 e par-a-par nas fases 2 e 3 abaixo.

%-------------------------------------------------------------------
\section{Fase 1 — Testes Globais (Mann-Whitney, Kruskal-Wallis e Wilcoxon)}
%-------------------------------------------------------------------
Para cada métrica de conectividade, calculamos a diferença Pós-Pré por participante, comparando as condições \textit{cathodic} versus \textit{sham} com testes não-paramétricos. Os resultados foram corrigidos por múltiplas comparações (Bonferroni, $\alpha_{\mathrm{corr}}=0{,}005$).

\begin{itemize}
  \item \textbf{EEG-EEG:} A PLV indicou diferenças significativas em todas as bandas (alpha, beta, delta, gamma e theta). A PLI confirmou efeitos significativos em alpha, delta, gamma e theta, mas não em beta.
  \item \textbf{EEG-ECG:} O CF-PLM revelou significância em delta, beta e gamma (com $p$-valores corrigidos $< 0{,}005$), sendo a banda delta a mais robusta. Alpha e theta não atingiram significância global.
  \item Em relação aos tamanhos de efeito e suas direções: aumentos em alpha (positivo) e reduções em delta/gamma (negativo) para EEG-EEG; reduções generalizadas no CF-PLM (positivo, pois sham > cathodic).
\end{itemize}

\subsection{Resultados para Conectividade Intra-Frequencial (EEG-EEG)}

\inputtable{tabelas/nonparametric_tests_results_EEG_EEG.tex}
{Resultados dos testes não-paramétricos (PLV e PLI) por faixa de frequência — grupo EEG-EEG}
{nonparametric_results_EEG_EEG}
{Elaborado pelo autor (2025). Nota: * indica significância estatística ($p < \alpha_{\mathrm{corr}}=0{,}005$).}

Os testes não-paramétricos aplicados às métricas de conectividade intra-frequencial (\textit{median\_plv\_diff} e \textit{median\_pli\_diff}) revelaram padrões consistentes de modulação induzida pela estimulação catódica com HD-tDCS.

A métrica PLV apresentou diferenças significativas entre as condições \textit{cathodic} e \textit{sham} em todas as cinco faixas de frequência analisadas ($\alpha$, $\beta$, $\delta$, $\theta$, $\gamma$), conforme indicado pelos testes de Mann-Whitney U e Kruskal-Wallis (com $p < \alpha_{\mathrm{corr}}=0{,}005$). A direção do efeito pode ser inferida pelo sinal da correlação bisserial de postos (RBC) nos resultados de Mann-Whitney:
\begin{itemize}
  \item A banda $\alpha$ apresentou um leve efeito negativo (RBC = $-0{,}29$), sugerindo uma tendência de redução da conectividade sob a condição catódica.
  \item As bandas $\beta$, $\delta$, $\theta$ e $\gamma$ mostraram efeitos positivos (RBC entre $+0{,}57$ e $+1{,}00$), indicando um aumento da sincronia funcional sob catódica em relação à sham.
\end{itemize}

A métrica PLI também evidenciou significância estatística em quatro das cinco bandas ($\alpha$, $\delta$, $\theta$, $\gamma$), com exceção da banda $\beta$, que apresentou valores marginais (por exemplo, $p = 0{,}063$ no teste de Mann-Whitney). Os sinais dos efeitos, quando disponíveis, indicam:
\begin{itemize}
  \item Aumento de sincronia em $\alpha$, $\theta$ e $\delta$ sob catódica (efeitos positivos).
  \item Redução de sincronia em $\gamma$ (RBC = $-0{,}71$) e possível queda em $\beta$ (embora não significativa).
\end{itemize}

Os testes de Wilcoxon signed-rank (aplicados para avaliar mudanças pareadas entre Pré e Pós dentro de cada condição) corroboraram a presença de efeitos nas bandas significativas, embora, por construção, não permitam inferência da direção (só magnitude).

No geral, os resultados sugerem que a HD-tDCS catódica promove modulações seletivas da sincronia intra-frequencial: há indícios de aumento da conectividade funcional em faixas como $\delta$, $\theta$, $\beta$ e $\gamma$ segundo a PLV, e de aumento em $\alpha$ e $\theta$ segundo o PLI. A ausência de significância para a banda $\beta$ na PLI, contrastando com o achado positivo na PLV, pode refletir a maior sensibilidade da PLV a mudanças de fase, enquanto a PLI é mais conservadora por descartar acoplamentos de fase com atraso zero.

Esses achados indicam que, embora ambas as métricas captem aspectos parecidos da sincronização funcional, a PLV é mais sensível por não descartar acoplamentos de fase próximos a zero (que pode ou não significar ruído de \textit{volume conduction}).

\subsection{Resultados para Acoplamento Cross-Frequency (EEG-ECG)}

\inputtable{tabelas/nonparametric_tests_results_EEG_ECG.tex}
{Resultados dos testes não-paramétricos (CF‑PLM) por faixa de frequência — grupo EEG-ECG}
{nonparametric_results_EEG_ECG}
{Elaborado pelo autor (2025). Nota: * indica significância estatística ($p < \alpha_{\mathrm{corr}}=0{,}01$).}

Os testes não paramétricos aplicados à métrica \texttt{median\_cf\_plm\_diff} — que quantifica o acoplamento entre sinais de EEG e ECG em diferentes faixas de frequência — revelaram padrões específicos de modulação induzida pela HD-tDCS catódica. Utilizou-se o teste de Mann-Whitney U com correção de Bonferroni para múltiplas comparações ($\alpha_{\mathrm{corr}}=0{,}01$). Foram observados os seguintes achados principais:

\begin{itemize}
  \item \textbf{Delta e Beta} apresentaram efeitos altamente significativos ($p < 0{,}01$ após correção), com tamanhos de efeito negativos de magnitude moderada a elevada (respectivamente $-0{,}467$ e $-0{,}236$), sugerindo uma redução do acoplamento cérebro-coração (CF-PLM) na condição catódica em relação ao sham.
  \item \textbf{Gamma} também demonstrou significância marginal ($p = 0{,}005$; $p_{\text{corr}} = 0{,}025$), com tamanho de efeito negativo ($-0{,}120$), indicando tendência de redução do acoplamento, embora não atingindo o limiar de significância após correção múltipla.
  \item \textbf{Alpha e Theta} não apresentaram diferenças significativas ($p > 0{,}05$), com tamanhos de efeito pequenos (respectivamente $-0{,}085$ e $-0{,}025$), sugerindo que essas bandas não foram substancialmente moduladas pela estimulação catódica.
\end{itemize}

Em conjunto, os resultados reforçam a sensibilidade do acoplamento cross-frequency (EEG-ECG) nas bandas \textbf{Delta} e \textbf{Beta} à modulação neuromodulatória, com efeito mais fraco porém presente em \textbf{Gamma}. A direção negativa dos efeitos indica que, em média, a condição catódica esteve associada a uma diminuição da coerência de fase entre sinais neurais e cardíacos.

%-------------------------------------------------------------------
\section{Fase 2 — Análise Bootstrap e Distribuições de Efeito}
%-------------------------------------------------------------------

Complementando os testes globais apresentados anteriormente, esta fase explorou os efeitos da neuromodulação com maior granularidade por meio de reamostragens \textit{bootstrap} BCa com 10.000 iterações. Foram avaliadas todas as combinações possíveis de pares de canais (EEG-EEG para PLI e EEG-ECG para CF-PLM), extraindo:
\begin{itemize}
  \item Intervalos de confiança corrigidos (BCa);
  \item Estatísticas de Wilcoxon pareado ($W$);
  \item Correlação bisserial de postos (RBC);
  \item Tamanho de efeito de Hedges' $g$;
  \item Correções múltiplas (Bonferroni, Holm e FDR-BH).
\end{itemize}

\subsection{Resultados Bootstrap — EEG-EEG (PLI)}

Os resultados do \textit{bootstrap} para conectividade intra-frequencial revelaram padrões distintos de modulação da sincronia cerebral:

\begin{itemize}
  \item A banda \textbf{alpha} exibiu um predomínio robusto de conexões com \textbf{RBC positivo}, sobretudo em trajetórias fronto-parietais diagonais (ex.: Fp2-PO8), indicando um \textbf{aumento da sincronia de fase} induzido pela estimulação catódica.
  \item Em \textbf{theta}, observaram-se padrões semelhantes, com conexões RBC +1 concentradas à esquerda e algumas RBC -1 na fronteira frontal direita, sugerindo uma modulação assimétrica.
  \item As bandas \textbf{delta} e \textbf{gamma} apresentaram \textbf{predomínio de conexões RBC -1}, com destaque para o canal Fp2 e áreas fronto-temporais, sugerindo \textbf{redução de sincronia} em regiões específicas sob a estimulação.
  \item A banda \textbf{beta} mostrou comportamento mais heterogêneo: conexões positivas e negativas coexistiram, com leve predominância de \textbf{RBC negativo} após remoção de outliers.
\end{itemize}

Esses resultados indicam que a HD-tDCS catódica não afeta uniformemente a sincronia entre regiões cerebrais, mas tende a \textbf{aumentar a conectividade nas bandas alpha e theta} e a \textbf{reduzir nas bandas delta e gamma}, com variações topológicas evidentes por banda.

\subsection{Resultados Bootstrap — EEG-ECG (CF-PLM)}

O acoplamento \textit{cross-frequency} entre sinais cerebrais e cardíacos revelou efeitos mais localizados:

\begin{itemize}
  \item Todas as conexões significativas identificadas apresentaram \textbf{RBC positivo}, sugerindo um \textbf{aumento consistente no acoplamento cérebro-coração} induzido pela estimulação catódica.
  \item A banda \textbf{beta} destacou-se com múltiplas conexões significativas envolvendo a região parietal esquerda (CP1, CP3, CP5 e TPP7h).
  \item Em \textbf{delta}, observaram-se pares adicionais, com envolvimento de regiões frontais (Fp2, FC4), fronto-temporais (FTT8h) e parietais (CP3, P1).
  \item As bandas \textbf{theta} e \textbf{gamma} exibiram um único par significativo cada (F6-ECG e CP3-ECG, respectivamente).
  \item A banda \textbf{alpha} não apresentou nenhuma conexão significativa.
\end{itemize}

O padrão unidirecional de RBC +1 sugere que a neuromodulação intensifica seletivamente o acoplamento EEG-ECG nas faixas de frequência com significância, possivelmente refletindo mecanismos autonômicos de integração corpo-cérebro.

\subsection{Síntese dos Resultados \textit{Bootstraped} Par-a-par}

Com base nos achados descritos, é possível sintetizar os seguintes pontos:

\begin{itemize}
  \item A HD-tDCS catódica produz \textbf{efeitos direcionais distintos por banda}: sincronia aumentada nas faixas alpha e theta, e reduzida nas faixas delta e gamma (para EEG-EEG); aumento consistente de acoplamento em todas as faixas com efeito significativo (para EEG-ECG).
  \item Os \textbf{valores de RBC tendem a ser extremos} (próximos de $+1$ ou $-1$), indicando modulações pronunciadas e robustas mesmo após correção por múltiplos testes.
  \item A \textbf{estrutura topológica das redes} obtidas via bootstrap revelou a formação de ``\textit{hubs} de modulação'' (canais que concentram pares significativamente alterados) e variou por banda e localização cortical.
  \item A remoção de \textit{outliers} pouco afetou a estrutura geral das redes, indicando que os achados são robustos e replicáveis.
\end{itemize}

As visualizações completas das distribuições de RBC, Hedges' $g$ e $p$-valores corrigidos encontram-se nos Capítulos~\ref{chap:analise_de_rede} e~\ref{chap:analise_centralidade_de_grafos}.


\section{Fase 3 — Centralidade de Grafos e Hierarquia Funcional}
\label{sec:fase3_centralidade}

Nesta fase, partimos dos resultados estatísticos da Fase 2 para calcular medidas de centralidade utilizando as redes de efeito da HD-tDCS na sincronicidade de fase baseadas no índice PLI (EEG-EEG), onde cada aresta representa a magnitude (positiva ou negativa) do efeito da estimulação catódica em comparação ao sham, quantificada pelo \textit{rank-biserial correlation} (RBC). Com base nessas redes ponderadas, estimamos três métricas clássicas de centralidade (\textit{Degree}, \textit{Betweenness} e \textit{Eigenvector}) para investigar a organização hierárquica dos canais cerebrais como \textit{hubs} funcionais, e também explorar o efeito da remoção de \textit{outliers} na estrutura da rede.

A análise foi conduzida separadamente para cada banda de frequência, considerando dois cenários: com e sem remoção de \textit{outliers}. Canais com valores elevados de centralidade aparecem destacados em vermelho e com maior tamanho nos grafos. Embora esta análise não avalie diretamente a direção das conexões, a combinação com os resultados da Fase 2 permite inferir possíveis padrões de entrada e saída de informação.

Os resultados da Fase 3 reforçam e complementam os achados da Fase 2, oferecendo uma perspectiva topológica de centralidade sobre o efeito da HD-tDCS catódica no DLPFC esquerdo em comparação à sham. Os principais pontos observados incluem:

\begin{itemize}
    \item \textbf{Fp2} foi o canal mais central na banda alpha em todas as métricas, sugerindo forte modulação frontal mesmo após remoção de \textit{outliers}, em linha com os efeitos esperados da estimulação no DLPFC esquerdo.
    \item Na banda beta, \textbf{TPP7h} emergiu como \textit{hub} dominante em quase todas as métricas, indicando um possível redirecionamento da influência funcional para regiões temporo-parietais.
    \item A banda delta apresentou mudança marcante: \textbf{Fp2} foi o mais central com \textit{outliers}, mas os canais parietais \textbf{CP5} e \textbf{CP4} assumiram o protagonismo sem \textit{outliers}, apontando para uma redistribuição mais estável de centralidade nessas regiões.
    \item Nas bandas gamma e theta, houve maior variabilidade. Ainda assim, canais como \textbf{F3}, \textbf{FT10} e \textbf{TP9} se destacaram, com reorganizações relevantes entre os cenários. Esses padrões podem refletir a sensibilidade das redes de alta frequência à variabilidade interindividual.
    \item A remoção de \textit{outliers} alterou a ordem de hierarquia em algumas bandas, mas não mudou os principais \textit{hubs}, evidenciando a robustez dos achados.
\end{itemize}

\section{Discussão Integrada}
% --- síntese entre intrafreq e cross-freq ---
Em ambas as análises, intrafrequencial (PLI) e \textit{cross-frequency} (CF-PLM), o ritmo alpha emergiu como o mais sensível à HD-tDCS catódica, tanto estatisticamente quanto em tamanho de efeito. Já a banda beta mostrou um padrão heterogêneo, com significância marginal nos testes globais mas efeitos moderados em pares selecionados. Essa dissociação reforça a ideia de que diferentes circuitos e escalas de acoplamento (local \textit{versus} corpo-coração) respondem de maneira distinta à neuromodulação, possivelmente por diferenças em sua arquitetura anatômica e fisiologia de geração de oscilação.

% --- implicações de rede e hubs ---
Na construção da rede, cada aresta reflete o quanto a estimulação catódica alterou a sincronia em relação ao sham (diferença cathodic-sham) após 10.000 reamostragens BCa e correção de Bonferroni. Ou seja, não se trata de uma rede da sincronia basal em repouso, mas de uma rede das mudanças induzidas pela HD-tDCS.

Nas medidas de centralidade, os canais de maior centralidade correspondem, portanto, àqueles que participaram de um maior número de pares em que o efeito da HD-tDCS sobre a sincronia foi mais pronunciado. Na banda \textbf{alpha} (8-13 Hz), o canal \textbf{Fp2} emergiu como hub de maior centralidade em todas as métricas (\textit{Betweenness}, \textit{Degree} e \textit{Eigenvector}), tanto com outliers quanto sem outliers, indicando que é justamente nessa região que as ocorrências de pares com maiores diferenças cathodic-sham foram mais frequentes.

Em outras faixas de frequência, observamos reorganizações na hierarquia de \textit{hubs}: por exemplo, \textbf{TPP7h} na banda \textbf{beta} e, no cenário sem remoção de outliers, \textbf{CP5} e \textbf{CP4} na banda \textbf{delta}.

% --- link com a literatura ---
Esses achados corroboram o conceito de que a HD-tDCS induz mudanças agudas e persistentes na sincronização cortical (ou seja, uma forma de plasticidade funcional induzida pela corrente contínua) conforme demonstrado por \citeonline{kunze2014high}, e o conceito de ``integração corpo-cérebro'' de \citeonline{criscuolo2022cognition}, ampliando-os ao contexto de atletas de elite em \emph{resting-state}.

% --- brainstorming quanto ao futuro ---
Futuros estudos poderão investigar se esses \emph{hubs} de modulação de sincronia, especialmente o reforço da sincronização alpha em Fp2 e o aumento do acoplamento cérebro-coração, traduzem-se em ganhos funcionais. Por exemplo, seria interessante testar se atletas que exibem maiores correlações bisseriais de postos (RBC) nesses canais também apresentam melhor desempenho em tarefas.

Além disso, protocolos de neuromodulação poderiam ser otimizados de forma individualizada, aplicando estimulação focalizada nesses \textit{hubs} e em faixas de frequência específicas, para potencializar processos neurofisiológicos associados à alta performance esportiva. Essa abordagem de \textit{stimulation-to-performance} alinha-se ao modelo Body-Brain Dynamic System (BBDS) de \citeonline{criscuolo2022cognition}, que postula que oscilações corticais lentas modulam a excitabilidade neural em sincronia com o ciclo cardíaco. Assim, ao intensificar seletivamente o acoplamento alfa e corpo-cérebro, a HD-tDCS personalizada poderia melhorar a alocação de recursos autonômicos em situações de alta demanda competitiva, abrindo caminho para intervenções neuromodulatórias de precisão no esporte de elite.

\subsection{Limitações e Considerações Metodológicas}
As sessões de estimulação (\emph{sham} e \emph{cathodic}) foram realizadas em dias distintos, o que pode introduzir variáveis confundidoras (estado fisiológico, cansaço, fatores ambientais). Embora tivéssemos padronizado horários e instruções, recomenda-se um desenho \emph{crossover} contrabalanceado em futuros estudos para minimizar esse viés.

O recrutamento de atletas de elite limitou o número de participantes, reduzindo a potência estatística e a possibilidade de generalizar os achados a populações não atléticas. Estudos posteriores devem incluir amostras maiores e grupos-controle não atletas.

Apesar do uso de ICA e filtros avançados, artefatos residuais (especialmente na banda gamma) não podem ser totalmente descartados. A análise paralela com e sem remoção de outliers mostrou estabilidade geral, mas sugere a necessidade de métodos ainda mais robustos de pré-processamento.

Observou-se que as bandas delta e gamma apresentaram leve sensibilidade à remoção de outliers (\(d\approx5\)\%) dos pares, indicando maior variabilidade intrínseca ou suscetibilidade a artefatos. Cautela é recomendada ao interpretar efeitos isolados nessas faixas.

Embora proporcione focalidade superior, o custo e a complexidade do HD-tDCS podem limitar sua adoção em laboratórios com recursos restritos. Comparações diretas com tDCS de esponja são necessárias para avaliar \textit{trade-offs} entre eficácia e viabilidade.