\chapter{Resultados e Discussões}
\label{chap:resultados_e_discussoes}

A compreensão dos mecanismos neurobiológicos subjacentes aos efeitos da neuromodulação não-invasiva é uma das fronteiras atuais da neurociência aplicada ao esporte de alto rendimento. Neste capítulo, retomamos os principais achados apresentados nas análises anteriores (Capítulos~\ref{chap:analise_de_rede} e~\ref{chap:analise_centralidade_de_grafos}), discutindo suas possíveis interpretações funcionais. Os dados completos estão disponíveis no repositório público\footnote{\url{https://github.com/dantebarross/efeito-da-neuromodulacao-na-sincronicidade-eeg-ecg}}.

A apresentação dos resultados segue três fases analíticas:

\begin{itemize}
  \item \textbf{Fase 1 - Testes Estatísticos Globais (macro)}: comparações das diferenças \textbf{pós-pré} entre \textit{cathodic} e \textit{sham} em cada banda.
  \item \textbf{Fase 2 - Análise \textit{bootstrap} par a par (micro)}: identificação dos pares de canais com efeitos consistentes.
  \item \textbf{Fase 3 - Centralidade de Grafos (micro)}: hierarquia de canais nos grafos de efeito.
\end{itemize}

Todas as análises globais (Fase 1) foram feitas \textit{com} e \textit{sem} remoção de outliers pelo método ECOD (ver Seção~\ref{chap:analise_estatistica_np}), para avaliar a robustez dos achados. Para as métricas EEG-EEG (PLV e PLI) houve pequenas diferenças entre cenários. Aqui apresentamos os dois resultados lado a lado. Para EEG-ECG (CF-PLM), nenhum valor foi identificado como outlier (0\% removido), logo exibimos só a versão consolidada (Tabela~\ref{tab:nonparametric_results_EEG_ECG}).

\section{Fase 1 - Testes Globais (Mann-Whitney, Kruskal-Wallis e Wilcoxon)}

Em cada métrica foram realizadas cinco comparações (uma por banda) e aplicada correção de Bonferroni
\[
  \alpha_{\mathrm{corr}} = \frac{0{,}05}{5} = 0{,}01.
\]
Resultados com \(p_{\mathrm{raw}} < \alpha_{\mathrm{corr}}\) são considerados significativos.

\subsection{EEG-EEG (PLV e PLI) - com \textit{versus} sem outliers}

\inputtable{tabelas/nonparametric_tests_results_EEG_EEG_with_outlier.tex}
{Resultados EEG-EEG \textit{com} outliers (PLV e PLI) por faixa de frequência}
{nonparametric_results_EEG_EEG_with}
{Elaborado pelo autor (2025). Nota: * indica $p<p_{\mathrm corr}$.}

\inputtable{tabelas/nonparametric_tests_results_EEG_EEG_without_outlier.tex}
{Resultados EEG-EEG \textit{sem} outliers (PLV e PLI) por faixa de frequência}
{nonparametric_results_EEG_EEG_without}
{Elaborado pelo autor (2025). Nota: * indica $p<p_{\mathrm corr}$.}

Comparando as versões:
\begin{itemize}
  \item \textbf{Wilcoxon em beta:}  
    \begin{itemize}
      \item \emph{Com outliers:} $p_{\mathrm corr}<0{,}01$ (significativo).  
      \item \emph{Sem outliers:} $p_{\mathrm corr}<0{,}01$ (significativo).  
    \end{itemize}
    Indica que o efeito em beta é robusto e não depende de valores extremos.
  \item \textbf{Wilcoxon em delta e theta:}  
    \begin{itemize}
      \item \emph{Com outliers:} $\mathrm{RBC}_{\text{delta}} \approx 0{,}11$, $\mathrm{RBC}_{\text{theta}} \approx 0{,}13$.
      \item \emph{Sem outliers:} $\mathrm{RBC}_{\text{delta}} \approx 0{,}38$, $\mathrm{RBC}_{\text{theta}} \approx 0{,}31$.
    \end{itemize}
    A remoção de outliers fortaleceu muito os efeitos em delta e theta.
  \item \textbf{Mann-Whitney em delta:}  
    \begin{itemize}
      \item \emph{Com outliers:} $\mathrm{RBC}_{\text{delta}} \approx 0{,}16$, mantendo $p_{\mathrm{corr}} < 0{,}01$.
      \item \emph{Sem outliers:} $\mathrm{RBC}_{\text{delta}} \approx 0{,}31$, mantendo $p_{\mathrm{corr}} < 0{,}01$.
    \end{itemize}
    Mostra que o efeito intragrupal em delta se intensifica sem perder significância.
\end{itemize}

Em suma, a remoção de cerca de 5\% dos pontos atípicos em EEG-EEG tornou alguns efeitos (em delta, theta) mais claros e revelou a fragilidade do achado em beta com Wilcoxon.

\subsection{EEG-ECG (CF-PLM)}

\inputtable{tabelas/nonparametric_tests_results_EEG_ECG_with_outlier.tex}
{Resultados EEG-ECG (CF-PLM) por faixa de frequência}
{nonparametric_results_EEG_ECG}
{Elaborado pelo autor (2025). Nota: * indica $p<p_{\mathrm corr}=0{,}01$.}

Para EEG-ECG não houve remoção de outliers via ECOD (0\% dos dados), e os resultados permanecem idênticos aos já reportados na Seção~\ref{chap:analise_estatistica_np}: significância em delta, beta e gamma, todos com efeitos negativos (\textit{sham} $>$ \textit{cathodic}).

\bigskip
\noindent Nas próximas fases (bootstrap e centralidade) aprofundamos esses efeitos no nível par a par e topológico, usando os parâmetros mais robustos identificados aqui.

\subsection{Resultados para Acoplamento Cross-Frequency (EEG-ECG)}

\vspace{0.3em}
\noindent\textit{Nota:} Como a remoção de \textit{outliers} pelo ECOD resultou em 0\% de exclusões nesse grupo (EEG-ECG), os resultados apresentados aqui correspondem a ambos os cenários (com e sem \textit{outliers}).

Os testes não paramétricos aplicados à métrica \texttt{median\_cf\_plm\_diff} - que quantifica o acoplamento entre sinais de EEG e ECG em diferentes faixas de frequência - revelaram padrões específicos de modulação induzida pela HD-tDCS catódica. Utilizou-se o teste de Mann-Whitney U com correção de Bonferroni para múltiplas comparações (alpha\textsubscript{corr} = 0{,}01). Foram observados os seguintes achados principais:

\begin{itemize}
  \item \textbf{Delta e beta} apresentaram efeitos altamente significativos ($p < 0{,}01$ após correção), com tamanhos de efeito negativos de magnitude moderada a elevada (respectivamente $-0{,}467$ e $-0{,}236$), sugerindo uma redução do acoplamento cérebro-coração (CF-PLM) na condição catódica em relação ao \textit{sham}.
  \item \textbf{Gamma} também demonstrou significância marginal ($p = 0{,}005$; $p_{\text{corr}} = 0{,}025$), com tamanho de efeito negativo ($-0{,}120$), indicando tendência de redução do acoplamento, embora não atingindo o limiar de significância após correção múltipla.
  \item \textbf{Alpha e theta} não apresentaram diferenças significativas ($p > 0{,}05$), com tamanhos de efeito pequenos (respectivamente $-0{,}085$ e $-0{,}025$), sugerindo que essas bandas não foram substancialmente moduladas pela estimulação catódica.
\end{itemize}

Em conjunto, os resultados reforçam a sensibilidade do acoplamento cross-frequency (EEG-ECG) nas bandas \textbf{delta} e \textbf{beta} à modulação neuromodulatória, com efeito mais fraco porém presente em \textbf{gamma}. A direção negativa dos efeitos indica que, em média (no macro), a condição catódica (comparada à \textit{sham}) esteve associada a um efeito negativo da sincronicidade de fase entre sinais de EEG e ECG.

Esses achados se mostraram estáveis à remoção de \textit{outliers}. Vale lembrar que, no grupo EEG-ECG, nenhum dado foi excluído pelo método ECOD.

%-------------------------------------------------------------------
\section{Fase 2 - Análise Bootstrap e Distribuições de Efeito}
%-------------------------------------------------------------------

Complementando os testes globais apresentados anteriormente, esta fase explorou os efeitos da neuromodulação com maior granularidade por meio de reamostragens \textit{bootstrap} BCa com 10.000 iterações. Foram avaliadas todas as combinações possíveis de pares de canais (EEG-EEG para PLI e EEG-ECG para CF-PLM), extraindo:
\begin{itemize}
  \item Intervalos de confiança corrigidos (BCa);
  \item Estatísticas de Wilcoxon pareado ($W$);
  \item Correlação bisserial de postos (RBC);
  \item Tamanho de efeito de Hedges' $g$;
  \item Correções múltiplas (Bonferroni, Holm e FDR-BH).
\end{itemize}

\subsection{Resultados Bootstrap - EEG-EEG (PLI)}

Os resultados do \textit{bootstrap} para conectividade intra-frequencial revelaram padrões distintos de modulação da sincronia cerebral:

\begin{itemize}
  \item A banda \textbf{alpha} exibiu um predomínio robusto de conexões com \textbf{RBC positivo}, sobretudo em trajetórias fronto-parietais diagonais (ex.: Fp2-PO8), indicando um \textbf{efeito positivo na sincronia de fase} induzido pela estimulação catódica com \textit{sham} em referência.
  \item Em \textbf{theta}, observaram-se padrões semelhantes, com conexões RBC +1 concentradas à esquerda e algumas RBC -1 na fronteira frontal direita, sugerindo uma modulação assimétrica.
  \item As bandas \textbf{delta} e \textbf{gamma} apresentaram \textbf{predomínio de conexões RBC -1}, com destaque para o canal Fp2 e áreas fronto-temporais, sugerindo \textbf{redução de sincronia} em regiões específicas sob a estimulação.
  \item A banda \textbf{beta} mostrou comportamento mais heterogêneo: conexões positivas e negativas coexistiram, com leve predominância de \textbf{RBC negativo} após remoção de outliers.
\end{itemize}

Esses resultados indicam que a HD-tDCS catódica não afeta uniformemente a sincronia entre regiões cerebrais, mas tende a \textbf{aumentar a conectividade nas bandas alpha e theta} e a \textbf{reduzir nas bandas delta e gamma}, com variações topológicas evidentes por banda.

\subsection{Resultados Bootstrap - EEG-ECG (CF-PLM)}

Como não foram identificados outliers neste grupo (0\% removido pelo ECOD), apresentamos apenas a versão consolidada, sem necessidade de comparação entre cenários. O acoplamento \textit{cross-frequency} entre sinais cerebrais e cardíacos revelou efeitos mais localizados:

\begin{itemize}
  \item A maioria das conexões com efeito significativo apresentou \textbf{RBC positivo}, sugerindo efeito positivo localizado da sincronicidade fásica EEG-ECG; entretanto, na banda \textbf{theta} o par significativo F6-ECG exibiu \textbf{RBC negativo}, indicando efeito negativo pontual nessa frequência.
  \item A banda \textbf{beta} destacou-se com múltiplas conexões com efeito significativo envolvendo a região parietal esquerda (CP1, CP3, CP5 e TPP7h, par a par com ECG).
  \item Em \textbf{delta}, observaram-se pares adicionais, com envolvimento de regiões frontais (Fp2-ECG, FC4-ECG), fronto-temporais (FTT8h) e parietais (CP3-ECG, P1-ECG).
  \item As bandas \textbf{theta} e \textbf{gamma} exibiram um único par significativo cada: F6-ECG (negativo) e CP3-ECG (positivo), respectivamente.
  \item A banda \textbf{alpha} não apresentou nenhuma conexão significativa.
\end{itemize}

O padrão unidirecional de RBC +1 sugere que a HD-tDCS catódica no DLPFC esquerdo causou um efeito positivo na sincronicidade de fase nos casos com significância, apesar do efeito negativo no único par da banda theta.

\subsection{Síntese dos Resultados \textit{Bootstraped} par a par}

Com base nos achados descritos, é possível sintetizar os seguintes pontos:

\begin{itemize}
  \item A HD-tDCS catódica produz \textbf{efeitos direcionais distintos por banda}. \textbf{Para EEG-EEG}, efeito positivo na sincronia para as faixas alpha e theta, e negativo nas faixas delta e gamma. \textbf{Para EEG-ECG}, sem efeito significativo em alpha, efeito positivo em beta, gamma (região centro-parietal) e delta (mais espalhado), e negativo em theta (F6).
  \item Os \textbf{valores de RBC tendem a ser extremos} (próximos de $+1$ ou $-1$), indicando modulações pronunciadas e robustas mesmo após correção por múltiplos testes.
  \item A \textbf{estrutura topológica das redes} obtidas via bootstrap revelou a formação de ``\textit{hubs} de modulação'' (canais que concentram pares significativamente alterados) e variou por banda e localização cortical.
  \item A remoção de \textit{outliers} pouco afetou a estrutura geral das redes, indicando que os achados são robustos e replicáveis.
\end{itemize}

As visualizações completas das distribuições de RBC, Hedges' $g$ e $p$-valores corrigidos encontram-se nos Capítulos~\ref{chap:analise_de_rede} e~\ref{chap:analise_centralidade_de_grafos}.

%-------------------------------------------------------------------
\section{Fase 3 - Centralidade de Grafos e Hierarquia Funcional}
\label{sec:fase3_centralidade}
%-------------------------------------------------------------------

Partindo da rede de pares \emph{significativamente alterados} (ver Sec.~\ref{sec:met_centralidade}), calculámos as centralidades de \textit{Degree}, \textit{Betweenness} e \textit{Eigenvector}\footnote{Definições formais em \cite{rubinov2010complex,bullmore2009complex}; aplicação em EEG em \citeonline{lohmann2010eigenvector}.} em cada banda (delta-gamma), nos cenários \emph{com} e \emph{sem} outliers. Para \textit{Degree} e \textit{Betweenness} usamos o grafo binarizado (presença/ausência de aresta); para \textit{Eigenvector} empregámos o grafo ponderado pelos valores de RBC. Nós de maior centralidade surgem em vermelho e com tamanho proporcional, numa escala de cores fixa (0 - máx.\ global) em todas as figuras\footnote{Escala cromática unificada evita aparentes diferenças de cor devidas apenas ao ajuste automático do \textit{colormap}.}.  
Além disso, a semelhança das hierarquias \emph{com} vs.\ \emph{sem} outliers foi alta (mediana $\rho_{\text{Spearman}}\approx0{,}97$, $p<0{,}001$), confirmando a robustez dos rankings.  
Note-se que uma centralidade elevada pode refletir tanto aumento (RBC >\,0) quanto redução (RBC <\,0) de sincronia; sempre que pertinente indicamos essa predominância no texto.


\subsection*{Síntese dos principais achados}

\begin{itemize}
    \item \textbf{Alpha -} \textit{Fp2} lidera todas as métricas; \textit{PO8} e \textit{TP10} funcionam como pontes (betweenness), enquanto \textit{FC3}/\textit{CP4} surgem como nós de segundo nível em eigenvector, delineando um eixo frontal $\rightarrow$ centro-parietal $\rightarrow$ occipital.
    \item \textbf{Beta -} Com outliers, o núcleo de maior centralidade concentra-se no cluster centro-parietal (\textit{CP5-CP2-F5}); após a remoção de outliers, \textit{TPP7h} passa a liderar todas as métricas (grau de centralidade $\approx 0{,}17$), reforçando a interpretação de uma reorquestração temporo-parietal estável.
    \item \textbf{Delta -} O hub desloca-se de \textit{Fp2} (com outliers) para \textit{CP5/CP4} (sem outliers).
    \item \textbf{Gamma -} Apesar de maior dispersão global, dois focos persistem: um cluster \emph{fronto-temporal} (F3-FT10) que lidera grau e eigenvector em ambos os cenários, e nós centrais adicionais (C5, FT8h) que perdem força ao excluir outliers; a malha de betweenness torna-se mais compacta, com \textit{F3 $\rightarrow$ C5} concentrando as rotas mínimas.
    \item \textbf{Theta -} Dois sub-clusters tornam-se evidentes: (i) um núcleo \emph{occipito-parietal} (PO4/PO7 + FC1) que domina betweenness e degree, e (ii) um foco \emph{temporo-parietal} (TP9) que assume a liderança em eigenvector após remover outliers; no cenário original F8/AF4 ocupam esse papel, indicando reconfiguração das rotas occipito \leftrightarrow  temporal consoante a métrica e a robustez da amostra.
    \item \textbf{Robustez -} A remoção de outliers muda algumas posições (p.\,ex., troca \textit{CP5} $\rightarrow$ \textit{TPP7h} em beta), mas preserva os hubs estruturais; \textit{CP5/CP4} (delta) e \textit{TP10/Fp1} (alpha) continuam atuando como \emph{connector hubs} entre módulos fronto-parietal e parieto-temporal.
\end{itemize}

Estes resultados complementam as Fases 1 e 2: identificam não só \emph{onde} os efeitos foram significativos, mas \emph{quais} canais assumem papéis de hubs locais (\textit{Degree}), pontes globais (\textit{Betweenness}) ou nós influentes ligados a outros hubs (\textit{Eigenvector}), compondo um mapa funcional da modulação induzida pela HD-tDCS catódica.

\section{Discussão Integrada}
% --- síntese entre intrafreq e cross-freq ---
Em ambas as análises, intrafrequencial (PLI) e \textit{cross-frequency} (CF-PLM), o ritmo alpha emergiu como o mais sensível à HD-tDCS catódica, tanto estatisticamente quanto em tamanho de efeito. Já a banda beta mostrou um padrão heterogêneo, com significância marginal nos testes globais mas efeitos moderados em pares selecionados. Essa dissociação reforça a ideia de que diferentes circuitos e escalas de acoplamento respondem de maneira distinta à neuromodulação, possivelmente por diferenças em sua arquitetura anatômica e fisiologia de geração de oscilação. Nota-se que Fp2, TPP7h e CP5/CP4 se destacam em pelo menos duas métricas nas bandas em que apresentam efeito, indicando consistência de sua relevância.

Salienta-se que as centralidades de \textit{Degree} e \textit{Betweenness} foram calculadas em grafos binarizados (presença/ausência de aresta significativa), enquanto a centralidade de \textit{Eigenvector} considerou os valores RBC como pesos das arestas. Dessa forma, um nó de alta centralidade pode refletir tanto aumentos (RBC>0) quanto reduções (RBC<0) de sincronia em múltiplas conexões.

% --- implicações de rede e hubs ---
Na construção da rede, cada aresta reflete o quanto a estimulação catódica alterou a sincronia em relação ao \textit{sham} (diferença \textit{cathodic}-\textit{sham}) após 10.000 reamostragens BCa e correção de Bonferroni. Ou seja, não se trata de uma rede da sincronia basal em repouso, mas de uma rede das mudanças induzidas pela HD-tDCS.

Nas medidas de centralidade, os canais de maior centralidade correspondem, portanto, àqueles que participaram de um maior número de pares em que o efeito da HD-tDCS sobre a sincronia foi mais pronunciado. Na banda \textbf{alpha} (8-13 Hz), o canal \textbf{Fp2} emergiu como hub de maior centralidade em todas as métricas (\textit{Betweenness}, \textit{Degree} e \textit{Eigenvector}), tanto com outliers quanto sem outliers, indicando que é justamente nessa região que as ocorrências de pares com maiores diferenças \textit{cathodic}-\textit{sham} foram mais frequentes.

Em outras faixas de frequência, observamos reorganizações na hierarquia de \textit{hubs}: por exemplo, \textbf{TPP7h} na banda \textbf{beta} e, no cenário sem remoção de outliers, \textbf{CP5} e \textbf{CP4} na banda \textbf{delta}.

% --- link com a literatura ---
Esses achados corroboram o conceito de que a HD-tDCS induz mudanças agudas e persistentes na sincronização cortical (ou seja, uma forma de plasticidade funcional induzida pela corrente contínua) conforme demonstrado por \citeonline{kunze2014high}, e o conceito de ``integração corpo-cérebro'' de \citeonline{criscuolo2022cognition}, ampliando-os ao contexto de atletas de elite em \textit{resting-state}.

% --- brainstorming quanto ao futuro ---
Futuros estudos deverão explorar se os \textit{hotspots} de acoplamento EEG-ECG identificados na análise de pares, em especial os pares parietais (TPP7h, CP5, CP3, CP1) que exibiram RBC $\approx +1$ na banda \textbf{beta}, estão relacionados a ganhos funcionais, por exemplo em controle autonômico ou desempenho motor. 

Vale salientar que, nos testes globais, a HD-tDCS catódica resultou em \textbf{reduções médias do acoplamento cérebro-coração} nas bandas \textbf{delta}, \textbf{beta} e \textbf{gamma} (por exemplo, \(\mathrm{RBC}\approx -0{,}47\) em delta); assim, os hubs positivos que emergem no bootstrap representam aumentos \textit{locais} que contrastam com essa tendência geral e merecem investigação funcional específica.

Investigar se atletas com RBC mais elevados nesses pares específicos apresentam melhor performance (ou recuperação) poderá elucidar o papel funcional dessas conexões pontuais.

Além disso, protocolos de neuromodulação poderiam ser otimizados de forma individualizada, aplicando estimulação focalizada nesses \textit{hubs} e em faixas de frequência específicas, para potencializar processos neurofisiológicos associados à alta performance esportiva. Essa abordagem de \textit{stimulation-to-performance} alinha-se ao modelo Body-Brain Dynamic System (BBDS) de \citeonline{criscuolo2022cognition}, que postula que oscilações corticais lentas modulam a excitabilidade neural em sincronia com o ciclo cardíaco. Assim, ao intensificar seletivamente o acoplamento alfa e corpo-cérebro, a HD-tDCS personalizada poderia melhorar a alocação de recursos autonômicos em situações de alta demanda competitiva, abrindo caminho para intervenções neuromodulatórias de precisão no esporte de elite.

Como não usamos as coordenadas 10-20 reais para posicionar os eletrodos, os ``\textit{hubs}'' identificados pelas métricas de centralidade devem ser interpretados como eletrodos \emph{sensoriais} centrais na rede funcional de diferenças, não necessariamente como regiões corticais específicas.

Em outras palavras, tratam-se de \textit{hubs} funcionais em nível de sensor EEG (por exemplo, ``Fp2'' indica alta centralidade do eletrodo Fp2 na rede de efeitos) não de localização exata no córtex.

\subsection{Limitações e Considerações Metodológicas}
As sessões de estimulação (\textit{sham} e \textit{cathodic}) foram realizadas em dias distintos, o que pode introduzir variáveis confundidoras (estado fisiológico, cansaço, fatores ambientais). Embora tivéssemos padronizado horários e instruções, recomenda-se um desenho \textit{crossover} contrabalanceado em futuros estudos para minimizar esse viés.

O recrutamento de atletas de elite limitou o número de participantes, reduzindo a potência estatística e a possibilidade de generalizar os achados a populações não atléticas. Estudos posteriores devem incluir amostras maiores e grupos-controle não atletas.

Apesar do uso de ICA e filtros avançados, artefatos residuais (especialmente na banda gamma) não podem ser totalmente descartados. A análise paralela com e sem remoção de outliers mostrou estabilidade geral, mas sugere a necessidade de métodos ainda mais robustos de pré-processamento.

Observou-se que as bandas delta e gamma apresentaram leve sensibilidade à remoção de outliers (\(d\approx5\)\%) dos pares, indicando maior variabilidade intrínseca ou suscetibilidade a artefatos. Cautela é recomendada ao interpretar efeitos isolados nessas faixas.

Embora proporcione focalidade superior, o custo e a complexidade do HD-tDCS podem limitar sua adoção em laboratórios com recursos restritos. Comparações diretas com tDCS de esponja são necessárias para avaliar \textit{trade-offs} entre eficácia e viabilidade.