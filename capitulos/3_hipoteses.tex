\chapter{Hipóteses}
\label{chap:hipoteses}

A hipótese central deste estudo é que a aplicação da estimulação transcraniana por corrente contínua de alta definição (HD-tDCS) catódica sobre o córtex pré-frontal dorsolateral (DLPFC) esquerdo modula significativamente os padrões de sincronicidade neural em atletas de elite de basquetebol feminino, durante o estado de repouso (\textit{resting-state}). Especificamente, espera-se que a estimulação catódica altere significativamente a conectividade funcional intrafrequencial (EEG-EEG) dentro das bandas delta, theta, alpha, beta e gamma, bem como a conectividade funcional cross-frequency entre o EEG (nessas mesmas bandas) e o ECG. Essas mudanças serão quantificadas por diferenças nos índices de sincronicidade, tanto no Phase Lag Index (PLI) para conexões EEG-EEG quanto no CF-PLM para conexões EEG-ECG, ao comparar as condições pré e pós-estimulação nas sessões catódica e controle (sham). Assim, espera-se obter evidências claras da influência da HD-tDCS na interação entre os sistemas neural e cardiovascular em atletas, contribuindo para o entendimento dos mecanismos neurofisiológicos subjacentes à neuromodulação.