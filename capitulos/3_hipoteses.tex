\chapter{Hipóteses}
\label{chap:hipoteses}
A hipótese central deste estudo é que a neuromodulação catódica, aplicada via HD-tDCS sobre o DLPFC esquerdo, modula significativamente os padrões de sincronização neural em atletas de elite de basquetebol feminino durante o repouso. Em termos específicos, espera-se que:
\begin{itemize}
    \item A conectividade intrafrequencial entre canais de EEG (nas bandas delta, theta, alpha, beta e gamma) seja alterada significativamente;
    \item O acoplamento \textit{cross-frequency} entre EEG e ECG seja modulado, refletindo mudanças na interação entre os ritmos cerebrais e o ciclo cardíaco.
\end{itemize}
