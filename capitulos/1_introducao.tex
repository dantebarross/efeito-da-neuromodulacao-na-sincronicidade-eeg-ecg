\chapter{Introdução}
\label{chap:introducao}
A neurociência tem avançado na compreensão da sincronização entre o cérebro e processos fisiológicos, destacando o papel das interações dinâmicas na integração entre sistemas corporais e neurais. Nesse contexto, o conceito de \textit{Body--Brain Dynamic System} (BBDS) tem ganhado espaço como uma abordagem para investigar como oscilações neurais se sincronizam com ritmos fisiológicos – como os da frequência cardíaca, respiratória, entre outros –, modulando a atividade cerebral \cite{cohen2017where,criscuolo2022cognition}. Esse entendimento tem impulsionado o desenvolvimento de intervenções de neuromodulação capazes de modificar padrões rítmicos e, consequentemente, a função cerebral.

O corpo humano possui uma capacidade intrínseca de sincronizar seus ritmos fisiológicos com estímulos ambientais e internos. Por exemplo, o ritmo respiratório pode alinhar-se a padrões de atividade sensorial e cognitiva \cite{haas1985effects}, e estados psicofisiológicos – como ansiedade, depressão e estresse – influenciam tanto a frequência cardíaca quanto a atividade neural \cite{criscuolo2022cognition}. Pesquisas recentes demonstram que a variabilidade dos ritmos cardíacos e respiratórios gera ciclos de alta e baixa excitabilidade, modulando a integração e a regulação neural. Estudos em neurocardiologia evidenciam que a interação entre mecanismos sensoriais (como barorreceptores e quimiorreceptores) e centros neurais superiores é essencial para a regulação dos padrões cardíacos \cite{marcondes2024linguagem}. Além disso, pesquisas com potenciais evocados pelo batimento cardíaco (HEPs) reforçam a importância da integração interoceptiva na formação da consciência corporal \cite{banelli2020skipping, mackinnon2013utilizing, park2018neural}.

Para explorar os efeitos da neuromodulação na conectividade neural, este projeto investiga como a estimulação transcraniana por corrente contínua de alta definição (HD-tDCS), aplicada de forma catódica sobre o córtex pré-frontal dorsolateral (DLPFC) esquerdo, impacta os padrões de sincronização cerebral em atletas de elite de basquetebol feminino em repouso (\textit{resting-state}). O delineamento experimental adotado é do tipo cruzado (\textit{cross-over}) e duplo-cego, permitindo que os mesmos participantes sejam submetidos tanto à estimulação ativa quanto à condição controle (\textit{sham}), reduzindo assim a influência de variáveis individuais e possibilitando uma análise mais precisa dos efeitos neuromodulatórios.

Neste estudo, além de analisar a sincronização intrafrequencial entre pares de canais de EEG, investigamos a sincronicidade \textit{cross-frequency} entre sinais de eletroencefalografia (EEG) e eletrocardiograma (ECG). Para isso, o sinal de ECG foi convertido em uma representação senoidal simples – baseada no pico R – que delimita de forma clara o ciclo cardíaco. Ao comparar, por exemplo, o canal Fp1 na banda alpha com esse sinal senoidal, é possível quantificar a sincronização entre a atividade cerebral e o ritmo cardíaco. Essa abordagem detalha o acoplamento entre oscilações cerebrais de diferentes frequências e o sinal cardíaco, contribuindo para a compreensão da interação entre os sistemas neural e cardiovascular, com especial relevância para intervenções neuromodulatórias em atletas. Ademais, modelos recentes exploram a previsibilidade dos sinais EEG alinhados com os batimentos cardíacos, ampliando o entendimento dos mecanismos de acoplamento entre a atividade neural e o ritmo cardíaco \cite{vergara2024exploring}.

\section{Neuromodulação e Modulação da Função Cerebral em Contextos Esportivos e Clínicos}
Esta seção apresenta uma visão abrangente das técnicas de neuromodulação, destacando suas aplicações em ambientes clínicos, esportivos e interpessoais, além de abordar estratégias multidimensionais e técnicas alternativas para o monitoramento e potencialização da função cerebral.

\subsection{Técnicas Convencionais de Neuromodulação}
A estimulação transcraniana por corrente contínua (tDCS) é uma técnica não invasiva que utiliza correntes elétricas fracas para modular a excitabilidade cortical e a atividade neuronal \cite{nitsche2000excitability, okano2013estimulacao, purpura1965intracellular, stagg2011physiological}. A aplicação de correntes anódicas geralmente aumenta a excitabilidade, enquanto a catódica a reduz. Para aumentar a focalidade da estimulação, foi desenvolvida a High-Definition tDCS (HD-tDCS), que utiliza arrays de eletrodos menores dispostos, por exemplo, na configuração \emph{4$\times$1} \cite{villamar2013hdtdcs}. Outras técnicas não invasivas – como a tACS (estimulação transcraniana por corrente alternada) e a rTMS (estimulação magnética transcraniana repetitiva) – também podem modificar a sincronização neural e a conectividade funcional, contribuindo para a modulação abrangente das redes cerebrais \cite{kunze2014high, scheler2019neuromodulation}.

\subsection{Aplicações Clínicas}
Diversos estudos demonstraram o potencial das técnicas convencionais de neuromodulação em contextos clínicos, evidenciando efeitos na restauração de circuitos neurais disfuncionais e na modulação da atividade cerebral. Por exemplo:
\begin{itemize}
    \item \citeonline{singh2024evaluating} identificaram alterações na topologia da rede de EEG em repouso de pacientes com transtorno depressivo maior, sugerindo que a tDCS pode restaurar circuitos neurais disfuncionais.
    \item \citeonline{toutant2024hdtdcs} demonstraram que a HD‑tDCS pode reorganizar o perfil espectral do EEG – com redução das frequências baixas (delta e theta) e aumento das frequências altas (beta e gamma) – em pacientes com epilepsia refratária.
    \item \citeonline{cukic2018shift} relataram que a tDCS, dependendo da polaridade, provoca deslocamentos no estado energético do cérebro, refletidos em alterações do \textit{Mean State Shift} (MSS) e da \textit{State Variance} (SV).
    \item \citeonline{dong2023efficacy} evidenciaram que a estimulação não invasiva pode melhorar o estado de consciência em pacientes com transtornos do nível de consciência.
\end{itemize}

\subsection{Aplicações em Contextos Esportivos, Emocionais e Interpessoais}
A neuromodulação tem sido aplicada não apenas em ambientes clínicos, mas também para melhorar o desempenho esportivo, aspectos emocionais e a interação interpessoal:
\begin{itemize}
    \item \citeonline{valenzuela2019enhancement} demonstraram que a tDCS pode melhorar o estado de humor em atletas.
    \item Estudos com tACS, como os de \citeonline{rostami2020transcranial}, indicam que a estimulação a 6 Hz sobre o córtex pré-frontal medial (mPFC) melhora a atenção sustentada, modulando a sincronia alpha e theta.
    \item \citeonline{spooner2020hdtdcs} mostraram que a HD-tDCS aplicada no DLPFC altera a conectividade na banda theta durante tarefas de atenção visual seletiva.
    \item Em contextos interpessoais, \citeonline{long2023transcranial} relataram que a tDCS aplicada no lobo temporal anterior direito pode reduzir a sincronização neural interpessoal e os níveis de empatia emocional.
    \item \citeonline{liu2023effects} constataram que a tDCS diminui oscilações delta e aumenta oscilações alpha em pacientes pós-AVC, enquanto \citeonline{han2022functional} observaram aumentos na conectividade funcional em pacientes com distúrbios de consciência após HD‑tDCS.
    \item Intervenções domiciliares com tDCS têm sido associadas ao aumento da sincronização neural em pacientes com depressão bipolar \cite{xiao2025enhanced}.
    \item Estudos com HD-tDCS aplicados no DLPFC sugerem que a estimulação pode alterar a conectividade parieto-frontal e a dinâmica dos circuitos neurais \cite{arif2021high}.
\end{itemize}

\subsection{Abordagens Multidimensionais, Modelos Matemáticos e Monitoramento Integrado}
Para aprofundar a compreensão dos efeitos da neuromodulação, diversas abordagens multidimensionais e modelos computacionais têm sido empregados, integrando análises quantitativas de EEG, modelos matemáticos e técnicas avançadas de monitoramento. Por exemplo:
\begin{itemize}
    \item \citeonline{zhang2022multidimensional} realizaram uma avaliação abrangente das métricas de EEG em intervenções neuromodulatórias.
    \item \citeonline{jones2017frontoparietal} demonstraram que a combinação de tDCS com treinamento de memória de trabalho aumenta a sincronia em redes fronto-parietais.
    \item Estudos com tDCS bilateral indicam que há remodelação das redes cerebrais em repouso, sugerindo indução de plasticidade em regiões não diretamente estimuladas \cite{pellegrino2018bilateral}.
    \item \citeonline{riedinger2022model} explicam como a tDCS modula a excitabilidade cerebral e a potência do EEG.
    \item Abordagens de controle em loop fechado para oscilações gamma, exploradas por \citeonline{zhang2024closed}, indicam que intervenções transcranianas – tanto por tDCS quanto por rTMS – podem aumentar as oscilações gamma.
    \item Análises de grafos aplicadas à tDCS revelam alterações na sincronização cortical que dependem da polaridade da estimulação \cite{mancini2016assessing, pellegrino2019transcranial, schollmann2019anodal}.
    \item A viabilidade do monitoramento simultâneo de EEG durante a tDCS foi demonstrada por \citeonline{schesatsky2013simultaneous}, facilitando a avaliação contínua da excitabilidade cortical.
\end{itemize}

\subsection{Abordagens Alternativas e Complementares}
Além das técnicas convencionais, diversas abordagens alternativas têm sido exploradas para ajustar a sincronização neural e ampliar a compreensão dos mecanismos moduladores:
\begin{itemize}
    \item \citeonline{boecker2024interpersonal} investigaram a sincronia neural interpessoal, demonstrando que intervenções envolvendo estimulação cerebral combinada com neurofeedback podem aprimorar a coordenação dos sinais durante interações sociais, especialmente em populações com transtorno do espectro autista (ASD), transtorno de ansiedade social (SAD) e condições correlatas.
    \item \citeonline{mcnaughton2020interpersonal} e \citeonline{baldwin2014evidence} apontam que indivíduos com ASD frequentemente exibem comportamentos temporalmente assíncronos em tarefas de integração sensorial, sugerindo o potencial de intervenções neuromodulatórias alternativas para restabelecer a sincronização funcional.
    \item Estudos adicionais \cite{gerloff2022autism, quinones2021dysfunction, key2022greater, tanabe2012hard} corroboram a hipótese de que a desincronia interpessoal pode estar associada a déficits em múltiplos níveis, sugerindo que abordagens complementares podem ser essenciais para restabelecer a comunicação neural.
    \item No campo do neurofeedback, evidências sugerem que essa abordagem pode reduzir sintomas de ansiedade generalizada e fobias específicas \cite{hou2021neurofeedback, zilverstand2015fmri}. Investigações utilizando neurofeedback baseado em NIRS demonstraram a viabilidade de treinar o controle do DLPFC em indivíduos com ansiedade social \cite{kimmig2019feasibility, direito2021training, steiner2014pilot, lamarca2018facilitating, catala2017treatment}, com análises econômicas indicando que os custos podem ser compatíveis com outras intervenções terapêuticas \cite{arnold2013eeg}.
    \item Em termos de potencialização de oscilações, \citeonline{maiella2022simultaneous} exploraram a combinação de tACS e \textit{TMS} para aumentar as oscilações gamma no DLPFC. De modo complementar, \citeonline{zrenner2020brain} demonstraram a eficácia da rTMS sincronizada com oscilações alpha no DLPFC para pacientes com depressão resistente, e \citeonline{konrad2024interpersonal} discutiram intervenções voltadas à sincronia neural interpessoal como uma via inovadora para tratamentos clínicos.
\end{itemize}

\section{Medidas Neurofisiológicas}
Para compreender a complexa interação entre o cérebro e o corpo, é possível integrar diversas técnicas que permitam registrar simultaneamente a atividade elétrica cerebral e sinais fisiológicos. A eletroencefalografia (EEG) fornece dados sobre a atividade neural, possibilitando a extração de oscilações que podem ser decompostas nas bandas clássicas – delta, theta, alpha, beta e gamma – cada uma associada a escalas temporais e funções cognitivas ou comportamentais específicas \cite{cohen2017where, bullmore2009complex}.

Além do EEG, técnicas como a eletromiografia (EMG) e o eletrocardiograma (ECG) são empregadas para monitorar a atividade muscular e os ritmos cardíacos, respectivamente. Em determinadas situações, a posição estratégica dos eletrodos de EMG pode possibilitar a detecção dos picos dos batimentos cardíacos – refletindo a despolarização ventricular (complexo QRS) – quando o ECG tradicional não está disponível ou quando se busca uma integração mais próxima com outros sinais fisiológicos. Essa abordagem permite, por exemplo, capturar a dinâmica do ciclo cardíaco a partir da atividade do músculo peitoral maior, conforme realizado neste estudo.

Esses registros permitem analisar o acoplamento de frequências cruzadas (\textit{cross-frequency coupling}, CFC) entre sinais de EEG e ECG, fenômeno no qual oscilações de diferentes frequências interagem entre si. O CFC possibilita compreender melhor processos de integração neural e corporal, como evidenciado por \citeonline{criscuolo2022cognition}, que investigaram interações entre atividade cerebral e sinais periféricos (\textit{brain-body coupling}) na modulação da cognição. Além disso, \citeonline{cohen2017where} destaca o papel das interações entre bandas rápidas (por exemplo, gamma) e lentas (como theta) em processos cognitivos fundamentais. Em conjunto, essas abordagens fundamentam conceitos e estratégias para a otimização de intervenções em contextos clínicos e de neuromodulação.