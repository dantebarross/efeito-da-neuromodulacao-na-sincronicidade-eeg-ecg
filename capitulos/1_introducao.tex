\chapter{Introdução}
\label{chap:introducao}
A neurociência tem avançado na compreensão da sincronização entre o cérebro e processos fisiológicos, destacando o papel das interações dinâmicas na integração entre sistemas corporais e neurais. Nesse contexto, o conceito de \textit{Body--Brain Dynamic System} (BBDS) tem ganhado espaço como uma abordagem para investigar como oscilações neurais se sincronizam com ritmos fisiológicos – como os da frequência cardíaca, respiratória, entre outros –, modulando a atividade cerebral \cite{cohen2017where,criscuolo2022cognition}. Esse entendimento tem impulsionado o desenvolvimento de intervenções de neuromodulação capazes de modificar padrões rítmicos e, consequentemente, a função cerebral.

O corpo humano possui uma capacidade intrínseca de sincronizar seus ritmos fisiológicos com estímulos ambientais e internos. Por exemplo, o ritmo respiratório pode alinhar-se a padrões de atividade sensorial e cognitiva \cite{haas1985effects}, e estados psicofisiológicos – como ansiedade, depressão e estresse – influenciam tanto a frequência cardíaca quanto a atividade neural \cite{criscuolo2022cognition}. Pesquisas recentes demonstram que a variabilidade dos ritmos cardíacos e respiratórios gera ciclos de alta e baixa excitabilidade, modulando a integração e a regulação neural. Estudos em neurocardiologia evidenciam que a interação entre mecanismos sensoriais (como barorreceptores e quimiorreceptores) e centros neurais superiores é essencial para a regulação dos padrões cardíacos \cite{marcondes2024linguagem}. Além disso, pesquisas com potenciais evocados pelo batimento cardíaco (HEPs) reforçam a importância da integração interoceptiva na formação da consciência corporal \cite{banelli2020skipping, mackinnon2013utilizing, park2018neural}.

Para explorar os efeitos da neuromodulação na conectividade neural, este projeto investiga como a estimulação transcraniana por corrente contínua de alta definição (HD-tDCS), aplicada de forma catódica sobre o córtex pré-frontal dorsolateral (DLPFC) esquerdo, impacta os padrões de sincronização cerebral em atletas de elite de basquetebol feminino em repouso (\textit{resting-state}). A escolha da estimulação catódica, que tipicamente reduz a excitabilidade cortical, fundamenta-se na hipótese de que a diminuição da atividade no DLPFC esquerdo pode modular o equilíbrio entre redes neurais, potencialmente reduzindo padrões de hiperconectividade e promovendo uma reorganização funcional mais eficiente. O DLPFC esquerdo foi selecionado como alvo por seu papel central em redes frontoparietais envolvidas em funções executivas e controle cognitivo-motor, processos fundamentais para o desempenho atlético em esportes coletivos. O delineamento experimental adotado é do tipo cruzado (\textit{cross-over}) e duplo-cego, permitindo que os mesmos participantes sejam submetidos tanto à estimulação ativa quanto à condição controle (\textit{sham}), reduzindo assim a influência de variáveis individuais e possibilitando uma análise mais precisa dos efeitos neuromodulatórios.

Neste estudo, além de analisar a sincronização intrafrequencial entre pares de canais de EEG, investigamos a sincronicidade \textit{cross-frequency} entre sinais de eletroencefalografia (EEG) e eletrocardiograma (ECG). Para isso, o sinal de ECG foi convertido em uma representação senoidal simples – baseada no pico R – que delimita de forma clara o ciclo cardíaco. Ao comparar, por exemplo, o canal Fp1 na banda alpha com esse sinal senoidal, é possível quantificar a sincronização entre a atividade cerebral e o ritmo cardíaco. Essa abordagem detalha o acoplamento entre oscilações cerebrais de diferentes frequências e o sinal cardíaco, contribuindo para a compreensão da interação entre os sistemas neural e cardiovascular, com especial relevância para intervenções neuromodulatórias em atletas. Ademais, modelos recentes exploram a previsibilidade dos sinais EEG alinhados com os batimentos cardíacos, ampliando o entendimento dos mecanismos de acoplamento entre a atividade neural e o ritmo cardíaco \cite{vergara2024exploring}.

\section{Neuromodulação e Modulação da Função Cerebral em Contextos Esportivos e Clínicos}
A neuromodulação não invasiva representa um campo em rápida expansão, oferecendo ferramentas poderosas para investigar e modular a atividade cerebral em diversos contextos. Esta seção apresenta uma revisão sistemática das técnicas de neuromodulação, partindo dos princípios neurofisiológicos fundamentais e avançando para aplicações específicas em ambientes clínicos, esportivos e interpessoais. Inicialmente, exploramos as bases neurofisiológicas da estimulação transcraniana por corrente contínua (tDCS) e suas variantes, como a HD-tDCS, detalhando os mecanismos celulares e de rede pelos quais estas técnicas modulam a excitabilidade cortical e a conectividade funcional. Em seguida, examinamos evidências de sua eficácia em contextos clínicos, destacando como a neuromodulação pode normalizar circuitos neurais disfuncionais em diversas condições neuropsiquiátricas. Posteriormente, abordamos aplicações em contextos esportivos e interpessoais, onde estas técnicas demonstram potencial para otimizar o desempenho e modular interações sociais. Finalmente, discutimos abordagens multidimensionais e técnicas alternativas que expandem o horizonte investigativo e terapêutico da neuromodulação. Esta progressão lógica permite compreender como os princípios básicos da neuromodulação se traduzem em aplicações práticas e como diferentes abordagens podem ser integradas para uma compreensão mais completa da dinâmica cerebral e sua modulação.

\subsection{Técnicas Convencionais de Neuromodulação}
A estimulação transcraniana por corrente contínua (tDCS) emergiu como uma técnica não invasiva de neuromodulação capaz de induzir alterações controladas na excitabilidade cortical. O trabalho pioneiro de \citeonline{nitsche2000excitability} estabeleceu os princípios fundamentais desta técnica, demonstrando que correntes elétricas fracas (1-2 mA) aplicadas através do crânio podem modular a excitabilidade do córtex motor humano de maneira polaridade-dependente. Seus experimentos revelaram que a estimulação anódica aumenta a excitabilidade cortical, enquanto a catódica a reduz, com efeitos que persistem por vários minutos após o término da estimulação. Estas alterações foram quantificadas através de potenciais evocados motores (MEPs) induzidos por estimulação magnética transcraniana (TMS), evidenciando o potencial da tDCS como ferramenta para modular a atividade cerebral de forma seletiva, reversível e indolor.

Os mecanismos neurofisiológicos subjacentes a estes efeitos foram inicialmente investigados por \citeonline{purpura1965intracellular}, que demonstraram, através de registros intracelulares, que a polarização anódica despolariza os corpos celulares de neurônios piramidais, aumentando sua excitabilidade, enquanto a polarização catódica os hiperpolariza, reduzindo sua atividade espontânea. Este trabalho fundamental revelou ainda que os efeitos da estimulação dependem da orientação das células em relação ao fluxo de corrente, explicando a especificidade dos efeitos observados em diferentes populações neuronais e regiões corticais.

Avançando na compreensão destes mecanismos, \citeonline{stagg2011physiological} elucidaram como a tDCS influencia a plasticidade sináptica no neocórtex, estabelecendo paralelos com processos de potencialização de longo prazo (LTP) e depressão de longo prazo (LTD). Sua revisão destacou o papel crucial de neurotransmissores como glutamato, GABA, dopamina e serotonina na mediação dos efeitos da tDCS, fornecendo uma base neuroquímica para compreender como esta técnica modifica a aprendizagem motora e a conectividade funcional cerebral.

No contexto brasileiro, \citeonline{okano2013estimulacao} contribuíram significativamente ao revisar as aplicações da tDCS na promoção da saúde e melhoria do desempenho físico. Seu trabalho sistematizou evidências sobre como a tDCS pode modular funções cardiovasculares, controle de apetite e percepção de esforço, destacando o potencial desta técnica como estratégia complementar para otimização do desempenho esportivo e intervenções terapêuticas.

Para superar limitações de focalidade da tDCS convencional, foi desenvolvida a estimulação transcraniana por corrente contínua de alta definição (HD-tDCS). \citeonline{villamar2013hdtdcs} descreveram a configuração \emph{4$\times$1}, na qual um eletrodo central (anódico ou catódico) é circundado por quatro eletrodos de retorno, permitindo uma estimulação mais focal e reduzindo a dispersão de corrente. Modelagens computacionais e estudos clínicos confirmaram que esta técnica restringe a modulação da excitabilidade cortical à região-alvo, produzindo efeitos mais específicos e potencialmente mais duradouros que a tDCS convencional.

Expandindo o escopo das técnicas de neuromodulação, \citeonline{kunze2014high} investigaram os efeitos da HD-tDCS no cérebro utilizando EEG simultâneo, demonstrando que a estimulação do córtex sensorimotor esquerdo gera mudanças agudas e persistentes na sincronização cortical. Estas alterações incluem modificações globais e locais na sincronização neural, com efeitos distintos para estimulação anódica e catódica, especialmente na sincronização relacionada à imaginação motora.

Complementando esta perspectiva, \citeonline{scheler2019neuromodulation} exploraram como a neuromodulação influencia a sincronização neural e a capacidade intrínseca de leitura em redes neuronais. Seu trabalho demonstrou que a neuromodulação pode transformar as propriedades topológicas de redes neurais, alterando a distribuição de conexões e influenciando propriedades intrínsecas dos neurônios através da modulação de canais iônicos específicos. Esta abordagem teórica fornece um arcabouço para compreender como intervenções neuromodulatórias podem ajustar o equilíbrio entre sincronização e heterogeneidade neural, afetando a capacidade de processamento de informações no córtex cerebral.

Além da tDCS, outras técnicas não invasivas como a estimulação transcraniana por corrente alternada (tACS) e a estimulação magnética transcraniana repetitiva (rTMS) também têm demonstrado capacidade de modificar a sincronização neural e a conectividade funcional, contribuindo para uma compreensão mais abrangente de como a neuromodulação pode ser utilizada para ajustar a dinâmica das redes cerebrais em contextos clínicos e experimentais.

\subsection{Aplicações Clínicas}
A aplicação de técnicas de neuromodulação em contextos clínicos tem revelado seu potencial terapêutico para uma variedade de condições neuropsiquiátricas, com evidências crescentes de sua capacidade de restaurar circuitos neurais disfuncionais e modular padrões de atividade cerebral patológicos. Estas intervenções têm sido particularmente promissoras no tratamento de transtornos resistentes às abordagens farmacológicas convencionais.

No campo dos transtornos de humor, \citeonline{singh2024evaluating} conduziram uma investigação inovadora sobre os efeitos da tDCS em pacientes com transtorno depressivo maior (MDD), utilizando análises avançadas de conectividade funcional em dados de EEG em repouso. Seus resultados revelaram que a tDCS induz modificações significativas na topologia das redes cerebrais, particularmente na banda beta, indicando uma redução na randomização e um aumento na propriedade de "small-worldness" após a intervenção. Esta reorganização da arquitetura funcional cerebral é consistente com a hipótese de que pacientes com depressão apresentam redes neurais excessivamente randomizadas, especialmente durante o processamento emocional. Os autores demonstraram, através de análises de Phase Lag Index (PLI) e medidas de "hubness", que a tDCS pode normalizar estes padrões disfuncionais, reduzindo a hiperconectividade nas bandas theta e alpha – associadas ao estado de relaxamento – e aumentando a conectividade na banda beta – associada ao estado de alerta. Estas alterações correlacionaram-se com a melhora clínica, sugerindo mecanismos neurobiológicos específicos pelos quais a tDCS pode exercer seus efeitos antidepressivos.

Em pacientes com epilepsia refratária, \citeonline{toutant2024hdtdcs} demonstraram que a HD-tDCS catódica pode atuar como uma intervenção anti-epiléptica através da dessincronização de redes neurais hipersincrônicas. Seu estudo revelou uma redução significativa na sincronização de baixa frequência (bandas delta e theta) após a aplicação de HD-tDCS, particularmente nas regiões frontocentrais e parietais, acompanhada por um discreto aumento na atividade de alta frequência (bandas beta e gamma). Esta modulação do perfil espectral do EEG sugere uma alteração na excitabilidade cortical que pode interromper os padrões de sincronização excessiva característicos da atividade epileptiforme. Análises de conectividade revelaram ainda uma diminuição na coerência entre pares de eletrodos sobre o foco epiléptico e regiões adjacentes, indicando uma disrupção das redes hipersincrônicas que contribuem para a atividade convulsiva. Os efeitos mais pronunciados foram observados em pacientes com epilepsia focal no córtex frontal, sugerindo uma especificidade anatômica na resposta à neuromodulação.

Investigando os mecanismos fundamentais pelos quais a tDCS modifica a dinâmica cerebral, \citeonline{cukic2018shift} utilizaram medidas de Recurrence Quantification Analysis para caracterizar as mudanças no estado cerebral induzidas pela estimulação. Seu estudo com 16 indivíduos saudáveis demonstrou que a tDCS exerce efeitos específicos de polaridade sobre a dinâmica cortical: a estimulação catódica resultou em valores significativamente menores de Mean State Shift (MSS) em comparação com a anódica, indicando uma transição do sistema cerebral para diferentes regiões do espaço de estados. Além disso, a estimulação catódica afetou a State Variance (SV), enquanto a anódica não produziu alterações detectáveis neste parâmetro. Os autores propõem um modelo teórico baseado em princípios termodinâmicos, sugerindo que o cérebro em repouso ocupa um estado de energia mínima com alta probabilidade, e a estimulação desloca o sistema para um estado de maior energia e menor probabilidade. Esta perspectiva oferece um arcabouço conceitual para compreender como intervenções neuromodulatórias podem induzir transições entre diferentes estados cerebrais, com implicações para o tratamento de condições caracterizadas por padrões disfuncionais de atividade neural.

Em uma abordagem mais ampla, \citeonline{dong2023efficacy} realizaram uma revisão sistemática e meta-análise sobre a eficácia da estimulação cerebral não invasiva (NIBS) em pacientes com transtornos de consciência (DoC). Analisando 17 estudos randomizados controlados com 377 pacientes, os autores concluíram que a NIBS melhora significativamente o estado de consciência em comparação com estimulação simulada. A análise por subgrupos revelou que a estimulação magnética transcraniana repetitiva (rTMS) aplicada ao córtex pré-frontal dorsolateral esquerdo (DLPFC) foi a modalidade mais eficaz, com efeitos mais pronunciados em pacientes no estado de consciência mínima (MCS) do que naqueles com síndrome de vigília não responsiva (UWS/VS). Além disso, protocolos com múltiplas sessões demonstraram resultados superiores aos tratamentos únicos, sugerindo um efeito dose-dependente e de longa duração. Estes achados destacam o potencial da neuromodulação não invasiva como uma abordagem terapêutica para melhorar a consciência em pacientes com DoC, especialmente quando aplicada de forma sistemática e direcionada a regiões cerebrais específicas.

Coletivamente, estes estudos demonstram que as técnicas de neuromodulação não invasiva podem induzir alterações significativas na atividade cerebral e na conectividade funcional, com potencial para normalizar circuitos neurais disfuncionais em diversas condições neuropsiquiátricas. A especificidade dos efeitos observados – dependentes da polaridade da estimulação, da região cerebral alvo, do estado clínico do paciente e dos parâmetros de estimulação – sugere que estas intervenções podem ser personalizadas para abordar mecanismos patofisiológicos específicos, abrindo caminho para abordagens terapêuticas mais precisas e eficazes.

\subsection{Aplicações em Contextos Esportivos, Emocionais e Interpessoais}
A aplicação de técnicas de neuromodulação tem transcendido o domínio clínico, expandindo-se para contextos esportivos, emocionais e interpessoais, onde demonstra potencial para otimizar o desempenho humano e modular interações sociais. Esta diversificação de aplicações reflete a versatilidade destas técnicas e sua capacidade de influenciar múltiplos aspectos do funcionamento cerebral.

No âmbito esportivo, \citeonline{valenzuela2019enhancement} conduziram um estudo cruzado, duplo-cego e controlado por placebo com triatletas de elite, investigando os efeitos da tDCS anódica sobre o córtex motor. Embora a estimulação não tenha melhorado o desempenho físico durante testes de natação de 800 metros, os autores observaram um aumento significativo no vigor percebido tanto antes quanto após o exercício, com tamanhos de efeito expressivos (1,14 e 0,98, respectivamente). Estes resultados sugerem que a tDCS pode exercer efeitos seletivos sobre aspectos psicológicos do desempenho atlético, potencialmente através da modulação de circuitos neurais envolvidos na regulação do humor e na percepção de esforço, mesmo quando medidas objetivas de desempenho permanecem inalteradas.

Complementando esta perspectiva, estudos com estimulação transcraniana por corrente alternada (tACS) têm revelado efeitos promissores sobre funções cognitivas relevantes para o desempenho esportivo. \citeonline{rostami2020transcranial} demonstraram que a tACS aplicada a 6 Hz sobre o córtex pré-frontal medial (mPFC) melhora significativamente a atenção sustentada, evidenciada por aumentos no número total de acertos e na sensibilidade ao alvo durante tarefas de processamento rápido de informações visuais. Análises de EEG revelaram que estes ganhos comportamentais foram acompanhados por um aumento na densidade espectral de potência na faixa theta nas regiões fronto-centrais e por modulações na sincronização de fase alfa na Rede de Atenção Dorsal (DAN). Estes achados são particularmente relevantes para modalidades esportivas que exigem atenção sustentada e processamento rápido de informações visuais, como esportes coletivos e de precisão.

Avançando na compreensão dos mecanismos subjacentes à modulação da atenção, \citeonline{spooner2020hdtdcs} utilizaram magnetoencefalografia (MEG) para investigar como a HD-tDCS aplicada ao córtex pré-frontal dorsolateral (DLPFC) altera a conectividade funcional dinâmica durante tarefas de atenção visual seletiva. Os autores observaram que a estimulação modifica especificamente a conectividade na banda theta entre regiões frontais e visuais, sugerindo um mecanismo pelo qual a neuromodulação pode influenciar processos atencionais fundamentais para o desempenho cognitivo-motor. Esta modulação da conectividade fronto-visual pode ser particularmente relevante para atletas que dependem de processos atencionais eficientes, como aqueles envolvidos em esportes de alta velocidade ou que exigem tomadas de decisão rápidas.

Em contextos interpessoais, a neuromodulação tem demonstrado capacidade de influenciar processos sociais complexos. \citeonline{long2023transcranial} investigaram como a tDCS aplicada ao lobo temporal anterior direito (rATL) afeta a sincronização neural interpessoal (INS) em casais durante interações sociais. Surpreendentemente, a estimulação reduziu a INS entre o rATL das mulheres e o córtex sensorimotor (SMC) dos homens, acompanhada por uma diminuição nos níveis de empatia emocional. Esta modulação da INS impactou a empatia através de comportamentos não verbais, sem alterar os padrões de interação verbal. Estes resultados sugerem que a INS está associada indiretamente aos processos mentais compartilhados através de inputs sensório-motores, apoiando a teoria de representações compartilhadas durante interações sociais e destacando o potencial da neuromodulação para investigar e potencialmente modular processos sociais fundamentais.

No contexto de reabilitação neurológica, \citeonline{liu2023effects} demonstraram que a tDCS pode induzir alterações significativas na potência do EEG e nas redes funcionais cerebrais em pacientes pós-AVC. Especificamente, a estimulação reduziu a potência das oscilações delta e enfraqueceu a conectividade global da rede delta, enquanto aumentou a potência das oscilações alfa e melhorou a conectividade global e local da rede alfa. Estas alterações nos padrões oscilatórios e na arquitetura de rede sugerem um possível mecanismo neurofisiológico pelo qual a tDCS pode contribuir para a recuperação funcional após lesões cerebrais, com potenciais aplicações em programas de reabilitação neurológica e treinamento cognitivo-motor.

De forma similar, \citeonline{han2022functional} investigaram os efeitos da HD-tDCS no DLPFC esquerdo em pacientes com distúrbios crônicos de consciência (DOC). Após dez sessões de estimulação, os pacientes que responderam ao tratamento apresentaram aumentos significativos na conectividade funcional em regiões frontais e parieto-occipitais nas bandas theta (4-8 Hz) e alfa (8-13 Hz), correlacionados com melhorias nos escores clínicos. Notavelmente, pacientes com aumento de conectividade funcional na banda alfa demonstraram melhor prognóstico a longo prazo, sugerindo que a HD-tDCS pode melhorar a consciência através da modulação da conectividade funcional cerebral e que padrões específicos de resposta neural podem servir como biomarcadores prognósticos.

Expandindo as aplicações para contextos domiciliares, \citeonline{xiao2025enhanced} investigaram o uso de tDCS domiciliar em pacientes com depressão bipolar, analisando alterações na sincronização cortical e desenvolvendo preditores de remissão clínica baseados em aprendizado profundo. Os resultados revelaram que a tDCS aumentou significativamente o Phase Locking Value (PLV) delta nas regiões frontal e temporoparietal, correlacionando-se com a melhora nos sintomas depressivos. Além disso, o PLV beta aumentou em pacientes que atingiram remissão e diminuiu naqueles que não responderam ao tratamento. Notavelmente, os autores conseguiram prever a remissão clínica com 69,45% de acurácia utilizando medidas basais de PLV nas bandas theta, beta e gamma, destacando o potencial de medidas de sincronização neural como biomarcadores preditivos da resposta ao tratamento.

Finalmente, \citeonline{arif2021high} investigaram como a HD-tDCS anódica aplicada ao DLPFC direito e esquerdo afeta a inteligência fluida e a conectividade parieto-frontal. A estimulação do DLPFC direito resultou em tempos de resposta significativamente mais rápidos em tarefas de raciocínio lógico, acompanhados por uma redução na conectividade parieto-frontal esquerda que se correlacionou positivamente com o desempenho. Estes achados corroboram a hipótese de eficiência neural, sugerindo que a estimulação do DLPFC direito pode melhorar a eficiência cognitiva ao reduzir o esforço neural necessário para o processamento de informações. Esta modulação da conectividade funcional entre regiões frontais e parietais pode ter implicações significativas para a otimização do desempenho cognitivo em contextos que exigem raciocínio lógico e tomada de decisão, como certos aspectos do desempenho esportivo e interações sociais complexas.

Coletivamente, estes estudos demonstram o potencial diversificado da neuromodulação para influenciar aspectos psicológicos, cognitivos e sociais do funcionamento humano, transcendendo aplicações puramente clínicas e abrindo novas perspectivas para a otimização do desempenho e bem-estar em múltiplos domínios da experiência humana.

\subsection{Abordagens Multidimensionais, Modelos Matemáticos e Monitoramento Integrado}
À medida que o campo da neuromodulação avança, torna-se evidente que abordagens unidimensionais são insuficientes para capturar a complexidade dos efeitos destas intervenções sobre a dinâmica cerebral. Neste contexto, pesquisadores têm desenvolvido metodologias multidimensionais que integram análises quantitativas de EEG, modelos matemáticos sofisticados e técnicas avançadas de monitoramento, permitindo uma compreensão mais profunda e nuançada dos mecanismos subjacentes à neuromodulação.

\citeonline{zhang2022multidimensional} realizaram uma investigação pioneira sobre os efeitos da HD-tDCS no córtex parietal posterior (Pz) em pacientes com distúrbios de consciência (DOC), empregando uma avaliação multidimensional das métricas de EEG. Após 14 dias de estimulação, os pacientes que responderam ao tratamento apresentaram aumentos significativos na conectividade funcional e nas métricas de rede cerebral, particularmente nas bandas alfa e beta. Estas alterações sugerem uma maior integração e eficiência nas redes neurais, potencialmente refletindo a restauração parcial de circuitos de consciência. Notavelmente, os autores desenvolveram um modelo preditivo baseado em aprendizado de máquina que alcançou 92,9\% de acurácia na identificação de pacientes responsivos, destacando a complexidade espacial normalizada (NSC) como um preditor robusto da resposta ao tratamento. Este estudo não apenas reforça a importância do córtex parietal posterior na manutenção da consciência, mas também demonstra o valor de abordagens multidimensionais na caracterização e previsão dos efeitos da neuromodulação.

Explorando a interface entre neuromodulação e plasticidade cognitiva, \citeonline{jones2017frontoparietal} investigaram como a tDCS anódica combinada com treinamento de memória de trabalho (WM) modula a atividade oscilatória cerebral e o desempenho cognitivo. Utilizando EEG de alta densidade, os autores demonstraram que participantes que receberam tDCS ativa direcionada às redes frontoparietais durante o treinamento apresentaram melhorias significativas no desempenho de WM, enquanto o grupo com estimulação simulada não mostrou mudanças. Estes ganhos comportamentais foram acompanhados por alterações eletrofisiológicas específicas: redução da potência alfa posterior e aumento da sincronia de fase nas bandas alfa e teta. Estas mudanças sugerem uma maior eficiência no processamento neural e um fortalecimento da conectividade funcional nas redes relevantes para a memória de trabalho. O estudo destaca como a combinação de neuromodulação e treinamento cognitivo pode induzir alterações sinérgicas na dinâmica cerebral, potencializando a plasticidade neural e a aprendizagem.

A capacidade da tDCS de remodelar redes cerebrais em repouso foi elegantemente demonstrada por \citeonline{pellegrino2018bilateral}, que utilizaram magnetoencefalografia (MEG) para investigar os efeitos da estimulação bilateral em indivíduos saudáveis. Com o ânodo posicionado sobre o córtex sensório-motor esquerdo e o cátodo no direito, a tDCS real, em comparação com a simulada, induziu uma redução na potência das frequências alfa, beta e gama no córtex frontal esquerdo, acompanhada por um aumento na conectividade global em múltiplas bandas de frequência (delta, alfa, beta e gama). Notavelmente, estes efeitos não se limitaram às regiões diretamente sob os eletrodos, mas se estenderam a áreas distantes, sugerindo uma reorganização ampla das redes cerebrais. Estes resultados destacam o potencial da tDCS para induzir plasticidade cerebral de longo alcance, com implicações significativas para o desenvolvimento de intervenções terapêuticas personalizadas em condições neurológicas.

Para elucidar os mecanismos biofísicos subjacentes aos efeitos da tDCS, \citeonline{riedinger2022model} desenvolveram um modelo matemático sofisticado do circuito córtico-talâmico-cortical (CTC), incorporando o Sistema Reticular Ascendente (ARAS). Este modelo teórico explica como a tDCS pode modular a excitabilidade cerebral em estimulações de curta duração e a potência do EEG em estímulos prolongados, estabelecendo conexões com processos de plasticidade de longo prazo (LTP). Aplicando este modelo a um paradigma de psicose induzida por cetamina, os autores conseguiram reproduzir as alterações de potência no EEG observadas experimentalmente sob tDCS, corroborando a hipótese da disfunção dos receptores NMDA na esquizofrenia. O trabalho destaca o papel crítico do ARAS e da sincronização do ritmo delta no circuito CTC, oferecendo insights valiosos sobre os mecanismos neurobiológicos da psicose precoce e como a neuromodulação pode normalizar circuitos disfuncionais.

Avançando para paradigmas de controle mais sofisticados, \citeonline{zhang2024closed} investigaram abordagens de controle em loop fechado para oscilações gama através de estimulações transcranianas. Utilizando um modelo de rede neural cortical e análises de EEG pré e pós-estimulação, os autores demonstraram que estimulações prolongadas, tanto por tDCS quanto por rTMS, podem aumentar significativamente as oscilações gama, promovendo a liberação de fator neurotrófico derivado do cérebro (BDNF) por astrócitos e, consequentemente, melhorando as conexões neuronais. Este mecanismo oferece uma explicação para os efeitos promotores destas intervenções em lesões traumáticas e doenças neurodegenerativas, estabelecendo uma ligação mecanística entre a neuromodulação, a plasticidade sináptica e a recuperação funcional.

A aplicação de análises de grafos tem proporcionado insights valiosos sobre como a tDCS modula a sincronização cortical e a organização topológica das redes cerebrais. \citeonline{mancini2016assessing} utilizaram EEG de 19 canais e métricas baseadas em Synchronization Likelihood (SL) para avaliar os efeitos imediatos da tDCS em redes funcionais cerebrais durante o repouso. Seus resultados revelaram efeitos específicos de polaridade: a tDCS anodal reduziu a sincronização em áreas frontocentrais na banda teta, enquanto a catodal aumentou a conectividade inter-hemisférica em áreas parieto-occipitais na banda alfa. Estas alterações na sincronização cortical foram acompanhadas por modificações nas propriedades de redes funcionais locais e globais, demonstrando o potencial da tDCS para modular a dinâmica de redes cerebrais de forma dependente da polaridade.

Complementando esta linha de investigação, \citeonline{pellegrino2019transcranial} focaram especificamente nos efeitos da tDCS na sincronização gama cortical, um ritmo neural crucial para diversos processos cognitivos. Utilizando MEG e estimulação auditiva de 40 Hz em um experimento randomizado, controlado por placebo e duplo-cego, os autores observaram que a tDCS bilateral (ânodo na região sensório-motora esquerda, cátodo na direita) reduziu significativamente a sincronização gama em 13 dos 15 participantes. Notavelmente, esta redução foi mais pronunciada em áreas distantes do local de estimulação, como o córtex centro-temporal direito, enquanto a sincronização gama basal e as respostas auditivas iniciais permaneceram inalteradas. Estes resultados sugerem que a tDCS inibe seletivamente a sincronização gama induzida externamente, destacando seu potencial para modular mecanismos de plasticidade cortical.

Em um contexto clínico, \citeonline{schollmann2019anodal} investigaram como a tDCS anódica (atDCS) modula a atividade cortical e a sincronização em pacientes com doença de Parkinson. Em um estudo duplo-cego controlado por placebo com 11 pacientes e 10 controles saudáveis, a atDCS aplicada sobre a área sensório-motora esquerda durante uma tarefa de precisão motora melhorou significativamente os sintomas motores e modulou a atividade e sincronização cortical na faixa beta alta (22-27 Hz). Especificamente, observou-se uma redução da atividade no córtex sensório-motor esquerdo e um aumento da sincronização córtico-cortical durante a execução da tarefa motora. Crucialmente, estes efeitos foram específicos do contexto, ocorrendo apenas durante o processamento motor ativo e não durante o repouso ou na condição placebo. Estes achados sugerem que a atDCS pode ajustar disfunções no circuito motor cortical de forma dependente do estado, oferecendo uma abordagem promissora para a reabilitação motora em condições neurodegenerativas.

Para facilitar a investigação em tempo real dos efeitos da neuromodulação, \citeonline{schesatsky2013simultaneous} desenvolveram um dispositivo inovador que permite o monitoramento simultâneo de EEG durante a aplicação de tDCS. Esta metodologia avançada possibilita a avaliação contínua da excitabilidade cortical durante a estimulação, fornecendo informações valiosas sobre os mecanismos de ação da tDCS e permitindo a otimização em tempo real dos parâmetros de estimulação. Esta abordagem integrada representa um avanço significativo na instrumentação para pesquisa em neuromodulação, facilitando estudos mais precisos e personalizados dos efeitos neurais da estimulação transcraniana.

Coletivamente, estas abordagens multidimensionais, modelos matemáticos e técnicas de monitoramento integrado têm expandido significativamente nossa compreensão dos mecanismos subjacentes à neuromodulação, revelando a complexidade e especificidade dos efeitos destas intervenções sobre a dinâmica cerebral. A integração de múltiplas modalidades de análise e a aplicação de frameworks teóricos sofisticados continuam a impulsionar o desenvolvimento de protocolos de neuromodulação mais eficazes e personalizados, com potencial para transformar o tratamento de diversas condições neurológicas e psiquiátricas.

\subsection{Abordagens Alternativas e Complementares}
Enquanto as técnicas convencionais de neuromodulação continuam a evoluir, o campo tem testemunhado o surgimento de abordagens alternativas e complementares que expandem o horizonte terapêutico e investigativo. Estas novas perspectivas não apenas oferecem caminhos adicionais para modular a atividade cerebral, mas também proporcionam insights únicos sobre os mecanismos neurobiológicos subjacentes à sincronização neural e sua relevância para a cognição, comportamento e estados patológicos.

\subsubsection{Sincronia Neural Interpessoal e Transtornos do Neurodesenvolvimento}
Um domínio particularmente promissor é o estudo da sincronia neural interpessoal (INS) – a coordenação temporal entre os sinais cerebrais de duas pessoas durante interações sociais. \citeonline{boecker2024interpersonal} realizaram uma análise abrangente deste fenômeno e seu potencial terapêutico em transtornos caracterizados por disfunções nas interações sociais. Sua revisão revelou que intervenções combinando estimulação cerebral e neurofeedback podem aprimorar significativamente a coordenação de sinais neurais durante interações sociais, com aplicações particularmente relevantes para transtornos do espectro autista (ASD), transtorno de apego reativo (RAD) e transtorno de ansiedade social (SAD). Os autores destacam que, embora estas abordagens mostrem resultados promissores para aumentar a sincronia comportamental e a conexão social, os protocolos ideais de estimulação e parâmetros de neurofeedback ainda precisam ser refinados através de investigações sistemáticas.

Aprofundando a compreensão da sincronização interpessoal no autismo, \citeonline{mcnaughton2020interpersonal} conduziram uma revisão abrangente abordando quatro domínios críticos: motor, conversacional, fisiológico e neural. Seus achados indicam que, embora a sincronização esteja presente em indivíduos com autismo, ela se manifesta de forma reduzida ou atípica em comparação com controles neurotípicos. Esta redução pode ser atribuída tanto a mecanismos intraindividuais, como controle motor e processamento temporal atípicos, quanto a fatores interindividuais, como diferenças no acoplamento social. Significativamente, os autores propõem que a sincronização interpessoal pode servir como um biomarcador valioso para estratificação e intervenção em autismo, especialmente com o advento de tecnologias como sensores vestíveis e análise automatizada de vídeo, que viabilizam novas abordagens para quantificar a sincronização em contextos naturalísticos.

Em contraste com esta perspectiva focada no neurodesenvolvimento, \citeonline{baldwin2014evidence} oferecem uma visão complementar ao revisar as diretrizes da British Association for Psychopharmacology para o tratamento farmacológico de transtornos de ansiedade, TEPT e TOC. Embora seu foco principal seja a farmacoterapia, os autores enfatizam a importância de abordagens não farmacológicas, como a terapia cognitivo-comportamental, que podem ser tão eficazes quanto medicamentos para muitos transtornos. Esta perspectiva reforça a necessidade de considerar múltiplas modalidades terapêuticas, incluindo técnicas de neuromodulação, dentro de um framework de tratamento integrado que considere a gravidade dos sintomas, comorbidades e preferências individuais dos pacientes.

Avançando metodologicamente na caracterização da sincronização neural interpessoal, \citeonline{gerloff2022autism} exploraram a aplicação de técnicas de aprendizado de máquina para classificar indivíduos com TEA com base em biomarcadores neurais diádicos. Utilizando dados de EEG hyperscanning durante interações sociais, os autores demonstraram que representações gráficas complexas derivadas da sincronização neural, especialmente quando analisadas através de métodos de aprendizado não supervisionado, podem melhorar substancialmente a discriminação entre grupos TEA e controles. Este avanço metodológico sugere que a modelagem baseada em grafos de dados interpessoais oferece uma abordagem promissora para desenvolver biomarcadores mais sensíveis para transtornos do neurodesenvolvimento.

Corroborando a relevância funcional da sincronização neural interpessoal, \citeonline{quinones2021dysfunction} investigaram especificamente se déficits nesta sincronização poderiam explicar dificuldades de comunicação em adultos com autismo. Utilizando espectroscopia funcional no infravermelho próximo (fNIRS) durante conversas naturais, os pesquisadores observaram que indivíduos com autismo apresentaram menor sincronização neural no giro temporoparietal (TPJ) em comparação com controles, e que esta redução estava associada a maiores dificuldades autodeclaradas na comunicação social. Estes achados sugerem que a disfunção na coordenação de respostas cerebrais com um parceiro social constitui um mecanismo biológico subjacente às dificuldades sociais no autismo, destacando a importância de estudar a atividade neural em contextos sociais dinâmicos.

Expandindo esta linha de investigação, \citeonline{key2022greater} exploraram a relação entre competência social e sincronização neural interpessoal em adolescentes com autismo, utilizando EEG hyperscanning durante interações sociais naturais. Seus resultados revelaram que a sincronização neural entre participantes com autismo e confederados neurotípicos foi significativamente maior durante conversas do que em repouso, e que esta sincronização em bandas de frequência theta, alpha e beta correlacionou-se negativamente com a gravidade dos sintomas sociais de autismo e positivamente com indicadores de melhor funcionamento social. Notavelmente, o aumento da sincronização neural foi mais consistente entre participantes do sexo feminino, sugerindo possíveis diferenças relacionadas ao sexo na expressão de habilidades sociais em indivíduos com autismo.

Investigando os mecanismos neurais subjacentes às dificuldades de interação social no autismo, \citeonline{tanabe2012hard} utilizaram fMRI dupla (hiperscanning) para examinar as bases neurais da interação olho no olho em indivíduos com TEA de alto funcionamento. Comparando pares TEA-controle com pares controle-controle, os autores observaram que a coerência inter-brain na região do giro frontal inferior direito (IFG) – associada à intenção compartilhada – estava reduzida nos pares TEA-controle. Além disso, a conectividade funcional intra-cérebro entre IFG e sulco temporal superior (STS) também foi mais fraca nos parceiros neurotípicos de indivíduos TEA, correlacionando-se positivamente com o desempenho em tarefas de atenção conjunta. Estes achados sugerem que a falha na sincronização neural entre parceiros pode constituir um mecanismo fundamental para as dificuldades de interação social no TEA, oferecendo um alvo potencial para intervenções neuromodulatórias.

\subsubsection{Neurofeedback como Abordagem Terapêutica}
Paralelamente aos avanços na compreensão da sincronização neural interpessoal, o neurofeedback tem emergido como uma abordagem terapêutica promissora para diversos transtornos neuropsiquiátricos. \citeonline{hou2021neurofeedback} investigaram a eficácia do treinamento de neurofeedback para aumentar a atividade alfa no lobo parietal em pacientes com transtorno de ansiedade generalizada (TAG). Em um estudo randomizado com 26 mulheres diagnosticadas com TAG, os autores demonstraram reduções significativas nos escores de ansiedade-estado, ansiedade-traço, depressão e insônia após dez sessões de treinamento, com melhorias que continuaram a aumentar quatro semanas após a intervenção. Estes resultados sugerem que o aumento da atividade alfa no córtex parietal pode modular efetivamente a atenção e reduzir sintomas ansiosos, possivelmente através da diminuição do viés atencional para estímulos ameaçadores.

Complementando esta abordagem, \citeonline{zilverstand2015fmri} exploraram o potencial do neurofeedback por fMRI para facilitar a regulação da ansiedade em mulheres com fobia de aranhas. Utilizando uma abordagem baseada em reavaliação cognitiva, os pesquisadores demonstraram que participantes que receberam feedback em tempo real sobre a ativação da ínsula direita e do córtex pré-frontal dorsolateral esquerdo conseguiram reduzir mais efetivamente a ativação da ínsula e relataram menor ansiedade ao final do treinamento, em comparação com o grupo controle. Notavelmente, as mudanças na ativação da ínsula durante o treinamento previram melhorias de longo prazo na redução da fobia, medidas até três meses depois, sugerindo que o neurofeedback pode acelerar e reforçar o aprendizado de estratégias de regulação emocional em transtornos de ansiedade.

Estendendo a aplicação do neurofeedback ao transtorno de ansiedade social, \citeonline{kimmig2019feasibility} avaliaram a viabilidade de um treinamento baseado em espectroscopia funcional no infravermelho próximo (NIRS) para modular a atividade do córtex pré-frontal dorsolateral (dlPFC). Após 15 sessões, os participantes demonstraram melhora significativa no desempenho do neurofeedback, redução dos sintomas de ansiedade social, ansiedade traço e depressão, além de diminuição do viés atencional para estímulos sociais ameaçadores. Particularmente relevante foi a observação de que a presença inicial de dificuldades em regular a atividade do dlPFC na presença de estímulos ameaçadores foi superada ao longo do treinamento, e que melhorias no neurofeedback correlacionaram-se com mudanças em padrões de ativação cerebral. A aceitabilidade desta abordagem foi alta, com 9 dos 12 participantes recomendando o tratamento, sugerindo seu potencial como intervenção terapêutica viável.

No contexto do autismo, \citeonline{direito2021training} investigaram a viabilidade e os efeitos de um treinamento de neurofeedback baseado em fMRI em tempo real, direcionado ao sulco temporal superior posterior (pSTS). Após cinco sessões ao longo de oito semanas, os 15 participantes com autismo demonstraram capacidade de modular a atividade do pSTS, com efeitos positivos imediatos e sustentados após seis meses, incluindo melhorias na identificação de expressões de medo e em medidas de comportamento adaptativo. A análise de neuroimagem revelou recrutamento de redes neurais relacionadas a saliência, controle emocional e aprendizado operante, como a ínsula anterior, o córtex cingulado anterior e o corpo do estriado. A ausência de eventos adversos e a alta taxa de adesão reforçam a viabilidade desta abordagem para populações com autismo.

Abordando desafios metodológicos específicos, \citeonline{steiner2014pilot} avaliaram a viabilidade de um protocolo padronizado de neurofeedback para crianças com autismo de alto funcionamento e dificuldades de atenção. Utilizando o sistema Play Attention®, os autores demonstraram que, após seis semanas de treinamento intensivo, a maioria das crianças conseguiu reduzir comportamentos mal-adaptativos, melhorar a concentração durante tarefas acadêmicas e aumentar o tempo de atenção nas sessões de neurofeedback. Embora a adaptação inicial tenha sido desafiadora, o uso de reforço positivo, pausas e exercícios de respiração auxiliou a manter o engajamento, resultando em melhorias significativas em testes de desempenho acadêmico e controle de resposta.

Para populações com autismo e deficiência intelectual, que tradicionalmente são excluídas de intervenções que exigem alta cooperação metodológica, \citeonline{lamarca2018facilitating} investigaram a viabilidade do uso da metodologia TAGteach para preparar estas crianças para participação em treinamento de neurofeedback. O estudo demonstrou que, em média, após cinco horas de TAGteach, crianças com QI ≤ 80 aprenderam habilidades pré-requisito necessárias, como tolerar a preparação de eletrodos e manter atenção visual. Esta abordagem representa um avanço significativo na inclusão de populações mais severamente afetadas em pesquisas e tratamentos baseados em neurofeedback.

Em uma perspectiva mais ampla, \citeonline{catala2017pharmacological} realizaram uma revisão sistemática com meta-análises em rede para comparar a eficácia e segurança de diversos tratamentos para TDAH. Analisando 190 ensaios randomizados com 26.114 participantes, os autores observaram que, embora intervenções como neurofeedback não tenham mostrado evidência convincente de eficácia em comparação com terapia comportamental e medicamentos, apresentaram taxas de aceitação comparáveis a outras intervenções, sugerindo seu potencial como opção terapêutica complementar. Esta perspectiva é particularmente relevante considerando as análises econômicas apresentadas por \citeonline{arnold2013eeg}, que estimam o custo total de uma intervenção de neurofeedback em aproximadamente 1.500 USD para 24 sessões, tornando-a potencialmente acessível em comparação com outras modalidades terapêuticas.

\subsubsection{Potencialização de Oscilações Neurais}
Uma terceira vertente de abordagens alternativas foca na potencialização direta de oscilações neurais específicas. \citeonline{maiella2022simultaneous} investigaram os efeitos da estimulação simultânea por corrente alternada transcraniana (tACS) e estimulação magnética transcraniana (TMS) na modulação das oscilações gama no córtex pré-frontal dorsolateral (DLPFC). Aplicando tACS em 40 Hz juntamente com TMS em participantes saudáveis, os pesquisadores demonstraram um aumento significativo nas oscilações gama no DLPFC após a estimulação combinada. Estes resultados sugerem que esta abordagem sinérgica pode potencializar a plasticidade neural de forma mais eficaz que cada técnica isoladamente, com implicações terapêuticas para distúrbios que envolvem disfunções nas oscilações gama.

Complementarmente, \citeonline{zrenner2020brain} demonstraram a viabilidade e segurança da rTMS sincronizada com oscilações alfa em tempo real no córtex pré-frontal dorsolateral esquerdo (DLPFC) para pacientes com depressão resistente a antidepressivos. Esta abordagem inovadora, que ajusta a estimulação ao estado oscilatório instantâneo do cérebro, reduziu a atividade alfa em repouso, aumentou oscilações beta induzidas por TMS e mostrou efeitos neuromodulatórios específicos não observados em protocolos tradicionais. Embora os resultados sejam preliminares, baseados em uma única sessão, a técnica destaca o potencial de tratamentos personalizados baseados no estado cerebral, representando um avanço significativo na precisão das intervenções neuromodulatórias.

Integrando estas diversas linhas de pesquisa, \citeonline{konrad2024interpersonal} exploraram como a sincronização neural interpessoal pode ser aplicada no tratamento de transtornos mentais caracterizados por disfunções sociais. Os autores destacam que, apesar dos avanços em tecnologias como neurofeedback e estimulação cerebral para manipular a INS, os estudos ainda estão em estágio inicial e carecem de comprovação robusta sobre eficácia e aplicabilidade clínica. Abordagens indiretas, como intervenções comportamentais e biofeedback, mostram potencial, mas a integração da INS como alvo terapêutico exige investigações adicionais para otimizar protocolos, avaliar efeitos de longo prazo e adaptar tratamentos a contextos específicos. Esta perspectiva integrativa sugere que a INS pode abrir caminhos inovadores para melhorar a coesão social e o tratamento de disfunções sociais em diversos transtornos mentais, representando uma fronteira promissora na interface entre neuromodulação e intervenções psicossociais.

Coletivamente, estas abordagens alternativas e complementares expandem significativamente o arsenal terapêutico e investigativo disponível para modular a atividade cerebral e a sincronização neural. Ao transcender as limitações das técnicas convencionais e abordar aspectos únicos da função cerebral – como a sincronização interpessoal, a autorregulação neural e a potencialização de oscilações específicas – estas abordagens oferecem novas perspectivas sobre os mecanismos neurobiológicos subjacentes a diversos transtornos neuropsiquiátricos e abrem caminhos promissores para intervenções mais personalizadas e eficazes.

\section{Medidas Neurofisiológicas e Análise de Sincronização}
A investigação dos efeitos da HD-tDCS sobre a sincronização cerebral requer metodologias robustas para capturar e quantificar as complexas interações entre sistemas neurais e fisiológicos. Neste contexto, a integração de múltiplas técnicas de registro e análise torna-se fundamental para uma compreensão abrangente dos mecanismos subjacentes à neuromodulação e seus impactos na dinâmica corpo-cérebro.

A eletroencefalografia (EEG) constitui a espinha dorsal metodológica deste estudo, fornecendo dados de alta resolução temporal sobre a atividade neural. Conforme destacado por \citeonline{cohen2017where}, o sinal de EEG reflete principalmente a soma dos potenciais pós-sinápticos de populações de neurônios piramidais, permitindo a extração de oscilações que podem ser decompostas nas bandas clássicas – delta (1-4 Hz), theta (4-8 Hz), alpha (8-13 Hz), beta (13-30 Hz) e gamma (>30 Hz). Cada uma destas bandas está associada a escalas temporais e funções cognitivas ou comportamentais específicas, formando a base para análises de conectividade funcional e sincronização. A abordagem de teoria dos grafos, conforme descrita por \citeonline{bullmore2009complex}, permite caracterizar as propriedades topológicas destas redes neurais, identificando hubs (nós com alta centralidade) e padrões de conectividade que são cruciais para a comunicação neural eficiente.

Complementando o EEG, o eletrocardiograma (ECG) fornece informações precisas sobre os ritmos cardíacos, permitindo investigar a interação entre sistemas neural e cardiovascular. Em nosso protocolo, o sinal de ECG foi convertido em uma representação senoidal baseada no pico R, delimitando claramente o ciclo cardíaco e facilitando análises de sincronização com sinais cerebrais. Esta abordagem alinha-se com estudos recentes que destacam a importância da integração cardio-neural na regulação cognitiva e emocional. Em situações onde o ECG tradicional apresenta limitações, a eletromiografia (EMG) estrategicamente posicionada pode capturar a atividade do músculo peitoral maior, refletindo indiretamente a despolarização ventricular (complexo QRS) e permitindo uma integração mais próxima com outros sinais fisiológicos.

A análise do acoplamento de frequências cruzadas (\textit{cross-frequency coupling}, CFC) entre sinais de EEG e ECG representa uma abordagem particularmente inovadora neste estudo. Este fenômeno, no qual oscilações de diferentes frequências interagem entre si, possibilita compreender processos de integração neural e corporal em múltiplas escalas temporais. \citeonline{criscuolo2022cognition} demonstraram a relevância destas interações entre atividade cerebral e sinais periféricos (\textit{brain-body coupling}) na modulação da cognição, enquanto \citeonline{cohen2017where} destacou o papel das interações entre bandas rápidas (como gamma) e lentas (como theta) em processos cognitivos fundamentais. A investigação de como a HD-tDCS modifica estes padrões de acoplamento pode revelar mecanismos pelos quais a neuromodulação influencia a integração corpo-cérebro.

Para quantificar precisamente a sincronização entre sinais neurais e cardíacos, empregamos o Phase Locking Value (PLV), uma métrica robusta que avalia a consistência da relação de fase entre dois sinais ao longo do tempo. O PLV varia entre 0 (ausência de acoplamento) e 1 (acoplamento perfeito), permitindo identificar padrões de sincronização entre diferentes regiões cerebrais ou entre sinais cerebrais e cardíacos. \citeonline{singh2024evaluating} demonstraram a utilidade do PLV para avaliar alterações na conectividade funcional induzidas por intervenções neuromodulatórias, revelando como a tDCS pode reorganizar redes neurais e modificar padrões de sincronização em pacientes com transtorno depressivo maior. Em nosso estudo com atletas, o PLV permite quantificar como a HD-tDCS catódica sobre o DLPFC esquerdo modifica a sincronização entre oscilações cerebrais e o ritmo cardíaco, potencialmente revelando mecanismos de integração corpo-cérebro específicos desta população.

Complementando o PLV, aplicamos medidas de teoria dos grafos para caracterizar propriedades topológicas das redes funcionais cerebrais e sua modulação pela HD-tDCS. Métricas como centralidade de grau, centralidade de intermediação e centralidade de autovetor permitem identificar nós críticos (hubs) nas redes neurais e avaliar como a neuromodulação altera sua importância funcional. Medidas globais como eficiência de rede, coeficiente de agrupamento e comprimento de caminho característico fornecem insights sobre a organização global da rede e sua capacidade de integração e segregação funcional. Estas análises, fundamentadas no trabalho de \citeonline{bullmore2009complex}, permitem uma caracterização multidimensional dos efeitos da HD-tDCS sobre a arquitetura funcional do cérebro e sua interação com o sistema cardiovascular.

Em conjunto, estas abordagens metodológicas fornecem um arcabouço robusto para investigar como a HD-tDCS catódica sobre o DLPFC esquerdo modifica a sincronização entre sistemas neurais e cardiovasculares em atletas de elite, potencialmente revelando mecanismos neurobiológicos subjacentes à integração corpo-cérebro e suas implicações para o desempenho cognitivo e motor.

\section{Integração Conceitual e Fundamentação do Estudo}
A revisão da literatura apresentada nas seções anteriores estabelece um arcabouço teórico e metodológico robusto para a investigação dos efeitos da HD-tDCS catódica sobre o DLPFC esquerdo na sincronização cerebral em atletas de basquetebol feminino. A escolha da estimulação catódica, em contraste com a anódica frequentemente utilizada em estudos anteriores, fundamenta-se nos achados de \citeonline{purpura1965intracellular} e \citeonline{cukic2018shift}, que demonstraram que a polarização catódica hiperpolariza os corpos celulares de neurônios piramidais e induz alterações específicas na dinâmica cortical, potencialmente reduzindo a hiperconectividade e promovendo uma transição do sistema cerebral para diferentes regiões do espaço de estados.

A seleção do DLPFC esquerdo como alvo da estimulação é respaldada por múltiplos estudos que destacam esta região como um nó crítico em redes frontoparietais envolvidas em funções executivas, atenção e controle cognitivo-motor \cite{dong2023efficacy, arif2021high, jones2017frontoparietal}. Estas funções são particularmente relevantes para o desempenho atlético em esportes coletivos como o basquetebol, que exigem tomada de decisão rápida, atenção seletiva e coordenação sensório-motora refinada. Além disso, a aplicação da HD-tDCS, em vez da tDCS convencional, permite uma estimulação mais focal e precisa, conforme demonstrado por \citeonline{villamar2013hdtdcs}, potencialmente maximizando os efeitos neuromodulatórios sobre circuitos neurais específicos.

A análise da sincronização entre sinais de EEG e ECG representa uma abordagem inovadora para investigar os efeitos da neuromodulação sobre a integração entre sistemas neurais e cardiovasculares. Esta perspectiva alinha-se ao conceito de \textit{Body--Brain Dynamic System} (BBDS) apresentado na introdução e é fundamentada por estudos que demonstram a importância da sincronização entre ritmos cerebrais e fisiológicos para a regulação cognitiva e emocional \cite{criscuolo2022cognition, vergara2024exploring}. A quantificação desta sincronização através de métricas como o Phase Locking Value (PLV) e análises de teoria dos grafos permite caracterizar de forma precisa como a HD-tDCS modifica a arquitetura funcional das redes cerebrais e sua interação com o sistema cardiovascular.

A população de atletas de elite de basquetebol feminino representa um grupo particularmente interessante para esta investigação, dado que atletas de alto rendimento frequentemente apresentam padrões distintos de conectividade funcional e regulação autonômica em comparação com não-atletas. Conforme sugerido por \citeonline{valenzuela2019enhancement}, intervenções neuromodulatórias podem exercer efeitos seletivos sobre aspectos psicológicos do desempenho atlético, mesmo quando medidas objetivas de desempenho permanecem inalteradas. Assim, a análise dos efeitos da HD-tDCS sobre a sincronização EEG-ECG em atletas pode revelar mecanismos neurobiológicos subjacentes à integração corpo-cérebro em indivíduos com alto nível de treinamento físico e cognitivo.

O delineamento experimental cruzado (\textit{cross-over}) e duplo-cego adotado neste estudo alinha-se às melhores práticas metodológicas identificadas na literatura revisada, permitindo controlar variáveis confundidoras e isolar os efeitos específicos da neuromodulação. A análise do estado de repouso (\textit{resting-state}), por sua vez, oferece uma janela para observar a organização intrínseca das redes cerebrais e sua modulação pela HD-tDCS, sem a influência de demandas cognitivas ou motoras específicas.

Em síntese, este estudo integra conceitos e metodologias avançadas de neuromodulação, eletrofisiologia e análise de conectividade para investigar como a HD-tDCS catódica sobre o DLPFC esquerdo modifica a sincronização entre sistemas neurais e cardiovasculares em atletas de elite. Os resultados desta investigação têm o potencial de expandir nossa compreensão sobre os mecanismos pelos quais a neuromodulação influencia a integração corpo-cérebro, com implicações tanto para a ciência básica quanto para aplicações práticas no contexto esportivo e clínico.

\section{Reprodutibilidade e Disponibilidade}
O material desta dissertação, incluindo o texto fonte em \LaTeX, códigos de análise, figuras e tabelas completas dos resultados, está disponível publicamente em \cite{barros2025repository} (\url{https://github.com/dantebarross/efeito-da-neuromodulacao-na-sincronicidade-eeg-ecg}). Esta abordagem visa garantir a transparência e reprodutibilidade do trabalho, permitindo que outros pesquisadores possam verificar, replicar ou expandir os resultados apresentados.