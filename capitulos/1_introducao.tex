\chapter{Introdução}
\label{chap:introducao}
Desde o início do século XXI, a neurociência vem avançando significativamente na compreensão da sincronização entre o cérebro e processos fisiológicos, evidenciando o papel das interações dinâmicas na integração entre sistemas corporais e neurais. Nesse contexto, o conceito de \textit{Body-Brain Dynamic System} (BBDS) tem ganhado espaço como uma abordagem para investigar como oscilações neurais se sincronizam com ritmos fisiológicos, como os da frequência cardíaca, respiratória, entre outros, modulando a atividade cerebral \cite{cohen2017where,criscuolo2022cognition}. Esse entendimento tem impulsionado o desenvolvimento de intervenções de neuromodulação capazes de modificar padrões rítmicos e, consequentemente, a função cerebral.

Inspirado especialmente pela síntese de \citeonline{criscuolo2022cognition}, adotamos aqui a visão do \textit{Body-Brain Dynamic System} como um sistema preditivo multimodal. Nessa moldura, os ritmos cardíaco e respiratório não são meros ruídos fisiológicos, mas geradores de ``janelas de excitabilidade'' que modulam a probabilidade de processamento sensorial e a alocação de recursos metabólicos.  Oscilações do locus coeruleus influenciam o acoplamento neurovascular e, em sinergia com baro e quimiorreceptores, ajustam o tônus das redes tálamo-corticais responsáveis por atenção, memória de trabalho e regulação emocional. Consequentemente, percepções auditivas ou visuais apresentadas na fase de diástole tendem a ser detectadas com maior latência, enquanto estímulos entregues em períodos de alta excitabilidade (logo após o R-peak) são processados mais rapidamente. Esses achados consolidam a ideia de que variabilidade cardio-respiratória e flutuações neurais formam um ciclo preditivo corpo-cérebro-comportamento, ponto de partida conceitual desta dissertação.

O corpo humano possui uma capacidade intrínseca de sincronizar seus ritmos fisiológicos com estímulos ambientais e internos. Por exemplo, o ritmo respiratório pode alinhar-se a padrões de atividade sensorial e cognitiva \cite{haas1985effects}, e estados psicofisiológicos (como ansiedade, depressão e estresse) influenciam tanto a frequência cardíaca quanto a atividade neural \cite{criscuolo2022cognition}. Pesquisas recentes demonstram que a variabilidade dos ritmos cardíacos e respiratórios gera ciclos de alta e baixa excitabilidade, modulando a integração e a regulação neural. Estudos em neurocardiologia evidenciam que a interação entre mecanismos sensoriais (como barorreceptores e quimiorreceptores) e centros neurais superiores é essencial para a regulação dos padrões cardíacos \cite{marcondes2024linguagem}. Além disso, pesquisas com potenciais evocados pelo batimento cardíaco (HEPs) reforçam a importância da integração interoceptiva na formação da consciência corporal \cite{banelli2020skipping, mackinnon2013utilizing, park2018neural}.

A neuromodulação compreende técnicas que aplicam estímulos elétricos, magnéticos ou farmacológicos para alterar, de forma controlada, a excitabilidade e a comunicação entre circuitos neurais. Entre as abordagens não invasivas, a estimulação transcraniana por corrente contínua (tDCS), e sua versão de alta definição (HD-tDCS), utiliza correntes de baixa intensidade (≈1-2 mA) administradas no couro cabeludo para polarizar membranas neuronais, produzindo ajustes duradouros nas redes corticais \cite{nitsche2000excitability,stagg2011physiological}. Evidências de registros simultâneos tDCS-EEG revelam que esses ajustes repercutem em padrões de sincronização intra e inter-cortical \cite{kunze2014high}, sustentando a visão contemporânea de cérebro e corpo como um sistema dinâmico integrado no qual alterações na excitabilidade neural podem propagar-se a processos fisiológicos periféricos \cite{criscuolo2022cognition}.

Para explorar os efeitos da neuromodulação na conectividade neural, este projeto investiga como a estimulação transcraniana por corrente contínua de alta definição (HD-tDCS), aplicada de forma catódica sobre o córtex pré-frontal dorsolateral (DLPFC) esquerdo, impacta os padrões de sincronização cerebral em atletas de elite de basquetebol feminino em repouso (\textit{resting-state}). A escolha da estimulação catódica, que tipicamente reduz a excitabilidade cortical, fundamenta-se na hipótese de que a diminuição da atividade no DLPFC esquerdo pode modular o equilíbrio entre redes neurais, potencialmente reduzindo padrões de hiperconectividade e promovendo uma reorganização funcional mais eficiente. O DLPFC esquerdo foi selecionado como alvo por seu papel central em redes frontoparietais envolvidas em funções executivas e controle cognitivo-motor, processos fundamentais para o desempenho atlético em esportes coletivos. O delineamento experimental adotado é do tipo cruzado (\textit{crossover}) e duplo-cego, permitindo que os mesmos participantes sejam submetidos tanto à estimulação ativa quanto à condição controle (\textit{sham}), reduzindo assim a influência de variáveis individuais e possibilitando uma análise mais precisa dos efeitos neuromodulatórios.

Esportistas de alto rendimento exibem padrões neurais mais eficientes e especializados, como sincronia frontoparietal ajustada e menor variabilidade cortical, que constituem um modelo de ``cérebro proficiente'' onde pequenas perturbações neuromodulatórias podem ser detectadas com maior sensibilidade \cite{yarrow2009inside,bertollo2016proficient}. Segundo, já se demonstrou que jogadoras de basquete de nível internacional apresentam diferenças marcantes de oscilação e acoplamento frontal em tarefas motoras e de atenção, sugerindo que a sincronização cerebral é um determinante direto do desempenho \cite{chuang2013differences}. Terceiro, estudos recentes com HD-tDCS em contextos esportivos mostram que esta população responde de forma mensurável à estimulação, tanto em termos de atividade elétrica cerebral quanto de acertos em lances livres, reforçando a relevância aplicada da presente investigação \cite{moscaleski2022hdtdcs}. Desse modo, focar em atletas de alto nível permite testar hipóteses sobre reorganização funcional em um sistema neural já otimizado, ao mesmo tempo em que gera conhecimento potencialmente transferível para intervenções de melhoria de performance.

Neste estudo, além de examinar a sincronização intrafrequencial entre pares de canais de eletroencefalografia (EEG), investigamos o acoplamento cross-frequency entre oscilações cerebrais e o eletrocardiograma (ECG). Partimos da hipótese de que um organismo em homeostase, isto é, um sistema nervoso que opera segundo princípios de predição eficiente, tende a exibir maior sincronização fase-fase EEG-ECG, pois o ritmo cardíaco fornece um compasso temporal que o córtex utiliza para antecipar demandas metabólicas e comportamentais. Nessa perspectiva, uma elevada coerência EEG-ECG indicaria um ``sistema menos perturbado'', já que processos perceptomotores automatizados permaneceriam a cargo de circuitos subcorticais e parietais, enquanto o córtex pré-frontal se manteria relativamente silencioso, evitando gasto energético desnecessário. Em contraste, uma redução dessa sincronia poderia indicar maior recrutamento pré-frontal, frequentemente associado a estados de ansiedade, ruminação ou sobrecarga cognitiva, o que, em um contexto esportivo de alta performance, seria potencialmente subótimo. Para quantificar essa relação, transformamos o ECG em uma onda senoidal ancorada nos picos R, definindo com precisão cada ciclo cardíaco; em seguida, calculamos o acoplamento fase-fase entre essa referência cardíaca e, por exemplo, o canal Fp1 na banda alpha, estimando assim a convergência temporoespectral entre atividade neural e ritmo cardíaco. Essa abordagem permite dissecar a interação dinâmico-energética entre os sistemas neural e cardiovascular e avaliar como intervenções neuromodulatórias modulam essa integração em atletas de elite.

Essa estratégia se ancora em evidências de que o próprio ciclo cardíaco ``reinicializa'' a fase das oscilações corticais, os potenciais evocados pelo batimento cardíaco (HEPs), criando janelas de excitabilidade que influenciam percepção e consciência corporal \cite{park2018neural}. Além disso, estudos com estímulos auditivos sincronizados ao batimento mostram que o cérebro utiliza esse sinal interoceptivo para gerar predições sobre eventos externos, modulando a amplitude dos HEPs 
\cite{banelli2020skipping}. Esses achados reforçam a relevância de ancorar as análises de fase EEG ao R-peak para investigar o acoplamento neuro-cardíaco em atletas.

Modelagens baseadas em previsibilidade temporal da potência, como as de \citeonline{vergara2024exploring}, complementam essa perspectiva ao mostrar que a regularidade do EEG varia sistematicamente ao longo do ciclo cardíaco, ampliando o leque de métricas para caracterizar a interação cérebro-coração.

\section{Neuromodulação e Modulação da Função Cerebral em Contextos Esportivos e Clínicos}
Nas duas últimas décadas, a produção sobre tDCS passou de apenas 16 artigos em 2004 para mais de 1000 publicações em 2024 (PubMed), evidenciando a rápida expansão da neuromodulação não invasiva. Esta seção apresenta uma revisão sistemática das técnicas de neuromodulação, partindo dos princípios neurofisiológicos fundamentais e avançando para aplicações específicas em ambientes clínicos, esportivos e interpessoais. Inicialmente, exploramos as bases neurofisiológicas da estimulação transcraniana por corrente contínua (tDCS) e suas variantes, como a HD-tDCS, detalhando os mecanismos celulares e de rede pelos quais estas técnicas modulam a excitabilidade cortical e a conectividade funcional. Em seguida, examinamos evidências de sua eficácia em contextos clínicos, destacando como a neuromodulação pode normalizar circuitos neurais disfuncionais em diversas condições neuropsiquiátricas. Posteriormente, abordamos aplicações em contextos esportivos e interpessoais, onde estas técnicas demonstram potencial para otimizar o desempenho e modular interações sociais. Finalmente, discutimos abordagens multidimensionais e técnicas alternativas que expandem o horizonte investigativo e terapêutico da neuromodulação. 

Esta progressão lógica permite compreender como os princípios básicos da neuromodulação se traduzem em aplicações práticas e como diferentes abordagens podem ser integradas para uma compreensão mais completa da dinâmica cerebral e sua modulação.

\subsection{Técnicas Convencionais de Neuromodulação}
A estimulação transcraniana por corrente contínua (tDCS) emergiu como uma técnica não invasiva de neuromodulação capaz de induzir alterações controladas na excitabilidade cortical. O trabalho pioneiro de \citeonline{nitsche2000excitability} estabeleceu os princípios fundamentais desta técnica, demonstrando que correntes elétricas fracas (1-2 mA) aplicadas através do crânio podem modular a excitabilidade do córtex motor humano de maneira polaridade-dependente. Seus experimentos revelaram que a estimulação anódica aumenta a excitabilidade cortical, enquanto a catódica a reduz, com efeitos que persistem por vários minutos após o término da estimulação. Estas alterações foram quantificadas através de potenciais evocados motores (MEPs) induzidos por estimulação magnética transcraniana (TMS), evidenciando o potencial da tDCS como ferramenta para modular a atividade cerebral de forma seletiva, reversível e indolor.

Os mecanismos neurofisiológicos subjacentes a estes efeitos foram inicialmente investigados por \citeonline{purpura1965intracellular}, que demonstraram, através de registros intracelulares, que a polarização anódica despolariza os corpos celulares de neurônios piramidais, aumentando sua excitabilidade, enquanto a polarização catódica os hiperpolariza, reduzindo sua atividade espontânea. Este trabalho fundamental revelou ainda que os efeitos da estimulação dependem da orientação das células em relação ao fluxo de corrente, explicando a especificidade dos efeitos observados em diferentes populações neuronais e regiões corticais.

Avançando na compreensão destes mecanismos, \citeonline{stagg2011physiological} elucidaram como a tDCS influencia a plasticidade sináptica no neocórtex, estabelecendo paralelos com processos de potencialização de longo prazo (LTP) e depressão de longo prazo (LTD). Sua revisão destacou o papel crucial de neurotransmissores como glutamato, GABA, dopamina e serotonina na mediação dos efeitos da tDCS, fornecendo uma base neuroquímica para compreender como esta técnica modifica a aprendizagem motora e a conectividade funcional cerebral.

No contexto brasileiro, o trabalho de \citeonline{okano2013estimulacao} representa contribuição relevante ao revisar as aplicações da estimulação transcraniana por corrente contínua (tDCS) na promoção da saúde e no aprimoramento do desempenho físico. Os autores organizaram evidências de que a tDCS pode modular parâmetros cardiovasculares, influenciar o controle do apetite e atenuar a percepção de esforço, reforçando o potencial da técnica como estratégia complementar tanto para otimizar o rendimento esportivo quanto para fins terapêuticos.

Para superar a limitação de focalidade da tDCS convencional, evidenciada por modelagens FEM que mostram dispersão ampla do campo elétrico com montagens bipolares distantes \cite{datta2008transcranial}, desenvolveu-se a estimulação transcraniana por corrente contínua de alta definição (HD-tDCS). \citeonline{villamar2013hdtdcs} descreveram a configuração \emph{4$\times$1}, na qual um eletrodo central (anódico ou catódico) é circundado por quatro eletrodos de retorno, permitindo uma estimulação mais focal e reduzindo a dispersão de corrente. Modelagens computacionais e estudos clínicos confirmam que essa técnica restringe a modulação da excitabilidade cortical à região-alvo, produzindo efeitos mais específicos e potencialmente mais duradouros do que a tDCS convencional.

Expandindo o escopo das técnicas de neuromodulação, \citeonline{kunze2014high} investigaram os efeitos da HD-tDCS no cérebro utilizando EEG simultâneo, demonstrando que a estimulação do córtex sensorimotor esquerdo gera mudanças agudas e persistentes na sincronização cortical. Estas alterações incluem modificações globais e locais na sincronização neural, com efeitos distintos para estimulação anódica e catódica, especialmente na sincronização relacionada à imaginação motora.

Complementando esta perspectiva, \citeonline{scheler2019neuromodulation} explorou como a neuromodulação influencia a sincronização neural e a capacidade intrínseca de leitura em redes neuronais. Seu trabalho demonstrou que a neuromodulação pode transformar as propriedades topológicas de redes neurais, alterando a distribuição de conexões e influenciando propriedades intrínsecas dos neurônios através da modulação de canais iônicos específicos. Esta abordagem teórica fornece um arcabouço para compreender como intervenções neuromodulatórias podem ajustar o equilíbrio entre sincronização e heterogeneidade neural, afetando a capacidade de processamento de informações no córtex cerebral.

Além da tDCS, outras técnicas não invasivas como a estimulação transcraniana por corrente alternada (tACS) e a estimulação magnética transcraniana repetitiva (rTMS) também têm demonstrado capacidade de modificar a sincronização neural e a conectividade funcional, contribuindo para uma compreensão mais abrangente de como a neuromodulação pode ser utilizada para ajustar a dinâmica das redes cerebrais em contextos clínicos e experimentais.

\subsection{Aplicações Clínicas}
A aplicação de técnicas de neuromodulação em contextos clínicos tem revelado seu potencial terapêutico para uma variedade de condições neuropsiquiátricas, com evidências crescentes de sua capacidade de restaurar circuitos neurais disfuncionais e modular padrões de atividade cerebral patológicos. Estas intervenções têm sido particularmente promissoras no tratamento de transtornos resistentes às abordagens farmacológicas convencionais.

No campo dos transtornos de humor, \citeonline{singh2024evaluating} investigaram os efeitos da tDCS em pacientes com transtorno depressivo maior utilizando métricas avançadas de conectividade funcional de EEG em repouso. O protocolo revelou modificações topológicas pronunciadas, sobretudo na banda beta, com aumento do índice de \textit{small-worldness} e redução da aleatoriedade das redes após a intervenção. Os autores mostraram, por meio do \textit{Phase Lag Index} (PLI) e de análises de \textit{hubness}, que a estimulação diminuiu o número de ligações nas bandas theta e alpha, enquanto concentrou e reforçou a conectividade na banda beta, padrão coerente com relatos de redes excessivamente randomizadas em depressão. Essas mudanças vieram acompanhadas de redução significativa nos escores da HAMD, sugerindo que a reorganização funcional pode estar relacionada aos benefícios clínicos observados.

Em pacientes com epilepsia refratária, \citeonline{toutant2024hdtdcs} demonstraram que a HD-tDCS catódica pode atuar como uma intervenção anti-epiléptica através da dessincronização de redes neurais hipersincrônicas. Seu estudo revelou uma redução significativa na sincronização de baixa frequência (bandas delta e theta) após a aplicação de HD-tDCS, particularmente nas regiões frontocentrais e parietais, acompanhada por um discreto aumento na atividade de alta frequência (bandas beta e gamma). Esta modulação do perfil espectral do EEG sugere uma alteração na excitabilidade cortical que pode interromper os padrões de sincronização excessiva característicos da atividade epileptiforme. Análises de conectividade revelaram ainda uma diminuição na coerência entre pares de eletrodos sobre o foco epiléptico e regiões adjacentes, indicando uma disrupção das redes hipersincrônicas que contribuem para a atividade convulsiva. Os efeitos mais pronunciados foram observados em pacientes com epilepsia focal no córtex frontal, sugerindo uma especificidade anatômica na resposta à neuromodulação.

Investigando os mecanismos fundamentais pelos quais a tDCS modifica a dinâmica cerebral, \citeonline{cukic2018shift} utilizaram medidas de Recurrence Quantification Analysis para caracterizar as mudanças no estado cerebral induzidas pela estimulação. Seu estudo com 16 indivíduos saudáveis demonstrou que a tDCS exerce efeitos específicos de polaridade sobre a dinâmica cortical: a estimulação catódica resultou em valores significativamente menores de \textit{Mean State Shift} (MSS) em comparação com a anódica, indicando uma transição do sistema cerebral para diferentes regiões do espaço de estados. Além disso, a estimulação catódica afetou a \textit{State Variance} (SV), enquanto a anódica não produziu alterações detectáveis neste parâmetro. Os autores propõem um modelo teórico baseado em princípios termodinâmicos, sugerindo que o cérebro em repouso ocupa um estado de energia mínima com alta probabilidade, e a estimulação desloca o sistema para um estado de maior energia e menor probabilidade. Esta perspectiva oferece um arcabouço conceitual para compreender como intervenções neuromodulatórias podem induzir transições entre diferentes estados cerebrais, com implicações para o tratamento de condições caracterizadas por padrões disfuncionais de atividade neural.

Em uma abordagem mais ampla, \citeonline{dong2023efficacy} realizaram uma revisão sistemática e meta-análise sobre a eficácia da estimulação cerebral não invasiva (NIBS) em pacientes com transtornos de consciência (DoC). Analisando 17 estudos randomizados controlados com 377 pacientes, os autores concluíram que a NIBS melhora significativamente o estado de consciência em comparação com a estimulação simulada. A análise por subgrupos revelou que a estimulação magnética transcraniana repetitiva (rTMS) aplicada ao córtex pré-frontal dorsolateral esquerdo (DLPFC) foi a modalidade mais eficaz, com efeitos mais pronunciados em pacientes no estado de consciência mínima (MCS) do que naqueles com síndrome de vigília não responsiva (UWS/VS). Além disso, protocolos com múltiplas sessões demonstraram resultados superiores aos tratamentos únicos, sugerindo um efeito dose-dependente e de longa duração. Estes achados destacam o potencial da neuromodulação não invasiva como uma abordagem terapêutica para melhorar a consciência em pacientes com DoC, especialmente quando aplicada de forma sistemática e direcionada a regiões cerebrais específicas.

Coletivamente, estes estudos demonstram que as técnicas de neuromodulação não invasiva podem induzir alterações significativas na atividade cerebral e na conectividade funcional, com potencial para normalizar circuitos neurais disfuncionais em diversas condições neuropsiquiátricas. A especificidade dos efeitos observados, dependentes da polaridade da estimulação, da região cerebral alvo, do estado clínico do paciente e dos parâmetros de estimulação, sugere que estas intervenções podem ser personalizadas para abordar mecanismos patofisiológicos específicos, abrindo caminho para abordagens terapêuticas mais precisas e eficazes.

A literatura revisada converge para alguns pilares que fundamentam nossas hipóteses:
\begin{itemize}
    \item Acoplamento corpo-cérebro: a síntese de \citeonline{criscuolo2022cognition} mostra que oscilações cardíacas e respiratórias definem ``janelas de excitabilidade'' capazes de modular atenção, memória e percepção, o que respalda analisar a sincronização EEG-ECG como alvo sensível de neuromodulação.
    \item Métricas de fase como marcadores: da noção de ``\textit{brainweb}'' em \citeonline{varela2001brainweb} à hipótese de ``\textit{communication-through-coherence}'' em \citeonline{fries2015rhythms}, a coerência de fase emerge como mecanismo-chave de coordenação em larga escala. Optamos, portanto, por índices baseados em fase (PLI para conectividade EEG-EEG intrafrequencial e CF-PLM para o acoplamento EEG-ECG). No caso cardíaco, a amplitude do ECG é dominada pelo complexo QRS; ao extrairmos apenas a fase a partir dos picos R tratamos o ciclo cardíaco como um oscilador endógeno lento, preservando o foco na sincronicidade rítmica relevante.
    \item Pertinência do recorte esportivo: em basquetebolistas profissionais, a HD-tDCS já alterou potência alfa/beta e elevou o aproveitamento em lances livres \cite{moscaleski2022hdtdcs}; atletas de elite constituem um modelo privilegiado para detectar efeitos sutis de reorganização funcional e, futuramente, relacioná-los ao desempenho.
    \item Neuromodulação não invasiva do córtex pré-frontal dorsolateral esquerdo (DLPFC): estudos utilizando rTMS sincronizada com ritmos alfa em tempo real, tDCS convencional e HD-tDCS no DLPFC esquerdo demonstraram alterações na conectividade funcional e na dinâmica oscilatória cortical, justificando sua inclusão neste protocolo \cite{zrenner2020brain,wu2019efficiency,han2022functional,arif2021high}.
\end{itemize}

Esses eixos, corpo-cérebro, métricas de fase, população altamente treinada e neuromodulação do DLPFC esquerdo, estruturam a pergunta central deste trabalho e orientam as hipóteses testadas.

Empregamos HD-tDCS (montagem 4 × 1) por oferecer maior focalidade e gradiente de corrente mais concentrado no DLPFC esquerdo, reduzindo a dispersão do campo elétrico típica da tDCS clássica com eletrodos esponja. Cabe lembrar porém que esta escolha não implica superioridade; a tDCS convencional continua vantajosa em termos de custo, simplicidade e cobertura de áreas extensas. Aqui, entretanto, a necessidade de isolar efeitos locais da região do DLPFC esquerda e minimizar variabilidade inter-sujeitos justificou a adoção do protocolo HD-tDCS.

\subsection{Aplicações em Contextos Esportivos, Emocionais e Interpessoais}
O interesse pela neuromodulação ultrapassou o domínio clínico tradicional, alcançando cenários de alta performance esportiva, regulação emocional e interação social. Ao mapear esses três eixos, observa-se um denominador comum: a capacidade de técnicas como tDCS, HD-tDCS, tACS ou combinações com TMS de ajustar, em escalas de segundos a dias, os padrões de sincronização de fase que sustentam processos cognitivo-motores, precisamente o foco desta dissertação.

Em triatletas de elite, a tDCS anódica sobre o córtex motor não se traduziu em ganhos físicos imediatos, mas aumentou expressivamente o vigor subjetivo antes e depois de um teste de natação (tamanhos de efeito \(d\approx1\)) \cite{valenzuela2019enhancement}. Evidências mais diretamente alinhadas ao basquete vêm de intervenções frontais: tACS a 6 Hz (mPFC) elevou a atenção sustentada e modulou potência-theta/sincronia-alpha em redes fronto-centrais \cite{rostami2020transcranial}, enquanto HD-tDCS catódica no DLPFC ajustou a conectividade theta fronto-visual durante tarefas de atenção seletiva \cite{spooner2020hdtdcs}. Esses resultados sugerem que neuromodular ritmos lentos (theta) pode refinar o circuito de seleção visuomotora crucial para arremessos precisos e decisões rápidas em quadra.

Ainda no domínio cognitivo, estimular o DLPFC direito com HD-tDCS anódica reduziu o tempo de resposta em problemas de lógica e diminuiu a conectividade parieto-frontal esquerda, um quadro compatível com a ``hipótese da eficiência neural'' \cite{arif2021high}. Tal redução de custo neural para resolver tarefas complexas reforça a noção de que manipular \textit{hubs} frontais pode liberar recursos atencionais em esportes de alta demanda.

Protocolos de \emph{state-dependent} neuromodulação ilustram um caminho complementar: tACS a 40 Hz sobreposta a TMS no DLPFC aumentou de forma sustentada a potência e a coerência gamma \cite{maiella2022simultaneous}, ao passo que rTMS sincronizada à fase-alpha reduziu a potência alfa basal e fortaleceu oscilações beta em depressão resistente \cite{zrenner2020brain}. Ao privilegiar o momento de máxima receptividade oscilatória, tais abordagens apontam para futuros esquemas \textit{closed-loop} em que se possa estabilizar o acoplamento fase-fase entre EEG e ECG, hipótese central do presente trabalho.

Quando dois cérebros interagem, a coerência inter-sujeitos emerge como possível substrato da empatia e da cooperação. Revisão recente destaca a INS como biomarcador promissor para transtornos sociais, mas ainda carente de padronização metodológica \cite{konrad2024interpersonal}. Em casais, tDCS no lobo temporal anterior direito reduziu a INS rATL-SMC e baixou a empatia emocional \cite{long2023transcranial}, confirmando que oscilações compartilhadas podem ser moduladas externamente. Embora a INS esteja fora do escopo direto desta dissertação, o paralelismo conceitual (coerência como ``cola'' de sistemas) reforça a relevância de medir a sincronia EEG-ECG como marcador de integração corpo-cérebro em atletas.

Em síntese, a literatura indica que ritmos frontais lentos são moduláveis, que protocolos \textit{state-dependent} refinam a dinâmica de fase e que a coerência (entre regiões corticais, entre cérebros ou entre cérebro e coração) é métrica-chave de eficiência funcional. Essa convergência justifica o uso, neste trabalho, de índices baseados em fase para investigar como a HD-tDCS catódica sobre o DLPFC esquerdo reorganiza a sincronia EEG-EEG e EEG-ECG em jogadoras de basquete de elite.

\subsection{Abordagens Multidimensionais, Modelos Matemáticos e Monitoramento Integrado}
À medida que o campo da neuromodulação evolui, reconhece-se que as abordagens unidimensionais forneceram avanços valiosos por capturar parte da complexidade envolvida nos efeitos dessas intervenções sobre a dinâmica cerebral. Entretanto, investigações recentes mostram que protocolos analíticos multidimensionais (integrando métricas de fase e potência do EEG, modelagem computacional de redes dinâmicas e indicadores periféricos como variabilidade cardíaca) oferecem uma lente mais abrangente para decifrar as adaptações neuroplásticas induzidas. Tais metodologias integrativas mitigam limitações das abordagens isoladas e possibilitam inferências mecanisticamente mais robustas, orientando intervenções terapêuticas e estratégias de otimização com maior precisão.

\citeonline{zhang2022multidimensional} aplicaram, durante 14 dias, um protocolo de HD-tDCS anódica centrado em Pz (2 mA; quatro eléctrodos de retorno a 3{,}5 cm) em 42 pacientes com distúrbios de consciência. Dos participantes, 32 foram classificados como respondedores pela variação da pontuação CRS-R. Nesses pacientes observou-se (i) incremento da energia espectral principalmente nas bandas alfa\textsubscript{2} e beta\textsubscript{1} em eléctrodos frontais e parietais; (ii) aumento da eficiência global, da eficiência local e do índice \textit{small-world} na banda alfa\textsubscript{1}, bem como elevação do índice \textit{small-world} na banda beta\textsubscript{1}; e (iii) acréscimo dos coeficientes de clusterização nos nodos Cz e Pz, denotando reforço de conectividade local. Adicionalmente, os autores treinaram um classificador SVM com a complexidade espacial normalizada (NSC) das sete bandas na linha de base e alcançaram 92{,}9\% de acurácia para prever a responsividade ao tratamento, com maior peso das bandas alfa e gama. O estudo evidencia, assim, que a neuromodulação do córtex parietal posterior pode induzir reorganizações funcionais mensuráveis por métricas multidimensionais de EEG e que essas mesmas métricas podem antecipar o desfecho clínico da intervenção.

Explorando a interface entre neuromodulação e plasticidade cognitiva, \citeonline{jones2017frontoparietal} investigaram como a tDCS anódica combinada com treinamento de memória de trabalho (\textit{working memory}, WM) modula a atividade oscilatória cerebral e o desempenho cognitivo. Utilizando EEG de alta densidade, os autores demonstraram que participantes que receberam tDCS ativa direcionada às redes frontoparietais durante o treinamento apresentaram melhorias significativas no desempenho de WM, enquanto o grupo com estimulação simulada não mostrou mudanças. Estes ganhos comportamentais foram acompanhados por alterações eletrofisiológicas específicas: redução da potência alfa posterior e aumento da sincronia de fase nas bandas alfa e teta. Estas mudanças sugerem uma maior eficiência no processamento neural e um fortalecimento da conectividade funcional nas redes relevantes para a memória de trabalho. O estudo destaca como a combinação de neuromodulação e treinamento cognitivo pode induzir alterações sinérgicas na dinâmica cerebral, potencializando a plasticidade neural e a aprendizagem.

A capacidade da tDCS de remodelar redes cerebrais em repouso foi demonstrada por \citeonline{pellegrino2018bilateral}, que utilizaram magnetoencefalografia (MEG) para investigar os efeitos da estimulação bilateral em indivíduos saudáveis. Com o ânodo posicionado sobre o córtex sensório-motor esquerdo e o cátodo no direito, a tDCS real, em comparação com a simulada, induziu uma redução na potência das frequências alfa, beta e gamma no córtex frontal esquerdo, acompanhada por um aumento na conectividade global em múltiplas bandas de frequência (delta, alfa, beta e gamma). Notavelmente, estes efeitos não se limitaram às regiões diretamente sob os eletrodos, mas se estenderam a áreas distantes, sugerindo uma reorganização ampla das redes cerebrais. Estes resultados destacam o potencial da tDCS para induzir plasticidade cerebral de longo alcance, com implicações significativas para o desenvolvimento de intervenções terapêuticas personalizadas em condições neurológicas.

Para elucidar os mecanismos biofísicos subjacentes aos efeitos da tDCS, \citeonline{riedinger2022model} desenvolveram um modelo matemático sofisticado do circuito córtico-talâmico-cortical (CTC), incorporando o Sistema Reticular Ascendente (ARAS). Este modelo teórico explica como a tDCS pode modular a excitabilidade cerebral em estimulações de curta duração e a potência do EEG em estímulos prolongados, estabelecendo conexões com processos de plasticidade de longo prazo (LTP). Aplicando este modelo a um paradigma de psicose induzida por cetamina, os autores conseguiram reproduzir as alterações de potência no EEG observadas experimentalmente sob tDCS, corroborando a hipótese da disfunção dos receptores NMDA na esquizofrenia. O trabalho destaca o papel crítico do ARAS e da sincronização do ritmo delta no circuito CTC, oferecendo insights valiosos sobre os mecanismos neurobiológicos da psicose precoce e como a neuromodulação pode normalizar circuitos disfuncionais.

Avançando para paradigmas de controle mais sofisticados, \citeonline{zhang2024closed} investigaram abordagens de controle em loop fechado para oscilações gamma através de estimulações transcranianas. Utilizando um modelo de rede neural cortical e análises de EEG pré e pós-estimulação, os autores demonstraram que estimulações prolongadas, tanto por tDCS quanto por rTMS, podem aumentar significativamente as oscilações gamma, promovendo a liberação de fator neurotrófico derivado do cérebro (BDNF) por astrócitos e, consequentemente, melhorando as conexões neuronais. Este mecanismo oferece uma explicação para os efeitos promotores destas intervenções em lesões traumáticas e doenças neurodegenerativas, estabelecendo uma ligação mecanística entre a neuromodulação, a plasticidade sináptica e a recuperação funcional.

A aplicação de análises de grafos tem proporcionado insights valiosos sobre como a tDCS modula a sincronização cortical e a organização topológica das redes cerebrais. \citeonline{mancini2016assessing} utilizaram EEG de 19 canais e métricas baseadas em \textit{Synchronization Likelihood} (SL) para avaliar os efeitos imediatos da tDCS em redes funcionais cerebrais durante o repouso. Seus resultados revelaram efeitos específicos de polaridade: a tDCS anódica reduziu a sincronização em áreas frontocentrais na banda teta, enquanto a catódica aumentou a conectividade inter-hemisférica em áreas parieto-occipitais na banda alfa. Estas alterações na sincronização cortical foram acompanhadas por modificações nas propriedades de redes funcionais locais e globais, demonstrando o potencial da tDCS para modular a dinâmica de redes cerebrais de forma dependente da polaridade.

Complementando esta linha de investigação, \citeonline{pellegrino2019transcranial} focaram especificamente nos efeitos da tDCS na sincronização gamma cortical, um ritmo neural crucial para diversos processos cognitivos. Utilizando MEG e estimulação auditiva de 40 Hz em um experimento randomizado, controlado por placebo e duplo-cego, os autores observaram que a tDCS bilateral (ânodo na região sensório-motora esquerda, cátodo na direita) reduziu significativamente a sincronização gamma em 13 dos 15 participantes. Notavelmente, esta redução foi mais pronunciada em áreas distantes do local de estimulação, como o córtex centro-temporal direito, enquanto a sincronização gamma basal e as respostas auditivas iniciais permaneceram inalteradas. Estes resultados sugerem que a tDCS inibe seletivamente a sincronização gamma induzida externamente, destacando seu potencial para modular mecanismos de plasticidade cortical.

Em um contexto clínico, \citeonline{schollmann2019anodal} investigaram como a tDCS anódica (atDCS) modula a atividade cortical e a sincronização em pacientes com doença de Parkinson. Em um estudo duplo-cego controlado por placebo com 11 pacientes e 10 controles saudáveis, a atDCS aplicada sobre a área sensório-motora esquerda durante uma tarefa de precisão motora melhorou significativamente os sintomas motores e modulou a atividade e sincronização cortical na faixa beta alta (22-27 Hz). Especificamente, observou-se uma redução da atividade no córtex sensório-motor esquerdo e um aumento da sincronização córtico-cortical durante a execução da tarefa motora. Crucialmente, estes efeitos foram específicos do contexto, ocorrendo apenas durante o processamento motor ativo e não durante o repouso ou na condição placebo. Estes achados sugerem que a atDCS pode ajustar disfunções no circuito motor cortical de forma dependente do estado, oferecendo uma abordagem promissora para a reabilitação motora em condições neurodegenerativas.

Para facilitar a investigação em tempo real dos efeitos da neuromodulação, \citeonline{schesatsky2013simultaneous} desenvolveram um dispositivo inovador que permite o monitoramento simultâneo de EEG durante a aplicação de tDCS. Esta metodologia avançada possibilita a avaliação contínua da excitabilidade cortical durante a estimulação, fornecendo informações valiosas sobre os mecanismos de ação da tDCS e permitindo a otimização em tempo real dos parâmetros de estimulação. Esta abordagem integrada representa um avanço significativo na instrumentação para pesquisa em neuromodulação, facilitando estudos mais precisos e personalizados dos efeitos neurais da estimulação transcraniana.

Coletivamente, estas abordagens multidimensionais, modelos matemáticos e técnicas de monitoramento integrado têm expandido significativamente nossa compreensão dos mecanismos subjacentes à neuromodulação, revelando a complexidade e especificidade dos efeitos destas intervenções sobre a dinâmica cerebral. A integração de múltiplas modalidades de análise e a aplicação de frameworks teóricos sofisticados continuam a impulsionar o desenvolvimento de protocolos de neuromodulação mais eficazes e personalizados, com potencial para transformar o tratamento de diversas condições neurológicas e psiquiátricas.

\subsection{Abordagens Alternativas e Complementares}
Enquanto as técnicas convencionais de neuromodulação continuam a evoluir, o campo tem testemunhado o surgimento de abordagens alternativas e complementares que expandem o horizonte terapêutico e investigativo. Estas novas perspectivas não apenas oferecem caminhos adicionais para modular a atividade cerebral, mas também proporcionam insights únicos sobre os mecanismos neurobiológicos subjacentes à sincronização neural e sua relevância para a cognição, comportamento e estados patológicos.

Vale notar que abordagens de realimentação em tempo real conhecidas como \textit{neurofeedback} vêm mostrando que oscilações neurais podem ser moduladas de forma endógena. Treinamentos para amplificar atividade alpha parietal reduziram ansiedade-estado e melhorararam regulação emocional \cite{hou2021neurofeedback}, enquanto esquemas híbridos que combinam tDCS e \textit{neurofeedback} em \textit{closed-loop} começam a emergir como estratégia para ajustar a excitabilidade cortico-subcortical de maneira personalizada \cite{zilverstand2015fmri}. A convergência metodológica (uso de métricas de fase, como PLV, para guiar a intervenção) reforça a pertinência de explorar, em trabalhos futuros, protocolos que integrem feedback online da sincronia EEG-ECG com a HD-tDCS catódica aqui investigada.

Uma vertente cada vez mais citada procura amplificar ritmos específicos para favorecer a coerência tanto intra-cortical quanto cérebro-coração. Protocolos que combinam tACS e TMS em regime sinérgico ilustram o potencial dessa estratégia: aplicando tACS a 40 Hz enquanto se entrega TMS no DLPFC, \citeonline{maiella2022simultaneous} observaram elevação sustentada de potência e coerência gamma, efeito provável de \emph{spike-timing} reforçado nos interneurônios GABAérgicos. De forma convergente, rTMS sincronizada em tempo real à fase das oscilações alpha no DLPFC esquerdo reduziu a potência alpha basal e intensificou a conectividade beta em pacientes com depressão resistente \cite{zrenner2020brain}, sugerindo que protocolos \emph{state-dependent} (nos quais o estímulo é gatilhado no ponto de máxima receptividade oscilatória) podem produzir ajustes finos nas redes de larga escala. Tais achados dialogam com a lógica desta dissertação: se a HD-tDCS catódica reorganiza a sincronia EEG-ECG ao rebaixar a excitabilidade pré-frontal, intervenções de potencialização dirigida poderiam, em um futuro \emph{closed-loop}, estabilizar o acoplamento fase-fase entre ciclos cardíacos e oscilações corticais, promovendo um regime energeticamente econômico de predição ativa.


\section{Medidas Neurofisiológicas e Análise de Sincronização}
A investigação dos efeitos da HD-tDCS sobre a sincronização cerebral requer metodologias robustas para capturar e quantificar as complexas interações entre sistemas neurais e fisiológicos. Neste contexto, a integração de múltiplas técnicas de registro e análise torna-se fundamental para uma compreensão abrangente dos mecanismos subjacentes à neuromodulação e seus impactos na dinâmica corpo-cérebro.

A eletroencefalografia (EEG) e o eletrocardiograma (ECG) constituem a espinha dorsal metodológica deste estudo. O EEG oferece resolução temporal de milissegundos e reflete, majoritariamente, a soma dos potenciais pós-sinápticos de populações de neurônios piramidais \cite{cohen2017where}; a decomposição espectral clássica isola ritmos delta (1-4 Hz), theta (4-8 Hz), alpha (8-13 Hz), beta (13-30 Hz) e gamma (> 30 Hz), cada qual associado a escalas temporais e funções cognitivas distintas. Já o ECG fornece a marcação precisa dos picos R, permitindo tratar o ciclo cardíaco como um oscilador endógeno lento, premissa fundamental para quantificar o acoplamento fase-fase cérebro-coração em análises de \textit{cross-frequency} \cite{criscuolo2022cognition,park2018neural}. Com esses dois sinais em sincronia, aplicamos a teoria dos grafos para caracterizar topologia e \textit{hubs} de redes neurais \cite{bullmore2009complex} e, simultaneamente, avaliar como a HD-tDCS catódica altera a coerência tanto intra-cortical (EEG-EEG) quanto cérebro-coração (EEG-ECG) em atletas de elite.

Complementando o EEG, o eletrocardiograma (ECG) fornece informações precisas sobre os ritmos cardíacos e permite investigar a interação entre sistemas neural e cardiovascular. Em nosso protocolo, o sinal de ECG é transformado em uma onda senoidal ancorada nos picos R, definindo claramente cada ciclo cardíaco e facilitando as análises de sincronização com sinais cerebrais. Esta abordagem está em linha com evidências de que a integração cardio-neural modula atenção, emoção e percepção, conforme demonstrado por estudos sobre o corpo-cérebro dinâmico \cite{criscuolo2022cognition}, potenciais evocados pelo batimento cardíaco \cite{park2018neural,banelli2020skipping}, efeitos de respiração ressonante nos HEPs \cite{mackinnon2013utilizing} e modulação dependente da fase cardíaca durante imaginação motora \cite{lai2024cardiac}.

A análise do acoplamento de frequências cruzadas (\textit{cross-frequency coupling}, CFC) entre sinais de EEG e ECG representa uma abordagem particularmente inovadora neste estudo. Este fenômeno, no qual oscilações de diferentes frequências interagem entre si, possibilita compreender processos de integração neural e corporal em múltiplas escalas temporais. \citeonline{criscuolo2022cognition} demonstraram a relevância destas interações entre atividade cerebral e sinais periféricos (\textit{brain-body coupling}) na modulação da cognição, enquanto \citeonline{cohen2017where} destacou o papel das interações entre bandas rápidas (como gamma) e lentas (como theta) em processos cognitivos fundamentais. A investigação de como a HD-tDCS modifica estes padrões de acoplamento pode revelar mecanismos pelos quais a neuromodulação influencia a integração corpo-cérebro.

Para quantificar precisamente a sincronização entre sinais neurais e cardíacos, empregamos o \textit{Phase Lag Index} (PLI), uma métrica que avalia a consistência da defasagem de fase (\textit{phase lag}) entre dois sinais, sendo insensível ao volume de condução. O PLI varia entre 0 (sem acoplamento de fase confiável) e 1 (acoplamento de fase altamente consistente com defasagem constante), o que o torna apropriado para identificar padrões de sincronização entre diferentes regiões cerebrais ou entre sinais cerebrais e cardíacos. Em nosso estudo com atletas, o PLI foi utilizado nas análises intrafrequenciais (entre sinais na mesma banda de frequência), permitindo avaliar como a HD-tDCS catódica sobre o DLPFC esquerdo altera a sincronização entre oscilações cerebrais e o ritmo cardíaco, revelando possíveis mecanismos de integração corpo-cérebro específicos dessa população.

Complementando essa abordagem, aplicamos o \textit{Cross-Frequency Phase Linearity Measurement} (CF-PLM) para investigar a conectividade \textit{cross-frequency} (entre frequências distintas), e em nosso caso entre diferentes bandas de EEG e o sinal de ECG. O CF-PLM possibilita detectar relações de fase entre oscilações cerebrais em diferentes faixas de frequência, oferecendo uma visão mais abrangente das dinâmicas neurocardíacas moduladas pela HD-tDCS.

Adicionalmente, utilizamos medidas de teoria dos grafos para caracterizar propriedades topológicas das redes funcionais cerebrais e sua modulação pela neuromodulação. Métricas como centralidade de grau, centralidade de intermediação e centralidade de autovetor permitem identificar nós críticos (\textit{hubs}) nas redes neurais e avaliar alterações na sua importância funcional induzidas pela tDCS. Medidas globais como eficiência de rede, coeficiente de agrupamento e comprimento de caminho característico fornecem insights sobre a organização geral da rede e sua capacidade de integrar e segregar informações de forma eficiente. Estas análises, fundamentadas no trabalho de \citeonline{bullmore2009complex}, oferecem uma caracterização multidimensional dos efeitos da HD-tDCS sobre a arquitetura funcional do cérebro e sua interação com o sistema cardiovascular.

\section{Integração Conceitual e Fundamentação do Estudo}
A revisão da literatura apresentada nas seções anteriores estabelece um arcabouço teórico e metodológico para a investigação dos efeitos da HD-tDCS catódica sobre o DLPFC esquerdo na sincronização cerebral em atletas de basquetebol feminino. A escolha da estimulação catódica, em contraste com a anódica frequentemente utilizada em estudos anteriores, fundamenta-se nos achados de \citeonline{purpura1965intracellular} e \citeonline{cukic2018shift}, que demonstraram que a polarização catódica hiperpolariza os corpos celulares de neurônios piramidais e induz alterações específicas na dinâmica cortical, potencialmente reduzindo a hiperconectividade e promovendo uma transição do sistema cerebral para diferentes regiões do espaço de estados.

A escolha do DLPFC esquerdo como alvo deve-se ao fato de esta área atuar como um nó estratégico das redes frontoparietais que sustentam funções executivas, atenção e controle cognitivo-motor \cite{dong2023efficacy,arif2021high,jones2017frontoparietal}. Tais funções são cruciais no basquetebol, que demanda decisão rápida, atenção seletiva e coordenação sensório-motora refinada. Para modulá-las, empregamos a HD-tDCS (montagem 4 × 1), que gera um gradiente de corrente mais concentrado e reduz a dispersão para regiões adjacentes, como mostrado em \citeonline{villamar2013hdtdcs}. Salienta-se, contudo, que a tDCS convencional continua válida em muitas aplicações: seu campo elétrico mais difuso pode ser desejável quando se busca engajar redes amplas, seu custo é menor e o equipamento é mais acessível. No presente estudo, a maior focalidade da HD-tDCS serve principalmente para isolar estatisticamente os efeitos da estimulação sobre o DLPFC, minimizando a contribuição de áreas vizinhas sem implicar superioridade absoluta sobre a técnica convencional.

A análise da sincronização fásica entre sinais de EEG e ECG representa uma nova abordagem para investigar os efeitos da neuromodulação sobre a integração entre sistemas neurais e cardiovasculares. Esta perspectiva alinha-se ao conceito de \textit{Body-Brain Dynamic System} (BBDS) apresentado na introdução e é sustentada por estudos que destacam a importância da coerência entre ritmos cerebrais e fisiológicos para a regulação cognitiva e emocional \cite{criscuolo2022cognition, vergara2024exploring}. Em nossas análises, utilizamos o Phase Lag Index (PLI) para avaliar a sincronização intrafrequencial entre EEG e ECG, e o Cross-Frequency Phase Linearity Measurement (CF-PLM) para identificar acoplamentos entre oscilações de diferentes bandas de frequência. Estas métricas complementares nos permitem examinar com maior precisão como a HD-tDCS pode reorganizar a dinâmica de acoplamento funcional entre cérebro e coração.

A população de atletas de elite de basquetebol feminino representa um grupo particularmente interessante para esta investigação, dado que atletas de alto rendimento frequentemente apresentam padrões distintos de conectividade funcional e regulação autonômica em comparação com não-atletas. Conforme sugerido por \citeonline{valenzuela2019enhancement}, intervenções neuromodulatórias podem exercer efeitos seletivos sobre aspectos psicológicos do desempenho atlético, mesmo quando medidas objetivas de desempenho permanecem inalteradas. Assim, a análise dos efeitos da HD-tDCS sobre a sincronização EEG-EEG (intrafrequencial) e EEG-ECG (\textit{cross-frequency}) em atletas pode revelar mecanismos neurobiológicos subjacentes à integração corpo-cérebro em indivíduos com alto nível de treinamento físico e cognitivo.

O delineamento experimental cruzado (\textit{crossover}) e duplo-cego adotado neste estudo alinha-se às melhores práticas metodológicas identificadas na literatura revisada, permitindo controlar variáveis confundidoras e isolar os efeitos específicos da neuromodulação. A análise do estado de repouso (\textit{resting-state}), por sua vez, oferece uma janela para observar a organização intrínseca das redes cerebrais e sua modulação pela HD-tDCS, sem a influência de demandas cognitivas ou motoras específicas.

Em síntese, este estudo integra conceitos e metodologias avançadas de neuromodulação, eletrofisiologia e análise de conectividade para investigar como a HD-tDCS catódica sobre o DLPFC esquerdo modula a sincronização fásica entre sistemas neurais e cardiovasculares em atletas de elite. Os resultados desta investigação têm o potencial de expandir nossa compreensão sobre os mecanismos pelos quais a neuromodulação influencia a integração corpo-cérebro, com implicações tanto para a ciência básica quanto para aplicações práticas no contexto esportivo e clínico.

\section{Reprodutibilidade e Disponibilidade}
O material desta dissertação, incluindo o texto fonte em \LaTeX, códigos de análise, figuras e tabelas completas dos resultados, está disponível publicamente em \citeonline{barros2025repository}\footnote{\url{https://github.com/dantebarross/efeito-da-neuromodulacao-na-sincronicidade-eeg-ecg}}. Esta abordagem visa garantir a transparência e reprodutibilidade do trabalho, permitindo que outros pesquisadores possam verificar, replicar ou expandir os resultados apresentados.