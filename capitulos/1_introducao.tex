\chapter{Introdução}
\label{chap:introducao}

A neurociência tem avançado na compreensão da sincronização entre o cérebro e processos fisiológicos, destacando o papel das interações dinâmicas nesse contexto. O conceito de sistema dinâmico cérebro-corpo (BBDS – Body Brain Dynamic System) vem ganhando espaço como uma abordagem para investigar essa integração e suas implicações na atividade neural.

O avanço de tecnologias como a eletroencefalografia (EEG) permitiu uma investigação detalhada das oscilações cerebrais tanto durante a execução de atividades quanto em resting state, um estado de repouso no qual a atividade neural ocorre espontaneamente, sem a influência direta de estímulos externos ou tarefas específicas. Essas oscilações, cuja origem e significado ainda são amplamente debatidos, têm demonstrado a capacidade de se sincronizar com ritmos corporais – como os da frequência cardíaca e respiratória – formando um padrão rítmico integrado que pode modular a atividade neural \cite{criscuolo2022cognition, cohen2017where}. Esse entendimento sobre a dinâmica das oscilações neurais tem sido fundamental para o desenvolvimento de intervenções de neuromodulação, que exploram a modificação desses padrões rítmicos para influenciar a atividade cerebral.

Além de analisar a sincronização intrafrequencial entre pares de EEG (EEG-EEG), o presente estudo investiga, de maneira inovadora, a sincronicidade cross-frequency entre sinais de EEG e ECG. Enquanto as análises EEG-EEG foram realizadas dentro da mesma banda de frequência — permitindo a avaliação da conectividade funcional entre diferentes regiões cerebrais — a análise cross-frequency focou na interação entre sinais de EEG, filtrados em bandas específicas (delta, theta, alpha, beta e gamma), e o sinal de ECG.

O sinal de ECG foi convertido em uma representação senoidal simples baseada no pico R, o que resulta em um sinal com uma única frequência intrínseca correspondente ao ciclo cardíaco. Assim, ao comparar, por exemplo, o canal Fp1 na banda alpha com o sinal de ECG, é possível quantificar como a atividade cerebral nessa banda se sincroniza com o ritmo cardíaco (que só possui uma mesma banda de frequência, por ser um sinal senoidal simples). Esse procedimento foi repetido para cada canal e banda de EEG, permitindo uma análise detalhada do acoplamento entre oscilações cerebrais de diferentes frequências e o sinal cardíaco. Dessa forma, o estudo contribui para uma compreensão mais aprofundada da interação entre os sistemas neural e cardiovascular, especialmente no contexto de neuromodulação aplicada ao sujeito atleta.

Este projeto visa explorar e compreender como a estimulação transcraniana por corrente contínua de alta definição (HD-tDCS), aplicada de forma catódica sobre o córtex pré-frontal dorsolateral (DLPFC) esquerdo, impacta os padrões de sincronização cerebral em atletas de elite de basquetebol feminino em repouso (\textit{resting state}). A pesquisa investiga os efeitos dessa neuromodulação sobre a conectividade neural, analisando a sincronicidade intrafrequencial entre pares de canais EEG-EEG e a sincronicidade \textit{cross-frequency} entre EEG e ECG. 

Neste estudo, adotamos um desenho experimental cruzado (cross-over), no qual os mesmos participantes foram submetidos tanto à estimulação catódica quanto à condição controle (sham) em sessões distintas, dentro de um protocolo duplo-cego. Esse delineamento permite investigar de forma mais precisa os efeitos da estimulação na conectividade neural, reduzindo a influência de variáveis individuais e proporcionando um controle interno mais rigoroso. Dessa forma, buscamos compreender como a neuromodulação pode modular os padrões de sincronização cerebral em atletas, sem pressupor efeitos benéficos ou deletérios a priori.

\section{Sincronicidade cérebro-corpo no esporte}  
Este estudo investiga como a estimulação transcraniana por corrente contínua de alta definição (HD-tDCS), aplicada de forma catódica ao DLPFC esquerdo, modula a conectividade neural em atletas de basquetebol feminino em repouso. Utilizando EEG, analisamos a sincronização entre diferentes regiões cerebrais (pares EEG-EEG) e também a relação entre a atividade cerebral e a oscilação do ciclo cardíaco (pares EEG-ECG). As análises EEG-EEG foram conduzidas dentro da mesma faixa de frequência (same-frequency), enquanto as análises EEG-ECG investigaram a relação entre diferentes bandas cerebrais e a oscilação cardíaca (cross-frequency).

Os seres humanos possuem uma capacidade intrínseca de sincronizar seus ritmos fisiológicos com estímulos ambientais e internos. Estudos indicam que o ritmo respiratório, por exemplo, pode se alinhar a padrões de atividade sensorial e cognitiva, refletindo uma integração dinâmica entre os sistemas autônomo e neural \cite{haas1985effects}. Além disso, estados psicofisiológicos, como ansiedade, depressão e estresse, bem como o controle consciente da respiração, influenciam a frequência cardíaca e a atividade cerebral \cite{criscuolo2022cognition}. Nesse contexto, pesquisas recentes, como a de Criscuolo et al. \cite{criscuolo2022cognition}, fundamentam o conceito do \textit{Body–Brain Dynamic System} (BBDS), demonstrando que a variabilidade dos ritmos cardíacos e respiratórios modula a atividade neural, criando ciclos de alta e baixa excitabilidade que podem afetar processos de integração e regulação neural.

Abordagens em neurocardiologia também evidenciam a importância do sistema nervoso autônomo na regulação dos padrões cardíacos, destacando como a influência de barorreceptores, quimiorreceptores e outros mecanismos sensoriais interage com os centros neurais superiores \cite{marcondes2024linguagem}. Pesquisas com potenciais evocados pelo batimento cardíaco (HEPs) reforçam essa visão, apontando para a relevância da integração interoceptiva na formação da consciência corporal \cite{park2018neural, banelli2020skipping, mackinnon2013utilizing}. Ademais, modelos recentes exploram a previsibilidade dos sinais EEG alinhados com os batimentos cardíacos, contribuindo para o entendimento dos mecanismos de acoplamento entre a atividade neural e o ritmo cardíaco \cite{vergara2024exploring}.

Observa-se, por exemplo, que o ciclo cardíaco pode modular a supressão das bandas alpha e beta no córtex, com evidências de que a fase diastólica favorece a preparação motora \cite{lai2024cardiac}. Essa modulação apoia a ideia de que o sistema corpo–cérebro transita de um estado padrão, predominantemente desacoplado, para um estado acoplado durante comportamentos ativos, otimizando a alocação de recursos cognitivos e motores \cite{criscuolo2022cognition}. Em paralelo, a investigação da sincronia neural interpessoal aponta que intervenções como neurofeedback podem melhorar a coerência entre indivíduos, sugerindo potenciais aplicações terapêuticas e, por extensão, implicações para a otimização do desempenho esportivo \cite{boecker2024interpersonal, konrad2024interpersonal}.

Por fim, técnicas de sincronização como o PLV e o PLI, amplamente discutidas na literatura \cite{seraj2018cerebral}, complementam essa abordagem integrativa ao fornecerem métodos quantitativos para avaliar a conectividade cerebral. Dado o papel fundamental da conectividade neural na atividade cerebral em repouso, o desenvolvimento de estratégias de neuromodulação pode oferecer uma base sólida para intervenções voltadas à modulação da sincronização entre regiões cerebrais. Essas abordagens não apenas abrem caminho para o aprimoramento de funções cognitivas e motoras em atletas, mas também podem ter implicações em reabilitação neurológica, transtornos psiquiátricos e até no aprimoramento do desempenho cognitivo em contextos acadêmicos e profissionais.

\section{Neuromodulação no esporte e modulação da função cerebral}

A estimulação transcraniana por corrente contínua (tDCS) tem se mostrado uma ferramenta promissora para modular a excitabilidade cortical e a função cerebral. Estudos iniciais demonstraram que a tDCS aumenta a excitabilidade do córtex motor e modula a percepção de esforço, promovendo alterações na atividade neural tanto em contextos clínicos quanto em experimentos controlados \cite{nitsche2000excitability, okano2013estimulacao}. Além disso, intervenções não invasivas – como tDCS, tACS (estimulação transcraniana por corrente alternada) e rTMS (estimulação magnética transcraniana repetitiva) – podem modificar a sincronização neural e a conectividade funcional, contribuindo para a modulação abrangente das redes cerebrais \cite{scheler2019neuromodulation, kunze2014high}. Estudos recentes, como o de Zhang et al. \cite{zhang2024closed}, indicam que essas intervenções também podem aumentar as oscilações gamma e promover a liberação de BDNF por astrócitos, melhorando as conexões neuronais e contribuindo para processos cognitivos.

\subsection{Transcranial Direct Current Stimulation (tDCS) e High-Definition tDCS: Fundamentos e Diferenças}\label{subsec:tdcs_hdtdcs}

A estimulação transcraniana por corrente contínua (tDCS) é uma técnica não invasiva que utiliza correntes elétricas fracas para modular a excitabilidade cortical e, consequentemente, a atividade neuronal \cite{stagg2011physiological}. Conforme demonstrado por Nitsche e Paulus \cite{nitsche2000excitability}, a aplicação de correntes DC fracas induz modificações polaridade-específicas: a estimulação anódica promove a despolarização dos neurônios, aumentando a excitabilidade cortical, enquanto a catódica resulta na hiperpolarização e na diminuição da excitabilidade. Esses efeitos são dependentes da intensidade e da duração do estímulo, podendo durar vários minutos após o término da estimulação, quando aplicados por períodos entre 10 e 30 minutos.

Para aumentar a focalidade da estimulação, foi desenvolvida a High-Definition tDCS (HD-tDCS), que utiliza arrays de eletrodos menores dispostos em configurações específicas. Um exemplo é a configuração \emph{4$\times$1}, na qual um eletrodo central, definindo a polaridade, é circundado por quatro eletrodos de retorno, gerando um campo elétrico mais concentrado sobre a região-alvo \cite{villamar2013hdtdcs}.

Além disso, a polaridade aplicada exerce efeitos distintos: a estimulação anódica tende a aumentar a excitabilidade cortical na região de foco, enquanto a catódica a diminui, conforme demonstrado em estudos clássicos \cite{purpura1965intracellular, stagg2011physiological}. Dessa forma, a HD-tDCS representa um avanço significativo na neuromodulação, possibilitando intervenções mais direcionadas e com potencial para aprimorar os efeitos clínicos e neurofisiológicos.

No presente estudo, utilizamos um protocolo de posicionamento baseado em modelagem computacional para garantir precisão e focalidade na estimulação, conforme descrito em detalhes na Seção~\ref{chap:metodologia}. O equipamento empregado foi um \textbf{estimulador digital MxN (Soterix Medical)}, com eletrodos de \textbf{Ag/AgCl} acoplados a uma touca de EEG, seguindo recomendações do fabricante quanto à intensidade de corrente e aos procedimentos de calibração. Essa abordagem assegurou uma distribuição focalizada da corrente no \emph{córtex pré-frontal dorsolateral} (DLPFC), na configuração catódica, de acordo com o delineamento do protocolo experimental descrito nos capítulos anteriores.

\subsection{Efeitos em contextos esportivos, cognitivos e clínicos}

Oscilações em bandas theta, alpha e gamma desempenham papéis essenciais na manutenção da memória de trabalho visuoespacial, sendo a sincronização fronto-parietal crucial para o processamento de informações complexas \cite{fell2011phase, fries2015rhythms, palva2010neuronal}. Nesse sentido, Zhang et al. \cite{zhang2022theta} demonstraram que a aplicação de HD-tACS, ajustada individualmente para a frequência de pico de theta, pode modificar a sincronização fronto-parietal e influenciar os mecanismos associados à memória de trabalho visuoespacial. Estudos com HD-tDCS aplicados no DLPFC (córtex pré-frontal dorsolateral) também sugerem que a estimulação pode alterar a conectividade parieto-frontal e a dinâmica dos circuitos neurais \cite{arif2021high}.

A eficácia da tDCS tem sido investigada em condições patológicas. Singh et al. (2024) \cite{singh2024evaluating} identificaram alterações significativas na topologia da rede de EEG em repouso de pacientes com transtorno depressivo maior, especialmente na faixa beta, sugerindo que a tDCS pode reduzir a aleatoriedade e favorecer uma organização de small-world, restaurando circuitos neurais disfuncionais. Em pacientes com epilepsia refratária, Toutant et al. (2024) \cite{toutant2024hdtdcs} demonstraram que a aplicação de HD‑tDCS altera o perfil espectral do EEG – com redução relativa das frequências baixas (delta e theta) e aumento das frequências altas (beta e gamma) – indicando uma reorganização neural benéfica para o controle de crises. Ademais, evidências de mudanças no estado cerebral foram relatadas por Čukić et al. (2018) \cite{cukic2018shift}, que observaram que a tDCS, dependendo da polaridade, provoca deslocamentos no estado energético do cérebro, refletidos na redução dos valores de Mean State Shift (MSS) e em alterações na State Variance (SV).

\subsection{Impacto nos aspectos emocionais, interpessoais e na reabilitação}

A neuromodulação também influencia aspectos emocionais e interpessoais. Valenzuela et al. \cite{valenzuela2019enhancement} demonstraram que a aplicação de tDCS em atletas pode melhorar o estado de humor, mesmo sem impactar diretamente a atividade motora. Revisões e meta-análises de técnicas de estimulação cerebral não invasiva (NIBS) sugerem ainda que essas intervenções podem melhorar o estado de consciência em pacientes com transtornos relacionados, evidenciando um potencial terapêutico abrangente.

Além disso, diferentes modalidades de estimulação têm sido associadas à modulação da sincronização de fase. Por exemplo, Rostami et al. \cite{rostami2020transcranial} evidenciaram que a tACS a 6 Hz sobre o córtex pré-frontal medial melhora a atenção sustentada ao modular a sincronia alpha e theta, enquanto Spooner et al. (2020) \cite{spooner2020hdtdcs} mostraram que a aplicação de HD-tDCS no DLPFC pode alterar a conectividade na banda theta durante tarefas de atenção visual seletiva. Em um contexto interpessoal, Long et al. \cite{long2023transcranial} relataram que a tDCS aplicada no lobo temporal anterior direito reduz a sincronização neural interpessoal e os níveis de empatia emocional.

No campo da reabilitação, a tDCS tem se mostrado eficaz na modulação das oscilações neurais. Liu et al. \cite{liu2023effects} constataram que a tDCS diminui as oscilações delta e aumenta as oscillações alpha em pacientes pós-AVC, enquanto Han et al. \cite{han2022functional} observaram aumentos na conectividade funcional em pacientes com distúrbios de consciência após HD‑tDCS. Estudos clínicos também indicam que a estimulação anódica sobre o córtex parietal posterior direito está associada ao aumento da conectividade de longa distância em ritmos gamma, sugerindo implicações para a atenção temporal, e que intervenções domiciliares com tDCS podem aumentar a sincronização neural em pacientes com depressão bipolar \cite{xiao2025enhanced}. Análises de grafos aplicadas à tDCS ainda revelam alterações na sincronização cortical que dependem da polaridade da estimulação \cite{mancini2016assessing, pellegrino2019transcranial, schollmann2019anodal}.

\subsection{Abordagens multidimensionais, modelos matemáticos e monitoramento integrado}

Diversas abordagens multidimensionais têm sido desenvolvidas para avaliar os efeitos da neuromodulação. Zhang et al. \cite{zhang2022multidimensional} realizaram uma avaliação abrangente das métricas de EEG em intervenções neuromodulatórias, enquanto Jones et al. \cite{jones2017frontoparietal} demonstraram que a combinação de tDCS com treinamento de memória de trabalho pode aumentar a sincronia em redes fronto-parietais. Estudos com tDCS bilateral indicam que há uma remodelação das redes cerebrais em repouso, sugerindo a indução de plasticidade cerebral em regiões além daquelas diretamente estimuladas \cite{pellegrino2018bilateral}.

Modelos matemáticos também têm contribuído para a compreensão dos mecanismos subjacentes à neuromodulação. Riedinger e Hutt \cite{riedinger2022model} propuseram um modelo do circuito córtico-talâmico que explica como a tDCS modula a excitabilidade cerebral e a potência do EEG. Adicionalmente, abordagens de controle em loop fechado para oscilações gamma, exploradas por Zhang et al. \cite{zhang2024closed}, demonstraram que intervenções transcranianas – tanto por tDCS quanto por rTMS – podem aumentar as oscilações gamma. 

Por fim, a viabilidade do monitoramento simultâneo de EEG durante a tDCS foi comprovada por Schestatsky et al. \cite{schesatsky2013simultaneous}, por meio de um dispositivo inovador que permite a estimulação e o registro do EEG em tempo real, facilitando a avaliação contínua da excitabilidade cortical e a otimização dos parâmetros terapêuticos.

\subsection{Abordagens alternativas e complementares}

Diversas técnicas alternativas de neuromodulação têm sido exploradas para ajustar a sincronização neural e ampliar o entendimento dos mecanismos moduladores. Estudos de Boecker et al. \cite{boecker2024interpersonal} investigaram a sincronia neural interpessoal em contextos clínicos, demonstrando que intervenções envolvendo estimulação cerebral e neurofeedback podem aprimorar a coordenação dos sinais durante interações sociais, especialmente em populações com transtorno do espectro autista (ASD), transtorno de ansiedade social (SAD) e condições correlatas. Revisões de McNaughton and Redcay \cite{mcnaughton2020interpersonal} e Baldwin et al. \cite{baldwin2014evidence} apontam que indivíduos com ASD tendem a apresentar comportamentos temporalmente assíncronos em tarefas de integração sensorial complexa. Estudos adicionais \cite{gerloff2022autism, quinones2021dysfunction, key2022greater, tanabe2012hard} reforçam a ideia de que a desincronia interpessoal pode estar associada a déficits em múltiplos níveis, sugerindo o potencial de intervenções neuromodulatórias alternativas para restabelecer a sincronização funcional.

No campo do neurofeedback, evidências apontam que essa abordagem pode reduzir sintomas de ansiedade generalizada e fobias específicas \cite{hou2021neurofeedback, zilverstand2015fmri}. Estudos pilotos com neurofeedback baseado em NIRS, como o de Kimmig et al. \cite{kimmig2019feasibility}, demonstraram a viabilidade de treinar o controle do DLPFC em indivíduos com ansiedade social, sendo corroborados por investigações que avaliaram a eficácia e aceitabilidade dessa técnica \cite{direito2021training, steiner2014pilot, lamarca2018facilitating, catala2017treatment}. Análises econômicas, como a de Arnold et al. \cite{arnold2013eeg}, sugerem que os custos associados ao neurofeedback podem ser compatíveis com outras abordagens terapêuticas.

Outras técnicas de neuromodulação também têm sido avaliadas em contextos clínicos. Dong et al. \cite{dong2023efficacy} evidenciaram que a estimulação não invasiva melhora o estado de consciência em pacientes com transtornos do nível de consciência, enquanto Zrenner et al. \cite{zrenner2020brain} demonstraram a eficácia da rTMS sincronizada com oscilações alpha no DLPFC para pacientes com depressão resistente. Além disso, Maiella et al. \cite{maiella2022simultaneous} exploraram a combinação de tACS e TMS para potencializar as oscilações gamma no DLPFC, e Konrad et al. \cite{konrad2024interpersonal} discutiram intervenções voltadas à sincronia neural interpessoal como uma via inovadora para tratamentos clínicos.

\section{Medidas Neurofisiológicas}

Para compreender a complexa interação entre o cérebro e o corpo, é necessário integrar diversas técnicas que permitam registrar simultaneamente tanto a atividade elétrica cerebral quanto os sinais fisiológicos. A eletroencefalografia (EEG) fornece dados sobre a atividade neural, possibilitando a extração de oscilações que podem ser decompostas nas bandas clássicas – delta, theta, alpha, beta e gamma –, conforme amplamente discutido na literatura \cite{cohen2017where, bullmore2009complex}. Essas bandas representam diferentes escalas temporais e estão associadas a funções cognitivas e comportamentais específicas.

Além do EEG, técnicas como a eletromiografia (EMG) e o eletrocardiograma (ECG) são empregadas para monitorar a atividade muscular e os ritmos cardíacos, respectivamente. Em alguns contextos, a posição estratégica dos eletrodos de EMG pode permitir a detecção dos picos dos batimentos cardíacos – refletindo a despolarização cardíaca – quando o ECG tradicional não está disponível ou quando se busca uma integração mais próxima com outros sinais fisiológicos. Essa abordagem possibilita, por exemplo, capturar a dinâmica do ciclo cardíaco a partir da atividade do músculo peitoral maior, como feito em nosso estudo.

Um aspecto fundamental desses registros é a capacidade de observar como oscilações em diferentes frequências se inter-relacionam. O fenômeno conhecido como acoplamento de frequências cruzadas (cross-frequency coupling, CFC) descreve a interação dinâmica entre oscilações de bandas distintas – por exemplo, a modulação da amplitude de uma frequência rápida (como a gama) pela fase de uma frequência mais lenta (como a theta) – e é considerado crucial para a integração entre processos neurais e funções fisiológicas \cite{criscuolo2022cognition, cohen2017where}. Essa propriedade permite investigar de forma mais abrangente como o cérebro integra informações provenientes de diferentes sistemas, fundamentando abordagens que visam otimizar tanto o desempenho esportivo quanto estratégias de neuromodulação e reabilitação.