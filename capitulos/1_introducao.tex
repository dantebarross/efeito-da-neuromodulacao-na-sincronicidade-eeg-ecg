\chapter{Introdução}
\label{chap:introducao}

A interação entre o cérebro e o corpo é fundamental para a compreensão das funções neurofisiológicas, e a neurociência tem avançado significativamente na elucidação dessa relação. No contexto esportivo, o conceito de sincronização cérebro-corpo, ou sistema dinâmico cérebro-corpo (BBDS, do inglês Body Brain Dynamic System), emerge como um elemento-chave para o entendimento da integração dinâmica entre atividades cerebrais e fisiológicas. 

Com o advento de tecnologias como a eletroencefalografia (EEG), tornou-se possível investigar detalhadamente as oscilações cerebrais durante diferentes atividades. Essas oscilações, embora sua origem e significado ainda sejam amplamente debatidos, têm mostrado potencial para se sincronizar com ritmos corporais – como os da frequência cardíaca e respiratória – formando um padrão rítmico integrado que pode modular a atividade neural \cite{criscuolo2022cognition, cohen2017where}. Esse avanço tecnológico abriu caminho para intervenções de neuromodulação que visam alterar padrões de atividade cerebral.

A neuromodulação, especialmente por meio da estimulação transcraniana por corrente contínua (tDCS), tem demonstrado potencial para alterar a excitabilidade cortical e, consequentemente, modular esse sistema dinâmico \cite{nitsche2000excitability, stagg2011physiological}. Apesar de seu amplo uso em contextos clínicos, ainda são escassos os estudos que exploram os efeitos da tDCS no contexto esportivo, especialmente entre atletas de basquetebol.

Este projeto visa explorar e compreender como a estimulação transcraniana por corrente contínua de alta definição (HD-tDCS) pode impactar os padrões de sincronização cerebral em atletas de elite de basquetebol feminino em repouso (resting state). A pesquisa busca evidenciar os efeitos da neuromodulação cerebral sobre a conectividade neural, fornecendo insights sobre possíveis aplicações dessa técnica na modulação da atividade cerebral em atletas.

\section{Sincronicidade cérebro-corpo no esporte}  
A sincronicidade entre oscilações cerebrais em diferentes regiões representa um fenômeno crucial para compreender os efeitos da neuromodulação. Por meio do EEG, este projeto investiga como a conectividade neural em atletas de basquetebol em repouso é modulada pela estimulação transcraniana por corrente contínua de alta definição (HD-tDCS), buscando compreender os mecanismos pelos quais essa técnica influencia a atividade cerebral.

Os seres humanos possuem uma capacidade intrínseca de sincronizar seus ritmos fisiológicos com estímulos ambientais e internos. Estudos indicam que o ritmo respiratório, por exemplo, pode se alinhar a padrões de atividade sensorial e cognitiva, refletindo uma integração dinâmica entre sistemas autônomo e neural \cite{haas1985effects}. Além disso, estados psicofisiológicos, como ansiedade, depressão e estresse, bem como o controle consciente da respiração, influenciam a frequência cardíaca e a atividade cerebral \cite{criscuolo2022cognitioncognition}. Nesse contexto, pesquisas recentes fundamentam o conceito do Body–Brain Dynamic System (BBDS), demonstrando que a variabilidade dos ritmos cardíacos e respiratórios modula a atividade neural, criando ciclos de alta e baixa excitabilidade que podem afetar processos de integração e regulação neural.

Abordagens em neurocardiologia também evidenciam a importância do sistema nervoso autônomo na regulação dos padrões cardíacos, destacando como a influência de barorreceptores, quimiorreceptores e outros mecanismos sensoriais interage com os centros neurais superiores \cite{marcondes2024linguagem}. Pesquisas com potenciais evocados pelo batimento cardíaco (HEPs) reforçam essa visão, apontando para a relevância da integração interoceptiva na formação da consciência corporal \cite{park2018neural, banelli2020skipping, mackinnon2013utilizing}. Ademais, modelos recentes exploram a previsibilidade dos sinais EEG alinhados com os batimentos cardíacos, contribuindo para o entendimento dos mecanismos de acoplamento entre a atividade neural e o ritmo cardíaco \cite{vergara2024exploring}.

Observa-se, por exemplo, que o ciclo cardíaco pode modular a supressão das bandas alfa e beta no córtex, com evidências de que a fase diastólica favorece a preparação motora \cite{lai2024cardiac}. Essa modulação apoia a ideia de que o sistema corpo–cérebro transita de um estado padrão, predominantemente desacoplado, para um estado acoplado durante comportamentos ativos, otimizando a alocação de recursos cognitivos e motores \cite{criscuolo2022cognition}. Em paralelo, a investigação da sincronia neural interpessoal aponta que intervenções como neurofeedback podem melhorar a coerência entre indivíduos, sugerindo potenciais aplicações terapêuticas e, por extensão, implicações para a otimização do desempenho esportivo \cite{boecker2024interpersonal, konrad2024interpersonal}.

Por fim, técnicas de sincronização como o PLV e o PLI, amplamente discutidas na literatura \cite{seraj2018}, complementam essa abordagem integrativa ao fornecerem métodos para quantificar a conectividade cerebral. Considerando a importância da conectividade neural na atividade cerebral em repouso, o desenvolvimento de estratégias de neuromodulação pode fornecer uma base sólida para intervenções que visem alterações na sincronização entre regiões cerebrais, abrindo caminho para estudos futuros sobre o aprimoramento de funções cognitivas e motoras em atletas.

\section{Neuromodulação no esporte e modulação da função cerebral}

A estimulação transcraniana por corrente contínua (tDCS) tem se mostrado uma ferramenta promissora para modular a excitabilidade cortical e a função cerebral. Estudos iniciais demonstraram que a tDCS aumenta a excitabilidade do córtex motor e modula a percepção de esforço, promovendo alterações na atividade neural tanto em contextos clínicos quanto em experimentos controlados \cite{nitsche2000excitability, okano2013estimulacao}. Além disso, intervenções não invasivas – incluindo tDCS, tACS (estimulação transcraniana por corrente alternada) e rTMS (estimulação magnética transcraniana repetitiva) – podem modificar a sincronização neural e a conectividade funcional, contribuindo para a modulação abrangente das redes cerebrais \cite{scheler2019neuromodulation, kunze2014high}.

\subsection{Efeitos em contextos esportivos, cognitivos e clínicos}

Estudos têm demonstrado que oscilações em bandas theta, alpha e gama desempenham papéis fundamentais na manutenção da memória de trabalho visuoespacial, com a sincronização fronto-parietal sendo crucial para o processamento dessas informações \cite{fell2011, fries2015, palva2010}. Nesse sentido, Zhang et al. (2022) \cite{zhang2022theta} mostraram que a aplicação de HD-tACS (Estimulação Alternada Transcraniana de Alta Definição), ajustada individualmente para a frequência de pico de teta, pode modificar a sincronização fronto-parietal e influenciar os mecanismos associados à memória de trabalho visuoespacial, reforçando a importância de abordagens personalizadas na modulação da conectividade neural. De forma complementar, estudos recentes com HD-tDCS no DLPFC (córtex pré-frontal dorsolateral) indicam que a estimulação pode alterar a conectividade parieto-frontal e influenciar a dinâmica dos circuitos neurais \cite{arif2021high}.

A eficácia da tDCS também vem sendo investigada em condições patológicas. Singh et al. (2024) \cite{singh2024evaluating} analisaram dados de EEG em repouso de pacientes com transtorno depressivo maior e identificaram modificações significativas na topologia da rede — especialmente na faixa beta —, sugerindo que a tDCS pode reduzir a aleatoriedade e favorecer propriedades de small-world, contribuindo para a restauração dos circuitos neurais disfuncionais. Em estudos com pacientes com epilepsia refratária, Toutant et al. (2024) \cite{toutant2024hdtdcs} demonstraram que a aplicação de HD‑tDCS promove a dessincronização do EEG, evidenciada por uma redução na sincronização de baixas frequências (delta e teta) e um aumento na atividade de altas frequências (beta e gama) nas regiões frontocentrais e parietais, sugerindo uma reorganização das redes neurais benéfica para o controle de crises.

Adicionalmente, a tDCS tem sido associada a mudanças significativas no estado cerebral. Čukić et al. (2018) \cite{cukic2018shift} observaram que a tDCS, dependendo da polaridade, provoca deslocamentos no estado cerebral — com a estimulação catódica produzindo valores significativamente menores de Mean State Shift (MSS) e alterando a State Variance (SV) — o que evidencia a modulação do estado energético do cérebro.

\subsection{Impacto nos aspectos emocionais, interpessoais e na reabilitação}

A influência da neuromodulação estende-se aos aspectos emocionais e cognitivos. Por exemplo, Valenzuela et al. \cite{valenzuela2019enhancement} demonstraram que a aplicação de tDCS em atletas melhora o estado de humor, mesmo sem impactar diretamente a atividade motora. Além disso, revisões e meta-análises de técnicas de estimulação cerebral não invasiva (NIBS) em pacientes com transtornos de consciência indicam que tais intervenções podem melhorar significativamente o estado de consciência, evidenciando o potencial dessas abordagens para modular a atividade neural de forma abrangente.

Modulações da sincronização de fase também foram observadas com outras modalidades de estimulação. Rostami et al. \cite{rostami2020transcranial} evidenciaram que a tACS a 6 Hz sobre o córtex pré-frontal medial (mPFC) melhora a atenção sustentada através da modulação da sincronia alfa e theta. Spooner et al. (2020) \cite{spooner2020hdtdcs} demonstraram que a aplicação de HD-tDCS no DLPFC pode modular a conectividade na banda teta, alterando a sincronização entre regiões frontais e visuais durante tarefas de atenção visual seletiva. Em um contexto interpessoal, Long et al. \cite{long2023transcranial} relataram que a tDCS aplicada no lobo temporal anterior direito reduz a sincronização neural interpessoal e os níveis de empatia emocional, sugerindo que a sincronização interpessoal também pode ser modulada por intervenções neuromodulatórias.

No campo da reabilitação, a tDCS tem se mostrado eficaz na modulação das oscilações neurais. Liu et al. \cite{liu2023effects} constataram que a tDCS reduz as oscilações delta e aumenta as alfa em pacientes pós-AVC, enquanto Han et al. \cite{han2022functional} observaram aumentos na conectividade funcional em pacientes com distúrbios de consciência após HD‑tDCS. Adicionalmente, a estimulação anódica sobre o córtex parietal posterior direito tem sido associada ao aumento da conectividade de longa distância em ritmos gama, sugerindo implicações para a atenção temporal \cite{tan2023selective}. Em contextos clínicos, além das intervenções em depressão maior \cite{singh2024evaluating}, estudos com tDCS domiciliares demonstraram aumento na sincronização neural em pacientes com depressão bipolar \cite{xiao2025enhanced}, e métodos de análise de grafos aplicados à tDCS revelaram alterações na sincronização cortical dependentes da polaridade da estimulação \cite{mancini2016assessing, pellegrino2019transcranial, schollmann2019anodal}.

\subsection{Abordagens multidimensionais, modelos matemáticos e monitoramento integrado}

Abordagens multidimensionais para a avaliação dos efeitos da neuromodulação têm sido propostas. Zhang et al. \cite{zhang2022multidimensional} realizaram uma avaliação abrangente das métricas de EEG em intervenções neuromodulatórias, enquanto Jones et al. \cite{jones2017frontoparietal} demonstraram que a combinação de tDCS com treinamento de memória de trabalho pode aumentar a sincronia em redes frontoparietais. Estudos com tDCS bilateral indicam uma remodelação das redes cerebrais em repouso, sugerindo a indução de plasticidade cerebral em regiões além das diretamente estimuladas \cite{pellegrino2018bilateral}.

Modelos matemáticos têm contribuído para a compreensão dos mecanismos subjacentes à neuromodulação. Riedinger e Hutt \cite{riedinger2022model} propuseram um modelo do circuito córtico-talâmico que explica como a tDCS modula a excitabilidade cerebral e a potência do EEG. Abordagens de controle em loop fechado para oscilações gama foram exploradas por Zhang et al. \cite{zhang2024closed}, os quais demonstraram que estimulações transcranianas — tanto por tDCS quanto por rTMS — podem aumentar as oscilações gama e promover a liberação de BDNF por astrócitos, melhorando as conexões neuronais e contribuindo para processos cognitivos. A viabilidade do monitoramento simultâneo de EEG durante a tDCS foi demonstrada por Schestatsky et al. \cite{schestatsky2013simultaneous}, através de um dispositivo inovador que realiza a estimulação e o registro do EEG em tempo real, permitindo a avaliação contínua da excitabilidade cortical e a otimização dos parâmetros terapêuticos.

\subsection{Abordagens alternativas e complementares}

Diversas abordagens alternativas de neuromodulação têm sido exploradas para ajustar a sincronização neural, ampliando a compreensão dos mecanismos moduladores. Estudos de Boecker et al. \cite{boecker2024interpersonal} investigaram a sincronia neural interpessoal em contextos clínicos, demonstrando que intervenções envolvendo estimulação cerebral e neurofeedback podem aprimorar a coordenação dos sinais cerebrais durante interações sociais, especialmente em populações com transtorno do espectro autista (ASD), transtorno de ansiedade social (SAD) e outras condições relacionadas. Revisões de McNaughton and Redcay \cite{McNaughton2020} e Baldwin et al. \cite{Baldwin2022} evidenciaram que indivíduos com ASD tendem a apresentar comportamentos temporalmente assíncronos durante tarefas que envolvem integração sensorial complexa. Estudos adicionais \cite{Gerloff2022a, QuinonesCamacho2021, Key2022, Tanabe2012} reforçaram que a desincronia interpessoal pode estar associada a déficits em múltiplos níveis, sugerindo o potencial de intervenções neuromodulatórias alternativas para restabelecer a sincronização funcional.

No campo do neurofeedback, diversas evidências apontam que essa abordagem pode reduzir sintomas de ansiedade generalizada e fobias específicas \cite{Hou2021, Zilverstand2015}. Estudos pilotos utilizando neurofeedback baseado em NIRS, como o de Kimmig et al. \cite{Kimmig2019}, demonstraram a viabilidade de treinar o controle do DLPFC em indivíduos com ansiedade social, sendo corroborados por investigações que avaliaram a aceitabilidade e eficácia dessa técnica \cite{Direito2021, Steiner2014, LaMarca2018, CatalaLopez2017}. Análises econômicas, como a de Arnold et al. \cite{Arnold2013}, sugerem que intervenções de neurofeedback podem ter custos compatíveis com outras abordagens terapêuticas.

Outras técnicas de neuromodulação também têm sido avaliadas em contextos clínicos. Dong et al. \cite{dong2023efficacy} evidenciaram que a estimulação não invasiva melhora o estado de consciência em pacientes com transtornos de consciência, enquanto Zrenner et al. \cite{zrenner2020brain} demonstraram a eficácia da rTMS sincronizada com oscilações alfa no DLPFC para pacientes com depressão resistente. Adicionalmente, Maiella et al. \cite{maiella2022simultaneous} exploraram a combinação de tACS e TMS para potencializar as oscilações gama no DLPFC, e Konrad et al. \cite{konrad2024interpersonal} discutiram intervenções voltadas à sincronia neural interpessoal como uma via inovadora para tratamentos clínicos.

Essas abordagens alternativas, embora não constituam o foco principal deste trabalho, ampliam a compreensão dos múltiplos métodos de neuromodulação e evidenciam que a modulação da sincronização neural é um fenômeno transversal, com aplicações que vão desde a melhoria da integração funcional até intervenções terapêuticas em condições neurológicas e psiquiátricas.

\section{Medidas Neurofisiológicas}

Para compreender a complexa interação entre o cérebro e o corpo, é essencial integrar diversas técnicas de medição que captem tanto a atividade cerebral quanto a fisiológica. A eletroencefalografia (EEG) fornece dados sobre a atividade elétrica do cérebro, enquanto a eletromiografia (EMG) e o eletrocardiograma (ECG) registram, respectivamente, a atividade muscular e os ritmos cardíacos. Essas técnicas permitem a observação de oscilações em diferentes frequências, que podem se sincronizar de maneira complexa – um fenômeno conhecido como acoplamento de frequências cruzadas (cross-frequency coupling) – e que é fundamental para compreender os mecanismos de integração entre os sistemas corporal e cerebral \cite{criscuolo2022cognition, cohen2017where}.

Moscaleski et al. \cite{moscaleski2022hdtdcs} ressaltam que "as ativações cerebrais diferenciadas em atletas de alta performance suportam mecanismos neuronais relevantes para o desempenho esportivo. A preparação para a ação motora envolve regiões corticais e subcorticais que podem ser moduladas de forma não invasiva por estimulação de corrente elétrica". O uso do EEG, por exemplo, possibilita identificar perfis específicos de atividade elétrica que se correlacionam com desempenhos de alto nível, oferecendo insights valiosos para a otimização do treinamento esportivo.

No presente estudo, investiga-se o efeito da estimulação transcraniana por corrente contínua de alta definição (HD-tDCS) na atividade elétrica cerebral em atletas profissionais de basquetebol em estado de repouso. Para tanto, serão exploradas as relações entre os biomarcadores espectrais do EEG, com foco especial na sincronicidade entre diferentes bandas de frequência e na conectividade funcional entre regiões corticais. O estudo busca compreender como a HD-tDCS modula padrões de atividade neural, contribuindo com insights para aplicações futuras em neuromodulação e treinamento cognitivo.