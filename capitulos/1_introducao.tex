\chapter{Introdução}
\label{chap:introducao}

A interação entre o cérebro e o corpo é fundamental para o desempenho atlético, e a neurociência tem avançado significativamente na compreensão dessa relação. No contexto esportivo, o conceito de sincronização cérebro-corpo, ou sistema dinâmico cérebro-corpo (BBDS, do inglês Body Brain Dynamic System), emerge como um elemento-chave na execução de tarefas motoras complexas. Esse sistema refere-se à integração dinâmica entre atividades cerebrais e fisiológicas que ocorrem durante a execução de movimentos, como no caso do arremesso livre no basquetebol.

Com o advento de tecnologias como a eletroencefalografia (EEG), tornou-se possível investigar detalhadamente as oscilações cerebrais que ocorrem durante diferentes atividades. Essas oscilações, embora sua origem e significado ainda sejam amplamente debatidos, têm mostrado potencial para se sincronizar com ritmos corporais, como os da frequência cardíaca e respiratória, formando um padrão rítmico integrado que pode influenciar diretamente o desempenho atlético \cite{criscuolo2022cognition, cohen2017where}.

A neuromodulação, especialmente a estimulação transcraniana por corrente contínua (tDCS), tem demonstrado potencial para alterar a excitabilidade cortical e, consequentemente, modular esse sistema dinâmico \cite{nitsche2000excitability, stagg2011physiological}. No entanto, ainda são escassos os estudos que exploram esses efeitos no contexto esportivo, particularmente em atletas de basquetebol.

Este projeto visa explorar e compreender como a tDCS, em sua modalidade de alta definição (HD-tDCS), pode impactar a sincronização cérebro-corpo em atletas de basquetebol, especificamente durante a execução de arremessos livres. A pesquisa busca evidenciar a existência e o papel do BBDS na modulação do desempenho esportivo, fornecendo insights sobre a potencial aplicação dessa técnica no aprimoramento de habilidades motoras e cognitivas no esporte.

\section{Sincronicidade cérebro-corpo no esporte}
A sincronicidade entre as oscilações cerebrais e as frequências corporais é um fenômeno que pode otimizar a execução de tarefas motoras complexas, como o arremesso livre no basquetebol. Utilizando o EEG para captar as oscilações cerebrais e correlacionando-as com sinais fisiológicos de frequência cardíaca e respiratória, este projeto investiga como essas interações ocorrem e de que forma são moduladas através da tDCS.

Os seres humanos ajustam naturalmente seu ritmo respiratório a ritmos musicais, atuando como um marcapasso biológico. Esse alinhamento pode ser ainda mais acentuado durante atividades rítmicas, como o simples ato de bater os dedos no ritmo da música \cite{haas1985effects}. A pesquisa de Haas e colegas demonstra que esse tipo de sincronização, ou entrainment, é mais provável de ocorrer quando o ritmo musical é claro e o indivíduo está envolvido ativamente, como no caso do tapping. Além disso, estados psicofísicos como ansiedade, depressão e estresse, bem como o controle consciente da respiração, podem influenciar tanto a frequência cardíaca quanto a atividade cerebral \cite{criscuolo2022cognition}.

Considerando a influência dos ritmos corporais na atividade cerebral, estratégias de modulação dessas frequências podem ser desenvolvidas e testadas para avaliar seu impacto no desempenho em tarefas cognitivas e motoras críticas, como o arremesso livre no basquetebol. Essa tarefa exige um alto grau de concentração, percepção espacial, controle de força e precisão de movimento, fatores que podem ser potencialmente aprimorados pela sincronicidade entre os ritmos cerebrais e corporais. Ao explorar a modulação dessas sincronicidades, espera-se identificar intervenções que possam otimizar o desempenho atlético, fornecendo uma base para aprimorar o treinamento e a preparação dos atletas em situações de alta pressão.

\section{Neuromodulação no esporte e modulação da função cerebral}
A estimulação transcraniana por corrente contínua (tDCS) tem se mostrado uma ferramenta promissora para modular a excitabilidade cortical e a função cerebral em áreas específicas, o que pode influenciar o desempenho em tarefas motoras. Estudos têm demonstrado que a tDCS pode aumentar a excitabilidade do córtex motor, modular a percepção de esforço e melhorar o desempenho em exercícios físicos, destacando seu potencial tanto em contextos clínicos quanto esportivos \cite{nitsche2000excitability, okano2013estimulacao}.

Polanía et al. \cite{polania2012modulating} demonstraram que a tDCS anódica aplicada sobre o córtex motor primário (M1) aumenta a conectividade funcional entre M1 e o tálamo ipsilateral, além de fortalecer a conexão entre o núcleo caudado e áreas associativas corticais, indicando uma modulação do circuito motor córtico-estriado-talâmico. Stagg e Nitsche \cite{stagg2011physiological} acrescentam que os efeitos da tDCS, particularmente após a estimulação, estão intimamente ligados à modulação sináptica, sugerindo que a tDCS pode influenciar a plasticidade cortical ao interagir com sinapses dependentes de receptores NMDA.

Estudos anteriores demonstraram que a tDCS pode melhorar o desempenho esportivo, aumentando a excitabilidade cortical e a conectividade funcional entre diferentes regiões cerebrais \cite{okano2015brain, moreira2021effect_male, moreira2021effect_female}. Além disso, a aplicação da tDCS, especificamente sobre o córtex pré-frontal dorsolateral esquerdo, mostrou efeitos positivos em medidas subjetivas de bem-estar, percepção de recuperação e controle autonômico em atletas de elite após competições oficiais \cite{moreira2021effect_male, moreira2021effect_female}. Esses achados sugerem que a tDCS pode ser uma estratégia promissora para otimizar a recuperação pós-exercício e melhorar a preparação psicológica e fisiológica dos atletas para competições subsequentes.

No entanto, ainda há muito a ser explorado sobre como diferentes configurações de eletrodos na estimulação transcraniana, como investigado por Datta et al. \cite{datta2008transcranial}, podem influenciar a modulação cortical e, por conseguinte, os efeitos neurológicos em contextos como o repouso e o esforço físico em atletas de basquetebol.

Este projeto explorará especificamente o efeito da tDCS catódica de alta definição (HD-tDCS) no córtex pré-frontal dorsolateral (CPFDL) e como essa modulação afeta a sincronicidade entre as atividades cerebrais e corporais durante o arremesso livre.

\section{Medidas neurofisiológicas}
Para entender a complexa interação entre o cérebro e o corpo durante atividades esportivas, é fundamental integrar diversas técnicas de medida. A eletroencefalografia (EEG) fornece dados sobre a atividade elétrica cerebral, enquanto a eletromiografia (EMG) e o eletrocardiograma (ECG) capturam informações sobre a atividade muscular e cardíaca, respectivamente. Essas técnicas permitem a observação de oscilações em diferentes frequências, que podem se sincronizar de maneira complexa, um fenômeno conhecido como acoplamento de frequências cruzadas (cross-frequency coupling), essencial para compreender como os ritmos corporais e cerebrais influenciam o desempenho \cite{criscuolo2022cognition}.

Moscaleski et al. \cite{moscaleski2022hdtdcs} destacam que "as ativações cerebrais diferenciadas em atletas de alta performance suportam mecanismos neuronais relevantes para o desempenho esportivo. A preparação para a ação motora envolve regiões corticais e subcorticais que podem ser moduladas de forma não invasiva por estimulação de corrente elétrica". O uso do EEG para examinar a atividade elétrica cerebral proporciona insights valiosos sobre o desempenho atlético e pode ajudar a identificar perfis específicos de EEG associados a desempenhos de alto nível.

Este estudo investiga o efeito da estimulação transcraniana por corrente contínua de alta definição (HD-tDCS) na atividade elétrica do cérebro em jogadoras profissionais de basquetebol durante o arremesso livre, explorando a relação entre os biomarcadores espectrais do EEG e a performance motora. Além disso, este estudo explora a interação entre os dados de EEG, EMG e ECG, com foco nas sincronicidades entre as bandas de frequência durante a preparação para o arremesso. Investigar esses fenômenos pode fornecer uma compreensão mais profunda sobre a integração cérebro-corpo, informando estratégias para melhorar o desempenho atlético e potencialmente beneficiando áreas como reabilitação e treinamento cognitivo.