\chapter{Análise de Distribuição e Normalidade}
\label{chap:analise_distribuicao_normalidade}
Nesta seção, investigamos a forma das distribuições das métricas de conectividade, tanto em seus valores ``puros'' quanto nas diferenças (\textit{median\_diff}) entre as condições (Pós - Pré). Inicialmente, apresentamos as distribuições originais das métricas específicas \textbf{(PLI para EEG-EEG e CF-PLM para EEG-ECG)} que serão mantidas para análise. Em seguida, explicamos que, para testar o efeito da estimulação, calculamos a diferença entre os valores pós e pré (por exemplo, pós-\textit{sham} menos pré-\textit{sham}), e por fim discutimos a escolha dos testes estatísticos com base nessas distribuições.

\section{Distribuição das Métricas de Conectividade}
Antes de subtrair os valores pré dos pós, as métricas de conectividade foram extraídas diretamente dos sinais, refletindo as medidas originais sem a influência do efeito de estimulação. As distribuições ``puras'' são ilustradas nas Figuras~\ref{fig:pli_eeg_eeg} (EEG-EEG, PLI) e~\ref{fig:cfplm_eeg_ecg} (EEG-ECG, CF-PLM):
\begin{itemize}
    \item \textbf{PLI (EEG-EEG):} Avalia a sincronização de fase intrafrequencial entre sinais de canais de EEG.
    \item \textbf{CF-PLM (EEG-ECG):} Mede o acoplamento \textit{cross-frequency} entre o EEG e o ciclo cardíaco obtido via ECG.
\end{itemize}

As faixas de frequência investigadas incluem: delta, theta, alpha, beta e gamma.

% -----------------------------------------------------------------------------  
% PLI (EEG-EEG)  
% -----------------------------------------------------------------------------  
\ultrawidefigure{figs/3_1_connectivity_metrics/Distribuição_de_PLI_(EEG-EEG)_por_Banda.png}
{Distribuição de PLI (EEG-EEG) por banda. Nota-se uma clara concentração de valores próximos a zero em todas as bandas de frequência, com a maioria situada entre 0 e 0.2. As bandas rápidas (beta e gamma) apresentam curvas mais estreitas e concentradas perto de zero. Já as bandas mais lentas (delta, theta e alpha), embora também centradas próximo a zero, exibem caudas superiores maiores, especialmente a banda delta (a mais lenta) onde se destacam alguns valores próximos a 1 (excepcionalmente altos nesta distribuição).}
{pli_eeg_eeg}


% -----------------------------------------------------------------------------  
% CF-PLM (EEG-ECG)  
% -----------------------------------------------------------------------------  
\ultrawidefigure{figs/3_1_connectivity_metrics/Distribuição_de_CF-PLM_(EEG-ECG)_por_Banda.png}
{Distribuição de CF-PLM (EEG-ECG) por banda. Observa-se que é praticamente inexistente valores muito próximos a zero. A maior parte encontra-se entre 0.05 e 0.20. A depender da banda de frequência analisada, é possível observar um deslocamento de seus dados em relação às demais. No geral, é possível observar que quanto mais rápidas as bandas de frequência (e.g. gamma), menores os valores de CF-PLM, e menos densos (curvas mais pontudas). Quanto mais lentas (e.g. delta), maiores os valores de CF-PLM, e mais densos (curvas menos pontudas).}
{cfplm_eeg_ecg}

No geral, observamos que:
\begin{itemize}
    \item \textbf{EEG-EEG (PLI)}: A distribuição exibe variabilidade entre as bandas. Enquanto a maioria dos valores se concentra próxima de zero (indicando baixa estabilidade de fase), algumas bandas, sobretudo alpha e gamma, apresentam caudas mais extensas, sugerindo pares de canais com maior \emph{phase-locking} (Figura~\ref{fig:pli_eeg_eeg}).
    \item \textbf{EEG-ECG (CF-PLM)}: A maior parte dos valores se acumula em torno de zero, evidenciando um acoplamento \textit{cross-frequency} tipicamente baixo entre o EEG e o ciclo cardíaco. Ainda assim, há um leve deslocamento em faixas mais lentas (especialmente delta e theta), o que indica a possibilidade de sincronia pontual em alguns pares (Figura~\ref{fig:cfplm_eeg_ecg}).
\end{itemize}

Essas observações fornecem uma visão inicial do comportamento das métricas ``puras'' de conectividade, servindo de base para a comparação entre as condições (Pós e Pré), que será apresentada a seguir.

\section{Distribuição das Diferenças (\textit{median\_diff})}
Para avaliar o efeito da estimulação \textit{cathodic} \textit{versus} \textit{sham}, calculamos a diferença entre os valores medidos após a intervenção (Pós) e os valores obtidos antes (Pré). Essa diferença representa a mudança ocorrida na métrica mediana do índice específico, sendo positiva para aumentos e negativa para reduções:

\[
\textit{median\_diff} = (\text{pós condição x}) - (\text{pré condição x}),
\]
o que visa isolar o efeito da intervenção, removendo variações comuns que estariam presentes independentemente da estimulação.

As distribuições dessas diferenças foram avaliadas por meio de histogramas com \textit{cross-frequency} (KDE) para as métricas utilizadas nesta análise (PLI para EEG-EEG e CF-PLM para EEG-ECG). Os resultados são apresentados nas Figuras~\ref{fig:pli_freq_eeg_eeg} e~\ref{fig:cf_plm_freq_eeg_ecg}.

\ultrawidefigure{figs/6_distribuicao_metricas_conectividade/Distribuição_da_Diferença_da_PLI_(Pós_-_Pré)_por_Faixa_de_Frequência_EEG_EEG.png}
{Distribuição da diferença da PLI (Pós - Pré) em EEG-EEG, por faixa de frequência.}
{pli_freq_eeg_eeg}

\ultrawidefigure{figs/6_distribuicao_metricas_conectividade/Distribuição_da_Diferença_da_CF-PLM_(Pós_-_Pré)_por_Faixa_de_Frequência_EEG_ECG.png}
{Distribuição da diferença da CF-PLM (Pós - Pré) em EEG-ECG, por faixa de frequência.}
{cf_plm_freq_eeg_ecg}

Na Figura~\ref{fig:pli_freq_eeg_eeg} (diferença da PLI em EEG-EEG), podemos observar:
\begin{itemize}
    \item \textbf{Alpha e Beta}: distribuições com picos relativamente bem definidos próximos de zero, porém com caudas que se estendem para valores positivos e negativos. Isso indica que, embora a maioria dos pares de canais não apresente mudanças extremas de \emph{phase-locking}, há casos em que a PLI sofre variações mais intensas (tanto de aumento quanto de redução).
    \item \textbf{Delta e Theta}: apresentam formatos mais ``achatados'' (\emph{platocúrticos}), com uma região larga de valores ao redor de zero em vez de um pico pontudo. Sugere-se, assim, que múltiplos pares sofrem pequenas variações dispersas, em vez de convergirem para um valor dominante.
    \item \textbf{Gamma}: exibe uma curva um pouco mais pontuda, ainda que com assimetria leve. Isso pode significar que, em alguns pares, a diferença entre pós e pré se concentra de forma mais coesa em torno de um desvio particular (positivo ou negativo), enquanto a maioria dos pares permanece próxima de zero.
\end{itemize}

Já na Figura~\ref{fig:cf_plm_freq_eeg_ecg} (diferença da CF-PLM em EEG-ECG), o comportamento difere:
\begin{itemize}
    \item \textbf{Gamma}: é a única banda que se aproxima de uma forma mais simétrica e quase gaussiana em torno de zero, indicando que os valores de diferença se distribuem de modo relativamente uniforme, com menor propensão a distorções ou caudas prolongadas.
    \item \textbf{Delta, Alpha e Theta}: exibem curvas mais ``achatadas'', sugerindo novamente distribuições \emph{platocúrticas}. O pico central menos pronunciado e a largura mais acentuada indicam variações difusas em torno de zero (seja aumento ou redução), sem um valor predominante.
    \item \textbf{Beta}: apresenta uma assimetria mais clara (cauda se estendendo em uma das direções), apontando que uma fração dos pares EEG-ECG tende a exibir diferenças (Pós $-$ Pré) mais extremas, enquanto a maioria se mantém próxima de zero.
\end{itemize}

Esses perfis de dispersão são fundamentais para a escolha dos testes estatísticos, que será abordada no próximo capítulo, além de fornecerem pistas sobre como as diferentes faixas de frequência podem responder à neuromodulação em termos de sincronização de fase.

\subsubsection{Exemplo Individual por Métrica e Banda}
Para ilustrar de forma mais específica o comportamento das distribuições em um caso individual, a Figura~\ref{fig:median_cf_plm_diff_ath4_alpha_eeg_ecg} exibe a diferença da métrica \texttt{median\_cf\_plm\_diff} (Pós - Pré) para o atleta 4, na banda alpha, considerando todos os pares EEG-ECG. Já a Figura~\ref{fig:median_pli_diff_ath4_alpha_eeg_eeg} apresenta a diferença da \textit{median\_pli\_diff} para o mesmo atleta e banda, mas agora em pares EEG-EEG.

\ultrawidefigure{figs/5_connectivity_metrics_individual_distribution/median_cf_plm_diff_athlete_4_alpha_EEG_ECG.png}
{Distribuição da \texttt{median\_cf\_plm\_diff} (Pós - Pré) para o atleta 4, banda alpha, em pares EEG-ECG.}
{median_cf_plm_diff_ath4_alpha_eeg_ecg}

\ultrawidefigure{figs/5_connectivity_metrics_individual_distribution/median_pli_diff_athlete_4_alpha_EEG_EEG.png}
{Distribuição da \textit{median\_pli\_diff} (Pós - Pré) para o atleta 4, banda alpha, em pares EEG-EEG.}
{median_pli_diff_ath4_alpha_eeg_eeg}

Nesses gráficos, cada curva KDE (azul para \textit{cathodic}, vermelha para \textit{sham}) reflete como os valores de diferença (Pós $-$ Pré) se distribuem entre todos os pares de canais daquele atleta e banda de frequência. Um deslocamento maior de uma curva em relação à outra indica uma tendência geral de aumento (deslocamento à direita) ou redução (deslocamento à esquerda) na métrica pós-estimulação, enquanto uma sobreposição acentuada sugere pouca variação entre as condições. 

No exemplo da Figura~\ref{fig:median_cf_plm_diff_ath4_alpha_eeg_ecg}, observa-se que a curva azul (\textit{cathodic}) está levemente à direita da curva vermelha (\textit{sham}), indicando um ligeiro aumento na \texttt{median\_cf\_plm\_diff} para a maioria dos pares. Já na Figura~\ref{fig:median_pli_diff_ath4_alpha_eeg_eeg}, as curvas mostram maior sobreposição, sugerindo que a diferença na \textit{median\_pli\_diff} entre \textit{cathodic} e \textit{sham} é menos pronunciada.

Embora não seja o foco principal deste estudo, a análise individual por atleta e banda oferece uma visão clara sobre a variabilidade intra e intersujeitos, além de evidenciar como a neuromodulação pode afetar a conectividade de modo heterogêneo. Posteriormente, analisamos cada par de canal em cada faixa de frequência, aprofundando a avaliação dos efeitos significativos da intervenção.

\section{Verificação de Normalidade e Escolha do Teste Estatístico}
Para determinar o tipo de teste estatístico mais adequado (paramétrico ou não-paramétrico), realizamos a verificação da normalidade das distribuições de interesse. Aplicamos os testes de Shapiro-Wilk e Kolmogorov-Smirnov para cada combinação de grupo de canais (EEG-EEG ou EEG-ECG) e faixa de frequência, tanto para as métricas de PLI quanto para as de CF-PLM. 
Além disso, considerando o grande número de observações em cada grupo, mesmo desvios leves podem resultar em p-valores extremamente baixos, justificando o uso de vários testes de normalidade para reforçar a robustez da análise.

Portanto, abaixo, apresentamos algumas considerações:
\begin{itemize}
    \item \textbf{Tamanho amostral:} Como dispomos de grande quantidade de observações após a agregação dos dados, mesmo desvios sutis em relação à normalidade podem levar à rejeição da hipótese nula nos testes de normalidade. 
    Adicionalmente, aplicamos diversos testes (Anderson-Darling, D'Agostino, Jarque-Bera, Lilliefors) para abranger diferentes perspectivas de assimetria e curtose.
    \item \textbf{Forma das distribuições:} Embora muitos histogramas apresentem uma simetria visual aparente, a análise considerando todos os atletas e condições agrupados revela que cada banda de frequência exibe um padrão próprio de distribuição. Isso sugere que, apesar da simetria, existem características distintas em cada banda, reforçando a importância de análises separadas por faixa de frequência para compreender melhor os efeitos da estimulação nas métricas de conectividade.
    \item \textbf{Interpretação:} Quando os valores de p na maioria das distribuições são inferiores a 0.05, adotamos testes não-paramétricos (como \textit{Wilcoxon signed-rank} ou \textit{Mann-Whitney}) para as análises de inferência.
\end{itemize}

Além desses testes de normalidade, avaliamos também as medidas de assimetria (\textit{skewness}) e curtose (\textit{kurtosis}) para confirmar a adequação de testes paramétricos ou reforçar o uso de métodos não-paramétricos. 
Observou-se que a remoção de \textit{outliers} via Intervalo Interquartil (IQR) ou \textit{Empirical Cumulative Distribution-based Outlier Detection} (ECOD) reduziu a amplitude de variação em alguns casos, mas não eliminou a não-normalidade das distribuições, mantendo inviável o uso de testes paramétricos.

Com base nesses resultados, prosseguimos para as etapas de análise estatística.

\section{Testes de Normalidade e Decisão sobre o Tipo de Teste Estatístico}
Dando seguimento aos procedimentos descritos, aplicamos uma série de testes de normalidade às métricas de conectividade (\textit{median\_pli\_diff} e \texttt{median\_cf\_plm\_diff}), considerando os grupos \texttt{EEG\_EEG} e \texttt{EEG\_ECG}. A Tabela~\ref{tab:normality_tests} apresenta os resultados completos dos testes de normalidade, incluindo Shapiro-Wilk, Kolmogorov-Smirnov, Anderson-Darling, D'Agostino's K-squared, Jarque-Bera e Lilliefors, tanto para os dados originais quanto após a remoção de \textit{outliers} pelos métodos IQR e ECOD.

\inputtable{tabelas/normality_by_shapiro.tex}
{Resultados dos testes de normalidade — Shapiro-Wilk (EEG-EEG: PLV/PLI; EEG-ECG: CF-PLM)}
{normality_by_shapiro}
{Elaborado pelo autor (2025).}

\inputtable{tabelas/normality_by_ks.tex}
{Resultados dos testes de normalidade — Kolmogorov-Smirnov (EEG-EEG: PLV/PLI; EEG-ECG: CF-PLM)}
{normality_by_ks}
{Elaborado pelo autor (2025).}

\inputtable{tabelas/normality_by_ad.tex}
{Resultados dos testes de normalidade — Anderson-Darling (EEG-EEG: PLV/PLI; EEG-ECG: CF-PLM)}
{normality_by_ad}
{Elaborado pelo autor (2025).}

\inputtable{tabelas/normality_by_dagostino.tex}
{Resultados dos testes de normalidade — D’Agostino’s K² (EEG-EEG: PLV/PLI; EEG-ECG: CF-PLM)}
{normality_by_dagostino}
{Elaborado pelo autor (2025).}

\inputtable{tabelas/normality_by_jarque_bera.tex}
{Resultados dos testes de normalidade — Jarque-Bera (EEG-EEG: PLV/PLI; EEG-ECG: CF-PLM)}
{normality_by_jarque_bera}
{Elaborado pelo autor (2025).}

\inputtable{tabelas/normality_by_lilliefors.tex}
{Resultados dos testes de normalidade — Lilliefors (EEG-EEG: PLV/PLI; EEG-ECG: CF-PLM)}
{normality_by_lilliefors}
{Elaborado pelo autor (2025).}



\begin{table}[htbp]
  \centering
  \caption{Estatísticas descritivas das diferenças pós-pré nas métricas de conectividade}
  \label{tab:descriptive_stats}
  \begin{adjustbox}{width=\textwidth}
    \small
    \begin{tabular}{llllllllllll}
\toprule
metric & outlier\_removal & n\_samples & mean & std & median & q1 & q3 & min & max & skew & kurtosis \\
\midrule
PLV & none & 118950 & -0.016000 & 0.168000 & 0.003000 & -0.075000 & 0.067000 & -0.922000 & 0.805000 & -0.900000 & 3.360000 \\
PLV & iqr & 107717 & 0.002000 & 0.105000 & 0.007000 & -0.057000 & 0.065000 & -0.288000 & 0.279000 & -0.190000 & 0.220000 \\
PLV & ecod & 113002 & -0.012000 & 0.128000 & 0.003000 & -0.069000 & 0.062000 & -0.462000 & 0.299000 & -0.770000 & 1.250000 \\
PLI & none & 118950 & 0.002000 & 0.057000 & 0.000000 & -0.016000 & 0.016000 & -0.497000 & 0.670000 & 1.560000 & 17.570000 \\
PLI & iqr & 106265 & -0.000000 & 0.023000 & 0.000000 & -0.014000 & 0.014000 & -0.064000 & 0.064000 & -0.020000 & 0.210000 \\
PLI & ecod & 113003 & 0.000000 & 0.031000 & 0.000000 & -0.015000 & 0.015000 & -0.100000 & 0.120000 & 0.220000 & 1.780000 \\
CF-PLM & none & 3965 & -0.002000 & 0.011000 & -0.002000 & -0.009000 & 0.003000 & -0.073000 & 0.055000 & 0.790000 & 3.650000 \\
CF-PLM & iqr & 3781 & -0.003000 & 0.009000 & -0.002000 & -0.009000 & 0.003000 & -0.026000 & 0.021000 & -0.140000 & -0.120000 \\
CF-PLM & ecod & 3767 & -0.002000 & 0.009000 & -0.002000 & -0.008000 & 0.003000 & -0.022000 & 0.025000 & 0.160000 & 0.140000 \\
\bottomrule
\end{tabular}

  \end{adjustbox}

  \vspace{0.5em}
  \footnotesize\itshape
  \textbf{Nota:} PLV e PLI em EEG-EEG ($n\approx119.000$) e CF-PLM em EEG-ECG ($n\approx3.800$). A média de todas as séries está muito próxima de zero; a dispersão (\textit{std}) é reduzida em 38-55\% pelo método IQR (PLV: de 0.168\textrightarrow0.105; PLI: de 0.057\textrightarrow0.023) e em 24-46\% pelo ECOD (PLV: 0.168\textrightarrow0.128; PLI: 0.057\textrightarrow0.031), indicando que o ECOD retira menos observações extremas, mas produz distribuições com curtose e assimetria mais próximas de uma gaussiana.
  \normalsize

  \vspace{0.5em}
  \source{Elaborado pelo autor (2025).}
\end{table}

Com base nos resultados apresentados nas Tabelas~\ref{tab:normality_by_shapiro},~\ref{tab:normality_by_ks},~\ref{tab:normality_by_ad},~\ref{tab:normality_by_dagostino},~\ref{tab:normality_by_jarque_bera} e~\ref{tab:normality_by_lilliefors}, observamos que:

\begin{itemize}
    \item Para o grupo \texttt{EEG\_EEG}, os valores p de todos os testes foram extremamente baixos ($p < 0.001$), tanto para PLI quanto para CF-PLM, rejeitando fortemente a hipótese de normalidade.
    
    \item A remoção de \textit{outliers} por IQR melhorou levemente os indicadores (especialmente para PLI, onde o estatístico Shapiro-Wilk aumentou de 0.745 para 0.993), mas ainda manteve a rejeição da normalidade na maioria dos testes.
    
    \item O método ECOD apresentou resultados intermediários entre os dados originais e a filtragem por IQR, sugerindo uma abordagem mais conservadora na remoção de valores extremos.
    
    \item Os valores de assimetria (\textit{skewness}) e curtose (\textit{kurtosis}) confirmam o desvio da normalidade, com várias distribuições apresentando caudas pesadas (curtose > 3) e assimetrias significativas.
\end{itemize}

As distribuições analisadas mostraram desvios notáveis de normalidade em todos os cenários. Isso sugere que a aplicação de testes paramétricos (por exemplo, ANOVA e t-test) seria inadequada, pois pressupõe dados aproximadamente gaussianos. 
Portanto, considerando tanto a alta sensibilidade dos testes de normalidade quanto a permanência de assimetrias e curtoses mesmo após remoção de \textit{outliers}, concluímos que a abordagem mais confiável seria a utilização de métodos não-paramétricos.

Assim, optamos por empregar testes não-paramétricos de inferência (como Wilcoxon e Mann-Whitney) para comparar as diferenças (Pós - Pré) em ambas as condições \textit{cathodic} e \textit{sham}. Essa abordagem evita conclusões equivocadas decorrentes de pressupostos de normalidade violados.
