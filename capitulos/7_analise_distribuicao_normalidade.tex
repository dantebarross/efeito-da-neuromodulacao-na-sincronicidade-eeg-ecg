\chapter{Análise de Distribuição e Normalidade}
\label{chap:analise_distribuicao_normalidade}
Nesta seção, investigamos a forma das distribuições das métricas de conectividade, tanto em seus valores ``puros'' quanto nas diferenças (\texttt{median\_diff}) entre as condições (Pós – Pré). Inicialmente, apresentamos as distribuições originais das métricas específicas \textbf{(PLI para EEG-EEG e CF-PLM para EEG-ECG)} que serão mantidas para análise. Em seguida, explicamos que, para testar o efeito da estimulação, calculamos a diferença entre os valores pós e pré (por exemplo, \emph{pós-sham} menos \emph{pré-sham}), e por fim discutimos a escolha dos testes estatísticos com base nessas distribuições.

\section{Distribuição das Métricas de Conectividade}
Antes de subtrair os valores pré dos pós, as métricas de conectividade foram extraídas diretamente dos sinais, refletindo as medidas originais sem a influência do efeito de estimulação. As distribuições ``puras'' são ilustradas nas Figuras~\ref{pli_eeg_eeg} (EEG-EEG, PLI) e~\ref{cfplm_eeg_ecg} (EEG-ECG, CF-PLM):
\begin{itemize}
    \item \textbf{PLI (EEG-EEG):} Avalia a sincronização de fase iso-frequencial entre sinais de canais de EEG.
    \item \textbf{CF-PLM (EEG-ECG):} Mede o acoplamento \textit{cross-frequency} entre o EEG e o ciclo cardíaco obtido via ECG.
\end{itemize}

As faixas de frequência investigadas incluem: delta, theta, alpha, beta e gamma.

% -----------------------------------------------------------------------------  
% CF-PLM (EEG-ECG)  
% -----------------------------------------------------------------------------  
\standardfigure{figs/3_1_connectivity_metrics/Distribuição_de_CF-PLM_(EEG-ECG)_por_Banda.png}
{Distribuição de CF-PLM (EEG--ECG) por banda. Observa-se que a maioria dos valores se concentra próximo de zero, indicando acoplamento \textit{cross-frequency} tipicamente baixo entre as oscilações cerebrais e o ritmo cardíaco. Ainda assim, as bandas \emph{delta} e \emph{theta} exibem picos levemente deslocados (em torno de 0,01--0,03), sugerindo que, em frequências mais lentas, o ECG e o EEG podem sincronizar-se de forma ligeiramente mais robusta. Em contrapartida, a banda \emph{gamma} atinge valores um pouco mais elevados, embora com densidade menor, o que indica a existência de alguns pares EEG--ECG com acoplamento mais pronunciado em frequências rápidas. No geral, esses resultados reforçam a predominância de valores muito baixos de CF-PLM entre ECG e EEG no estado de repouso, mas não descartam a ocorrência de sincronias pontuais em faixas específicas.}
{cfplm_eeg_ecg}

% -----------------------------------------------------------------------------  
% PLI (EEG-EEG)  
% -----------------------------------------------------------------------------  
\standardfigure{figs/3_1_connectivity_metrics/Distribuição_de_PLI_(EEG-EEG)_por_Banda.png}
{Distribuição de PLI (EEG--EEG) por banda. Nota-se que a maior parte dos valores se concentra em torno de zero, sugerindo que, para a maioria dos pares de canais, a defasagem de fase não se mantém estável. Entretanto, as bandas \emph{alpha} e \emph{gamma} exibem caudas mais extensas, chegando a valores acima de 0,2 ou 0,3, o que indica a existência de alguns pares com sincronização de fase mais consistente. Esse contraste entre bandas mais lentas (delta, theta, beta) e as mais rápidas (alpha, gamma) reforça a hipótese de que certos ritmos corticais podem sustentar um \emph{phase-locking} mais robusto em estado de repouso.}
{pli_eeg_eeg}

No geral, observamos que:
\begin{itemize}
    \item \textbf{EEG--EEG (PLI)}: A distribuição exibe variabilidade entre as bandas. Enquanto a maioria dos valores se concentra próxima de zero (indicando baixa estabilidade de fase), algumas bandas -- sobretudo \emph{alpha} e \emph{gamma} -- apresentam caudas mais extensas, sugerindo pares de canais com maior \emph{phase-locking} (Figura~\ref{pli_eeg_eeg}).
    \item \textbf{EEG--ECG (CF-PLM)}: A maior parte dos valores se acumula em torno de zero, evidenciando um acoplamento \textit{cross-frequency} tipicamente baixo entre o EEG e o ciclo cardíaco. Ainda assim, há um leve deslocamento em faixas mais lentas (especialmente \emph{delta} e \emph{theta}), o que indica a possibilidade de sincronia pontual em alguns pares (Figura~\ref{cfplm_eeg_ecg}).
\end{itemize}

Essas observações fornecem uma visão inicial do comportamento das métricas ``puras'' de conectividade, servindo de base para a comparação entre as condições (Pós e Pré), que será apresentada a seguir.

\section{Distribuição das Diferenças (\texttt{median\_diff})}
Para avaliar o efeito da estimulação \textit{cathodic} versus \textit{sham}, calculamos a diferença entre os valores medidos após a intervenção (Pós) e os valores obtidos antes (Pré). Essa diferença representa a mudança ocorrida na métrica mediana do índice específico, sendo positiva para aumentos e negativa para reduções:

\[
\texttt{median\_diff} = (\text{pós condição x}) - (\text{pré condição x}),
\]
o que visa isolar o efeito da intervenção, removendo variações comuns que estariam presentes independentemente da estimulação.

As distribuições dessas diferenças foram avaliadas por meio de histogramas com \textit{cross-frequency} (KDE) para as métricas utilizadas nesta análise (PLI para EEG--EEG e CF-PLM para EEG--ECG). Os resultados são apresentados nas Figuras~\ref{pli_freq_eeg_eeg} e~\ref{cf_plm_freq_eeg_ecg}.

\standardfigure{figs/6_distribuicao_metricas_conectividade/Distribuição_da_Diferença_da_PLI_(Pós_-_Pré)_por_Faixa_de_Frequência_EEG_EEG.png}
{Distribuição da diferença da PLI (Pós -- Pré) em EEG--EEG, por faixa de frequência.}
{pli_freq_eeg_eeg}

\standardfigure{figs/6_distribuicao_metricas_conectividade/Distribuição_da_Diferença_da_CF-PLM_(Pós_-_Pré)_por_Faixa_de_Frequência_EEG_ECG.png}
{Distribuição da diferença da CF-PLM (Pós -- Pré) em EEG--ECG, por faixa de frequência.}
{cf_plm_freq_eeg_ecg}

Na Figura \ref{pli_freq_eeg_eeg} (diferença da PLI em EEG--EEG), podemos observar:
\begin{itemize}
    \item \textbf{Alpha e Beta}: distribuições com picos relativamente bem-definidos próximos de zero, porém com caudas que se estendem para valores positivos e negativos. Isso indica que, embora a maioria dos pares de canais não apresente mudanças extremas de \emph{phase-locking}, há casos em que a PLI sofre variações mais intensas (tanto de aumento quanto de redução).
    \item \textbf{Delta e Theta}: apresentam formatos mais “achatados” (\emph{platocúrticos}), com uma região larga de valores ao redor de zero em vez de um pico pontudo. Sugere-se, assim, que múltiplos pares sofrem pequenas variações dispersas, em vez de convergirem para um valor dominante.
    \item \textbf{Gamma}: exibe uma curva um pouco mais pontuda, ainda que com assimetria leve. Isso pode significar que, em alguns pares, a diferença entre pós e pré se concentra de forma mais coesa em torno de um desvio particular (positivo ou negativo), enquanto a maioria dos pares permanece próxima de zero.
\end{itemize}

Já na Figura~\ref{cf_plm_freq_eeg_ecg} (diferença da CF-PLM em EEG--ECG), o comportamento difere:
\begin{itemize}
    \item \textbf{Gamma}: é a única banda que se aproxima de uma forma mais simétrica e quase gaussiana em torno de zero, indicando que os valores de diferença se distribuem de modo relativamente uniforme, com menor propensão a distorções ou caudas prolongadas.
    \item \textbf{Delta, Alpha e Theta}: exibem curvas mais “achatadas”, sugerindo novamente distribuições \emph{platocúrticas}. O pico central menos pronunciado e a largura mais acentuada indicam variações difusas em torno de zero (seja aumento ou redução), sem um valor predominante.
    \item \textbf{Beta}: apresenta uma assimetria mais clara (cauda se estendendo em uma das direções), apontando que uma fração dos pares EEG--ECG tende a exibir diferenças (Pós $-$ Pré) mais extremas, enquanto a maioria se mantém próxima de zero.
\end{itemize}

Esses perfis de dispersão são fundamentais para a escolha dos testes estatísticos, que será abordada no próximo capítulo, além de fornecerem pistas sobre como as diferentes faixas de frequência podem responder à neuromodulação em termos de sincronização de fase.

\subsubsection{Exemplo Individual por Métrica e Banda}
Para ilustrar de forma mais específica o comportamento das distribuições em um caso individual, a Figura~\ref{fig:median_cf_plm_diff_ath4_alpha_eeg_ecg} exibe a diferença da métrica \texttt{median\_cf\_plm\_diff} (Pós -- Pré) para o atleta 4, na banda \emph{alpha}, considerando todos os pares EEG--ECG. Já a Figura~\ref{fig:median_pli_diff_ath4_alpha_eeg_eeg} apresenta a diferença da \texttt{median\_pli\_diff} para o mesmo atleta e banda, mas agora em pares EEG--EEG.

\standardfigure{figs/5_connectivity_metrics_individual_distribution/median_cf_plm_diff_athlete_4_alpha_EEG_ECG.png}
{Distribuição da \texttt{median\_cf\_plm\_diff} (Pós -- Pré) para o atleta 4, banda \emph{alpha}, em pares EEG--ECG.}
{median_cf_plm_diff_ath4_alpha_eeg_ecg}

\standardfigure{figs/5_connectivity_metrics_individual_distribution/median_pli_diff_athlete_4_alpha_EEG_EEG.png}
{Distribuição da \texttt{median\_pli\_diff} (Pós -- Pré) para o atleta 4, banda \emph{alpha}, em pares EEG--EEG.}
{median_pli_diff_ath4_alpha_eeg_eeg}

Nesses gráficos, cada curva KDE (azul para \textit{cathodic}, vermelha para \textit{sham}) reflete como os valores de diferença (Pós $-$ Pré) se distribuem entre todos os pares de canais daquele atleta e banda de frequência. Um deslocamento maior de uma curva em relação à outra indica uma tendência geral de aumento (deslocamento à direita) ou redução (deslocamento à esquerda) na métrica pós-estimulação, enquanto uma sobreposição acentuada sugere pouca variação entre as condições. 

No exemplo da Figura~\ref{fig:median_cf_plm_diff_ath4_alpha_eeg_ecg}, observa-se que a curva azul (\textit{cathodic}) está levemente à direita da curva vermelha (\textit{sham}), indicando um ligeiro aumento na \texttt{median\_cf\_plm\_diff} para a maioria dos pares. Já na Figura~\ref{fig:median_pli_diff_ath4_alpha_eeg_eeg}, as curvas mostram maior sobreposição, sugerindo que a diferença na \texttt{median\_pli\_diff} entre \textit{cathodic} e \textit{sham} é menos pronunciada.

Embora não seja o foco principal deste estudo, a análise individual por atleta e banda oferece uma visão clara sobre a variabilidade intra e intersujeitos, além de evidenciar como a neuromodulação pode afetar a conectividade de modo heterogêneo. Posteriormente, analisamos cada par de canal em cada faixa de frequência, aprofundando a avaliação dos efeitos significativos da intervenção.

\section{Verificação de Normalidade e Escolha do Teste Estatístico}
Para determinar o tipo de teste estatístico mais adequado (paramétrico ou não-paramétrico), realizamos a verificação da normalidade das distribuições de interesse. Aplicamos os testes de Shapiro-Wilk e Kolmogorov-Smirnov para cada combinação de grupo de canais (EEG--EEG ou EEG--ECG) e faixa de frequência, tanto para as métricas de PLI quanto para as de CF-PLM. 
Além disso, considerando o grande número de observações em cada grupo, mesmo desvios leves podem resultar em p-valores extremamente baixos, justificando o uso de vários testes de normalidade para reforçar a robustez da análise.}

\paragraph{Considerações:}
\begin{itemize}
    \item \textbf{Tamanho amostral:} Como dispomos de grande quantidade de observações após a agregação dos dados, mesmo desvios sutis em relação à normalidade podem levar à rejeição da hipótese nula nos testes de normalidade. 
    Adicionalmente, aplicamos diversos testes (Anderson-Darling, D'Agostino, Jarque-Bera, Lilliefors) para abranger diferentes perspectivas de assimetria e curtose.}
    \item \textbf{Forma das distribuições:} Embora muitos histogramas apresentem uma simetria visual aparente, a análise considerando todos os atletas e condições agrupados revela que cada banda de frequência exibe um padrão próprio de distribuição. Isso sugere que, apesar da simetria, existem características distintas em cada banda, reforçando a importância de análises separadas por faixa de frequência para compreender melhor os efeitos da estimulação nas métricas de conectividade.
    \item \textbf{Interpretação:} Quando os valores de p na maioria das distribuições são inferiores a 0.05, adotamos testes não-paramétricos (como Wilcoxon signed-rank ou Mann-Whitney) para as análises de inferência.
\end{itemize}

Além desses testes de normalidade, avaliamos também as medidas de assimetria (\emph{skewness}) e curtose (\emph{kurtosis}) para confirmar a adequação de testes paramétricos ou reforçar o uso de métodos não-paramétricos. 
Observou-se que a remoção de outliers via Intervalo Interquartil (IQR) ou \textit{Empirical Cumulative Distribution-based Outlier Detection} (ECOD) reduziu a amplitude de variação em alguns casos, mas não eliminou a não-normalidade das distribuições, mantendo inviável o uso de testes paramétricos.

Com base nesses resultados, prosseguimos para as etapas de análise estatística.

\section{Testes de Normalidade e Decisão sobre o Tipo de Teste Estatístico}
Dando seguimento aos procedimentos descritos, aplicamos uma série de testes de normalidade às métricas de conectividade (\texttt{median\_pli\_diff} e \texttt{median\_cf\_plm\_diff}), considerando os grupos \texttt{EEG\_EEG} e \texttt{EEG\_ECG}. Além dos testes de Shapiro-Wilk e Kolmogorov-Smirnov, utilizamos também Anderson-Darling, D'Agostino's K-squared, Jarque-Bera e Lilliefors. Para avaliar o impacto de valores extremos, repetimos os testes antes e depois de remover outliers tanto pelo método do IQR quanto pelo ECOD.

\paragraph{Motivações e Procedimentos:}
\begin{itemize}
    \item \textbf{Objetivo:} Verificar se as distribuições das diferenças (Pós -- Pré) podem ser aproximadas por uma gaussiana e, em caso afirmativo, validar o uso de testes paramétricos. 
    A confirmação de normalidade permitiria, por exemplo, empregar ANOVA ou t-test para investigar diferenças entre condições.}
    \item \textbf{Condução dos testes:} 
    \begin{itemize}
        \item Os dados foram segmentados pelo tipo de par de canais (\texttt{EEG\_EEG} ou \texttt{EEG\_ECG}) e a métrica de conectividade (PLI ou CF-PLM).
        \item Avaliamos as faixas de frequência (delta, theta, alpha, beta, gamma), analisando se \texttt{(Pós -- Pré)} segue distribuição normal.
        \item Fizemos comparações em três cenários: (i) dados originais, (ii) dados sem outliers detectados por IQR e (iii) dados sem outliers removidos por ECOD.
    \end{itemize}
\end{itemize}

\paragraph{Principais Resultados:}
\begin{itemize}
    \item \textbf{Grupo EEG\_EEG:}
    \begin{itemize}
        \item Para \texttt{median\_pli\_diff}, todos os testes (Shapiro-Wilk, Kolmogorov-Smirnov, Anderson-Darling, D'Agostino, Jarque-Bera e Lilliefors) rejeitaram a hipótese de normalidade no conjunto original. Mesmo após remover aproximadamente 10\,\% de outliers via IQR ou 5\,\% via ECOD, as distribuições permaneceram distantes de uma forma gaussiana.
    \end{itemize}
    \item \textbf{Grupo EEG\_ECG:}
    \begin{itemize}
        \item Os resultados de \texttt{median\_cf\_plm\_diff} foram semelhantes: praticamente todas as combinações de testes indicaram rejeição da normalidade, tanto na base original quanto após filtrar outliers por IQR ou ECOD.
    \end{itemize}
\end{itemize}

\paragraph{Interpretação e Decisão Metodológica:}
As distribuições analisadas mostraram desvios notáveis de normalidade em todos os cenários. Isso sugere que a aplicação de testes paramétricos (por exemplo, ANOVA e t-test) seria inadequada, pois pressupõe dados aproximadamente gaussianos. 
Portanto, considerando tanto a alta sensibilidade dos testes de normalidade quanto a permanência de assimetrias e curtoses mesmo após remoção de outliers, concluímos que a abordagem mais confiável seria a utilização de métodos não-paramétricos.
Assim, optamos por empregar testes não-paramétricos de inferência (como Wilcoxon e Mann-Whitney) para comparar as diferenças (pós--pré) em ambas as condições \textit{cathodic} e \textit{sham}. Essa abordagem evita conclusões equivocadas decorrentes de pressupostos de normalidade violados.