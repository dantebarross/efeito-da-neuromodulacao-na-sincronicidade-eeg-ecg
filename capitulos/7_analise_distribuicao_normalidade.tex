\chapter{Análise de Distribuição e Normalidade}
\label{chap:analise_distribuicao_normalidade}

Nesta seção, investigamos a forma das distribuições das métricas de conectividade, tanto em seus valores "puros" quanto nas diferenças (\texttt{median\_diff}) entre as condições (Pós – Pré). Inicialmente, apresentamos as distribuições originais das métricas específicas \textbf{(PLI para EEG-EEG e CF-PLM para EEG-ECG)} que serão mantidas para análise. Em seguida, explicamos que, para testar o efeito da estimulação, calculamos a diferença entre os valores pós e pré (por exemplo, \emph{pós-sham} menos \emph{pré-sham}), e por fim discutimos a escolha dos testes estatísticos com base nessas distribuições.

\section{Distribuição das Métricas de Conectividade}

Antes de subtrair os valores pré dos pós, as métricas de conectividade foram extraídas diretamente dos sinais, refletindo as medidas originais sem a influência do efeito de estimulação. As figuras a seguir ilustram as distribuições "puras" para:

\begin{itemize}
    \item \textbf{PLI (EEG-EEG):} Avalia a sincronização de fase iso-frequencial entre canais cerebrais de EEG.
    \item \textbf{CF-PLM (EEG-ECG):} Mede o acoplamento \emph{cross-frequency} entre o EEG e o ciclo cardíaco obtido via ECG.
\end{itemize}

As faixas de frequência investigadas incluem: delta, theta, alpha, beta e gamma.

% -----------------------------------------------------------------------------  
% CF-PLM (EEG-ECG)  
% -----------------------------------------------------------------------------  
\standardfigure{figs/3_1_connectivity_metrics/Distribuição_de_CF-PLM_(EEG-ECG)_por_Banda.png}
{Distribuição de CF-PLM (EEG-ECG) por banda. Observa-se a concentração dos valores em faixas mais baixas (próximas de 0), com maior densidade para as bandas delta e theta.}
{cfplm_eeg_ecg}

% -----------------------------------------------------------------------------  
% PLI (EEG-EEG)  
% -----------------------------------------------------------------------------  
\standardfigure{figs/3_1_connectivity_metrics/Distribuição_de_PLI_(EEG-EEG)_por_Banda.png}
{Distribuição de PLI (EEG-EEG) por banda. Embora a maior parte dos valores se concentre em torno de zero, algumas bandas (alpha e gamma) apresentam caudas mais extensas, indicando pares de canais com defasagem de fase mais consistente.}
{pli_eeg_eeg}

No geral, observamos que:
\begin{itemize}
    \item \textbf{EEG-EEG (PLI):} A distribuição apresenta variabilidade entre bandas, com algumas exibindo valores mais elevados.
    \item \textbf{EEG-ECG (CF-PLM):} Os valores geralmente se concentram próximos a zero, embora haja variações que indicam acoplamento \emph{cross-frequency} pontual.
\end{itemize}

Essas observações fornecem uma visão inicial do comportamento das métricas "puras" de conectividade, servindo de base para a comparação entre as condições (Pós e Pré), que será apresentada a seguir.

\section{Distribuição das Diferenças (\texttt{median\_diff})}

Para avaliar o efeito da estimulação (cathodic versus sham), calculamos a diferença entre os valores medidos após a intervenção (Pós) e os valores obtidos antes (Pré). Essa diferença representa a mudança ocorrida na métrica mediana do índice específico, sendo positiva para aumentos e negativa para reduções:

\[
\texttt{median\_diff} = (\text{pós condição x}) - (\text{pré condição x})
\]

Esse procedimento visa isolar o efeito da intervenção, removendo variações comuns que estariam presentes independentemente da estimulação.

As distribuições das diferenças foram avaliadas por meio de histogramas com \emph{Kernel Density Estimation} (KDE) para as métricas utilizadas nesta análise (PLI para EEG-EEG e CF-PLM para EEG-ECG).

A seguir, ilustram-se exemplos desses histogramas:

% Exemplo: PLI (EEG-EEG) Diferença
\standardfigure{figs/6_distribuicao_metricas_conectividade/Distribuição_da_Diferença_da_PLI_(Pós_-_Pré)_por_Faixa_de_Frequência_EEG_EEG.png}
{Distribuição da diferença da PLI (Pós -- Pré) em EEG-EEG, por faixa de frequência.}
{pli_freq_eeg_eeg}

\standardfigure{figs/6_distribuicao_metricas_conectividade/Distribuição_da_Diferença_da_CF-PLM_(Pós_-_Pré)_por_Faixa_de_Frequência_EEG_ECG.png}
{Distribuição da diferença da CF-PLM (Pós -- Pré) em EEG-ECG, por faixa de frequência.}
{cf_plm_freq_eeg_ecg}

Esses histogramas permitem visualizar como cada métrica (PLI ou CF-PLM) varia entre as condições Pós e Pré em diferentes faixas de frequência e fundamentam a escolha dos testes estatísticos, que serão discutidos no próximo capítulo.

\subsubsection{Exemplo Individual por Métrica e Banda}
Para ilustrar de forma mais específica o comportamento das distribuições em um caso individual, a Figura~\ref{fig:median_cf_plm_diff_ath4_alpha_eeg_ecg} exibe a distribuição da diferença da métrica \texttt{median\_cf\_plm\_diff} (Pós -- Pré) para o atleta 4, na banda \emph{alpha}, em pares EEG-ECG. De forma semelhante, a Figura~\ref{fig:median_pli_diff_ath4_alpha_eeg_eeg} apresenta a distribuição da diferença da métrica \texttt{median\_pli\_diff} para o mesmo atleta e banda, porém em pares EEG-EEG.

\standardfigure{figs/5_connectivity_metrics_individual_distribution/median_cf_plm_diff_athlete_4_alpha_EEG_ECG.png}
{Distribuição da \texttt{median\_cf\_plm\_diff} (Pós -- Pré) para o atleta 4, banda alpha, em pares EEG-ECG.}
{median_cf_plm_diff_ath4_alpha_eeg_ecg}

\standardfigure{figs/5_connectivity_metrics_individual_distribution/median_pli_diff_athlete_4_alpha_EEG_EEG.png}
{Distribuição da \texttt{median\_pli\_diff} (Pós -- Pré) para o atleta 4, banda alpha, em pares EEG-EEG.}
{median_pli_diff_ath4_alpha_eeg_eeg}

Essas figuras exemplificam como as diferenças entre as condições \emph{cathodic} (azul) e \emph{sham} (vermelho) podem se sobrepor ou divergir. Em alguns casos, a curva KDE de uma condição desloca-se para a direita (indicando aumento da métrica após a estimulação) ou para a esquerda (indicando redução), enquanto em outros há significativa sobreposição, sugerindo pouca variação entre as condições. Esse tipo de análise individual é útil para verificar a variabilidade intra-sujeito e compreender se os efeitos observados são consistentes ou pontuais.

\section{Verificação de Normalidade e Escolha do Teste Estatístico}

Para determinar o tipo de teste estatístico mais adequado (paramétrico ou não-paramétrico), realizamos a verificação da normalidade das distribuições de interesse. Aplicamos os testes de Shapiro-Wilk e Kolmogorov-Smirnov para cada combinação de grupo de canais (EEG-EEG ou EEG-ECG) e faixa de frequência, tanto para as métricas de PLI quanto para as de CF-PLM.

\paragraph{Considerações:}
\begin{itemize}
    \item \textbf{Tamanho amostral:} Como dispomos de grande quantidade de observações após a agregação dos dados, mesmo desvios leves em relação à normalidade podem levar à rejeição da hipótese nula em testes de normalidade.
    \item \textbf{Forma das distribuições:} Muitos histogramas se mostram visualmente simétricos e unimodais, mas a presença de caudas mais longas ou assimetrias moderadas requer atenção ao uso de testes paramétricos.
    \item \textbf{Interpretação:} Quando os valores de p na maioria das distribuições são inferiores a 0.05, adotamos testes não-paramétricos (como Wilcoxon signed-rank ou Mann-Whitney) para as análises de inferência.
\end{itemize}

Além desses testes de normalidade, avaliamos também as medidas de assimetria (\emph{skewness}) e curtose (\emph{kurtosis}) para confirmar a adequação de testes paramétricos ou reforçar o uso de métodos não-paramétricos. Com base nesses resultados, prosseguimos para as etapas de análise estatística, respeitando as particularidades de cada métrica (PLI e CF-PLM).

\section{Testes de Normalidade e Decisão sobre o Tipo de Teste Estatístico}

Dando seguimento aos procedimentos descritos, aplicamos uma série de testes de normalidade às métricas de conectividade (\texttt{median\_pli\_diff} e \texttt{median\_cf\_plm\_diff}), considerando os grupos \texttt{EEG\_EEG} e \texttt{EEG\_ECG}. Além dos testes de Shapiro-Wilk e Kolmogorov-Smirnov, utilizamos também Anderson-Darling, D'Agostino's K-squared, Jarque-Bera e Lilliefors. Para avaliar o impacto de valores extremos, repetimos os testes antes e depois de remover outliers tanto pelo método do \emph{Interquartile Range} (IQR) quanto pelo \emph{Empirical Cumulative Distribution-based Outlier Detection} (ECOD).

\paragraph{Motivações e Procedimentos:}
\begin{itemize}
    \item \textbf{Objetivo:} Verificar se as distribuições das diferenças (Pós -- Pré) podem ser aproximadas por uma gaussiana e, em caso afirmativo, validar o uso de testes paramétricos.
    \item \textbf{Condução dos testes:} 
    \begin{itemize}
        \item Os dados foram segmentados pelo tipo de par de canais (\texttt{EEG\_EEG} ou \texttt{EEG\_ECG}) e a métrica de conectividade (PLI ou CF-PLM).
        \item Avaliamos as faixas de frequência (delta, theta, alpha, beta, gamma), analisando se \texttt{(Pós -- Pré)} segue distribuição normal.
        \item Fizemos comparações em três cenários: (i) dados originais, (ii) dados sem outliers detectados por IQR e (iii) dados sem outliers removidos por ECOD.
    \end{itemize}
\end{itemize}

\paragraph{Principais Resultados:}
\begin{itemize}
    \item \textbf{Grupo EEG\_EEG:}
    \begin{itemize}
        \item Para \texttt{median\_pli\_diff}, todos os testes (Shapiro-Wilk, Kolmogorov-Smirnov, Anderson-Darling, D'Agostino, Jarque-Bera e Lilliefors) rejeitaram a hipótese de normalidade no conjunto original. Mesmo após remover aproximadamente 10\,\% de outliers via IQR ou 5\,\% via ECOD, as distribuições permaneceram distantes de uma forma gaussiana.
    \end{itemize}
    \item \textbf{Grupo EEG\_ECG:}
    \begin{itemize}
        \item Os resultados de \texttt{median\_cf\_plm\_diff} foram semelhantes: praticamente todas as combinações de testes indicaram rejeição da normalidade, tanto na base original quanto após filtrar outliers por IQR ou ECOD.
    \end{itemize}
\end{itemize}

\paragraph{Interpretação e Decisão Metodológica:}
As distribuições analisadas mostraram desvios notáveis de normalidade em todos os cenários. Isso sugere que a aplicação de testes paramétricos (por exemplo, ANOVA e t-test) seria inadequada, pois pressupõe dados aproximadamente gaussianos. Assim, optamos por empregar métodos não-paramétricos de inferência (como Wilcoxon e Mann-Whitney) para comparar as diferenças (pós--pré) em ambas as condições (\emph{cathodic} e \emph{sham}). Essa abordagem evita conclusões equivocadas decorrentes de pressupostos de normalidade violados.