\chapter{Conclusões e Trabalhos Futuros}
\label{chap:conclusoes_e_trabalhos_futuros}

\section{Conclusões}
Este trabalho demonstrou que a neuromodulação catódica por HD-tDCS sobre o DLPFC esquerdo em atletas de elite de basquetebol, em condição de repouso, promoveu mudanças consistentes e robustas na conectividade funcional cerebral e na integração cérebro-coração. Todas as análises estatísticas foram realizadas sobre as diferenças entre as medições Pós e Pré para cada condição, comparando-se diretamente a condição \emph{cathodic} em relação à \emph{sham}. Em resumo:
\begin{itemize}
  \item A sincronização de fase intrafrequencial (EEG-EEG) foi significativamente modulada, sobretudo nas bandas alpha, delta e theta, com efeito marginal em beta.
  \item O acoplamento cross-frequency (EEG-ECG) apresentou aumento uniforme em alpha, beta e delta, indicando reforço da integração neural e autonômica.
  \item Testes não paramétricos (Mann-Whitney U, Wilcoxon signed-rank, Kruskal-Wallis) confirmaram as alterações mesmo após correção de múltiplas comparações e no cenário com e sem remoção de outliers.
  \item A fase \emph{bootstrap} par-a-par revelou dezenas de pares de canais com efeitos de grande magnitude, refletidos por correlações de postos frequentemente superiores a $\mathrm{RBC}\ge0{,}9$, fornecendo a base estatística para as análises de rede.
  \item A análise de grafos mostrou reorganização topológica: emergiram novos hubs, destacando-se o canal \textbf{Fp2} na banda alpha, sugerindo realocação funcional e resiliência da rede.
\end{itemize}
  
Esses achados corroboram modelos de plasticidade funcional induzida e de integração corpo-cérebro, e indicam que a HD-tDCS catódica pode ser usada para modular especificamente ritmos e circuitos relevantes ao desempenho esportivo de alto nível.

\section{Trabalhos Futuros}
Para aprofundar e ampliar estas descobertas, sugerimos:
\begin{itemize}
  \item \textbf{Análise do efeito imediato no desempenho:} correlacionar as alterações de conectividade em repouso com a performance nos arremessos livres coletada na mesma sessão, a fim de mapear relações entre estado neural e execução motora.
  \item \textbf{Estudo longitudinal:} acompanhar atletas ao longo da temporada para avaliar a estabilidade e a evolução dos efeitos da HD-tDCS na conectividade e no rendimento.
  \item \textbf{Paradigma \textit{event‑related} (\textit{trial‑based}):} desenhar experimentos baseados em eventos, por exemplo, apresentar estímulos ou comandos em trials controlados, para coletar múltiplas observações por participante em cada sessão, aumentando o poder estatístico e permitindo análises temporais finas das respostas neurais. Para um avanço do presente estudo, esta tarefa pode estar relacionada ao desempenho esportivo, como por exemplo \textit{trials} de lance livre de basquetebol.
  \item \textbf{Comparação entre protocolos:} confrontar HD-tDCS e tDCS convencional para identificar vantagens de focalidade, intensidade e duração dos efeitos.
  \item \textbf{Generalização a outras populações:} incluir atletas de diferentes modalidades e níveis de experiência, bem como voluntários não atletas, para testar a robustez e a especificidade dos resultados.
  \item \textbf{Integração de outras modalidades fisiológicas:} incorporar sinais de respiração (ciclo respiratório), variabilidade da frequência cardíaca e marcadores endócrinos para investigar interações multissistêmicas sob neuromodulação.
  \item \textbf{Avaliações comportamentais e cognitivas:} aplicar testes de atenção, memória e tomada de decisão antes e depois da estimulação para vincular mudanças de conectividade a impactos funcionais concretos.
  \item \textbf{Análises multimodais de neuroimagem:} combinar EEG de alta densidade com fMRI ou MEG para refinar a localização das alterações causadas pela HD-tDCS e entender seus mecanismos de ação em nível de fonte.
\end{itemize}