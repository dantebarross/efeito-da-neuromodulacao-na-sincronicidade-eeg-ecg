\chapter{Conclusões e Trabalhos Futuros}
\label{chap:conclusoes_e_trabalhos_futuros}
\section{Conclusões}
Este estudo demonstrou que a neuromodulação catódica, via HD-tDCS aplicada sobre o DLPFC, reconfigura significativamente a conectividade funcional em atletas de elite de basquetebol durante o repouso. Em síntese:
\begin{itemize}
    \item A intervenção altera a sincronização de fase tanto nas conexões intrafrequenciais (EEG--EEG) quanto no acoplamento \textit{cross-frequency} (EEG--ECG), com efeitos mais evidentes nas bandas \emph{alpha}, \emph{delta} e \emph{theta}.
    \item A robustez dos resultados foi confirmada por análises estatísticas não paramétricas, mesmo considerando cenários com e sem \textit{outliers}.
    \item A análise de rede revelou uma reorganização da topologia da conectividade, destacando a emergência de hubs como o canal \textbf{Fp2} na banda \emph{alpha}.
\end{itemize}
Essas conclusões reforçam que a neuromodulação pode influenciar a integração entre os sistemas neural e cardiovascular, contribuindo para a compreensão dos mecanismos subjacentes à modulação cortical em contextos de alta performance esportiva.

\section{Trabalhos Futuros}
Embora os resultados deste estudo ofereçam contribuições interessantes para a compreensão dos efeitos da neuromodulação na conectividade funcional, futuras investigações poderão expandir e aprofundar esse conhecimento:
\begin{itemize}
    \item \textbf{Exploração do Setup Trial:} Avaliar a correlação entre as alterações na sincronização neural, registradas em estado de repouso, e o desempenho nos arremessos livres. Uma análise integrada dos dados de desempenho (já coletados na mesma sessão) poderá revelar como o estado da conectividade funcional registrado um pouco antes de realizar os arremessos livres se relaciona com a eficácia dos mesmos, oferecendo insights sobre a influência da neuromodulação em funções motoras e cognitivas.
    
    \item \textbf{Estudo Longitudinal:} Investigar a persistência e a evolução dos efeitos da neuromodulação ao longo do tempo, acompanhando os sujeitos durante diferentes fases da temporada ou do treinamento, para verificar se as alterações na conectividade registrada se mantêm ou se adaptam.
    
    \item \textbf{Comparação de Protocolos de Estimulação:} Realizar estudos comparativos entre a HD-tDCS e a tDCS clássica.
    
    \item \textbf{Ampliação e Diversificação da Amostra:} Incluir outros grupos de atletas ou sujeitos com diferentes perfis esportivos para investigar a generalização dos resultados e aprofundar a compreensão dos efeitos da neuromodulação em populações mais diversas.

    \item \textbf{Integração do Sistema Respiratório:} Investigar a influência da neuromodulação sobre a conectividade funcional entre o ciclo respiratório e os sinais de EEG, bem como o efeito da neuromodulação no acoplamento entre ciclo respiratório e ciclo cardíaco.

    \item \textbf{Integração Multimodal:} Incorporar medidas adicionais, como avaliações cognitivas, métricas de desempenho técnico e outros biomarcadores, para enriquecer a análise e identificar correlações precisas entre os efeitos da neuromodulação e o desempenho esportivo.
\end{itemize}
