\chapter{Conclusões e Trabalhos Futuros}
\label{chap:conclusoes_e_trabalhos_futuros}

\section{Conclusões}
Este trabalho demonstrou que a neuromodulação catódica por HD-tDCS sobre o DLPFC esquerdo em atletas de elite de basquetebol, em condição de repouso, promoveu alterações específicas na conectividade intra-frequencial (EEG-EEG). Todas as análises estatísticas consideraram as diferenças Pós - Pré entre as condições \emph{cathodic} e \textit{sham}, com correção de Bonferroni (\alpha= 0,01).

\noindent\textbf{Análise Macro (Fase 1):} PLV apresentou aumento significativo de conectividade em \textbf{beta}, \textbf{delta}, \textbf{theta} e \textbf{gamma}; não houve mudança robusta em \textbf{alpha}. CF-PLM (EEG-ECG) mostrou redução significativa de acoplamento em \textbf{beta} e \textbf{delta} e tendência em \textbf{gamma}, enquanto \textbf{alpha} foi significativo apenas no Wilcoxon. PLI apresentou redução significativa em \textbf{alpha} e \textbf{theta} (efeitos negativos) e aumento em \textbf{delta} e \textbf{gamma} (efeitos positivos), com \textbf{beta} permanecendo neutra. A exclusão de outliers preservou a significância geral, mas reduziu os tamanhos de efeito, especialmente em \textbf{beta} (PLV) e \textbf{delta} (PLV, PLI).

\noindent\textbf{Análise Micro EEG-EEG (Fases 2 e 3):} No \textit{bootstrap}, foram identificados pares com RBC próximo a $+1$ e $-1$, evidenciando efeitos localizados da catódica: positivos em \textbf{alpha} e \textbf{theta}, negativos em \textbf{delta} e \textbf{gamma}, e um padrão misto em \textbf{beta}. Na \textit{centralidade}, o eletrodo \textbf{Fp2} liderou em \textbf{alpha}. Em \textbf{beta}, observou-se um cluster centro-parietal (\textit{CP5-F5}) com outliers, e \textit{TPP7h} após a limpeza. Para \textbf{delta}, destacaram-se \textit{CP5/CP4}; em \textbf{gamma}, \textit{F3} e \textit{FT10}; e em \textbf{theta}, uma combinação de \textit{PO4}, \textit{FC1} e \textit{TP9} dependendo da métrica e presença de outliers.

\noindent\textbf{Análise Micro EEG-ECG (Fase 2):} Os pares EEG-ECG apresentaram efeitos positivos predominantes em \textbf{beta}, \textbf{gamma} e \textbf{delta}, especialmente em regiões centro-parietais (\textit{CP1}, \textit{CP3}, \textit{CP5}, \textit{TPP7h}). Houve um único par com efeito negativo em \textbf{theta} (\textit{F6-ECG}) e ausência de efeitos significativos em \textbf{alpha}.

Esses resultados indicam que a HD-tDCS catódica:
\begin{itemize}
  \item Macro: modula de forma diferenciada os ritmos EEG-EEG, amplificando delta, theta e gamma (e beta no PLV) e reduzindo alpha.
  \item Micro: provoca mudanças topográficas de sincronia e hierarquia de canais, com \textit{hubs} funcionais parcialmente estáveis mesmo após remoção de outliers, como evidenciado pela alta correlação de rankings e preservação de estruturas principais.
\end{itemize}

\section{Trabalhos Futuros}
Para aprofundar e ampliar estas descobertas, sugerimos:
\begin{itemize}
  \item \textbf{Análise do efeito imediato no desempenho:} correlacionar as alterações de conectividade em repouso com a performance nos arremessos livres coletada na mesma sessão, a fim de mapear relações entre estado neural e execução motora.
  \item \textbf{Estudo longitudinal:} acompanhar atletas ao longo da temporada para avaliar a estabilidade e a evolução dos efeitos da HD-tDCS na conectividade e no rendimento.
  \item \textbf{Paradigma \textit{event-related} (\textit{trial-based}):} desenhar experimentos baseados em eventos, por exemplo, apresentar estímulos ou comandos em trials controlados, para coletar múltiplas observações por participante em cada sessão, aumentando o poder estatístico e permitindo análises temporais finas das respostas neurais. Para um avanço do presente estudo, esta tarefa pode estar relacionada ao desempenho esportivo, como por exemplo \textit{trials} de lance livre de basquetebol.
  \item \textbf{Validação funcional dos \textit{hubs} de modulação:} investigar se os canais mais centrais identificados nas redes moduladas (como Fp2 em alpha) estão relacionados a marcadores de desempenho atlético e mais a fundo a responsividade à neuromodulação.
  \item \textbf{Comparação entre protocolos:} confrontar HD-tDCS e tDCS convencional para identificar vantagens de focalidade, intensidade e duração dos efeitos.
  \item \textbf{Generalização a outras populações:} incluir atletas de diferentes modalidades e níveis de experiência, bem como voluntários não atletas, para testar a robustez e a especificidade dos resultados.
  \item \textbf{Integração de outras modalidades fisiológicas:} incorporar sinais de respiração (ciclo respiratório), variabilidade da frequência cardíaca e marcadores endócrinos para investigar interações multissistêmicas sob neuromodulação.
  \item \textbf{Avaliações comportamentais e cognitivas:} aplicar testes de atenção, memória e tomada de decisão antes e depois da estimulação para vincular mudanças de conectividade a impactos funcionais concretos.
  \item \textbf{Análises multimodais de neuroimagem:} combinar EEG de alta densidade com fMRI ou MEG para refinar a localização das alterações causadas pela HD-tDCS e entender seus mecanismos de ação em nível de fonte.
\end{itemize}