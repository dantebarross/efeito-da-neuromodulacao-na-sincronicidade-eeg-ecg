\chapter{Conclusões e Trabalhos Futuros}
\label{chap:conclusoes_e_trabalhos_futuros}

\section{Conclusões}
Este trabalho demonstrou que a neuromodulação catódica por HD-tDCS sobre o DLPFC esquerdo em atletas de elite de basquetebol, em condição de repouso, promoveu alterações específicas na conectividade intra-frequencial (EEG-EEG). Todas as análises estatísticas consideraram as diferenças Pós - Pré entre as condições \emph{cathodic} e \textit{sham}, com correção de Bonferroni ($\alpha\$ = 0,01).

\begin{itemize}
  \item[\textbf{Análise Macro (Fase 1)}]  
    \begin{itemize}
      \item \textbf{PLV:} aumento significativo de conectividade em \textbf{beta}, \textbf{delta}, \textbf{theta} e \textbf{gamma} (p < 0,01 em Wilcoxon e Mann–Whitney); sem mudança em \textbf{alpha}.
      \item \textbf{PLI:} redução significativa em \textbf{alpha} e \textbf{theta} (efeitos negativos, p < 0,01) e aumento significativo em \textbf{delta} e \textbf{gamma} (efeitos positivos, p < 0,01); \textbf{beta} permaneceu sem efeito robusto.
      \item Robustez: exclusão do outlier manteve as mesmas bandas significativas, embora os tamanhos de efeito em \textbf{beta} (PLV) e \textbf{delta} (PLV, PLI) tenham se reduzido.
    \end{itemize}

  \item[\textbf{Análise Micro (Fase 2 e 3)}]  
    \begin{itemize}
      \item \textbf{Bootstrap (Fase 2):} alguns pares de canais com RBC próximos a $+1$ e a $-1$, confirmando localizações específicas de efeito positivo da catódica em relação à \textit{sham} na sincronicidade de fase (alpha e theta) e negativo (delta e gamma).
      \item \textbf{Centralidade (Fase 3):} Fp2 emergiu como hub dominante em alpha, sinalizando reorganização topológica consistente em ambos os cenários; demais hubs variaram por banda (p. ex., TPP7h em beta, CP5/CP4 em delta).
    \end{itemize}
\end{itemize}

Esses resultados indicam que a HD-tDCS catódica:
\begin{itemize}
  \item Macro: modula de forma diferenciada os ritmos EEG-EEG—amplificando delta, theta e gamma (e beta no PLV) e reduzindo alpha.
  \item Micro: provoca mudanças topográficas de sincronia e hierarquia de canais, com hubs funcionais estáveis mesmo após remoção de outliers.
\end{itemize}

\section{Trabalhos Futuros}
Para aprofundar e ampliar estas descobertas, sugerimos:
\begin{itemize}
  \item \textbf{Análise do efeito imediato no desempenho:} correlacionar as alterações de conectividade em repouso com a performance nos arremessos livres coletada na mesma sessão, a fim de mapear relações entre estado neural e execução motora.
  \item \textbf{Estudo longitudinal:} acompanhar atletas ao longo da temporada para avaliar a estabilidade e a evolução dos efeitos da HD-tDCS na conectividade e no rendimento.
  \item \textbf{Paradigma \textit{event-related} (\textit{trial-based}):} desenhar experimentos baseados em eventos, por exemplo, apresentar estímulos ou comandos em trials controlados, para coletar múltiplas observações por participante em cada sessão, aumentando o poder estatístico e permitindo análises temporais finas das respostas neurais. Para um avanço do presente estudo, esta tarefa pode estar relacionada ao desempenho esportivo, como por exemplo \textit{trials} de lance livre de basquetebol.
  \item \textbf{Validação funcional dos \textit{hubs} de modulação:} investigar se os canais mais centrais identificados nas redes moduladas (como Fp2 em alpha) estão relacionados a marcadores de desempenho atlético e mais a fundo a responsividade à neuromodulação.
  \item \textbf{Comparação entre protocolos:} confrontar HD-tDCS e tDCS convencional para identificar vantagens de focalidade, intensidade e duração dos efeitos.
  \item \textbf{Generalização a outras populações:} incluir atletas de diferentes modalidades e níveis de experiência, bem como voluntários não atletas, para testar a robustez e a especificidade dos resultados.
  \item \textbf{Integração de outras modalidades fisiológicas:} incorporar sinais de respiração (ciclo respiratório), variabilidade da frequência cardíaca e marcadores endócrinos para investigar interações multissistêmicas sob neuromodulação.
  \item \textbf{Avaliações comportamentais e cognitivas:} aplicar testes de atenção, memória e tomada de decisão antes e depois da estimulação para vincular mudanças de conectividade a impactos funcionais concretos.
  \item \textbf{Análises multimodais de neuroimagem:} combinar EEG de alta densidade com fMRI ou MEG para refinar a localização das alterações causadas pela HD-tDCS e entender seus mecanismos de ação em nível de fonte.
\end{itemize}