\chapter{Análise de Rede}
\label{chap:analise_de_rede}
Esta seção apresenta a análise de conectividade em nível de rede, integrando as figuras geradas para os índices de sincronização PLI (para conexões EEG-EEG) e CF-PLM (para conexões EEG-ECG). As legendas foram elaboradas de modo a discorrer sobre os principais pontos de cada figura. 

As conexões representadas por linhas refletem esses pares, sendo exibidos apenas aqueles que são significativos após a correção de Bonferroni. Os valores de RBC indicam a direção e tamanho do efeito (RBC +1 para um efeito positivo na sincronia sob estimulação catódica e RBC -1 para um efeito negativo na sincronia, com \textit{sham} em referência).

Os nós pintados em verde indicam canais que possuem ao menos um par significativo.

\section{Rede de Conectividade via PLI (EEG-EEG)}
Nesta parte, analisamos a rede de pares de canais onde os efeitos da estimulação catódica foram signficativamente diferentes se comparados à \textit{sham} na sincronicidade de fase entre canais de EEG, obtida através do índice PLI. Todas as figuras são dispostas uma após a outra, primeiro a versão \textbf{com \textit{outliers}}, e logo em seguida a versão \textbf{sem \textit{outliers}} (remoção via ECOD). Nas legendas, descrevemos o principal a ser observado para cada banda de frequência:

%\clearpage
\ultrawidefigure
{figs/7_bootstrap_results_analysis/2_network_graphs/PLI_EEG-EEG_Com_Outliers/Banda_Alpha_(8_Hz_a_13_Hz)_-_Análise_de_Rede_-_PLI_EEG-EEG_Com_Outliers.png}
{Na banda alpha, observa-se um predomínio de conexões em vermelho (RBC +1) que se originam na região frontal e se estendem em diagonal até a área occipital direita, culminando aproximadamente no canal PO8. Essas conexões indicam que a estimulação catódica eleva a sincronia de fase na maior parte do eixo frontal-parietal-occipital, em contraste com um grupo menor de conexões azuis (RBC -1) no lado oposto.}
{rede_alpha_pli_com}

\ultrawidefigure
{figs/7_bootstrap_results_analysis/2_network_graphs/PLI_EEG-EEG_Sem_Outliers/Banda_Alpha_(8_Hz_a_13_Hz)_-_Análise_de_Rede_-_PLI_EEG-EEG_Sem_Outliers.png}
{Versão sem \textit{outliers}. Comparada à análise com \textit{outliers}, nota-se que alguns canais apresentaram um aumento discreto no número de ocorrências significativas, enquanto outros registraram uma leve diminuição. Além disso, há a aparição de pares significativos em canais (como O2 e F3) que não estavam presentes na versão com \textit{outliers} e a desaparecimento de conexões anteriormente notadas (por exemplo, no canal FC5).}
{rede_alpha_pli_sem}

\ultrawidefigure
{figs/7_bootstrap_results_analysis/2_network_graphs/PLI_EEG-EEG_Com_Outliers/Banda_Beta_(13_Hz_a_30_Hz)_-_Análise_de_Rede_-_PLI_EEG-EEG_Com_Outliers.png}
{Na banda beta, a rede apresenta um número menor de conexões em comparação com a banda alpha. Conexões vermelhas (RBC +1) e azuis (RBC -1) coexistem em áreas dispersas, com uma leve concentração na região delimitada por F5, CP5, CPz e Fz.}
{rede_beta_pli_com}

\ultrawidefigure
{figs/7_bootstrap_results_analysis/2_network_graphs/PLI_EEG-EEG_Sem_Outliers/Banda_Beta_(13_Hz_a_30_Hz)_-_Análise_de_Rede_-_PLI_EEG-EEG_Sem_Outliers.png}
{Versão sem \textit{outliers}. Observa-se um aumento significativo no número de pares com o canal FFT7h (todos RBC -1) e, no geral, uma elevação nas ocorrências de conexões com RBC -1. Alguns canais que não possuíam conexões significativas na versão com \textit{outliers} passam a exibir tais ocorrências (ex.: PO3, P1, P4 e F6).}
{rede_beta_pli_sem}

\ultrawidefigure
{figs/7_bootstrap_results_analysis/2_network_graphs/PLI_EEG-EEG_Com_Outliers/Banda_Delta_(0.5_a_4_Hz)_-_Análise_de_Rede_-_PLI_EEG-EEG_Com_Outliers.png}
{Nesta banda, todas as conexões significativas apresentam RBC -1. Destaca-se a alta incidência de conexões envolvendo o canal Fp2 (20 ocorrências), além de conexões intensas em outros canais frontais, e um menor número de conexões na região occipital. Os canais TTP8h (11 ocorrências) e AF4 (10 ocorrências) também se sobressaem.}
{rede_delta_pli_com}

\ultrawidefigure
{figs/7_bootstrap_results_analysis/2_network_graphs/PLI_EEG-EEG_Sem_Outliers/Banda_Delta_(0.5_a_4_Hz)_-_Análise_de_Rede_-_PLI_EEG-EEG_Sem_Outliers.png}
{Versão sem \textit{outliers}. O cenário mostra uma redução drástica no número de conexões significativas para os mesmos canais (Fp2, TTP8h e AF4), embora permaneça o predomínio de conexões com RBC -1, com a adição isolada de um caso com RBC +1.}
{rede_delta_pli_sem}

\ultrawidefigure
{figs/7_bootstrap_results_analysis/2_network_graphs/PLI_EEG-EEG_Com_Outliers/Banda_Gamma_(30_Hz_a_60_Hz)_-_Análise_de_Rede_-_PLI_EEG-EEG_Com_Outliers.png}
{Na banda gamma, observa-se a predominância de ocorrências com RBC -1, com conexões de longa distância entre áreas opostas. Destacam-se os canais FT10 e TT8h (com 10 e 7 ocorrências, respectivamente), e uma clara divisão entre regiões com RBC -1 e regiões com RBC +1, sendo este último associado à parte parietal direita (canal TPP8h) e à occipital direita (canal O1).}
{rede_gamma_pli_com}

\ultrawidefigure
{figs/7_bootstrap_results_analysis/2_network_graphs/PLI_EEG-EEG_Sem_Outliers/Banda_Gamma_(30_Hz_a_60_Hz)_-_Análise_de_Rede_-_PLI_EEG-EEG_Sem_Outliers.png}
{Versão sem \textit{outliers}. A configuração é similar à versão com \textit{outliers}, com um ligeiro aumento nas ocorrências de conexões RBC -1 e um acréscimo modesto em RBC +1. Alguns canais que não apresentavam conexões significativas na versão com \textit{outliers} passam a exibi-las (por exemplo, FC2 e TTP8h).}
{rede_gamma_pli_sem}

\ultrawidefigure
{figs/7_bootstrap_results_analysis/2_network_graphs/PLI_EEG-EEG_Com_Outliers/Banda_Theta_(4_Hz_a_8_Hz)_-_Análise_de_Rede_-_PLI_EEG-EEG_Com_Outliers.png}
{A rede em theta apresenta um predomínio de conexões positivas (RBC +1), concentradas sobretudo na parte esquerda, embora espalhadas por todas as regiões. Em contraste, as conexões com RBC -1 se concentram entre canais da região frontal esquerda e seus pares na área fronto-central e fronto-temporal direita.}
{rede_theta_pli_com}

\ultrawidefigure
{figs/7_bootstrap_results_analysis/2_network_graphs/PLI_EEG-EEG_Sem_Outliers/Banda_Theta_(4_Hz_a_8_Hz)_-_Análise_de_Rede_-_PLI_EEG-EEG_Sem_Outliers.png}
{Versão sem \textit{outliers}. A configuração geral permanece inalterada, com um ligeiro aumento nas ocorrências significativas e a aparição de conexões em canais adicionais (como P5, AFz e C4) que não estavam presentes na análise com \textit{outliers}.}
{rede_theta_pli_sem}


%%%%%%%%%%%%%%%% CF-PLM %%%%%%%%%%%%%%%%%%%%%%%%%%%%%%%%%%%%%%%%%%%%%%%%%%%%
%\clearpage
\section{Rede de Conectividade via CF-PLM (EEG-ECG)}
A seguir, a análise de rede baseada no índice CF-PLM, que avalia o efeito da estimulação HD-tDCS catódica no acoplamento \textit{cross-frequency} entre sinais de EEG e ECG. Nesta abordagem, os pares significativos foram identificados para diferentes bandas de frequência. Como o método ECOD não detectou \textit{outliers} no conjunto de dados (0\% removido), apresentamos apenas a versão consolidada com outliers.

\ultrawidefigure
{figs/7_bootstrap_results_analysis/2_network_graphs/CF-PLM_EEG-ECG_Com_Outliers/Banda_Alpha_(8_Hz_a_13_Hz)_-_Análise_de_Rede_-_CF-PLM_EEG-ECG_Com_Outliers.png}
{Nesta banda, não foram identificadas ocorrências significativas.}
{rede_alpha_cfplm_com}

\ultrawidefigure
{figs/7_bootstrap_results_analysis/2_network_graphs/CF-PLM_EEG-ECG_Com_Outliers/Banda_Beta_(13_Hz_a_30_Hz)_-_Análise_de_Rede_-_CF-PLM_EEG-ECG_Com_Outliers.png}
{Nesta banda, os pares significativos são formados pelos canais TPP7h, CP5, CP3 e CP1, todos localizados na região parietal esquerda, apresentando RBC +1.}
{rede_beta_cfplm_com}

\ultrawidefigure
{figs/7_bootstrap_results_analysis/2_network_graphs/CF-PLM_EEG-ECG_Com_Outliers/Banda_Delta_(0.5_a_4_Hz)_-_Análise_de_Rede_-_CF-PLM_EEG-ECG_Com_Outliers.png}
{Aqui, cinco ocorrências significativas foram identificadas, envolvendo pares entre o ECG e os canais CP3, P1, Fp2, FC4 e FTT8h (predominantemente na região parieto-central esquerda e também na região frontal, fronto-central e fronto-temporal direita), todas com RBC +1.}
{rede_delta_cfplm_com}

\ultrawidefigure
{figs/7_bootstrap_results_analysis/2_network_graphs/CF-PLM_EEG-ECG_Com_Outliers/Banda_Gamma_(30_Hz_a_60_Hz)_-_Análise_de_Rede_-_CF-PLM_EEG-ECG_Com_Outliers.png}
{Apenas um par significativo foi identificado, conectando o canal de EEG CP3 ao ECG na região centro-parietal esquerda, com RBC +1.}
{rede_gamma_cfplm_com}

\ultrawidefigure
{figs/7_bootstrap_results_analysis/2_network_graphs/CF-PLM_EEG-ECG_Com_Outliers/Banda_Theta_(4_Hz_a_8_Hz)_-_Análise_de_Rede_-_CF-PLM_EEG-ECG_Com_Outliers.png}
{O único par significativo nesta banda é o formado entre o EEG (canal F6) e o ECG, localizado na região frontal direita.}
{rede_theta_cfplm_com}