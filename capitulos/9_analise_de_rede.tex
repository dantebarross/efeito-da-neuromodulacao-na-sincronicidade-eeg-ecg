\chapter{Análise de Rede}

Esta seção apresenta a análise de conectividade em nível de rede, integrando as figuras geradas para os índices de sincronização PLI (para conexões EEG–EEG) e CF‐PLM (para conexões EEG–ECG). Todas as figuras são dispostas lado a lado, com o cenário \textbf{à esquerda} representando a análise \emph{com outliers} e o cenário \textbf{à direita} a análise \emph{sem outliers}. As legendas foram elaboradas de modo a discorrer sobre os principais pontos de cada figura. 

As conexões representadas por linhas refletem esses pares, sendo exibidos apenas aqueles que são significativos após a correção de Bonferroni. Os valores de Rank-Biserial Correlation (RBC) indicam a direção e tamanho do efeito (RBC +1 para um aumento da sincronia de pré para pós sob estimulação catódica e RBC –1 para uma diminuição na sincronia de pré para pós catódica, com sham em referência).

\section{Rede de Conectividade via PLI (EEG–EEG)}

Nesta parte, analisamos a rede de sincronização de fase entre canais de EEG, obtida com o índice PLI.  A seguir, descrevemos os resultados para cada banda de frequência:

\subsection{Banda Alpha (8–13 Hz)}
\begin{figure}[H]
    \centering
    \begin{subfigure}[b]{0.48\textwidth}
        \includegraphics[width=\textwidth]{figs/7_bootstrap_results_analysis/2_network_graphs/PLI_EEG-EEG_Com_Outliers/Banda_Alpha_(8_Hz_a_13_Hz)_-_Análise_de_Rede_-_PLI_EEG-EEG_Com_Outliers.png}
        \caption{\small Versão com outliers. Na banda alpha, observa-se um predomínio de conexões em vermelho (RBC +1) que se originam na região frontal e se estendem em diagonal até a área occipital direita, culminando aproximadamente no canal PO8. Essas conexões indicam que a estimulação catódica eleva a sincronia de fase na maior parte do eixo frontal–parietal–occipital, em contraste com um grupo menor de conexões azuis (RBC –1) no lado oposto.}
    \end{subfigure} 
    \hfill
    \begin{subfigure}[b]{0.48\textwidth}
        \includegraphics[width=\textwidth]{figs/7_bootstrap_results_analysis/2_network_graphs/PLI_EEG-EEG_Sem_Outliers/Banda_Alpha_(8_Hz_a_13_Hz)_-_Análise_de_Rede_-_PLI_EEG-EEG_Sem_Outliers.png}
        \caption{\small Versão sem outliers. Comparada à análise com outliers, nota-se que alguns canais apresentaram um aumento discreto no número de ocorrências significativas, enquanto outros registraram uma leve diminuição. Além disso, há a aparição de pares significativos em canais (como O2 e F3) que não estavam presentes na versão com outliers e a desaparecimento de conexões anteriormente notadas (por exemplo, no canal FC5).}
    \end{subfigure}
    \caption[Análise de Rede – Banda Alpha (PLI EEG–EEG)]{\small \textbf{Banda Alpha (8–13 Hz):} A análise de rede via PLI revela uma clara organização de conexões, destacando uma forte sincronização na região frontal que se projeta para áreas parietal e occipital direita, com diferenças sutis na configuração quando os outliers são removidos.}
    \label{fig:rede_alpha_pli}
\end{figure}

\subsection{Banda Beta (13–30 Hz)}
\begin{figure}[H]
    \centering
    \begin{subfigure}[b]{0.48\textwidth}
        \includegraphics[width=\textwidth]{figs/7_bootstrap_results_analysis/2_network_graphs/PLI_EEG-EEG_Com_Outliers/Banda_Beta_(13_Hz_a_30_Hz)_-_Análise_de_Rede_-_PLI_EEG-EEG_Com_Outliers.png}
        \caption{\small Versão com outliers. Na banda beta, a rede apresenta um número menor de conexões em comparação com a banda alpha. Conexões vermelhas (RBC +1) e azuis (RBC –1) coexistem em áreas dispersas, com uma leve concentração na região delimitada por F5, CP5, CPz e Fz.}
    \end{subfigure}
    \hfill
    \begin{subfigure}[b]{0.48\textwidth}
        \includegraphics[width=\textwidth]{figs/7_bootstrap_results_analysis/2_network_graphs/PLI_EEG-EEG_Sem_Outliers/Banda_Beta_(13_Hz_a_30_Hz)_-_Análise_de_Rede_-_PLI_EEG-EEG_Sem_Outliers.png}
        \caption{\small Versão sem outliers. Observa-se um aumento significativo no número de pares com o canal FFT7h (todos RBC –1) e, no geral, uma elevação nas ocorrências de conexões com RBC –1. Alguns canais que não possuíam conexões significativas na versão com outliers passam a exibir tais ocorrências (ex.: PO3, P1, P4 e F6).}
    \end{subfigure}
    \caption[Análise de Rede – Banda Beta (PLI EEG–EEG)]{\small \textbf{Banda Beta (13–30 Hz):} A análise evidencia uma rede menos densa do que na banda alpha, com uma reorganização das conexões e um aumento nas ocorrências de pares com RBC –1 na versão sem outliers, especialmente envolvendo o canal FFT7h e outros canais adicionais.}
    \label{fig:rede_beta_pli}
\end{figure}

\subsection{Banda Delta (0.5–4 Hz)}
\begin{figure}[H]
    \centering
    \begin{subfigure}[b]{0.48\textwidth}
        \includegraphics[width=\textwidth]{figs/7_bootstrap_results_analysis/2_network_graphs/PLI_EEG-EEG_Com_Outliers/Banda_Delta_(0.5_a_4_Hz)_-_Análise_de_Rede_-_PLI_EEG-EEG_Com_Outliers.png}
        \caption{\small Versão com outliers. Nesta banda, todas as conexões significativas apresentam RBC –1. Destaca-se a alta incidência de conexões envolvendo o canal Fp2 (20 ocorrências), além de conexões intensas em outros canais frontais, e um menor número de conexões na região occipital. Os canais TTP8h (11 ocorrências) e AF4 (10 ocorrências) também se sobressaem.}
    \end{subfigure}
    \hfill
    \begin{subfigure}[b]{0.48\textwidth}
        \includegraphics[width=\textwidth]{figs/7_bootstrap_results_analysis/2_network_graphs/PLI_EEG-EEG_Sem_Outliers/Banda_Delta_(0.5_a_4_Hz)_-_Análise_de_Rede_-_PLI_EEG-EEG_Sem_Outliers.png}
        \caption{\small Versão sem outliers. O cenário mostra uma redução drástica no número de conexões significativas para os mesmos canais (Fp2, TTP8h e AF4), embora permaneça o predomínio de conexões com RBC –1, com a adição isolada de um caso com RBC +1.}
    \end{subfigure}
    \caption[Análise de Rede – Banda Delta (PLI EEG–EEG)]{\small \textbf{Banda Delta (0.5–4 Hz):} A rede nesta banda é caracterizada quase exclusivamente por conexões negativas (RBC –1), com forte predominância na região frontal e uma diminuição geral dos pares significativos quando os outliers são removidos.}
    \label{fig:rede_delta_pli}
\end{figure}

\subsection{Banda Gamma (30–60 Hz)}
\begin{figure}[H]
    \centering
    \begin{subfigure}[b]{0.48\textwidth}
        \includegraphics[width=\textwidth]{figs/7_bootstrap_results_analysis/2_network_graphs/PLI_EEG-EEG_Com_Outliers/Banda_Gamma_(30_Hz_a_60_Hz)_-_Análise_de_Rede_-_PLI_EEG-EEG_Com_Outliers.png}
        \caption{\small Versão com outliers. Na banda gamma, observa-se a predominância de ocorrências com RBC –1, com conexões de longa distância entre áreas opostas. Destacam-se os canais FT10 e TT8h (com 10 e 7 ocorrências, respectivamente), e uma clara divisão entre regiões com RBC –1 e regiões com RBC +1, sendo este último associado à parte parietal direita (canal TPP8h) e à occipital direita (canal O1).}
    \end{subfigure}
    \hfill
    \begin{subfigure}[b]{0.48\textwidth}
        \includegraphics[width=\textwidth]{figs/7_bootstrap_results_analysis/2_network_graphs/PLI_EEG-EEG_Sem_Outliers/Banda_Gamma_(30_Hz_a_60_Hz)_-_Análise_de_Rede_-_PLI_EEG-EEG_Sem_Outliers.png}
        \caption{\small Versão sem outliers. A configuração é similar à versão com outliers, com um ligeiro aumento nas ocorrências de conexões RBC –1 e um acréscimo modesto em RBC +1. Alguns canais que não apresentavam conexões significativas na versão com outliers passam a exibi-las (por exemplo, FC2 e TTP8h).}
    \end{subfigure}
    \caption[Análise de Rede – Banda Gamma (PLI EEG–EEG)]{\small \textbf{Banda Gamma (30–60 Hz):} A rede na banda gamma é dominada por conexões negativas de longa distância, com distinção clara entre regiões associadas a RBC –1 e RBC +1, mantendo-se relativamente estável entre os cenários com e sem outliers.}
    \label{fig:rede_gamma_pli}
\end{figure}

\subsection{Banda Theta (4–8 Hz)}
\begin{figure}[H]
    \centering
    \begin{subfigure}[b]{0.48\textwidth}
        \includegraphics[width=\textwidth]{figs/7_bootstrap_results_analysis/2_network_graphs/PLI_EEG-EEG_Com_Outliers/Banda_Theta_(4_Hz_a_8_Hz)_-_Análise_de_Rede_-_PLI_EEG-EEG_Com_Outliers.png}
        \caption{\small Versão com outliers. A rede em theta apresenta um predomínio de conexões positivas (RBC +1), concentradas sobretudo na parte esquerda, embora espalhadas por todas as regiões. Em contraste, as conexões com RBC –1 se concentram entre canais da região frontal esquerda e seus pares na área fronto-central e fronto-temporal direita.}
    \end{subfigure}
    \hfill
    \begin{subfigure}[b]{0.48\textwidth}
        \includegraphics[width=\textwidth]{figs/7_bootstrap_results_analysis/2_network_graphs/PLI_EEG-EEG_Sem_Outliers/Banda_Theta_(4_Hz_a_8_Hz)_-_Análise_de_Rede_-_PLI_EEG-EEG_Sem_Outliers.png}
        \caption{\small Versão sem outliers. A configuração geral permanece inalterada, com um ligeiro aumento nas ocorrências significativas e a aparição de conexões em canais adicionais (como P5, AFz e C4) que não estavam presentes na análise com outliers.}
    \end{subfigure}
    \caption[Análise de Rede – Banda Theta (PLI EEG–EEG)]{\small \textbf{Banda Theta (4–8 Hz):} A rede em theta é marcada por uma predominância de conexões positivas, com distribuição espacial que favorece a região esquerda e uma manutenção do padrão global entre as análises com e sem outliers.}
    \label{fig:rede_theta_pli}
\end{figure}

%%%%%%%%%%%%%%%%%%%%%%%%%%%%%%%%%%%%%%%%%%%%%%%%%%%%%%%%%%%%%%%%%%%%
\section{Rede de Conectividade via CF‐PLM (EEG–ECG)}

A seguir, a análise de rede baseada no índice CF‐PLM, que avalia o acoplamento \emph{cross‐frequency} entre sinais de EEG e ECG. Nesta abordagem, os pares significativos foram identificados para diferentes bandas de frequência, observando a mesma disposição em cenários com e sem outliers.

\subsection{Banda Alpha (8–13 Hz)}
\begin{figure}[H]
    \centering
    \begin{subfigure}[b]{0.48\textwidth}
        \includegraphics[width=\textwidth]{figs/7_bootstrap_results_analysis/2_network_graphs/CF-PLM_EEG-ECG_Com_Outliers/Banda_Alpha_(8_Hz_a_13_Hz)_-_Análise_de_Rede_-_CF-PLM_EEG-ECG_Com_Outliers.png}
        \caption{\small Versão com outliers. Nesta banda, não foram identificadas ocorrências significativas.}
    \end{subfigure}
    \hfill
    \begin{subfigure}[b]{0.48\textwidth}
        \includegraphics[width=\textwidth]{figs/7_bootstrap_results_analysis/2_network_graphs/CF-PLM_EEG-ECG_Sem_Outliers/Banda_Alpha_(8_Hz_a_13_Hz)_-_Análise_de_Rede_-_CF-PLM_EEG-ECG_Sem_Outliers.png}
        \caption{\small Versão sem outliers. O mesmo resultado é observado, com ausência de conexões significativas.}
    \end{subfigure}
    \caption[Análise de Rede – Banda Alpha (CF‐PLM EEG–ECG)]{\small \textbf{Banda Alpha (8–13 Hz):} Tanto nas análises com quanto sem outliers, não se detectaram pares significativos entre EEG e ECG utilizando o CF‐PLM nesta banda.}
    \label{fig:rede_alpha_cfplm}
\end{figure}

\subsection{Banda Beta (13–30 Hz)}
\begin{figure}[H]
    \centering
    \begin{subfigure}[b]{0.48\textwidth}
        \includegraphics[width=\textwidth]{figs/7_bootstrap_results_analysis/2_network_graphs/CF-PLM_EEG-ECG_Com_Outliers/Banda_Beta_(13_Hz_a_30_Hz)_-_Análise_de_Rede_-_CF-PLM_EEG-ECG_Com_Outliers.png}
        \caption{\small Versão com outliers. Nesta banda, os pares significativos são formados pelos canais TPP7h, CP5, CP3 e CP1, todos localizados na região parietal esquerda, apresentando RBC +1.}
    \end{subfigure}
    \hfill
    \begin{subfigure}[b]{0.48\textwidth}
        \includegraphics[width=\textwidth]{figs/7_bootstrap_results_analysis/2_network_graphs/CF-PLM_EEG-ECG_Sem_Outliers/Banda_Beta_(13_Hz_a_30_Hz)_-_Análise_de_Rede_-_CF-PLM_EEG-ECG_Sem_Outliers.png}
        \caption{\small Versão sem outliers. O padrão é idêntico, com os mesmos pares significativos entre TPP7h, CP5, CP3 e CP1 (todos RBC +1).}
    \end{subfigure}
    \caption[Análise de Rede – Banda Beta (CF‐PLM EEG–ECG)]{\small \textbf{Banda Beta (13–30 Hz):} Os pares significativos entre EEG e ECG nesta banda, concentrados na região parietal esquerda, permanecem inalterados entre os cenários com e sem outliers.}
    \label{fig:rede_beta_cfplm}
\end{figure}

\subsection{Banda Delta (0.5–4 Hz)}
\begin{figure}[H]
    \centering
    \begin{subfigure}[b]{0.48\textwidth}
        \includegraphics[width=\textwidth]{figs/7_bootstrap_results_analysis/2_network_graphs/CF-PLM_EEG-ECG_Com_Outliers/Banda_Delta_(0.5_a_4_Hz)_-_Análise_de_Rede_-_CF-PLM_EEG-ECG_Com_Outliers.png}
        \caption{\small Versão com outliers. Aqui, cinco ocorrências significativas foram identificadas, envolvendo pares entre o ECG e os canais CP3, P1, Fp2, FC4 e FTT8h – predominantemente na região parieto-central esquerda e também na região frontal, fronto-central e fronto-temporal direita – todas com RBC +1.}
    \end{subfigure}
    \hfill
    \begin{subfigure}[b]{0.48\textwidth}
        \includegraphics[width=\textwidth]{figs/7_bootstrap_results_analysis/2_network_graphs/CF-PLM_EEG-ECG_Sem_Outliers/Banda_Delta_(0.5_a_4_Hz)_-_Análise_de_Rede_-_CF-PLM_EEG-ECG_Sem_Outliers.png}
        \caption{\small Versão sem outliers. O padrão se mantém idêntico, com os mesmos pares significativos e RBC +1.}
    \end{subfigure}
    \caption[Análise de Rede – Banda Delta (CF‐PLM EEG–ECG)]{\small \textbf{Banda Delta (0.5–4 Hz):} As conexões significativas entre EEG e ECG ocorrem consistentemente com RBC +1, envolvendo pares que conectam a região parieto-central esquerda a áreas frontais, permanecendo inalteradas entre as duas versões.}
    \label{fig:rede_delta_cfplm}
\end{figure}

\subsection{Banda Gamma (30–60 Hz)}
\begin{figure}[H]
    \centering
    \begin{subfigure}[b]{0.48\textwidth}
        \includegraphics[width=\textwidth]{figs/7_bootstrap_results_analysis/2_network_graphs/CF-PLM_EEG-ECG_Com_Outliers/Banda_Gamma_(30_Hz_a_60_Hz)_-_Análise_de_Rede_-_CF-PLM_EEG-ECG_Com_Outliers.png}
        \caption{\small Versão com outliers. Apenas um par significativo foi identificado, conectando o canal de EEG CP3 ao ECG na região centro-parietal esquerda, com RBC +1.}
    \end{subfigure}
    \hfill
    \begin{subfigure}[b]{0.48\textwidth}
        \includegraphics[width=\textwidth]{figs/7_bootstrap_results_analysis/2_network_graphs/CF-PLM_EEG-ECG_Sem_Outliers/Banda_Gamma_(30_Hz_a_60_Hz)_-_Análise_de_Rede_-_CF-PLM_EEG-ECG_Sem_Outliers.png}
        \caption{\small Versão sem outliers. O mesmo par significativo (ECG–CP3, RBC +1) é observado.}
    \end{subfigure}
    \caption[Análise de Rede – Banda Gamma (CF‐PLM EEG–ECG)]{\small \textbf{Banda Gamma (30–60 Hz):} A análise indica um único par significativo entre o EEG e o ECG, localizado na região centro-parietal esquerda, com comportamento idêntico em ambas as versões.}
    \label{fig:rede_gamma_cfplm}
\end{figure}

\subsection{Banda Theta (4–8 Hz)}
\begin{figure}[H]
    \centering
    \begin{subfigure}[b]{0.48\textwidth}
        \includegraphics[width=\textwidth]{figs/7_bootstrap_results_analysis/2_network_graphs/CF-PLM_EEG-ECG_Com_Outliers/Banda_Theta_(4_Hz_a_8_Hz)_-_Análise_de_Rede_-_CF-PLM_EEG-ECG_Com_Outliers.png}
        \caption{\small Versão com outliers. O único par significativo nesta banda é o formado entre o EEG (canal F6) e o ECG, localizado na região frontal direita.}
    \end{subfigure}
    \hfill
    \begin{subfigure}[b]{0.48\textwidth}
        \includegraphics[width=\textwidth]{figs/7_bootstrap_results_analysis/2_network_graphs/CF-PLM_EEG-ECG_Sem_Outliers/Banda_Theta_(4_Hz_a_8_Hz)_-_Análise_de_Rede_-_CF-PLM_EEG-ECG_Sem_Outliers.png}
        \caption{\small Versão sem outliers. O padrão se mantém, com o único par significativo entre EEG (F6) e ECG na região frontal direita.}
    \end{subfigure}
    \caption[Análise de Rede – Banda Theta (CF‐PLM EEG–ECG)]{\small \textbf{Banda Theta (4–8 Hz):} A análise revela que, para ambas as versões, o único par significativo ocorre entre o EEG e o ECG, especificamente envolvendo o canal F6, com RBC +1.}
    \label{fig:rede_theta_cfplm}
\end{figure}


\section{Síntese e Considerações Finais}

Em ambas as metodologias de análise de rede (PLI para conexões EEG–EEG e CF‐PLM para conexões EEG–ECG), observa-se que os padrões de conectividade se mantêm, em grande parte, semelhantes entre as análises com e sem outliers. De maneira geral, os pares significativos se agrupam em dois clusters principais: 
\begin{itemize}
    \item \textbf{Região Temporo-parietal/Parietal/Parieto-central esquerda:} Onde se concentram várias conexões negativas (RBC –1) e, em alguns casos, conexões positivas, sobretudo na análise CF‐PLM em banda beta e delta.
    \item \textbf{Região Frontal, Fronto-central e Frontotemporal direita:} Onde predominam conexões positivas (RBC +1), especialmente evidentes na banda alpha (no contexto EEG–EEG) e na banda theta.
\end{itemize}

Essa configuração sugere que a estimulação catódica sobre o DLPFC promove uma modulação diferenciada da sincronia de fase, elevando a conectividade em certas regiões enquanto reduz em outras, conforme evidenciado pelos diferentes padrões de RBC observados em cada banda de frequência.

Adicionalmente, ao comparar as bandas, observa-se um padrão distinto entre as metodologias: na análise via PLI, as bandas alpha e theta tendem a exibir predominância de conexões positivas (RBC +1) – sugerindo aumento da sincronia sob a estimulação catódica – enquanto as bandas delta (e, parcialmente, a gama) apresentam maior ocorrência de conexões negativas (RBC –1), indicando uma redução na sincronia em determinadas regiões. Por outro lado, para o CF‐PLM, todas as conexões significativas identificadas são invariavelmente positivas (RBC +1), sugerindo que a neuromodulação promove um aumento consistente no acoplamento cross-frequency entre EEG e ECG nas bandas onde a significância é detectada.

A comparação entre as versões com e sem outliers reforça a robustez dos achados, já que as configurações gerais das redes não se alteram de forma substancial após a remoção dos pontos atípicos.

Esta análise integrada contribui para uma compreensão mais aprofundada dos efeitos da neuromodulação na dinâmica da conectividade cerebral e na interação cérebro–coração, fornecendo bases sólidas para interpretações futuras dos mecanismos subjacentes à modulação de rede em contextos esportivos e clínicos.