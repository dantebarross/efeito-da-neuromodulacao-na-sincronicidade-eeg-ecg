\chapter{Métodos de Análise de Sincronização de Fase}
\label{chap:6_metodos_de_analise_de_sincronizacao_de_fase}
Neste capítulo, apresentamos os fundamentos teóricos e práticos dos métodos empregados para analisar a sincronização de fase entre sinais fisiológicos. Para este estudo, optamos por utilizar o PLI para quantificar a sincronização entre canais de EEG (mesma frequência) e o CF-PLM para avaliar o acoplamento \textit{cross-frequency} entre EEG e ECG. Adicionalmente, o tradicional PLV foi testado para comparação, cujos resultados encontram-se disponíveis no anexo.

\section{Fundamentos dos Métodos de Análise de Sincronização}
A conectividade funcional cerebral emerge da interação dinâmica e contínua entre diferentes regiões neurais, um fenômeno fundamental para a coordenação de respostas comportamentais complexas. Como destacado por \citeonline{sorrentino2022detection}, esta comunicação neural transcende a simples transmissão de informações, manifestando-se através de padrões específicos de sincronização que permitem a integração funcional entre áreas cerebrais distribuídas. Embora diversas métricas tenham sido desenvolvidas para detectar interações entre frequências distintas (\textit{cross-frequency coupling}, CFC), a maioria concentra-se no acoplamento fase-amplitude (\textit{phase-amplitude coupling}, PAC), capturando apenas parcialmente a riqueza das interações neurais.

A comunicação neural, entretanto, envolve mecanismos de feedback que exigem relações temporais extraordinariamente precisas entre populações neuronais. Um desses mecanismos fundamentais, denominado \emph{reentry} por Seraj (2018), amplifica sinais de feedback ao sincronizar as taxas de disparo de neurônios em uma assembleia funcional. Esta sincronização representa um mecanismo neural essencial para a integração de informações, permitindo que regiões cerebrais distantes coordenem sua atividade de forma coerente. A precisão temporal necessária para este processo torna a análise de fase particularmente valiosa, pois permite detectar o \textit{timing} exato entre disparos neuronais com uma resolução que análises baseadas apenas em amplitude não conseguem alcançar \cite{seraj2018cerebral, ren2022multi}.

A análise de sincronização de fase visa quantificar a consistência da diferença de fase entre dois sinais ao longo do tempo, um aspecto crucial para compreender como oscilações neurais se coordenam. Conforme detalhado por \citeonline{seraj2018cerebral}, esta abordagem pode ser implementada através de métodos baseados em modelos estatísticos ou técnicas \textit{data-driven}, cada um com vantagens específicas dependendo do contexto experimental. A análise de fase tem se mostrado particularmente robusta para investigar a conectividade cerebral, pois fornece informações confiáveis sobre os padrões de comunicação neural mesmo na presença de ruído ou variações na amplitude do sinal. Como destacado por \citeonline{seraj2018cerebral}, a fase é geralmente considerada uma característica ``mais pura'' e menos contaminada que a amplitude dos registros cerebrais, sendo menos influenciada pela impedância do crânio ou por artefatos como movimentos oculares ou musculares faciais.

Para extrair a fase dos sinais, técnicas como a Transformada de Hilbert são amplamente empregadas na literatura. Alternativamente, a Transformada de Fourier e os wavelets (particularmente o wavelet de Morlet) oferecem uma decomposição tempo-frequencial dos sinais que pode ser vantajosa em determinados contextos. \citeonline{singh2024evaluating} observam que, embora a Transformada de Fourier seja eficaz para sinais estacionários, sua interpretação torna-se mais complexa quando o conteúdo em frequência varia com o tempo, uma característica comum em sinais cerebrais. Para estes sinais dinâmicos e não-estacionários, a decomposição tempo-frequência torna-se essencial para preservar as vantagens de ambos os domínios, ainda que com algum sacrifício na precisão temporal e frequencial.

Abordagens baseadas em modelos estatísticos, como as propostas por \citeonline{nadalin2019statistical}, oferecem um framework rigoroso para avaliar o acoplamento fase-amplitude (PAC) e o acoplamento amplitude-amplitude (AAC). No entanto, estas metodologias podem não capturar completamente as relações dinâmicas entre as fases dos sinais, especialmente em dados de estado de repouso (\textit{resting-state}). A dependência de pressuposições estatísticas e o rigor necessário no controle de covariáveis podem limitar a identificação de padrões de sincronização mais complexos, particularmente quando o acoplamento de fase (\textit{phase-phase coupling}, PPC) entre diferentes bandas de EEG e ECG tende a ser mais variável e não-linear.

Uma vantagem significativa das análises contemporâneas é a utilização de gravações multicanais, que possibilitam a identificação de padrões fisiologicamente interpretáveis de acoplamento entre frequências. \citeonline{cohen2017multivariate} desenvolveu técnicas de redução de dimensionalidade e separação de fontes que permitem isolar padrões de atividade que seriam difíceis de detectar em sinais monofacetados. O framework de decomposição generalizada (\textit{Generalized Eigenvalue Decomposition}, gedCFC), apresentado por \citeonline{cohen2017multivariate}, propõe-se a superar limitações dos métodos tradicionais, especialmente quando se trata de sinais não-estacionários e multicanais.

Entretanto, embora o gedCFC ofereça amplas possibilidades para a extração de componentes e descoberta de padrões de CFC, sua grande flexibilidade implica uma multiplicidade de configurações e parâmetros, o que pode torná-lo impraticável para abordagens exploratórias sem hipóteses bem refinadas~\cite{cohen2017multivariate}. Diante disso, optamos pelo CF-PLM, método que, apesar de ser menos abrangente que o gedCFC, apresenta uma implementação mais direta, simples e adequada para nossa análise exploratória de sincronização entre ECG e múltiplos canais de EEG.

A crescente evidência científica demonstra que a fase desempenha um papel essencial na organização e transmissão da informação neural. \citeonline{canolty2010functional} evidenciaram que a fase de oscilações de baixa frequência organiza temporalmente a atividade rápida, criando janelas específicas para a integração de informação. Complementarmente, \citeonline{hyafil2015neural} demonstraram que padrões específicos de alinhamento fase-frequência (como theta-gamma) possuem funções cognitivas especializadas, como o processamento da fala, enquanto \citeonline{lisman2013theta} propuseram que tais acoplamentos organizam sequências de informação, com cada ciclo gamma aninhado em um ciclo theta representando um item distinto na memória de trabalho.

Esta ênfase na fase como mecanismo organizador da atividade neural é corroborada pelo modelo \textit{Communication Through Coherence} (CTC) proposto por \citeonline{fries2015rhythms}, que destaca que a sincronização de fase entre regiões cerebrais é decisiva para uma transmissão eficaz de informação. Empiricamente, \citeonline{siegel2012spectral} demonstraram que padrões estáveis de coerência de fase entre regiões são características marcantes das interações de larga escala que sustentam processos cognitivos, indicando que a organização temporal mediada pela fase é crucial para integrar a atividade em frequências mais altas. Assim, a literatura converge ao apontar que a fase, e não apenas a amplitude, desempenha um papel essencial na codificação e transferência de informação no cérebro.

\citeonline{ren2022multi} revisaram extensivamente o uso do acoplamento fase-fase (PPC) na montagem de redes cerebrais, demonstrando que abordagens multi-granulares podem revelar padrões robustos de sincronização \textit{cross-frequency}. Diversos estudos têm evidenciado que a interação entre diferentes bandas de frequência ocorre em regiões críticas como o hipocampo, o córtex pré-frontal e o sensorial, tanto em humanos quanto em primatas não-humanos~\cite{mormann2005phase, canolty2006high, jensen2007cross, khamechian2020decoding}. Pesquisas conduzidas por \citeonline{dimitriadis2015cognitive} e \citeonline{davoudi2021prefrontal} exploraram o acoplamento entre bandas theta e alpha, bem como mecanismos de acoplamento alpha-gamma durante tarefas cognitivas, revelando padrões específicos de sincronização associados a diferentes demandas mentais.

\citeonline{chen2023multiple} demonstraram que alterações nos padrões de acoplamento \textit{cross-frequency} podem atuar como biomarcadores relevantes em contextos neurofisiológicos. Estas modificações refletem ajustes dinâmicos na comunicação neural entre regiões cerebrais e podem indicar estados funcionais específicos, como processos atencionais, cognitivos ou até mesmo condições patológicas. Assim, a análise do CFC tem sido amplamente empregada para compreender melhor os mecanismos subjacentes à organização funcional do cérebro.

Entretanto, como destacado por \citeonline{ren2022multi}, extrair características efetivas do CFC é substancialmente mais desafiador do que analisar o acoplamento intrafrequencial, pois os dados \textit{cross-frequency} são intrinsecamente mais complexos e contêm informações ocultas que requerem análises sofisticadas para elucidar os mecanismos fisiológicos subjacentes~\cite{ren2022multi}. Esta complexidade ressalta a importância do desenvolvimento de métodos robustos para explorar os padrões de sincronização \textit{cross-frequency}. Apesar da ampla variedade de métricas existentes para PAC, há uma notável escassez de métodos que investiguem especificamente o PPC nessas interações. O CF-PLM, método adotado neste estudo, é um dos poucos que se propõe a capturar essa forma de sincronização, oferecendo uma alternativa mais direcionada para a análise da conectividade neural baseada em fase.

\subsection{\textit{Phase Lag Index} (PLI)}
O \textit{Phase Lag Index} (PLI) representa uma evolução significativa nas métricas de sincronização de fase, desenvolvido especificamente para superar limitações críticas de métodos anteriores. Conforme detalhado por \citeonline{seraj2018cerebral}, o PLI foi concebido para quantificar a sincronização entre sinais operando na mesma faixa de frequência, como os canais de EEG dentro de uma mesma banda, enquanto minimiza a detecção de sincronizações espúrias. A principal inovação do PLI reside em sua robustez contra o fenômeno de condução de volume (\textit{volume conduction}), um problema persistente em análises de EEG que ocorre quando uma única fonte neural gera potenciais elétricos detectados simultaneamente por múltiplos eletrodos, criando uma falsa impressão de sincronização.

Diferentemente do tradicional \textit{Phase Locking Value} (PLV), que pode ser inflacionado por efeitos de condução de volume, o PLI considera exclusivamente o sinal (positivo ou negativo) das diferenças de fase, ignorando completamente valores próximos a zero. Esta abordagem é fundamentada na observação de que sincronizações genuínas entre regiões cerebrais distintas tipicamente apresentam diferenças de fase consistentemente diferentes de zero, enquanto sincronizações espúrias causadas por fontes comuns tendem a produzir diferenças de fase centradas em zero \cite{seraj2018cerebral}. Matematicamente, o PLI é expresso como:

\[
\text{PLI} = |\langle \text{sign}(\Delta\phi(t)) \rangle|
\]

onde \(\Delta\phi(t)\) representa a diferença instantânea entre as fases de dois sinais, e \(\langle \cdot \rangle\) denota a média ao longo do tempo. A função \texttt{sign} extrai apenas o sinal da diferença de fase, retornando +1 para diferenças positivas, -1 para diferenças negativas e 0 para diferenças exatamente iguais a zero. Ao calcular o valor absoluto da média desses sinais, o PLI quantifica a assimetria na distribuição das diferenças de fase, um indicador robusto de acoplamento funcional genuíno.

Uma evolução do PLI, o \textit{\textbf{weighted Phase Lag Index}} (wPLI), introduz refinamentos adicionais ao incorporar pesos às diferenças de fase com base na magnitude da parte imaginária do espectro cruzado. Esta abordagem, também descrita por \citeonline{seraj2018cerebral}, é formulada como:

\[
\text{wPLI} = \frac{|\langle \Im(Z) \rangle|}{\langle |\Im(Z)| \rangle}
\]

onde \(Z\) representa o sinal espectral cruzado complexo entre dois sinais, e \(\Im(Z)\) denota sua parte imaginária. Ao ponderar as contribuições pela magnitude da parte imaginária, o wPLI atribui maior importância a diferenças de fase mais distantes de zero e de π, reduzindo ainda mais a influência de ruído e de efeitos de condução de volume.

Embora o wPLI ofereça potencialmente maior sensibilidade para detectar acoplamentos funcionais genuínos em ambientes altamente contaminados por ruído, optamos pelo PLI tradicional no presente estudo. Esta escolha metodológica se justifica pela maior rigorosidade do PLI ao descartar totalmente interações com diferenças de fase próximas a zero, proporcionando uma medida mais conservadora que minimiza o risco de falsos positivos. Como destacado por \citeonline{seraj2018cerebral}, em contextos onde a prioridade é evitar a detecção de sincronizações espúrias, o PLI tradicional pode oferecer vantagens sobre variantes mais sensíveis como o wPLI.

O PLI produz valores no intervalo de 0 a 1, onde 0 indica ausência completa de acoplamento de fase não-zero (sugerindo sincronização espúria ou ausência de sincronização) e 1 indica um acoplamento de fase consistente e assimétrico (sugerindo interação neural genuína). Esta propriedade torna o PLI particularmente adequado para nosso estudo, onde buscamos identificar padrões robustos de sincronização entre canais de EEG, minimizando a influência de artefatos e efeitos de condução de volume que poderiam comprometer a interpretação dos resultados.

\subsection{\textit{Cross-Frequency Phase Linearity Measurement} (CF-PLM)}
O \textit{Cross-Frequency Phase Linearity Measurement} (CF-PLM) representa uma inovação metodológica significativa no campo da análise de sincronização neural, desenvolvido especificamente para investigar o acoplamento fase-fase (PPC) entre sinais que operam em diferentes faixas de frequência. Introduzido por \citeonline{sorrentino2022detection}, este método preenche uma lacuna importante na literatura, onde a maioria das métricas existentes concentra-se no acoplamento fase-amplitude, negligenciando as interações fase-fase que podem ser cruciais para a compreensão da comunicação neural em diferentes escalas temporais.

A principal vantagem do CF-PLM reside em sua capacidade de detectar sincronizações de fase entre sinais de frequências distintas sem necessidade de hipóteses prévias sobre as bandas específicas envolvidas no acoplamento. Esta característica é particularmente valiosa para nossa investigação, que busca avaliar a relação entre os ritmos neurais de alta frequência registrados pelo EEG (4-50 Hz) e os sinais cardiovasculares de baixa frequência capturados pelo ECG (tipicamente próximos a 1 Hz). Como destacado por \citeonline{sorrentino2022detection}, o CF-PLM expande o conceito original de \textit{Phase Linearity Measurement} (PLM) para acomodar sincronizações de ordem \(n:m\), onde \(n\) ciclos de um sinal em uma determinada frequência se alinham consistentemente com \(m\) ciclos de outro sinal em uma frequência distinta.

Esta abordagem representa um avanço significativo em relação a métodos tradicionais que frequentemente pressupõem relações harmônicas fixas entre os sinais, limitando sua aplicabilidade em contextos exploratórios. O CF-PLM, ao eliminar tais pressuposições, permite a detecção de padrões de acoplamento mais flexíveis e adaptáveis, essenciais para capturar a complexidade das interações entre sistemas neurais e fisiológicos. Embora outras abordagens avançadas, como o framework de decomposição generalizada (gedCFC) proposto por \citeonline{cohen2017multivariate}, também tenham sido desenvolvidas para identificar PPC em sinais multicanais complexos e não-estacionários, o CF-PLM oferece um equilíbrio ideal entre sofisticação analítica e implementação prática para nossos objetivos específicos.

O procedimento metodológico do CF-PLM envolve quatro etapas principais, cada uma fundamentada em princípios matemáticos robustos:

\begin{enumerate}
    \item \textbf{Extração das fases instantâneas via sinais analíticos}: Inicialmente, para os sinais temporais \(x(t)\) e \(y(t)\), obtêm-se suas representações analíticas \(x_{\mathrm{an}}(t)\) e \(y_{\mathrm{an}}(t)\) através da Transformada de Hilbert. Esta transformação gera um sinal complexo cuja parte real é o sinal original e a parte imaginária é a transformada de Hilbert do sinal original, permitindo a extração das fases instantâneas \(\phi_x(t)\) e \(\phi_y(t)\) com alta precisão temporal. A Transformada de Hilbert é particularmente adequada para esta aplicação por preservar o conteúdo espectral do sinal original enquanto fornece uma representação analítica que facilita a análise de fase.

    \item \textbf{Construção do sinal interferométrico normalizado}: Na segunda etapa, calcula-se o sinal interferométrico \(z(t)\), definido matematicamente como:
    \[
    z(t) = \frac{x_{\mathrm{an}}(t)\, y_{\mathrm{an}}^*(t)}{\lvert x_{\mathrm{an}}(t)\rvert\, \lvert y_{\mathrm{an}}(t)\rvert} = e^{i\Delta \phi(t)},
    \]
    onde \(\Delta \phi(t) = \phi_x(t) - \phi_y(t)\) representa a diferença de fase instantânea entre os sinais, e \(y_{\mathrm{an}}^*(t)\) denota o conjugado complexo de \(y_{\mathrm{an}}(t)\). A normalização pelos módulos \(\lvert x_{\mathrm{an}}(t)\rvert\) e \(\lvert y_{\mathrm{an}}(t)\rvert\) garante que o sinal resultante tenha amplitude unitária, isolando exclusivamente a informação de fase e eliminando qualquer influência das amplitudes dos sinais originais. Esta propriedade é crucial para a análise de sincronização de fase pura, independente de variações de amplitude que poderiam confundir a interpretação dos resultados.

    \item \textbf{Análise espectral do sinal interferométrico}: Utilizando a Transformada de Fourier, calcula-se a densidade espectral de potência (PSD) do sinal interferométrico \(z(t)\). Esta análise revela características fundamentais sobre o tipo de acoplamento presente: em cenários de acoplamento iso-frequencial (mesma frequência), o pico da PSD está centrado em \(f = 0\), indicando uma diferença de fase estável; em contraste, nos casos de acoplamento \textit{cross-frequency}, o pico se desloca para uma frequência correspondente à diferença entre as frequências dominantes dos sinais originais. Esta propriedade, também discutida por \citeonline{seraj2018cerebral}, permite identificar não apenas a presença de acoplamento, mas também caracterizar sua natureza específica.

    \item \textbf{Quantificação do acoplamento via índice CF-PLM}: Finalmente, o índice CF-PLM é calculado integrando a PSD em uma faixa estreita \([f_\Delta - B, f_\Delta + B]\) ao redor do pico identificado, e normalizando pelo poder total da PSD:
    \[
    \text{CF-PLM} = \frac{\displaystyle\int_{f_\Delta - B}^{f_\Delta + B} S_Z(f) \, df}{\displaystyle\int_{-\infty}^{+\infty} S_Z(f) \, df}.
    \]
    Nesta expressão, \(f_\Delta\) representa a frequência do pico identificado, \(B\) é a largura de banda considerada para a integração, e \(S_Z(f)\) é a densidade espectral de potência do sinal interferométrico. O resultado é um índice normalizado entre 0 e 1, onde valores próximos a 1 indicam forte acoplamento de fase na frequência específica \(f_\Delta\), enquanto valores próximos a 0 sugerem ausência de acoplamento consistente.
\end{enumerate}

É importante ressaltar que, diferentemente do PLI, o CF-PLM não implementa mecanismos específicos para descartar sincronizações próximas de zero (\textit{zero-lag}), que poderiam decorrer de efeitos como o \textit{volume conduction}. Esta característica metodológica implica que o CF-PLM, por si só, não oferece proteção explícita contra estas interferências, um aspecto que deve ser cuidadosamente considerado durante sua aplicação e interpretação. No entanto, no contexto específico de nossa investigação, onde analisamos a sincronização entre sinais de EEG e ECG — registrados por sistemas distintos e representando processos fisiológicos diferentes — o risco de contaminação por condução de volume é substancialmente reduzido, minimizando esta limitação potencial.

Em nossa implementação específica — análise exploratória de sincronização entre sinais de EEG e ECG obtidos por equipamentos distintos —, enfrentamos um desafio técnico relacionado à precisão temporal absoluta das gravações. Apesar de utilizarmos marcadores claros para o início das sessões de coleta, não é possível garantir sincronização com precisão milimétrica entre as séries temporais destes dois sistemas de registro. Entretanto, esta limitação não compromete a validade do CF-PLM em nosso estudo, uma vez que o método não depende da precisão absoluta dos tempos iniciais dos sinais, mas sim da \textit{consistência temporal} da diferença de fase ao longo do período analisado. Assim, mesmo na presença de um atraso fixo ou variável entre as séries EEG e ECG devido à sincronização técnica imperfeita, desde que este atraso permaneça relativamente estável, o método consegue capturar efetivamente a presença e magnitude da sincronização \textit{cross-frequency}.

A escolha do CF-PLM como metodologia central para nossa análise de acoplamento \textit{cross-frequency} é, portanto, fundamentada tanto em suas propriedades matemáticas robustas quanto em sua adequação específica às nossas condições experimentais. Este método oferece uma abordagem inovadora e flexível para investigar as interações fase-fase entre sistemas neurais e cardiovasculares em condições de repouso (\textit{resting-state}), potencialmente revelando mecanismos de integração corpo-cérebro que permanecem inexplorados por técnicas convencionais focadas exclusivamente em acoplamento fase-amplitude.


\subsection{Comparação com o \textit{Phase Locking Value} (PLV)}
O \textit{Phase Locking Value} (PLV) representa um dos métodos pioneiros e mais amplamente utilizados para quantificar a sincronização de fase entre sinais neurais. Apesar do desenvolvimento de métricas mais sofisticadas como o PLI e o CF-PLM, o PLV continua sendo um padrão de referência na literatura neurocientífica, justificando sua inclusão em nosso estudo para fins comparativos. Esta abordagem permite contextualizar nossos resultados no amplo corpo de pesquisas existentes que empregam esta métrica tradicional, facilitando comparações diretas com estudos anteriores.

O PLV quantifica a consistência da diferença de fase entre dois sinais ao longo do tempo, sendo matematicamente definido como:

\[
\text{PLV} = \left| \frac{1}{N} \sum_{j=1}^{N} e^{i\Delta\phi(j)} \right|
\]

onde \(\Delta\phi(j)\) representa a diferença de fase entre dois sinais no instante \(j\), e \(N\) corresponde ao número total de amostras temporais analisadas. Esta formulação resulta em um índice normalizado que varia entre 0 e 1, onde 0 indica completa ausência de sincronização (diferenças de fase distribuídas uniformemente) e 1 indica sincronização perfeita (diferença de fase absolutamente constante ao longo do tempo).

A principal limitação do PLV, amplamente reconhecida na literatura e discutida extensivamente por \citeonline{seraj2018cerebral}, reside em sua incapacidade de distinguir entre sincronizações neurais genuínas e aquelas artificialmente induzidas por fenômenos como o \textit{volume conduction}. Este fenômeno ocorre quando uma única fonte neural gera potenciais elétricos que são simultaneamente detectados por múltiplos eletrodos, criando uma falsa impressão de sincronização com diferença de fase próxima a zero. Como o PLV considera igualmente todas as diferenças de fase, independentemente de seu valor absoluto, ele tende a produzir resultados inflacionados em análises de EEG, onde o \textit{volume conduction} é particularmente prevalente devido à condutividade do tecido cerebral e do crânio.

Apesar destas limitações metodológicas, o PLV mantém seu valor como ferramenta comparativa, especialmente em contextos onde o \textit{volume conduction} é menos preocupante. Em nossa investigação específica da sincronização entre sinais de EEG e ECG, que são registrados por sistemas distintos e representam processos fisiológicos diferentes, o risco de contaminação por condução de volume é substancialmente reduzido. Neste cenário particular, o PLV pode fornecer informações complementares valiosas, permitindo uma triangulação metodológica que fortalece a robustez de nossas conclusões.


\subsection{Escolha e Justificativa dos Métodos}
A seleção criteriosa dos métodos de análise representa um passo fundamental para garantir a validade e a confiabilidade dos resultados em estudos de conectividade neural. Nesta investigação, a escolha metodológica foi orientada por uma avaliação sistemática da literatura científica contemporânea, considerando as especificidades dos sinais neurofisiológicos analisados e os objetivos específicos da pesquisa.

A literatura recente tem demonstrado que diferentes métricas de sincronização capturam aspectos distintos da conectividade funcional cerebral. \citeonline{abubaker2021working} investigaram o acoplamento cruzado entre frequências em tarefas de memória de trabalho utilizando o PLV, revelando que padrões específicos de sincronização podem estar intrinsecamente associados ao desempenho cognitivo. Entretanto, como destacado por \citeonline{seraj2018cerebral} e \citeonline{zhang2014phase}, a sensibilidade do PLV a ruídos e efeitos de \textit{volume conduction} pode comprometer significativamente sua aplicabilidade em determinados contextos experimentais, especialmente em análises de EEG onde a condução de volume é particularmente problemática. Esta limitação metodológica fundamentou nossa decisão de adotar o PLI como métrica principal para quantificar acoplamentos na mesma frequência, aproveitando sua robustez contra sincronizações espúrias e sua capacidade de discriminar interações neurais genuínas.

A análise do acoplamento \textit{cross-frequency} constituiu uma preocupação central deste estudo, reconhecendo sua importância fundamental para a compreensão da comunicação neural em diferentes escalas temporais e funcionais. Como evidenciado por \citeonline{hulsemann2019quantification}, diversas métricas têm sido propostas para quantificar essa comunicação, incluindo o acoplamento fase-amplitude (PAC), a bicoerência e várias formas de \textit{phase-locking}. No entanto, identificamos uma notável escassez de métodos especificamente desenvolvidos para investigar o acoplamento fase-fase (PPC) em interações \textit{cross-frequency}, uma lacuna metodológica significativa considerando a relevância deste tipo de acoplamento para a integração de informações entre sistemas neurais operando em diferentes frequências.

Neste contexto, o CF-PLM emerge como uma solução metodológica particularmente adequada, distinguindo-se pela sua capacidade de detectar padrões de sincronização sem assumir relações harmônicas fixas entre as frequências envolvidas. Esta característica, destacada por \citeonline{sorrentino2022detection}, \citeonline{seraj2018cerebral} e \citeonline{chen2023multiple}, torna o CF-PLM especialmente apropriado para a análise das interações entre os ritmos cerebrais de alta frequência (EEG) e os sinais cardiovasculares de baixa frequência (ECG), permitindo uma exploração mais flexível e realista das dinâmicas de acoplamento entre estes sistemas fisiológicos distintos.

Embora abordagens mais avançadas, como o framework de decomposição generalizada (gedCFC) proposto por \citeonline{cohen2017where}, ofereçam capacidades analíticas expandidas para explorar a conectividade neural multicanal, sua implementação envolve considerações práticas importantes. O gedCFC amplia significativamente a capacidade de identificar padrões de PPC mesmo em sinais altamente não estacionários, reforçando a evidência de que a fase pode codificar informações mais ricas e complexas do que a amplitude. Contudo, como observado por \citeonline{cohen2017multivariate}, a flexibilidade inerente ao gedCFC implica um aumento substancial na complexidade computacional e na necessidade de parametrização cuidadosa, aspectos que podem limitar sua aplicabilidade em análises exploratórias sem hipóteses bem refinadas, como é o caso do presente estudo.

A decisão metodológica final foi orientada por quatro critérios fundamentais que embasaram a escolha do CF-PLM e do PLI como métricas principais, com o PLV incluído apenas como referência comparativa:

\begin{itemize}
  \item \textbf{Robustez contra artefatos e volume conduction:} O PLI foi selecionado por sua capacidade superior de minimizar a detecção de sincronizações triviais (\textit{zero-lag}), frequentemente associadas a artefatos induzidos pelo volume conduction. Esta propriedade, extensivamente documentada por \citeonline{seraj2018cerebral} e \citeonline{zhang2014phase}, é crucial para garantir que as sincronizações detectadas reflitam interações neurais genuínas e não artefatos metodológicos.

  \item \textbf{Capacidade de detectar PPC \textit{cross-frequency}:} O CF-PLM foi adotado por sua capacidade única de capturar relações flexíveis entre frequências distintas, sem impor pressupostos restritivos sobre harmônicos fixos. Esta característica, destacada por \citeonline{sorrentino2022detection}, \citeonline{seraj2018cerebral} e \citeonline{chen2023multiple}, permite uma análise mais realista e abrangente das interações entre os ritmos cerebrais e cardiovasculares, essencial para os objetivos desta investigação.

  \item \textbf{Adequação ao tipo de sinal analisado:} A simplicidade operacional e a robustez analítica do CF-PLM e do PLI os tornam particularmente adequados para a análise exploratória em condições de \textit{resting-state}, especialmente considerando as limitações práticas na sincronização temporal absoluta entre os sistemas de registro de EEG e ECG. Embora métodos mais sofisticados como o gedCFC ofereçam maior flexibilidade para dados multicanais, sua complexidade implementacional poderia introduzir variáveis confundidoras desnecessárias neste contexto específico.

  \item \textbf{Caracterização robusta de padrões de conectividade:} Nossa abordagem priorizou a caracterização geral dos padrões de conectividade funcional ao longo do período analisado, em detrimento de testes estatísticos detalhados sobre variações temporais específicas. Esta decisão metodológica alinha-se perfeitamente com as propriedades do PLI e do CF-PLM, que oferecem medidas robustas e confiáveis da sincronização global, mesmo na presença de flutuações temporais localizadas.
\end{itemize}

A integração destas considerações metodológicas resultou na adoção de uma estratégia analítica complementar, combinando o PLI para a quantificação da sincronização iso-frequencial entre canais de EEG e o CF-PLM para a avaliação do acoplamento \textit{cross-frequency} entre EEG e ECG. O PLV, apesar de suas limitações reconhecidas, foi incluído como referência comparativa, facilitando o diálogo com o amplo corpo de literatura que utiliza esta métrica tradicional. Esta abordagem metodológica integrada proporciona uma caracterização abrangente e robusta da conectividade funcional entre sistemas neurais e cardiovasculares, maximizando a validade e a confiabilidade dos resultados obtidos.


\section{Validação Experimental com Injeção de Sinais}
A validação rigorosa de métodos de análise de sincronização constitui um passo essencial para garantir a confiabilidade e a interpretabilidade dos resultados em estudos neurofisiológicos. Para avaliar sistematicamente a sensibilidade e a especificidade das métricas empregadas nesta investigação, desenvolvemos um protocolo experimental de validação baseado na injeção controlada de sinais senoidais sobre dados reais de ECG e EEG coletados durante as sessões experimentais. Esta abordagem permite quantificar precisamente a capacidade dos índices CF-PLM, PLV e PLI em detectar e caracterizar diferentes padrões e intensidades de sincronização de fase artificialmente introduzidos.

O protocolo de validação foi estruturado em quatro etapas metodológicas complementares:

\begin{enumerate}
    \item \textbf{Seleção de segmentos representativos}: Inicialmente, foram identificados e extraídos segmentos representativos dos sinais originais de ECG e EEG, garantindo a inclusão de características típicas destes sinais fisiológicos, como complexos QRS no ECG e ritmos oscilatórios característicos no EEG. Esta seleção cuidadosa assegurou que a validação fosse realizada em condições realistas, preservando as propriedades intrínsecas dos sinais biológicos.

    \item \textbf{Geração de sinais senoidais controlados}: Foram sintetizados sinais senoidais com parâmetros precisamente definidos (frequência, fase e amplitude), utilizando funções senoidais puras. Para o cenário \textit{cross-frequency}, geramos sinais com frequências distintas (1 Hz para ECG e 40 Hz para EEG), enquanto para o cenário \textit{same-frequency}, utilizamos a mesma frequência para ambos os sinais, com defasagens controladas. Esta abordagem permitiu simular diferentes tipos de acoplamento fase-fase que poderiam ocorrer naturalmente entre sistemas fisiológicos.

    \item \textbf{Aplicação gradual dos sinais artificiais}: Os sinais senoidais foram incorporados aos sinais originais através de um sistema de máscaras de injeção que permitiu controlar com precisão a proporção de contribuição do sinal artificial. Foram implementados cinco níveis de injeção (0\%, 25\%, 50\%, 75\% e 100\%), criando um espectro contínuo desde o sinal original puro até o sinal completamente dominado pela componente artificial. Esta gradação permitiu avaliar a sensibilidade dos métodos em função da intensidade do acoplamento.

    \item \textbf{Análise comparativa dos índices de sincronização}: Finalmente, calculamos os índices CF-PLM, PLV e PLI para cada combinação de sinais modificados, quantificando sistematicamente como cada métrica responde aos diferentes cenários de sincronização. Esta análise comparativa revelou as características específicas de cada método em termos de sensibilidade, especificidade e robustez frente a diferentes tipos e intensidades de acoplamento fase-fase.
\end{enumerate}

Para ilustrar o processo de injeção controlada, as Figuras~\ref{fig:ecg_injection} e~\ref{fig:eeg_injection} apresentam exemplos visuais dos sinais antes e após a injeção no cenário \textit{cross-frequency}. Estas visualizações demonstram como os sinais artificiais com frequências distintas (1 Hz para ECG e 40 Hz para EEG) foram incorporados aos sinais fisiológicos originais, permitindo uma avaliação controlada da capacidade dos métodos em detectar acoplamentos entre oscilações de diferentes frequências.

\standardfigure{figs/3_2_testing_connectivity_metrics/1_ECG_Original_vs_Injecao_Cross-frequency.png}
{ECG: comparação entre o sinal original (azul) e o sinal senoidal injetado de 1~Hz (vermelho) no cenário \textit{cross-frequency}. A injeção controlada permite avaliar a sensibilidade dos métodos de sincronização a acoplamentos de baixa frequência.}
{ecg_injection}

\standardfigure{figs/3_2_testing_connectivity_metrics/3_EEG_Original_vs_Injecao_Cross-frequency.png}
{EEG: comparação entre o sinal original (azul) e o sinal senoidal injetado de 40~Hz (vermelho) no cenário \textit{cross-frequency}. A alta frequência do sinal injetado simula ritmos gama cerebrais, permitindo avaliar a detecção de acoplamentos entre oscilações rápidas cerebrais e ritmos cardíacos mais lentos.}
{eeg_injection}

Para complementar a avaliação no cenário \textit{cross-frequency}, implementamos também um cenário \textit{same-frequency}, onde tanto o ECG quanto o EEG receberam sinais senoidais com frequências idênticas, mas com uma defasagem controlada de \(\pi/4\) radianos. Este cenário é particularmente relevante para avaliar a capacidade dos métodos em detectar sincronizações iso-frequenciais com defasagens específicas, um padrão que pode ocorrer em determinados estados fisiológicos onde diferentes regiões cerebrais ou sistemas corporais oscilam na mesma frequência, mas com relações de fase estáveis. As Figuras~\ref{fig:eeg_original_vs_injection_samefreq} e~\ref{fig:eeg_injected_samefreq} ilustram este cenário experimental, evidenciando tanto os sinais de injeção quanto o resultado da incorporação controlada destes sinais aos dados originais.

\standardfigure{figs/3_2_testing_connectivity_metrics/10_EEG_Original_vs_Injecao_Same-frequency.png}
{Visualização do cenário \textit{same-frequency}: sinal de EEG original (azul) e sinal senoidal de injeção (vermelho) com defasagem controlada de \(\pi/4\) radianos. Esta configuração permite avaliar a sensibilidade dos métodos a sincronizações com defasagens específicas.}
{eeg_original_vs_injection_samefreq}

\standardfigure{figs/3_2_testing_connectivity_metrics/11_EEG_Injetado_Same-frequency.png}
{Resultado da injeção controlada no cenário \textit{same-frequency}: sinal de EEG após a incorporação da componente senoidal (verde), comparado ao sinal original sem injeção (azul). A modificação controlada do sinal permite quantificar precisamente a resposta dos métodos de sincronização.}
{eeg_injected_samefreq}

Após a preparação dos sinais com diferentes níveis de injeção, procedemos à análise de sincronização propriamente dita. O primeiro passo metodológico consistiu na extração das fases instantâneas utilizando a Transformada de Hilbert, seguida pela geração do \textit{sinal interferométrico}, uma representação complexa cuja fase corresponde à diferença instantânea entre as fases dos dois sinais analisados. Matematicamente, este sinal é definido como:

\[
z(t) \;=\; \frac{x_{\mathrm{an}}(t)\,\overline{y_{\mathrm{an}}(t)}}{\bigl|x_{\mathrm{an}}(t)\bigr|\;\bigl|y_{\mathrm{an}}(t)\bigr|},
\]

onde \(x_{\mathrm{an}}(t)\) e \(y_{\mathrm{an}}(t)\) representam as representações analíticas (obtidas via Transformada de Hilbert) dos sinais originais, e \(\overline{y_{\mathrm{an}}(t)}\) denota o conjugado complexo de \(y_{\mathrm{an}}(t)\). Esta formulação matemática possui propriedades particularmente valiosas para a análise de sincronização: a normalização pelos módulos \(\bigl|x_{\mathrm{an}}(t)\bigr|\) e \(\bigl|y_{\mathrm{an}}(t)\bigr|\) elimina completamente a dependência de amplitude, concentrando toda a informação relevante na fase \(\phi_z(t) = \phi_x(t) - \phi_y(t)\), que oscila no intervalo de \(-\pi\) a \(+\pi\).

A interpretação desta diferença de fase é direta e intuitiva: valores positivos indicam que o sinal \(x(t)\) está adiantado em relação a \(y(t)\), enquanto valores negativos indicam que \(x(t)\) está atrasado. É importante observar que, mesmo no cenário \textit{same-frequency} com injeção de sinais perfeitamente periódicos, a presença de ruído nos sinais originais, pequenas discrepâncias de frequência ou diferenças nas fases iniciais fazem com que \(\phi_z(t)\) apresente variações ao longo do tempo, refletindo a natureza dinâmica e complexa das interações entre sinais biológicos.

As Figuras~\ref{fig:fases_instantaneas_samefreq} e~\ref{fig:sinal_interferometrico_samefreq} ilustram, respectivamente, as fases instantâneas extraídas dos sinais modificados (EEG e ECG, ambos injetados com a mesma frequência) e o sinal interferométrico resultante da diferença entre estas fases:

\standardfigure{figs/3_2_testing_connectivity_metrics/12_Passo1_Fases_Same-frequency.png}
{Fases instantâneas extraídas dos sinais EEG e ECG no cenário \textit{same-frequency}. A visualização das fases desenroladas permite observar a evolução temporal da relação de fase entre os sinais, revelando padrões de sincronização potencial.}
{fases_instantaneas_samefreq}

\standardfigure{figs/3_2_testing_connectivity_metrics/13_Passo2_Interferometrico_Same-frequency.png}
{Sinal interferométrico resultante da diferença de fase instantânea no cenário \textit{same-frequency}. As flutuações neste sinal refletem a dinâmica temporal da sincronização entre os sinais analisados, com períodos de maior e menor estabilidade na relação de fase.}
{sinal_interferometrico_samefreq}

Uma característica fundamental a ser observada nestas visualizações é que, em caso de sincronização perfeita com diferença de fase absolutamente constante, o sinal \(\phi_z(t)\) apareceria como uma linha horizontal. No entanto, na maioria dos dados reais (e mesmo em simulações com componentes de ruído), a fase varia de forma dinâmica, refletindo a evolução temporal da sincronização e dessincronização entre os sinais. Esta variabilidade temporal na relação de fase constitui precisamente o fenômeno que métricas como CF-PLM, PLI e PLV procuram quantificar, cada uma com abordagens matemáticas distintas, para avaliar a coerência ou o acoplamento entre ritmos fisiológicos.

Para quantificar o grau de sincronização entre os sinais, aplicamos a Transformada de Fourier (FFT) ao sinal interferométrico, obtendo sua densidade espectral de potência (PSD). Esta análise espectral constitui um passo fundamental no cálculo do índice CF-PLM e revela características importantes sobre a natureza do acoplamento entre os sinais. A Figura~\ref{fig:fft_psd_samefreq} ilustra este processo analítico no cenário \textit{same-frequency}, evidenciando um pico proeminente em 0 Hz, característico de sinais com a mesma frequência fundamental e diferença de fase relativamente estável.

\standardfigure{figs/3_2_testing_connectivity_metrics/14_Passo3_FFT_PSD_Same-frequency.png}
{Densidade espectral de potência (PSD) do sinal interferométrico no cenário \textit{same-frequency}, revelando um pico característico em 0 Hz. Este padrão espectral é indicativo de uma diferença de fase relativamente estável entre sinais de mesma frequência, permitindo quantificar o grau de sincronização através da concentração de energia espectral.}
{fft_psd_samefreq}

Para completar nossa avaliação sistemática, implementamos um cenário experimental adicional de particular relevância teórica e metodológica: o cenário \textit{Same-frequency com Phase Lag Zero}. Nesta configuração, os sinais injetados possuem não apenas a mesma frequência, mas também fases perfeitamente alinhadas, simulando uma condição de sincronização absoluta que pode ocorrer em fenômenos como o \textit{volume conduction} em registros de EEG. Este cenário é especialmente valioso para avaliar a capacidade do PLI em discriminar entre sincronizações genuínas e aquelas potencialmente espúrias decorrentes de fontes comuns.

As Figuras~\ref{fig:zerolag_phases_final} e~\ref{fig:zerolag_difference_final} ilustram as características deste cenário especial. Na primeira visualização, observamos as fases desenroladas dos dois sinais praticamente sobrepostas, indicando sincronização quase perfeita. Na segunda figura, a diferença de fase permanece consistentemente próxima a zero ao longo de todo o período analisado, confirmando a ausência de defasagem significativa entre os sinais.

\standardfigure{figs/3_2_testing_connectivity_metrics/15_ZeroLag_Fases_Same-frequency Com Phase Lag Zero.png}
{Fases instantâneas desenroladas no cenário \textit{Same-frequency com Phase Lag Zero}, demonstrando sobreposição quase exata dos sinais. Esta configuração simula condições de sincronização perfeita que podem ocorrer em fenômenos como o \textit{volume conduction}, permitindo avaliar a especificidade dos diferentes métodos de sincronização.}
{zerolag_phases_final}

\standardfigure{figs/3_2_testing_connectivity_metrics/16_ZeroLag_Diferenca_Fase_Same-frequency Com Phase Lag Zero.png}
{Diferença de fase no cenário \textit{Same-frequency com Phase Lag Zero}, mantendo-se consistentemente próxima a zero ao longo do tempo. Este padrão é característico de sincronizações potencialmente espúrias decorrentes de fontes comuns, representando um desafio metodológico importante para métricas de sincronização.}
{zerolag_difference_final}

A análise comparativa sistemática dos três índices de sincronização (CF-PLM, PLV e PLI) nos diferentes cenários experimentais e níveis de injeção é sintetizada na Figura~\ref{fig:comparativo_metricas}. Esta visualização integrada permite avaliar simultaneamente a sensibilidade e a especificidade de cada método em função da intensidade do acoplamento e do tipo de sincronização, oferecendo uma base empírica robusta para a seleção das métricas mais adequadas para cada aplicação específica.

\standardfigure{figs/3_2_testing_connectivity_metrics/17_Comparativo_Subplots_Experimentos.png}
{Análise comparativa integrada dos índices CF-PLM, PLV e PLI nos três cenários experimentais (\textit{Cross-frequency}, \textit{Same-frequency} e \textit{Same-frequency com Phase Lag Zero}), em função da porcentagem de injeção aplicada. Os padrões de resposta distintos revelam as características específicas de cada método em termos de sensibilidade e especificidade para diferentes tipos de acoplamento fase-fase.}
{comparativo_metricas}

Os resultados desta validação experimental fornecem evidências empíricas robustas que fundamentam nossa escolha metodológica: o CF-PLM demonstrou excelente sensibilidade para detectar acoplamentos \textit{cross-frequency} entre EEG e ECG, respondendo de forma consistente e proporcional à intensidade da sincronização; o PLI, por sua vez, mostrou-se particularmente adequado para quantificar sincronizações em frequências iguais, com a vantagem crucial de discriminar eficientemente entre acoplamentos genuínos e aqueles potencialmente espúrios decorrentes de fenômenos como o \textit{volume conduction} (phase lag zero). O PLV, embora sensível a todos os tipos de sincronização, não demonstrou a mesma especificidade do PLI para discriminar acoplamentos triviais, justificando sua inclusão apenas como referência complementar para fins comparativos.

\section{Análise de Conectividade ao Longo do Tempo}
\label{sec:connectivity_over_time}
A caracterização da dinâmica temporal da sincronização neural representa um aspecto fundamental para compreender a estabilidade e a evolução dos padrões de conectividade funcional durante estados fisiológicos específicos. Para investigar sistematicamente esta dimensão temporal em nosso estudo, implementamos uma abordagem metodológica baseada na segmentação dos sinais em janelas temporais consecutivas, permitindo avaliar a evolução da sincronização ao longo de toda a sessão experimental.

Os registros contínuos de EEG e ECG, com duração total de 4 minutos e 30 segundos para cada condição experimental, foram segmentados em janelas consecutivas de 10 segundos. A escolha deste tamanho específico de janela foi fundamentada em evidências empíricas recentes, como as apresentadas por \citeonline{didaci2024how}, que demonstraram que o tamanho da janela temporal influencia significativamente a performance e a confiabilidade das métricas de conectividade em análises de EEG. Estes estudos indicam que janelas entre 8 e 12 segundos oferecem um equilíbrio ideal entre precisão estatística e estabilidade dos dados, minimizando tanto o ruído associado a janelas excessivamente curtas quanto a variabilidade indesejada introduzida por janelas demasiadamente longas.

Para cada janela temporal, calculamos exaustivamente as medidas de sincronização para todas as combinações relevantes de variáveis experimentais: cada par de canais (EEG-EEG para análises iso-frequenciais e EEG-ECG para análises \textit{cross-frequency}), cada banda de frequência (delta, theta, alpha, beta e gamma), cada condição experimental (\textit{cathodic} e \textit{sham}), e cada participante. Esta abordagem multidimensional gerou um conjunto de dados de alta granularidade, permitindo caracterizar a conectividade funcional com resolução temporal precisa ao longo de toda a sessão experimental.

Para sintetizar esta rica informação multidimensional de forma estatisticamente robusta, adotamos a mediana como medida de tendência central para cada janela temporal. Esta escolha metodológica foi motivada pela conhecida resistência da mediana a valores extremos e \textit{outliers}, garantindo que nossa caracterização da dinâmica temporal não fosse indevidamente influenciada por flutuações transitórias ou artefatos pontuais nos dados. Assim, para cada janela de 10 segundos (\textit{time\_window}), agregamos todos os valores de conectividade calculados para cada configuração específica (todos os pares de canais EEG-EEG ou EEG-ECG, dentro de cada banda de frequência e condição experimental), extraindo a mediana como representação estatisticamente robusta do estado de sincronização naquele intervalo temporal.

Esta abordagem metodológica oferece duas vantagens analíticas complementares: (1) permite visualizar a evolução temporal da sincronização ao longo da sessão experimental, revelando potenciais padrões dinâmicos ou tendências progressivas; e (2) possibilita avaliar a estabilidade temporal dos diferentes índices de sincronização, um aspecto particularmente relevante no contexto de registros em \textit{resting-state}, onde esperamos relativa estabilidade nos padrões de conectividade funcional. Esta segunda vantagem proporcionou um importante controle de qualidade metodológico, aumentando nossa confiança na robustez dos métodos empregados, uma vez que a observação de estabilidade temporal em condições de repouso alinha-se com as expectativas teóricas para este tipo de registro.

As Figuras~\ref{fig:cfplm_time_cat}, \ref{fig:pli_time_cat} e \ref{fig:plv_time_cat} apresentam as séries temporais obtidas para as três métricas principais empregadas neste estudo, cada uma capturando aspectos complementares da sincronização neural:

\begin{itemize}
    \item \textbf{CF-PLM (EEG-ECG):} A Figura~\ref{fig:cfplm_time_cat} ilustra a evolução temporal da mediana do CF-PLM para a condição \textit{cathodic}, quantificando a sincronização \textit{cross-frequency} entre os ritmos cerebrais (EEG) e cardiovasculares (ECG). Esta visualização permite avaliar a estabilidade do acoplamento fase-fase entre sistemas fisiológicos distintos operando em diferentes faixas de frequência.

    \item \textbf{PLI (EEG-EEG):} A Figura~\ref{fig:pli_time_cat} apresenta a dinâmica temporal da mediana do PLI para a condição \textit{cathodic}, caracterizando a sincronização iso-frequencial entre diferentes regiões cerebrais. Esta métrica, resistente a efeitos de \textit{volume conduction}, revela padrões de conectividade funcional genuína entre áreas corticais distintas.

    \item \textbf{PLV (EEG-EEG):} A Figura~\ref{fig:plv_time_cat} exibe a evolução temporal da mediana do PLV para a condição \textit{cathodic}, incluída como referência comparativa. Embora mais sensível a ruídos e efeitos de \textit{volume conduction}, o PLV oferece uma perspectiva complementar que, quando interpretada em conjunto com o PLI, enriquece a compreensão dos padrões de sincronização cerebral.
\end{itemize}

Cada ponto nestas séries temporais representa a mediana dos valores de conectividade calculados para todos os pares de canais relevantes dentro de uma janela específica de 10 segundos, para cada banda de frequência e condição experimental (pré-sham, pós-sham, pré-\textit{cathodic} ou pós-\textit{cathodic}). Esta representação estatisticamente robusta minimiza o impacto de variações pontuais e \textit{outliers}, oferecendo uma caracterização confiável da dinâmica temporal da sincronização neural ao longo da sessão experimental.

\standardfigure{figs/4_connectivity_over_time/Mediana_do_CF-PLM_ao_longo_do_tempo_(EEG_ECG)_Catódica.png}
{Evolução temporal da mediana do CF-PLM para a condição \textit{cathodic} (EEG-ECG). Cada ponto representa a mediana dos valores de sincronização calculados em janelas consecutivas de 10 segundos, revelando a dinâmica do acoplamento \textit{cross-frequency} entre ritmos cerebrais e cardiovasculares ao longo da sessão experimental. A relativa estabilidade observada é consistente com o contexto de registro em \textit{resting-state}.}
{cfplm_time_cat}

\standardfigure{figs/4_connectivity_over_time/Mediana_do_PLI_ao_longo_do_tempo_(EEG_EEG)_Catódica.png}
{Dinâmica temporal da mediana do PLI para a condição \textit{cathodic} (EEG-EEG). O gráfico ilustra a evolução da sincronização iso-frequencial entre diferentes regiões cerebrais ao longo da gravação, com padrões distintos para cada banda de frequência. A métrica PLI, resistente a efeitos de \textit{volume conduction}, revela padrões de conectividade funcional genuína entre áreas corticais.}
{pli_time_cat}

\standardfigure{figs/4_connectivity_over_time/Mediana_do_PLV_ao_longo_do_tempo_(EEG_EEG)_Catódica.png}
{Evolução temporal da mediana do PLV para a condição \textit{cathodic} (EEG-EEG). Incluído como referência comparativa, o PLV apresenta padrões temporais que, quando contrastados com o PLI, permitem inferências sobre a natureza das sincronizações detectadas. Valores consistentemente mais elevados do PLV em comparação ao PLI podem sugerir contribuições de \textit{volume conduction} para a sincronização aparente.}
{plv_time_cat}