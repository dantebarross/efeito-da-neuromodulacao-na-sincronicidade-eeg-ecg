\chapter{Métodos de Análise de Sincronização de Fase}
\label{chap:6_metodos_de_analise_de_sincronizacao_de_fase}

Neste capítulo, apresentamos os fundamentos teóricos e práticos dos métodos empregados para analisar a sincronização de fase entre sinais fisiológicos. Para este estudo, optamos por utilizar o Phase Lag Index (PLI) para quantificar a sincronização entre canais de EEG (mesma frequência) e o Cross-Frequency Phase Linearity Measurement (CF-PLM) para avaliar o acoplamento cross-frequency entre EEG e ECG. Adicionalmente, o tradicional Phase Locking Value (PLV) foi testado para comparação, cujos resultados encontram-se disponíveis no anexo.

\section{Fundamentos dos Métodos de Análise de Sincronização}

A conectividade funcional pode ser explorada sob a ótica da interação contínua entre diferentes regiões cerebrais. Conforme descrito em \cite{sorrentino2022detection}, diversas áreas precisam transferir informações constantemente para suportar respostas comportamentais complexas. Embora existam várias métricas para detectar interações \textit{cross-frequency}, a maioria é baseada em acoplamentos \textit{phase-amplitude}.

Contudo, a comunicação neural envolve mecanismos de feedback que exigem relações temporais extremamente precisas entre os neurônios. Um desses mecanismos, denominado \emph{reentry}, amplifica sinais de feedback ao impor que os neurônios de uma assembleia sincronicem suas taxas de disparo, garantindo que o timing exato entre esses disparos seja captado pelas técnicas de análise de fase \cite{seraj2018cerebral, ren2022multi}. Em contraste com a amplitude, que pode ser fortemente contaminada por ruídos, variações na impedância do crânio e artefatos decorrentes de movimentos oculares ou musculares, a fase dos sinais apresenta uma qualidade “pura” e menos suscetível a tais interferências. Por essa razão, os métodos que se baseiam exclusivamente na análise fase-fase – como o Phase Lag Index (PLI), o Phase Locking Value (PLV) e o Cross-Frequency Phase Linearity Measurement (CF-PLM) – são particularmente indicados para investigar a comunicação e a integração entre diferentes áreas cerebrais.

A análise de sincronização de fase visa quantificar a consistência da diferença de fase entre dois sinais ao longo do tempo. Diversas técnicas foram desenvolvidas para mensurar essa sincronização, abrangendo abordagens baseadas em modelos e métodos \textit{data-driven} \cite{seraj2018cerebral}. A sincronização de fase é definida como uma relação estável entre as fases de dois sinais dentro de uma determinada faixa de frequência, independentemente de suas amplitudes estarem correlacionadas \cite{seraj2018cerebral}. Esse conceito tem sido amplamente empregado na investigação da conectividade cerebral, pois fornece informações robustas sobre os padrões de comunicação neural, mesmo na presença de ruído ou variações na amplitude.

Para extrair a fase dos sinais, técnicas como a Transformada de Hilbert são amplamente empregadas. Alternativamente, a Transformada de Fourier e os wavelets (por exemplo, o wavelet de Morlet) oferecem uma decomposição tempo-frequencial dos sinais. Embora a Transformada de Fourier seja eficaz para sinais estacionários, sua interpretação torna-se mais complexa quando o conteúdo em frequência varia com o tempo \cite{singh2024evaluating}. Este estudo focou na análise de sincronização de fase utilizando PLI, PLV e CF-PLM, sendo os dois últimos os principais métodos aplicados.

Embora abordagens baseadas em modelos estatísticos, como as propostas por Nadalin et al. \cite{nadalin2019statistical}, sejam úteis para avaliar acoplamentos \textit{phase-amplitude} e amplitude-amplitude, elas podem não capturar completamente as relações dinâmicas entre as fases dos sinais. A dependência de pressuposições estatísticas e o rigor no controle de covariáveis podem limitar a identificação de padrões de sincronização mais complexos, especialmente em dados de estado de repouso, onde os acoplamentos fase-fase entre diferentes bandas de EEG e ECG tendem a ser mais variáveis e não lineares.

Outra vantagem das análises modernas é o uso de gravações multicanais, que possibilitam identificar padrões fisiologicamente interpretáveis de acoplamento entre frequências. Técnicas de redução de dimensionalidade e separação de fontes, conforme descrito em \cite{cohen2017multivariate}, permitem isolar padrões de atividade que seriam difíceis de detectar em sinais monofacetados. Ademais, o framework de decomposição generalizada (gedCFC), apresentado por Cohen (2017) \cite{cohen2017multivariate}, se propõe a superar limitações dos métodos tradicionais, especialmente quando se trata de sinais não estacionários.

Entretanto, embora o gedCFC ofereça amplas possibilidades para a extração de componentes e descoberta de padrões de CFC, sua grande flexibilidade implica uma multiplicidade de configurações e parâmetros, o que pode torná-lo impraticável para abordagens exploratórias sem hipóteses bem refinadas~\cite{cohen2017multivariate}. Diante disso, optamos pelo CF-PLM, método que, apesar de ser menos abrangente que o gedCFC, apresenta uma implementação mais direta, simples e adequada para nossa análise exploratória de sincronização entre ECG e múltiplos canais de EEG.

Essa escolha é fundamentada na crescente evidência de que a fase desempenha um papel essencial na organização e transmissão da informação neural. Estudos como os de Canolty e Knight~\cite{canolty2010functional} demonstram que a fase de oscilações de baixa frequência organiza temporalmente a atividade rápida, criando janelas para a integração de informação. Na mesma linha, Hyafil et al.~\cite{hyafil2015neural} reforçam que padrões específicos de alinhamento fase-frequência (e.g., theta-gamma) possuem funções cognitivas especializadas, como o parsing de fala, enquanto Lisman \& Jensen~\cite{lisman2013theta} propõem que tais acoplamentos organizam sequências, com cada ciclo gamma aninhado em um ciclo theta representando um item distinto. Essa ênfase na fase também aparece no modelo Communication Through Coherence (CTC) de Fries (2015), que destaca que a sincronização de fase entre regiões cerebrais é decisiva para uma transmissão eficaz de informação. Empiricamente, estudos como os de Siegel et al.~\cite{siegel2012spectral} evidenciam que padrões estáveis de coerência de fase entre regiões são características marcantes das interações de larga escala que sustentam processos cognitivos, indicando que a organização temporal mediada pela fase é crucial para integrar a atividade em frequências mais altas. Assim, a literatura converge ao apontar que a fase, e não apenas a amplitude, desempenha um papel essencial na codificação e transferência de informação no cérebro.

Ren et al.~\cite{ren2022multi} revisaram o uso do acoplamento de fase (\textit{phase-phase coupling}) na montagem de redes cerebrais, demonstrando que abordagens multi-granulares podem revelar padrões robustos de sincronização \textit{cross-frequency}. Diversos estudos têm mostrado que a interação entre diferentes bandas de frequência --- conhecida como \textit{cross-frequency coupling} (CFC) --- ocorre em regiões como o hipocampo, o córtex pré-frontal e o sensorial, tanto em humanos quanto em primatas não-humanos~\cite{mormann2005phase, canolty2006high, jensen2007cross, khamechian2020decoding}. Pesquisas de Dimitriadis et al.~\cite{dimitriadis2015cognitive} e Davoudi et al.~\cite{davoudi2021prefrontal} exploraram o acoplamento entre bandas theta e alpha, bem como mecanismos de acoplamento alpha-gamma durante tarefas cognitivas. No contexto de pacientes com AVC, estudos focados na análise do CFC durante movimentos ou tarefas executivas evidenciam a necessidade de investigar processos de imaginação motora para capturar interações mais amplas entre regiões cerebrais.

Além disso, estudos como os de Chen et al.~\cite{chen2023multiple} demonstraram que mudanças nos padrões de acoplamento cross-frequency podem atuar como biomarcadores relevantes em contextos neurofisiológicos. Essas alterações refletem ajustes dinâmicos na comunicação neural entre regiões cerebrais e podem indicar estados funcionais específicos, como processos atencionais, cognitivos ou até mesmo estados patológicos. Assim, a análise do CFC tem sido amplamente empregada para compreender melhor os mecanismos subjacentes à organização funcional do cérebro.

Entretanto, extrair características efetivas do CFC é mais desafiador do que analisar o acoplamento intrafrequencial, pois os dados \textit{cross-frequency} são mais complexos e contêm informações ocultas que requerem análises profundas para elucidar os mecanismos fisiológicos subjacentes~\cite{ren2022multi}. Essa dificuldade ressalta a importância do desenvolvimento de métodos robustos para explorar os padrões de sincronização \textit{cross-frequency}. Apesar da ampla variedade de métricas existentes para acoplamento \textit{phase-amplitude}, há uma notável escassez de métodos que investiguem especificamente o acoplamento fase-fase nessas interações. O CF-PLM é um dos poucos métodos que se propõe a capturar essa forma de sincronização, oferecendo uma alternativa mais direcionada para a análise da conectividade neural baseada em fase.

\subsection{Phase Lag Index (PLI)}

O \textbf{Phase Lag Index (PLI)} é um índice amplamente utilizado para medir a sincronização de fase entre sinais operando na mesma faixa de frequência, como os canais de EEG dentro de uma mesma banda. Ao contrário do \textbf{Phase Locking Value (PLV)}, o PLI reduz fortemente os efeitos de condução de volume, pois considera exclusivamente o sinal das diferenças de fase, ignorando completamente valores próximos a zero que frequentemente indicam sincronizações espúrias causadas por fontes comuns ou artefatos \cite{seraj2018cerebral}.

Segundo \citeauthor{seraj2018cerebral} (\citeyear{seraj2018cerebral}), a fórmula do PLI é dada por:

\[
\text{PLI} = |\langle \text{sign}(\Delta\phi(t)) \rangle|
\]

em que \(\Delta\phi(t)\) é a diferença instantânea entre as fases de dois sinais.

Uma variação do PLI, o \textbf{weighted Phase Lag Index (wPLI)}, introduz pesos às diferenças de fase com base na magnitude da parte imaginária dessas diferenças, sendo formulado da seguinte forma \cite{seraj2018cerebral}:

\[
\text{wPLI} = \frac{|\langle \Im(Z) \rangle|}{\langle |\Im(Z)| \rangle}
\]

onde \(Z\) representa o sinal espectral cruzado complexo entre dois sinais.

Embora o wPLI seja uma versão refinada que pode aumentar a sensibilidade para detectar acoplamentos funcionais genuínos em ambientes altamente contaminados por ruído, optamos por utilizar o PLI tradicional no presente estudo. Essa escolha se justifica pela maior rigorosidade do método ao descartar totalmente interações com diferenças de fase próximas a zero, reduzindo ainda mais o risco de detectar sincronizações espúrias. Além disso, dada a característica dos nossos dados, em que os efeitos de ruído e \textit{volume conduction} são moderados, a abordagem tradicional do PLI oferece simplicidade interpretativa sem introduzir complexidade adicional que surgiria com a ponderação realizada pelo wPLI.

Em resumo, o PLI calcula a média do sinal da parte imaginária (utilizando a função \texttt{sign}) e varia entre 0 e 1, com 0 indicando ausência de acoplamento efetivo e 1 indicando um acoplamento de fase consistente.

\subsection{Cross-Frequency Phase Linearity Measurement (CF-PLM)}

O \textbf{CF-PLM} é um método desenvolvido para analisar a sincronização \textit{fase-fase} entre sinais que operam em diferentes faixas de frequência, permitindo a detecção do acoplamento \textit{cross-frequency} baseado na relação entre suas fases. Essa abordagem é especialmente útil para avaliar a relação entre as diferentes bandas de frequência dos ritmos neurais registrados pelo EEG e sinais fisiológicos como o ritmo cardíaco obtido pelo ECG, cujas frequências típicas ficam próximas a 1 Hz.

De acordo com \citeauthor{sorrentino2022detection} (\citeyear{sorrentino2022detection}), o método expande o conceito de \textit{Phase Linearity Measurement} (PLM) para sincronizações de ordem \(n:m\), onde \(n\) ciclos de um sinal em uma determinada frequência se alinham consistentemente com \(m\) ciclos de outro sinal em uma frequência distinta. Essa abordagem elimina a necessidade de pressuposições sobre relações harmônicas fixas entre os sinais, permitindo a detecção de padrões de acoplamento mais flexíveis e adaptáveis. Outras abordagens avançadas, como o framework de decomposição generalizada (gedCFC) proposto por Cohen (\citeyear{cohen2017multivariate}), também têm sido empregadas para identificar acoplamentos fase-fase mesmo em sinais multicanais complexos e não estacionários.

O procedimento do CF-PLM envolve quatro etapas principais:

\begin{enumerate}
    \item \textbf{Cálculo dos sinais analíticos}: Para os sinais \(x(t)\) e \(y(t)\), obtém-se suas representações analíticas \(x_{\mathrm{an}}(t)\) e \(y_{\mathrm{an}}(t)\) por meio da Transformada de Hilbert. Essas representações fornecem, respectivamente, as fases instantâneas \(\phi_x(t)\) e \(\phi_y(t)\).

    \item \textbf{Construção do sinal interferométrico}: Calcula-se o sinal interferométrico \(z(t)\), definido como:
    \[
    z(t) = \frac{x_{\mathrm{an}}(t)\, y_{\mathrm{an}}^*(t)}{\lvert x_{\mathrm{an}}(t)\rvert\, \lvert y_{\mathrm{an}}(t)\rvert} = e^{i\Delta \phi(t)},
    \]
    em que \(\Delta \phi(t) = \phi_x(t) - \phi_y(t)\) é a diferença de fase instantânea entre os sinais, e a amplitude unitária de \(z(t)\) garante que somente a informação de fase seja analisada.

    \item \textbf{Análise da densidade espectral de potência (PSD)}: Utilizando a Transformada de Fourier — também discutida em \cite{seraj2018cerebral} — obtém-se a PSD do sinal interferométrico \(z(t)\). Em cenários de acoplamento iso-frequencial, o pico da PSD está centrado em \(f = 0\), enquanto em casos de acoplamento \textit{cross-frequency}, o pico se desloca correspondendo à diferença entre frequências dominantes dos sinais.

    \item \textbf{Cálculo do índice CF-PLM}: O índice é obtido integrando a PSD em torno do pico identificado, na faixa estreita \([f_\Delta - B, f_\Delta + B]\), e normalizando pelo poder total da PSD:
    \[
    \text{CF-PLM} = \frac{\displaystyle\int_{f_\Delta - B}^{f_\Delta + B} S_Z(f) \, df}{\displaystyle\int_{-\infty}^{+\infty} S_Z(f) \, df}.
    \]
\end{enumerate}

Cabe destacar que o CF-PLM, diferentemente do PLI, não implementa nenhuma etapa específica para descartar sincronizações próximas de zero (\textit{zero-lag}), que poderiam decorrer de efeitos como o \textit{volume conduction}. Portanto, o CF-PLM por si só não oferece proteção explícita contra essas interferências, o que precisa ser levado em consideração durante sua aplicação e interpretação.

Em nosso contexto específico — análise exploratória de sincronização entre sinais de EEG e ECG obtidos por equipamentos distintos —, enfrentamos uma limitação técnica relevante quanto à precisão temporal absoluta das gravações. Apesar de possuirmos marcadores claros para o início das sessões de coleta, não é possível garantir uma precisão milimétrica na sincronização absoluta das séries temporais desses dois sinais. Entretanto, tal limitação não afeta negativamente a aplicabilidade ou validade do CF-PLM em nosso estudo, uma vez que o método não depende da precisão absoluta dos tempos iniciais dos sinais. O CF-PLM mede, essencialmente, a \textit{consistência temporal} da diferença de fase ao longo do período analisado. Portanto, mesmo que exista um atraso fixo ou variável entre as séries EEG e ECG devido à sincronização técnica imperfeita, desde que esse atraso permaneça relativamente estável, o método será capaz de capturar efetivamente a presença e magnitude da sincronização \textit{cross-frequency} genuína.

Dessa forma, o uso do CF-PLM é justificado e adequado às nossas condições experimentais e aos objetivos analíticos propostos, sendo a nossa base metodológica para avaliar a interação cortical (via EEG) e cardiovascular (via ECG) em condições \textit{resting-state}.


\subsection{Comparação com o Phase Locking Value (PLV)}

Para fins de validação e comparação, também testamos o PLV, um método tradicional amplamente utilizado para a análise de sincronização de fase. Conforme descrito em \cite{seraj2018cerebral}, o PLV mede a consistência da diferença de fase entre dois sinais na mesma faixa de frequência; entretanto, ele é sensível a ruídos e aos efeitos de volume conduction, o que pode levar à detecção de sincronizações espúrias. Por exemplo, estudos como o de \cite{abubaker2021working} investigaram o acoplamento cruzado entre frequências em tarefas de memória de trabalho utilizando o PLV, sugerindo que padrões específicos de sincronização podem estar associados ao desempenho cognitivo. Contudo, a ausência de técnicas avançadas para mitigar artefatos pode limitar sua aplicabilidade em contextos clínicos.

Adicionalmente, Zhang et al. (2014) \cite{zhang2014phase} investigaram a sincronização de fase durante tarefas cognitivas prolongadas e constataram que, na faixa beta (13--30 Hz), os valores médios de PLV diminuem significativamente tanto em pares inter-hemisféricos (especialmente nas regiões central e parietal) quanto em pares intra-hemisféricos (como entre os eletrodos frontal-parietal, central-parietal e frontal-central). Esses achados ilustram como a sensibilidade do PLV pode ser afetada por estados de fadiga mental, ressaltando suas limitações diante de variações no processamento cognitivo.

É importante ainda considerar que a Transformada de Hilbert foi utilizada neste estudo para extrair a fase instantânea dos sinais, permitindo o cálculo do PLV, PLI e CF-PLM. No caso do CF-PLM, adicionalmente empregamos a Transformada de Fourier (FFT) para analisar a densidade espectral de potência do sinal interferométrico, garantindo a identificação dos acoplamentos \textit{cross-frequency}. Embora os wavelets, como o de Morlet, ofereçam uma alternativa para decomposição tempo-frequencial, sua aplicação não foi necessária neste contexto, uma vez que a Hilbert e a FFT já fornecem as informações requeridas.

Além disso, conforme relatado em \cite{sorrentino2022detection}, as áreas cerebrais precisam transferir informações constantemente para suportar respostas comportamentais complexas, e diversas métricas foram propostas para quantificar essa comunicação. Entre elas, a análise de acoplamento \textit{cross-frequency} é essencial para entender como diferentes ritmos interagem. Vários estudos \cite{hulsemann2019quantification} destacam que métodos como acoplamento \textit{phase-amplitude}, bicoerência e \textit{phase-locking} apresentam vantagens e desafios específicos, o que exige uma escolha cuidadosa do método baseado no tipo de dados e na hipótese experimental. Por fim, abordagens que utilizam registros multicanais, como a decomposição por autovalores generalizada (gedCFC) \cite{cohen2017where}, ampliam as possibilidades de identificação de padrões de conectividade, reforçando evidências de que a fase pode codificar mais informações do que a amplitude.

Essas considerações ressaltam a necessidade de selecionar cuidadosamente os métodos utilizados, levando em conta suas vantagens e limitações. A seguir, apresentamos os critérios que embasaram a escolha dos métodos adotados neste estudo.

\subsection{Escolha e Justificativa dos Métodos}

A escolha dos métodos empregados neste estudo foi fundamentada em critérios que garantem a adequação à análise de sincronização de fase em sinais neurofisiológicos.

Métodos adaptativos, como os propostos por Van Zaen et al.~\cite{vanzaen2013adaptive}, foram investigados para aprimorar a detecção de acoplamentos cruzados em sinais não estacionários.

Adicionalmente, estudos como o de Abubaker et al.~\cite{abubaker2021working} investigaram o acoplamento cruzado entre frequências em tarefas de memória de trabalho utilizando o PLV, sugerindo que padrões específicos de sincronização podem estar associados ao desempenho cognitivo. Contudo, a sensibilidade do PLV a ruídos e efeitos de volume conduction pode limitar sua aplicabilidade em alguns contextos. Por essa razão, o PLI foi escolhido como uma alternativa mais robusta para medir acoplamentos na mesma frequência, pois reduz a influência de sincronizações espúrias~\cite{seraj2018cerebral, zhang2014phase}.

A análise do acoplamento \textit{cross-frequency} foi uma preocupação central deste estudo, especialmente devido à sua importância para a compreensão da comunicação neural em diferentes escalas temporais e funcionais. Diferentes métricas foram propostas para quantificar essa comunicação, como o acoplamento \textit{phase-amplitude}, a bicoerência e o \textit{phase-locking}~\cite{hulsemann2019quantification}. No entanto, há uma notável escassez de métodos que investiguem especificamente o acoplamento fase-fase em interações \textit{cross-frequency}. O CF-PLM se destaca nesse aspecto por permitir a detecção de padrões de sincronização sem assumir relações harmônicas fixas entre as frequências envolvidas, sendo mais adequado para os sinais estudados~\cite{sorrentino2022detection, seraj2018cerebral, chen2023multiple}.

Além disso, abordagens mais avançadas, como a decomposição por autovalores generalizada (\textit{Generalized Eigenvalue Decomposition} - gedCFC)~\cite{cohen2017where}, foram desenvolvidas para explorar a conectividade neural multicanal. Essa técnica amplia a capacidade de identificar padrões de sincronização fase-fase mesmo em sinais altamente não estacionários, reforçando a evidência de que a fase pode codificar mais informações do que a amplitude. Apesar das vantagens do gedCFC, sua flexibilidade implica um aumento na complexidade computacional e na necessidade de parametrização cuidadosa, o que pode torná-lo menos adequado para análises exploratórias sem hipóteses bem refinadas~\cite{cohen2017multivariate}.

Com base nessas considerações, a escolha do CF-PLM e do PLI como principais métricas de análise, complementadas pelo PLV para referência comparativa, foi justificada pelos seguintes aspectos fundamentais:

\begin{itemize}
  \item \textbf{Robustez contra artefatos e volume conduction:} O PLI foi selecionado devido à sua capacidade de minimizar a detecção de sincronizações triviais (\textit{zero-lag}), evitando artefatos induzidos pelo volume conduction~\cite{seraj2018cerebral, zhang2014phase}.
  
  \item \textbf{Capacidade de detectar acoplamento fase-fase cross-frequency:} O CF-PLM foi escolhido por ser um dos poucos métodos que capturam relações flexíveis entre frequências distintas, sem pressupor harmônicos fixos, permitindo uma análise mais realista das interações entre EEG e ECG~\cite{sorrentino2022detection, seraj2018cerebral, chen2023multiple}.

  \item \textbf{Adequação ao tipo de sinal analisado:} Enquanto métodos mais sofisticados, como o gedCFC~\cite{cohen2017where}, oferecem maior flexibilidade para dados multicanais, a simplicidade operacional do CF-PLM e do PLI é mais adequada para uma análise exploratória em resting-state, onde há limitações na sincronização temporal absoluta dos sinais.

  \item \textbf{Exploração de padrões globais de sincronização:} A escolha por métodos tradicionais se justifica pelo fato de que, no cenário de \textit{resting-state}, o objetivo principal é detectar padrões de sincronização persistentes ao longo do tempo, sem necessidade de rastrear variações temporais detalhadas~\cite{vanzaen2013adaptive, seraj2018cerebral}.
\end{itemize}

Assim, a combinação do PLI para sincronização iso-frequencial e do CF-PLM para acoplamento \textit{cross-frequency} constitui a abordagem mais adequada para este estudo. O PLV, por sua vez, foi utilizado como referência comparativa, dada sua ampla adoção em estudos anteriores. Dessa forma, os métodos selecionados proporcionam uma análise robusta e confiável da conectividade funcional entre EEG e ECG.


\section{Validação Experimental com Injeção de Sinais}

Para validar os métodos utilizados neste estudo, foram realizados experimentos com injeção controlada de sinais senoidais sobre dados reais de ECG e EEG, coletados durante as sessões experimentais. O objetivo desta validação foi verificar a capacidade dos índices CF-PLM, PLV e PLI em identificar corretamente diferentes níveis e tipos de sincronização de fase artificialmente introduzidos.

A técnica empregada consistiu nas seguintes etapas:
\begin{enumerate}
    \item Seleção de segmentos representativos dos sinais originais de ECG e EEG.
    \item Geração de sinais senoidais com frequências e fases específicas utilizando o modelo de Kuramoto.
    \item Aplicação controlada dos sinais senoidais sobre os sinais originais por meio de máscaras de injeção, variando a porcentagem de contribuição (0\%, 25\%, 50\%, 75\% e 100\%) dos sinais injetados.
    \item Cálculo dos índices CF-PLM, PLV e PLI sobre os sinais modificados para avaliar o desempenho dos métodos de sincronização em diferentes cenários.
\end{enumerate}

Foram conduzidos três cenários principais:
\begin{itemize}
    \item \textbf{Cross-frequency:} Destinado a testar especialmente o índice CF-PLM (contrastado com PLV e PLI) em situações onde o acoplamento ocorre entre frequências distintas.
    \item \textbf{Same-frequency com defasagem:} Avalia a sensibilidade e desempenho de todos os índices quando os sinais possuem a mesma frequência, mas apresentam uma defasagem de fase fixa configurada (\(\pi/4\)).
    \item \textbf{Same-frequency com phase lag zero:} Cenário idealizado para demonstrar a robustez do PLI contra sincronizações triviais decorrentes de volume conduction.
\end{itemize}

Exemplos dos sinais antes e após a injeção no cenário Cross-frequency são mostrados nas Figuras~\ref{fig:ecg_injection} e~\ref{fig:eeg_injection}, que ilustram a adição de sinais artificiais com frequências distintas sobre o sinal original.

\standardfigure{figs/3_2_testing_connectivity_metrics/1_ECG_Original_vs_Injecao_Cross-frequency.png}
{ECG: comparação entre o sinal original e o sinal senoidal injetado (1~Hz), cenário Cross-frequency.}
{ecg_injection}

\standardfigure{figs/3_2_testing_connectivity_metrics/3_EEG_Original_vs_Injecao_Cross-frequency.png}
{EEG: comparação entre o sinal original e o sinal senoidal injetado (40~Hz), cenário Cross-frequency.}
{eeg_injection}


No cenário Same-frequency, onde tanto o ECG quanto o EEG recebem sinais senoidais com a mesma frequência (10~Hz) mas com uma defasagem de \(\pi/4\), os resultados são ilustrados nas Figuras~\ref{fig:eeg_original_vs_injection_samefreq} e~\ref{fig:eeg_injected_samefreq}. Esses gráficos evidenciam a interferência gerada pela injeção controlada.

\standardfigure{figs/3_2_testing_connectivity_metrics/10_EEG_Original_vs_Injecao_Same-frequency.png}
{EEG original (azul) e sinal de injeção de 10 Hz (vermelho) com pequena defasagem (\(\pi/4\)).}
{eeg_original_vs_injection_samefreq}

\standardfigure{figs/3_2_testing_connectivity_metrics/11_EEG_Injetado_Same-frequency.png}
{Sinal de EEG após a injeção controlada de uma senóide (verde), comparado ao sinal original sem injeção (azul).}
{eeg_injected_samefreq}


Em seguida, extraímos as fases instantâneas utilizando a Transformada de Hilbert e geramos o sinal interferométrico, conforme exemplificado nas Figuras~\ref{fig:fases_instantaneas_samefreq} e~\ref{fig:sinal_interferometrico_samefreq}.

\standardfigure{figs/3_2_testing_connectivity_metrics/12_Passo1_Fases_Same-frequency.png}
{Fases instantâneas extraídas dos sinais EEG e ECG injetados (same-frequency).}
{fases_instantaneas_samefreq}

\standardfigure{figs/3_2_testing_connectivity_metrics/13_Passo2_Interferometrico_Same-frequency.png}
{Sinal interferométrico gerado pela diferença de fase instantânea (cenário Same-frequency).}
{sinal_interferometrico_samefreq}


A seguir, calculamos o índice CF-PLM utilizando a Transformada de Fourier (FFT) sobre o sinal interferométrico. A Figura~\ref{fig:fft_psd_samefreq} exemplifica a análise, mostrando o pico em 0 Hz, que ocorre devido ao offset constante de fase entre os sinais de mesma frequência.

\standardfigure{figs/3_2_testing_connectivity_metrics/14_Passo3_FFT_PSD_Same-frequency.png}
{PSD do sinal interferométrico, indicando o pico em 0 Hz devido ao offset constante de fase entre os sinais de mesma frequência.}
{fft_psd_samefreq}

Por fim, analisamos o cenário especial "Same-frequency com Phase Lag Zero", onde as fases dos sinais estão perfeitamente sincronizadas. As Figuras~\ref{fig:zerolag_phases_final} e~\ref{fig:zerolag_difference_final} demonstram que, nesse caso, as fases desenroladas aparecem sobrepostas, indicando sincronização completa. Além disso, como não há diferença entre as fases, a linha correspondente à diferença de fase permanece em zero ao longo do tempo.

\standardfigure{figs/3_2_testing_connectivity_metrics/15_ZeroLag_Fases_Same-frequency Com Phase Lag Zero.png}
{Fases desenroladas em cenário sem defasagem (10 Hz), com sobreposição quase exata dos sinais.}
{zerolag_phases_final}

\standardfigure{figs/3_2_testing_connectivity_metrics/16_ZeroLag_Diferenca_Fase_Same-frequency Com Phase Lag Zero.png}
{Diferença de fase próxima a zero, indicando ausência completa de defasagem no cenário de Phase Lag Zero.}
{zerolag_difference_final}


O comportamento geral dos índices CF-PLM, PLV e PLI nos diferentes cenários e níveis de injeção é resumido na Figura~\ref{fig:comparativo_metricas}.

\standardfigure{figs/3_2_testing_connectivity_metrics/17_Comparativo_Subplots_Experimentos.png}
{Comparação dos índices CF-PLM, PLV e PLI nos três cenários estudados, em função da porcentagem de injeção aplicada.}
{comparativo_metricas}

Esses resultados justificam a escolha do CF-PLM para a análise de sincronização entre EEG e ECG (cross-frequency) e do PLI para sincronização em frequências iguais, evitando a detecção de acoplamentos triviais, como os decorrentes de volume conduction (phase lag zero). O PLV é reportado apenas para referência complementar, dada sua elevada sensibilidade em condições triviais.

\section{Análise de Conectividade ao Longo do Tempo}
\label{sec:connectivity_over_time}

Para investigar a dinâmica da sincronização ao longo da sessão experimental, os sinais foram segmentados em janelas de 10 segundos, considerando a gravação total de 4 minutos e 30 segundos de cada coleta. Em cada janela, foi calculada uma medida de sincronicidade (PLI, PLV ou CF-PLM) para cada par de canais, para cada banda de frequência, para cada condição (cathodic e sham) e para cada atleta. Em seguida, extraiu-se a mediana desses valores, fornecendo uma medida robusta da conectividade ao longo do tempo para cada configuração. Estudos recentes, como os de Didaci et al.~\cite{didaci2024how}, demonstraram que o tamanho da janela influencia significativamente a performance das métricas de conectividade em sistemas biométricos baseados em EEG, indicando que janelas entre 8 e 12 segundos oferecem um equilíbrio ideal entre precisão e estabilidade dos dados. Essa evidência reforça a estratégia adotada neste estudo.

Essa abordagem permite visualizar a evolução temporal da sincronização e avaliar a estabilidade dos diferentes índices ao longo do tempo. Para cada janela de tempo, foi calculada a mediana da métrica de sincronicidade, e a partir desses valores, extraímos a mediana total para cada condição experimental (e.g., pré-sham, pós-sham, pré-cathodic, pós-cathodic). Essa segmentação possibilita uma representação mais robusta da conectividade, sem realizar análises adicionais sobre mudanças ao longo do tempo além do gráfico apresentado.

A seguir, apresentamos as séries temporais obtidas para três métricas principais:

\begin{itemize}
    \item \textbf{CF-PLM (EEG-ECG):} A Figura~\ref{fig:cfplm_time_cat} mostra a mediana do CF-PLM ao longo do tempo para a condição cathodic, refletindo a sincronização cross-frequency entre EEG e ECG.
    \item \textbf{PLI (EEG-EEG):} A Figura~\ref{fig:pli_time_cat} apresenta a mediana do PLI ao longo do tempo para a condição cathodic, indicando a sincronização iso-frequencial entre canais cerebrais.
    \item \textbf{PLV (EEG-EEG):} A Figura~\ref{fig:plv_time_cat} exibe a mediana do PLV ao longo do tempo para a condição cathodic, utilizada aqui para comparação, considerando que o PLV é mais sensível a ruídos e aos efeitos de volume conduction.
\end{itemize}

Cada gráfico é construído a partir dos valores calculados em janelas de 10 segundos, onde a mediana de cada janela representa a medida de sincronicidade de forma robusta, minimizando o impacto de variações pontuais e outliers.

\standardfigure{figs/4_connectivity_over_time/Mediana_do_CF-PLM_ao_longo_do_tempo_(EEG_ECG)_Catódica.png}
{Mediana do CF-PLM ao longo do tempo para a condição cathodic (EEG-ECG). Cada ponto representa a mediana da medida de sincronicidade calculada em janelas de 10 segundos, evidenciando a evolução do acoplamento cross-frequency entre EEG e ECG.}
{cfplm_time_cat}

\standardfigure{figs/4_connectivity_over_time/Mediana_do_PLI_ao_longo_do_tempo_(EEG_EEG)_Catódica.png}
{Mediana do PLI ao longo do tempo para a condição cathodic (EEG-EEG). O gráfico mostra como a sincronização iso-frequencial entre canais cerebrais varia ao longo da gravação.}
{pli_time_cat}

\standardfigure{figs/4_connectivity_over_time/Mediana_do_PLV_ao_longo_do_tempo_(EEG_EEG)_Catódica.png}
{Mediana do PLV ao longo do tempo para a condição cathodic (EEG-EEG). Este índice é apresentado para comparação com o PLI, embora seja mais sensível a ruídos e efeitos de volume conduction.}
{plv_time_cat}
