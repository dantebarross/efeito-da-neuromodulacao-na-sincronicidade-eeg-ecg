\chapter{Pré-processamento de Dados}
\label{chap:preprocessamento_de_dados}

\section{Preparação dos Dados}
Para este estudo, os dados foram coletados em sessões experimentais com atletas profissionais de basquetebol feminino, submetidas a duas condições: estimulação transcraniana por corrente contínua de alta definição (HD-tDCS) catódica e uma condição de controle (sham). Neste capítulo, descrevemos os procedimentos adotados para organizar, sincronizar e preparar os sinais de eletroencefalografia (EEG) e eletrocardiografia (ECG) para análise.

\subsection{Organização Inicial dos Dados}
Os dados de EEG e ECG foram armazenados em arquivos separados, correspondentes a cada atleta e a cada condição experimental (\textit{pre\_sham}, \textit{post\_sham}, \textit{pre\_cathodic}, \textit{post\_cathodic}). Os sinais de EEG foram originalmente registrados a 1000 Hz, enquanto os sinais de ECG – obtidos a partir do EMG do músculo peitoral maior – foram registrados a 1111,11 Hz. Nesta etapa, os arquivos passaram por:
\begin{itemize}
    \item Identificação e associação a cada atleta e condição;
    \item Renomeação e normalização para garantir consistência;
    \item Verificação de integridade para evitar erros no processamento.
\end{itemize}

\subsection{Sincronização Temporal entre EEG e ECG}

Devido à ausência de sincronização inicial nas gravações, foi necessário alinhar os sinais de EEG e ECG temporalmente. Para tanto, utilizando marcadores do período de repouso, foram realizados os seguintes procedimentos:

\begin{itemize}
    \item Extração dos tempos de início e término do período de repouso para cada modalidade;
    \item Ajuste dos timestamps para que os sinais iniciassem simultaneamente;
    \item Remoção dos primeiros e últimos 15 segundos de cada gravação, a fim de minimizar artefatos de borda.
\end{itemize}

Essa abordagem garante que as análises se baseiem em um estado de repouso sincronizado, eliminando inconsistências temporais entre os registros.

\subsection{Estruturação dos Dados}
Após a sincronização, os dados foram organizados para o processamento subsequente. Os sinais de EEG foram armazenados em formato FIF, preservando as informações dos canais e os metadados, enquanto os dados de ECG foram exportados em formato CSV.
\section{Pré-processamento dos Sinais}

Esta seção descreve os procedimentos aplicados para garantir a qualidade dos sinais, abrangendo etapas de filtragem, remoção de artefatos e segmentação, com foco especial no processamento dos dados de EEG.

\subsection{Pré-processamento do EEG}

O processamento dos dados de EEG compreende duas etapas principais: a preparação dos dados brutos e a limpeza de artefatos utilizando técnicas de decomposição.

\subsubsection{Filtragem, Reamostragem e Preparação dos Dados de EEG}
Para a preparação dos dados de EEG, foram realizadas as seguintes etapas:
\begin{itemize}
    \item \textbf{Aquisição e Carregamento:} Os dados brutos foram baixados do Google Drive e carregados utilizando a biblioteca MNE-Python a partir de arquivos exportados do software BrainVision (ou, alternativamente, em formato EEGLAB). Após o carregamento, os canais foram renomeados conforme uma convenção padronizada (por exemplo, Fp1, Fz, F3, etc.) e canais irrelevantes foram removidos. (Ver Figura~\ref{fig:exemplo_sinais_eeg})
    \item \textbf{Aplicação de Montage:} Aplicou-se o montage padrão \textit{standard\_1005} para assegurar a correta localização dos eletrodos, conforme ilustrado no diagrama do sistema 10-20. (Ver Figura~\ref{fig:sistema_10_20})
    \item \textbf{Definição do Período de Análise:} Com base nas informações extraídas dos arquivos do período de repouso, os dados foram recortados para excluir os primeiros e últimos 15 segundos da gravação, reduzindo artefatos de borda.
    \item \textbf{Filtragem:} Foi aplicado um filtro passa-banda entre 0,5 e 60 Hz para eliminar ruídos fora do intervalo de interesse, seguido de um filtro notch em 60 Hz para remover interferências da rede elétrica (\textit{line noise}).
    \item \textbf{Reamostragem:} Para padronizar a taxa de amostragem e assegurar o alinhamento com outros sinais (como o ECG), bem como o funcionamento das medidas de sincronicidade, os dados de EEG foram reamostrados para 500 Hz.
\end{itemize}

\standardfigure{figs/1_preprocessamento_eeg/2_exemplo_sinais_canais_eeg.png}
{Exemplo de sinais brutos de EEG, ilustrando a variação natural dos canais ao longo do tempo.}
{fig:exemplo_sinais_eeg}

\standardfigure{figs/1_preprocessamento_eeg/1_sistema_10_20.png}
{Ilustração do sistema 10-20, representando a distribuição padronizada dos eletrodos no couro cabeludo. Ressalta-se que o posicionamento apresentado é meramente ilustrativo e não reflete com exatidão a disposição real dos eletrodos.}
{fig:sistema_10_20}



\subsubsection{Limpeza de Artefatos e Remoção de Componentes de Ruído (ICA)}
Para aprimorar a qualidade dos sinais de EEG, empregou-se a Análise de Componentes Independentes (ICA) seguindo os seguintes passos:
\begin{itemize}
    \item \textbf{Identificação de Canais Afetados por Ruído:} Antes da aplicação do ICA, foram identificados canais cujos sinais estavam significativamente comprometidos. Esses canais foram determinados com base em inspeção visual, considerando sinais com ruído excessivo, desvios severos em relação à morfologia típica do EEG ou padrões incompatíveis com atividade neural esperada. Tais canais foram excluídos da decomposição para evitar que influenciassem negativamente a separação dos componentes, e para que não participassem das análises posteriores.
    \item \textbf{Definição dos Componentes:} O número de componentes foi definido igual ao número de canais restantes após a remoção dos canais afetados por ruído, garantindo uma decomposição eficiente e representativa dos sinais neurais. 
    \item \textbf{Aplicação do ICA:} Utilizou-se o método FastICA para decompor o sinal, preservando as informações essenciais para a análise de sincronicidade. 
    \item \textbf{Identificação de Sinais Afastados da Distribuição Geral:} Utilizou-se a biblioteca \textit{autoreject} do Python para detectar automaticamente sinais que apresentavam desvios significativos em relação aos demais. Esse método não foi empregado para aplicar limiares rígidos de rejeição, mas sim para priorizar a inspeção dos canais e épocas mais discrepantes. Todos os sinais foram inspecionados manualmente, com ênfase nos identificados pelo \textit{autoreject}.
    \item \textbf{Identificação Automática de Artefatos:} Componentes associados a artefatos oculares (utilizando canais como Fp1 e Fp2) e à atividade cardíaca foram identificados automaticamente utilizando métodos da biblioteca MNE-Python. Adicionalmente, uma análise baseada na curtose foi realizada para detectar componentes com alta amplitude, indicativos de artefatos. 
    \item \textbf{Inspeção Visual e Seleção:} Foram gerados gráficos das propriedades dos componentes e visualizações interativas, incluindo animações, para auxiliar na verificação do sinal de todos os canais de EEG com e sem determinados componentes específicos. Por exemplo, foi possível comparar o comportamento do sinal sem os componentes [0, 1, 2] e sem os componentes [0, 1, 2, 3], permitindo avaliar a influência da adição ou remoção do componente 3 no sinal de todos os canais. Além disso, a avaliação dos ICs foi realizada considerando sua distribuição espacial, com auxílio de \textit{topomaps} (\textit{heatmaps} da cabeça), identificando em quais regiões os componentes estavam mais concentrados. Uma forte concentração em áreas comumente associadas à geração de ruídos, como a região próxima aos olhos, pode ser um indicativo de artefato e, portanto, um critério para sua remoção.
    \item \textbf{Aplicação e Salvamento:} Após a definição dos componentes a serem removidos, o ICA foi aplicado para eliminar os artefatos identificados, gerando um sinal de EEG considerado pré-processado, pronto para ser analisado estatisticamente. O sinal resultante foi salvo em formato FIF para futuras análises.
\end{itemize}


\standardfigure{figs/1_preprocessamento_eeg/3_exemplo_compomentes_pos_ICA.png}
{Exemplo de componentes obtidos após a aplicação do ICA, com mapas topográficos que evidenciam a distribuição espacial de cada componente.}
{fig:componentes_pos_ICA}

\standardfigure{figs/1_preprocessamento_eeg/4_exemplo_ICA_component_analysis.png}
{Exemplo de análise detalhada de um componente ICA, apresentando o mapa topográfico, o espectro de frequência e outras características relevantes para a identificação de artefatos.}
{fig:exemplo_ICA_component_analysis}


\subsection{Pré-processamento do Sinal de ECG}
\label{subsec:preprocess_ecg}

O processamento do sinal de ECG teve como objetivo obter uma versão refinada do sinal, permitindo a detecção precisa dos picos R, a extração de informações de fase e a obtenção do ciclo cardíaco com base nos picos R. Este ciclo foi utilizado no cálculo de sincronicidade de fase. Para isso, os procedimentos foram organizados em três etapas principais: aquisição e segmentação, limpeza do sinal com detecção de picos e aplicação de filtros complementares.

\subsubsection{Aquisição e Segmentação dos Dados de ECG}
\begin{itemize}
    \item \textbf{Aquisição:} Os dados de ECG foram coletados juntamente com informações sobre os tempos de início e fim do período de repouso para cada condição experimental, garantindo a correta identificação dos intervalos de interesse.
    \item \textbf{Segmentação:} Com base nos tempos extraídos dos arquivos do período de repouso, o sinal bruto foi segmentado para selecionar apenas o intervalo correspondente à condição de interesse. Para reduzir artefatos nas bordas, os primeiros e últimos 15 segundos foram removidos.
\end{itemize}

\subsubsection{Limpeza do Sinal e Detecção de Picos}
\begin{itemize}
    \item \textbf{Limpeza:} Utilizou-se a biblioteca \textit{NeuroKit2}, uma biblioteca em Python, para processar o sinal segmentado, removendo ruídos e gerando uma versão refinada do ECG.
    \item \textbf{Detecção Automática de Picos:} Utilizou-se um método da biblioteca \textit{NeuroKit2}, implementado em Python, para a detecção automática dos picos R (\textit{R-peaks}) no sinal processado. Esses picos correspondem aos momentos de despolarização ventricular e são essenciais para a extração do ciclo cardíaco. A Figura~\ref{fig:ecg_picos_detectados} ilustra um exemplo, no qual o sinal bruto (cinza) é sobreposto ao sinal processado (azul), com os picos R destacados em vermelho.
    \item \textbf{Correção Manual dos Picos R:} Após a detecção automática, uma inspeção visual cuidadosa foi realizada utilizando gráficos interativos para:
    \begin{itemize}
        \item Inserir manualmente os picos faltantes, identificados pela ausência de eventos em locais esperados;
        \item Remover manualmente picos falsos, identificados por timestamps incorretos.
    \end{itemize}
    Esse ajuste garante a acurácia na identificação dos batimentos, evitando omissões e inclusões indevidas.
\end{itemize}


\standardfigure{figs/2_preprocessamento_ecg/1_Sinal_de_ECG_-_Picos_Detectados_zoom.png}
{Exemplo de sinal de ECG com picos detectados. O sinal bruto (cinza) representa o sinal de EMG coletado, enquanto o sinal processado (azul) corresponde ao ECG filtrado, com os picos R marcados em vermelho.}
{fig:ecg_picos_detectados}


\subsubsection{Aplicação de Filtros Complementares}
Para refinar a definição dos eventos de pico R no ECG, foram aplicados filtros adicionais:
\begin{itemize}
    \item \textbf{Filtro de Janela:} Extração de uma janela de \(\pm50\) ms ao redor de cada pico, isolando os segmentos de interesse.
    \item \textbf{Filtro de Cruzamento pelo Zero:} Identificação dos pontos de cruzamento pelo zero nos segmentos próximos aos picos, permitindo um ajuste fino dos limites dos eventos.
\end{itemize}

A combinação desses filtros resultou em um sinal final filtrado, que destaca de forma mais clara a morfologia do ECG. A Figura~\ref{fig:ecg_filtros_aplicados} exemplifica o efeito dos filtros, comparando o sinal limpo inicial (linha colorida) com o sinal final filtrado, bem como os picos detectados.

\standardfigure{figs/2_preprocessamento_ecg/2_Sinal_ECG_-_Aplicação_de_Filtros_zoom.png}
{Exemplo de aplicação de filtros complementares ao sinal de ECG, destacando a morfologia dos picos R (em vermelho).}
{fig:ecg_filtros_aplicados}


\subsubsection{Geração de Sinais Senoidais e Análise de Fase}

Para viabilizar a análise de sincronização de fase entre os sinais de EEG e ECG, o sinal de ECG foi transformado em uma representação senoidal. Essa transformação apresenta diversos benefícios:
\begin{itemize}
    \item \textbf{Definição Clara do Ciclo Cardíaco:} Ao utilizar os R-peaks para delimitar cada ciclo, a conversão em uma onda senoidal permite definir de forma inequívoca o início e o fim do ciclo cardíaco, fornecendo um marcador preciso para segmentação dos períodos de interesse.
    \item \textbf{Extração Precisa da Fase:} Uma onda senoidal exibe uma variação linear de fase ao longo do tempo, o que facilita a extração da fase instantânea por meio da Transformada de Hilbert, proporcionando uma determinação robusta e consistente.
    \item \textbf{Facilitação da Análise de Sincronização:} A eficácia dos métodos baseados em fase depende de uma definição clara da fase dos sinais. A conversão do sinal de ECG para uma representação senoidal aprimora essa definição, permitindo que os algoritmos de análise captem com maior precisão a sincronização entre os ritmos neurais (EEG) e o ritmo cardíaco. Isso melhora a robustez das métricas de sincronização, reduzindo ambiguidades decorrentes de distorções na extração da fase.
    \item \textbf{Robustez à Variabilidade e Ruído:} A conversão do ECG, que apresenta picos acentuados e variabilidade, para uma forma senoidal completamente baseada no pico R suaviza essas irregularidades, melhorando a robustez do método de extração de fase mesmo na presença de ruídos ou artefatos.
    \item \textbf{Integração com a Análise de EEG:} Como os sinais de EEG são frequentemente filtrados para realçar componentes oscilatórios em bandas de frequência específicas, a padronização da representação do ECG permite uma comparação mais coerente entre os dois sinais. Essa abordagem pode facilitar a aplicação de métricas de sincronização e análise cross-frequency durante a investigação de possíveis interações entre ritmos cardíacos e cerebrais.
\end{itemize}

Além desses benefícios, estudos como o de Mollakazemi et al. \cite{mollakazemi2021eeg} demonstram que a sincronização entre EEG e ECG pode influenciar a detecção de efeitos funcionais em sinais cerebrais, reforçando a relevância de abordagens que consideram a interação entre ritmos cardíacos e neurais na análise de fase.

A Figura~\ref{fig:ecg_comparacao_fase} ilustra a comparação de fase entre o sinal de ECG filtrado (azul), o sinal senoidal gerado (verde) e um sinal simulado (vermelho), evidenciando a coerência de fase entre eles.


\standardfigure{figs/2_preprocessamento_ecg/3_Comparação_de_Fase_entre_Sinais.png}
{Exemplo de comparação de fase entre o ECG filtrado (azul), o ECG senoidal (verde) e um sinal simulado (vermelho). A boa concordância entre as fases indica a consistência do procedimento de geração do sinal senoidal e da extração de fase.}
{fig:ecg_comparacao_fase}


Em suma, a transformação do ECG em um sinal senoidal não apenas define claramente o ciclo cardíaco, mas também possibilita a extração de uma fase contínua, essencial para a análise de sincronização de fase entre EEG e ECG utilizando métodos de extração de fase empregados neste estudo.

\subsubsection{Estrutura do Dado Final e Armazenamento}

O conjunto final de dados resultante do processamento do ECG foi estruturado em um DataFrame que integra as seguintes variáveis:
\begin{itemize}
    \item \textbf{Tempo:} Timestamps sincronizados.
    \item \textbf{Sinal Bruto (EMG):} Valor original do sinal.
    \item \textbf{Sinal Limpo (ECG):} Versão filtrada do sinal.
    \item \textbf{Picos:} Indicador binário dos R-peaks detectados.
    \item \textbf{First Filtered:} Sinal obtido após a aplicação do filtro de janela (\(\pm50\) ms).
    \item \textbf{Final Filtered:} Sinal final obtido após a combinação dos filtros aplicados.
    \item \textbf{ECG Senoidal:} Sinal senoidal derivado dos R-peaks.
\end{itemize}

Este DataFrame foi exportado em formato CSV para facilitar o acesso e a análise subsequente dos dados.
