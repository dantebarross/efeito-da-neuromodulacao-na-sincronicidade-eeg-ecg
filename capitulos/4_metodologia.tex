\chapter{Metodologia}
\label{chap:metodologia}
Este estudo utiliza dados previamente coletados de atletas profissionais de basquetebol feminino em repouso, abrangendo medidas de EEG e EMG. O objetivo principal é investigar a sincronicidade entre sinais neurofisiológicos e musculares e avaliar como a HD-tDCS catódica, aplicada sobre o DLPFC esquerdo, pode modular a atividade neural em condições de repouso, fornecendo insights sobre potenciais efeitos neuromodulatórios em atletas

\section{Participantes e Coleta de Dados}
O estudo foi aprovado pelo Comitê de Ética em Pesquisa da UFABC (protocolo: 08070819.1.0000.5594) e conduzido em conformidade com os princípios éticos estabelecidos pela Declaração de Helsinque para experimentos envolvendo seres humanos. Todas as participantes assinaram o Termo de Consentimento Livre e Esclarecido (TCLE) antes do início da coleta de dados.

Foram selecionadas atletas de elite, caracterizadas por:
\begin{itemize}
    \item Participação regular no programa de treinamento da equipe;
    \item Regime de treinamento superior a 10 horas semanais;
    \item Ausência de doenças ou lesões que pudessem interferir na execução do protocolo.
\end{itemize}

A amostra foi detalhadamente caracterizada por meio da mensuração de parâmetros como massa corporal e estatura, além da coleta de informações pessoais e esportivas (nome, data de nascimento, categoria, experiência esportiva, posição no time, fase da temporada, membro dominante e estilo de arremesso). Devido a problemas técnicos durante a coleta, apenas os dados de 6 atletas foram incluídos nas análises. O número de sujeitos reflete não apenas a disponibilidade das participantes, mas também a especificidade do grupo estudado (atletas de basquetebol feminino de elite da mesma equipe) e a complexidade dos protocolos, que exigem múltiplas visitas ao laboratório e a sincronização de diversos equipamentos.

\section{Delineamento Experimental}
O estudo utilizou um delineamento experimental randomizado, cruzado e duplo-cego. Cada participante foi submetida às condições catódica (HD-tDCS) e \textit{sham} em momentos distintos, e a análise foi realizada considerando os pares de canais de forma agrupada.

Antes do início das sessões experimentais, foi realizada uma sessão de familiarização na qual as atletas receberam informações detalhadas sobre os objetivos, procedimentos, riscos e benefícios do estudo, além de treinamento prático para se familiarizarem com o protocolo e os equipamentos utilizados.

\standardfigure{figs/0_intro_e_desenho_experimental/desenho_experimental_drawio.png}
{Fluxo do protocolo experimental, incluindo a sessão de familiarização, as duas sessões experimentais (com estimulação catódica ou \textit{sham}), as coletas de EEG/ECG em repouso e a execução de 100 arremessos pré-estimulação e 100 arremessos pós-estimulação por sessão.}
{desenho_experimental}

As sessões experimentais seguiram a seguinte estrutura:
\begin{itemize}
    \item \textbf{Sessão 1:} Familiarização com os dispositivos e procedimentos do estudo.
    \item \textbf{Sessões 2 e 3:} Execução do protocolo experimental, na qual cada atleta realizou 100 arremessos pré-estimulação e 100 arremessos pós-estimulação, totalizando 200 arremessos por sessão (ou 400 no total).
\end{itemize}

Para garantir a padronização das condições experimentais, todas as sessões foram realizadas no mesmo local e horário habitual de treinamento das atletas, sob condições controladas de iluminação, temperatura e ruído. A ordem das condições (HD-tDCS ativa e \textit{sham}) foi definida aleatoriamente para cada participante por meio de um gerador de números aleatórios, assegurando imparcialidade no delineamento cruzado. Antes de cada sessão, os equipamentos de EEG e ECG foram calibrados e tiveram suas impedâncias verificadas, garantindo a qualidade dos registros. As participantes foram continuamente monitoradas quanto a possíveis efeitos adversos ou desconfortos decorrentes da estimulação, e os dados foram armazenados digitalmente com protocolos rigorosos de backup e verificação de integridade, assegurando sua reprodutibilidade e confiabilidade para as análises subsequentes.

Em cada sessão experimental, registramos EEG/ECG em repouso por 5 min antes (pré) e 5 min após (pós) a aplicação da HD-tDCS (ativa ou sham). Todas as análises subsequentes foram realizadas sobre a diferença pós-pré em cada condição, de modo a quantificar explicitamente o efeito da estimulação catódica em relação à sham.

\section{Questionários e Escalas}
Durante a coleta dos dados experimentais, foram aplicados questionários e escalas com o intuito de avaliar aspectos subjetivos relacionados ao estado psicológico e fisiológico das atletas. Embora esses instrumentos tenham sido incluídos no protocolo inicial com a finalidade de fornecer uma visão complementar às medidas neurofisiológicas, os dados obtidos não foram utilizados nas análises estatísticas ou nos resultados apresentados neste estudo. Os instrumentos aplicados foram:

\begin{itemize}
    \item \textbf{Escala de Qualidade Total de Recuperação (TQR):} Avalia a percepção subjetiva do estado geral de recuperação física e mental após as sessões experimentais.
    \item \textbf{Escala de Percepção Subjetiva de Esforço (PSE):} Mensura o esforço percebido pelas participantes durante as sessões experimentais.
    \item \textbf{Sport Competition Anxiety Test (SCAT):} Identifica níveis de ansiedade competitiva das participantes.
    \item \textbf{Questionário de Motivação Relacionado ao Exercício:} Investiga os fatores motivacionais das participantes durante o período experimental.
\end{itemize}

Dessa forma, apesar de terem sido coletados, os dados desses questionários não são apresentados nem discutidos, dado o escopo específico das análises neurofisiológicas deste trabalho.

\subsection{Protocolo de HD-tDCS}
\label{subsec:hdtdcs_protocol}
A HD-tDCS foi aplicada com um estimulador digital MxN da Soterix Medical, utilizando eletrodos de Ag/AgCl integrados a uma touca de EEG. Diferentemente da configuração clássica 4$\times$1 descrita em \citeonline{villamar2013hdtdcs}, optou-se por um arranjo customizado, em que o posicionamento dos eletrodos foi determinado por modelagem computacional, visando otimizar a focalidade da corrente no DLPFC esquerdo.

Nesta montagem, um eletrodo principal foi posicionado sobre o DLPFC, enquanto eletrodos de retorno foram distribuídos de forma a contornar a região-alvo, garantindo uma distribuição concentrada da corrente. Foram aplicadas duas condições experimentais:
\begin{itemize}
    \item \textbf{Estimulação Catódica (Ativa):} Corrente contínua de 2\,mA aplicada por 20 minutos.
    \item \textbf{Condição \textit{sham} (Simulada):} A corrente foi interrompida após os 30\,segundos iniciais, de modo a reproduzir a sensação inicial sem promover a modulação cortical efetiva.
\end{itemize}

A calibração do dispositivo e a verificação da impedância (mantida abaixo de 5\,k$\Omega$) foram realizadas antes de cada sessão, assegurando a qualidade e a confiabilidade dos dados registrados. Essa abordagem, fundamentada em parâmetros estabelecidos na literatura \cite{datta2008transcranial, stagg2011physiological}, permitiu uma estimulação focalizada e segura, adequada ao delineamento experimental adotado no presente estudo.


\subsection{Processamento e Análise de Dados}
O processamento e a análise dos dados seguiram um fluxo estruturado que abrange desde a coleta e organização dos arquivos até a extração das métricas de sincronização e a aplicação de testes estatísticos para avaliar as diferenças entre as condições experimentais. A Figura~\ref{fig:fluxo_processamento} apresenta um diagrama geral dessas etapas.

\standardfigure{figs/0_intro_e_desenho_experimental/diagrama_processamento_e_analise_drawio.png}
{Fluxo geral de processamento e análise de dados, desde a coleta e organização dos arquivos, passando pelas etapas de pré-processamento (EEG e ECG), sincronização temporal, extração de métricas de sincronização (PLI, PLV e CF-PLM) e, por fim, análises estatísticas e de conectividade em rede.}
{fluxo_processamento}

\subsubsection{Pré-processamento de Dados}
Os sinais de EEG e EMG foram submetidos a etapas de pré-processamento que incluíram a filtragem de ruídos e a remoção de artefatos por meio de \textit{Independent Component Analysis} (ICA). Paralelamente, os sinais de ECG passaram por processos de detecção de picos e extração do ciclo cardíaco, garantindo assim a qualidade e a consistência dos dados analisados.

\subsubsection{Sincronização de Sinais}
Para possibilitar uma análise integrada, os sinais de EEG e ECG foram alinhados temporalmente, assegurando que as medidas extraídas estivessem sincronizadas e pudessem ser comparadas corretamente.

\subsubsection{Cálculo de Sincronização Funcional}
A sincronização entre os sinais foi avaliada utilizando três métricas. O \textit{Phase Lock Value} (PLV) e o \textit{Phase Lag Index} (PLI) foram aplicados para quantificar a conectividade na mesma faixa de frequência, enquanto o \textit{Cross-Frequency Phase Linearity Measurement} (CF-PLM) foi empregada para avaliar o acoplamento entre frequências distintas. Antes de sua aplicação nos dados experimentais, esses métodos foram testados e validados utilizando sinais simulados.

\subsubsection{Análise Estatística} 
Para cada par de canais e para cada condição (catódica e sham), computamos a diferença de sincronização \(\Delta = \text{valor}_{\text{pós}} - \text{valor}_{\text{pré}}\). Essas diferenças foram então submetidas aos testes estatísticos.

Para investigar as diferenças entre as condições experimentais, aplicamos métodos estatísticos robustos, incluindo testes de normalidade, testes não paramétricos e correções para comparações múltiplas. Adicionalmente, foram utilizadas medidas de centralidade em redes para avaliar a conectividade funcional entre diferentes regiões corticais. Os padrões de conectividade foram representados por meio de gráficos e redes, facilitando a visualização e a interpretação dos resultados.

Esse fluxo sistemático permitiu uma abordagem rigorosa para explorar a sincronicidade cerebral e avaliar o impacto da estimulação transcraniana na conectividade neural das atletas em estado de repouso.
