% ---------------------------------------------------------------------------
% Template simplificado para Dissertação/UFABC
% ---------------------------------------------------------------------------
\documentclass[
  12pt,
  openany,
  twoside,
  a4paper,
  english,
  brazil
]{abntex2}

% -----------------------------------------
% Pacotes OBRIGATÓRIOS
% -----------------------------------------
\usepackage{lmodern}      % Usa a fonte Latin Modern
\usepackage[T1]{fontenc}  % Codificação de fonte
\usepackage[utf8]{inputenc}    % Reconhece acentos
\usepackage{indentfirst}  % Indenta o primeiro parágrafo de cada seção
\usepackage{color}        % Controle de cores
\usepackage{graphicx}     % Inclusão de gráficos
\usepackage{epsfig,subfig}% Inclusão de figuras
\usepackage{lastpage}     % Usado na ficha catalográfica
\usepackage{microtype}    % Melhora a justificação

% -----------------------------------------
% Pacotes ADICIONAIS
% -----------------------------------------
\usepackage{float}
\usepackage{lipsum}                 % Texto de exemplo
\usepackage{amsmath,amssymb,mathrsfs} 
\usepackage{setspace}               % Espaçamento
\usepackage{verbatim}               % Ambiente "comment"
\usepackage{tabularx}               % Tabelas autoajustáveis
\usepackage{afterpage}              % Comandos após o fim da página
\usepackage{url}                    % Formatação de URLs

% -----------------------------------------
% Pacotes de CITAÇÕES (ABNT)
% -----------------------------------------
\usepackage[brazilian,hyperpageref]{backref}
\usepackage[alf]{abntex2cite}  % ou [num] se preferir citações numéricas

\usepackage{hyperref}
\pdfstringdefDisableCommands{%
  \def\uppercase#1{#1}%
}

% -----------------------------------------
% Configurações
% -----------------------------------------
\input{extras/conf_citacoes}  % Ajustes de citações
% ---
% Informações de dados para CAPA e FOLHA DE ROSTO
% ---
\titulo{Efeito da Neuromodulação Não Invasiva na Sincronicidade Cérebro-Corpo em Atletas do Basquetebol}
\autor{Danilo Cavalcante Brambila de Barros}
\local{São Bernardo do Campo - SP}
\data{2025}
\orientador{Prof. Dr. Alexandre Hideki Okano}
\coorientador{Prof. Dr. Edgard Morya}
\instituicao{%
  Universidade Federal do ABC -- UFABC
  \par
  Centro de Matemática, Computação e Cognição 
  \par
  Programa de Pós-Graduação em Neurociência e Cognição}
\tipotrabalho{Dissertação (Mestrado)}
% O preambulo deve conter o tipo do trabalho, o objetivo,
% o nome da instituição e a área de concentração
\preambulo{\textbf{Dissertação de Mestrado} apresentada ao Programa de Pós-Graduação em Neurociência e Cognição, como parte dos requisitos necessários para a obtenção do Título de Mestre em Neurociência e Cognição.}
% ---          % Título, autor, orientador etc.
\input{extras/conf_pdf}       % Aparência do PDF final

\setlength{\parindent}{1.3cm}   % Tamanho da indentação
\setlength{\parskip}{0.2cm}     % Espaçamento entre parágrafos
\makeindex                       % Compilar índice remissivo
% -----------------------------------------

\begin{document}
% Remove espaço extra entre frases
\frenchspacing

% ----------------------------------------------------------
% ELEMENTOS PRÉ-TEXTUAIS
% ----------------------------------------------------------
\pretextual

% Capa
\input{pretextual/capa}

% Folha de rosto
\imprimirfolhaderosto*

% Ficha Catalográfica
% ---
% Ficha Catalográfica
% ---
% Exemplo de Ficha Catalográfica (Dados Internacionais de Catalogação).
% Se a biblioteca enviar um PDF oficial da ficha, basta:
%
% \begin{fichacatalografica}
%     \includepdf{fig_ficha_catalografica.pdf}
% \end{fichacatalografica}
%
% Caso contrário, use o modelo abaixo.
% ---
% Ficha Catalográfica
% ---

\begin{fichacatalografica}
    \vspace*{\fill}
    \hrule
    \begin{center}
    \begin{minipage}[c]{12cm}

    \imprimirautor

    \hspace{0.5cm} \imprimirtitulo /\\
    \hspace{0.5cm} \imprimirautor. -- \imprimirlocal, \imprimirdata.

    \hspace{0.5cm} \pageref{LastPage} p. : il. color. ; 30 cm.

    \hspace{0.5cm} \imprimirorientadorRotulo~\imprimirorientador

    \hspace{0.5cm}
    \parbox[t]{0.9\textwidth}{%
      \imprimirtipotrabalho~--~\imprimirinstituicao,
      \imprimirdata.
    }

    \hspace{0.5cm} Desenvolvido com apoio da Coordenação de Aperfeiçoamento de Pessoal de Nível Superior - Brasil (CAPES) - Código de Financiamento 001.

    \hspace{0.5cm}
    1. Neuromodulação. 2. Sincronização Cérebro-Corpo. 3. Basquetebol.\\
    I. Orientador. II. Universidade Federal do ABC. III. Programa de Pós-Graduação em Neurociência e Cognição. IV. Título.

    \hspace{8.75cm} CDU 02:141:005.7

    \end{minipage}
    \end{center}
    \hrule
\end{fichacatalografica}

% Folha de assinaturas
\input{pretextual/assinaturas}

% Dedicatória
\begin{dedicatoria}
   \vspace*{\fill}
   \centering
   \noindent
   \textit{Dedico este trabalho a todos os estudantes e pesquisadores que, movidos pela curiosidade e pela paixão pelo conhecimento, buscam entender os mistérios do cérebro e do corpo humano. Que esta pesquisa possa servir como inspiração para novas descobertas e avanços na ciência.} \vspace*{\fill}
\end{dedicatoria}

% Agradecimentos
\chapter*{Agradecimentos}
\addcontentsline{toc}{chapter}{Agradecimentos}

Agradeço à minha \textbf{família}, pelo amor e apoio incondicional, fundamentais para eu chegar até aqui.  
Ao meu \textbf{orientador} e \textbf{coorientador}, pela oportunidade de explorar novas ideias, pelo incentivo contínuo e por ser fonte de inspiração e aprendizado ao longo desta caminhada.  
Aos \textbf{colegas de laboratório}, pela convivência, troca de ideias e colaboração, que tornaram essa jornada mais enriquecedora e inspiradora.  
Aos \textbf{professores e colaboradores} que, de alguma forma, contribuíram para a realização deste estudo.

\bigskip
\noindent

O presente trabalho foi realizado com apoio da Coordenação de Aperfeiçoamento de Pessoal de Nível Superior - Brasil (CAPES) - Código de Financiamento 001.

% Epígrafe
% ---
% Epígrafe
% ---
\begin{epigrafe}
    \vspace*{\fill}
	\begin{flushright}
		\textit{``Se o cérebro humano fosse simples o suficiente para que pudéssemos entendê-lo, seríamos tão simples que não conseguiríamos entendê-lo.''\\
		          (Emerson Pugh)}
	\end{flushright}
\end{epigrafe}
% ---

% Resumos (Resumo e Abstract)
% ---
% RESUMOS
% ---

% RESUMO em português
\setlength{\absparsep}{18pt} % ajusta o espaçamento dos parágrafos do resumo
\begin{resumo}
A sincronização entre oscilações neurais e ritmos fisiológicos, conceito central do \textit{Body--Brain Dynamic System} (BBDS), desempenha um papel fundamental na regulação e integração neural. Este estudo investigou os efeitos da neuromodulação catódica por estimulação transcraniana por corrente contínua de alta definição (\textit{High-Definition transcranial Direct Current Stimulation}, HD-tDCS), aplicada sobre o córtex pré-frontal dorsolateral (DLPFC) esquerdo, na sincronização cerebral em atletas de elite de basquetebol feminino em repouso (\textit{resting-state}). Utilizando um delineamento experimental cruzado e duplo-cego, com sessões ativa (catódica, 2 mA por 20 min) e simulada (\textit{sham}), analisou-se o acoplamento fásico intrafrequencial entre pares de canais de eletroencefalografia (EEG), bem como o acoplamento fásico \textit{cross-frequency} entre EEG e eletrocardiograma (ECG). Os índices utilizados foram o \textit{Phase Lag Index} (PLI) para conexões EEG--EEG na mesma banda de frequência e o \textit{Cross-Frequency Phase Linearity Measurement} (CF-PLM) para conexões EEG--ECG entre frequências distintas. Os resultados demonstraram que a neuromodulação catódica alterou significativamente os padrões de sincronização cerebral intrafrequencial e \textit{cross-frequency}, indicando efeito na modulação dos diversos acoplamentos EEG--EEG e EEG--ECG. Estes achados ressaltam o potencial da HD-tDCS como ferramenta para modulação direcionada da conectividade funcional cerebral e cardiovascular, com implicações tanto para contextos esportivos quanto clínicos.

\textbf{Palavras-chave}: Neuromodulação catódica. HD-tDCS. EEG. ECG. Conectividade funcional. Sincronização de fase. Atletas.

\end{resumo}


% ABSTRACT in english
\begin{resumo}[Abstract]
\begin{otherlanguage*}{english}
Synchronization between neural oscillations and physiological rhythms, a central concept of the Body–Brain Dynamic System (BBDS), plays a fundamental role in neural regulation and integration. This study investigated the effects of cathodal neuromodulation using High-Definition transcranial Direct Current Stimulation (HD-tDCS), applied over the left dorsolateral prefrontal cortex (DLPFC), on brain synchronization in elite female basketball athletes during resting-state. Employing a double-blind, crossover experimental design with active (cathodal, 2 mA for 20 min) and sham sessions, intra-frequency phase coupling between pairs of electroencephalography (EEG) channels, as well as cross-frequency phase coupling between EEG and electrocardiogram (ECG), were analyzed. The indices used were the Phase Lag Index (PLI) for EEG–EEG connections within the same frequency band and Cross-Frequency Phase Linearity Measurement (CF-PLM) for EEG–ECG connections across distinct frequencies. Results demonstrated that cathodal neuromodulation significantly altered intra-frequency and cross-frequency brain synchronization patterns, indicating modulatory effects on EEG–EEG and EEG–ECG couplings. These findings highlight the potential of HD-tDCS as a targeted tool for modulating functional brain and cardiovascular connectivity, with implications for both sports and clinical contexts.

\vspace{\onelineskip}
 
\noindent 
\textbf{Keywords}: Cathodal neuromodulation. HD-tDCS. EEG. ECG. Functional connectivity. Phase synchronization. Athletes.
\end{otherlanguage*}
\end{resumo}

% Lista de Figuras
\pdfbookmark[0]{\listfigurename}{lof}
\listoffigures*
\cleardoublepage

% Lista de Tabelas
\pdfbookmark[0]{\listtablename}{lot}
\listoftables*
\cleardoublepage

\begin{siglas}
  \item[BBDS] \textit{Body Brain Dynamic System}
  \item[EEG] Eletroencefalografia
  \item[EMG] Eletromiografia
  \item[ECG] Eletrocardiograma
  \item[tDCS] Estimulação transcraniana por corrente contínua
  \item[HD-tDCS] Estimulação transcraniana por corrente contínua de alta definição
  \item[CPFDL] Córtex Pré-frontal Dorsolateral
  \item[PLI] \textit{Phase Lag Index}
  \item[PLV] \textit{Phase Locking Value}
  \item[CF-PLM] \textit{Cross-Frequency Phase Linearity Measurement}
  \item[SCAT] \textit{Sport Competition Anxiety Test}
  \item[TQR] Escala de Qualidade Total de Recuperação (\textit{Total Quality Recovery})
  \item[PSE] Escala de Percepção Subjetiva de Esforço
  \item[NMDA] \textit{N-methyl-D-aspartate}
  \item[ICA] \textit{Independent Component Analysis}
  \item[KDE] \textit{Kernel Density Estimation}
  \item[ECOD] \textit{Empirical Cumulative Distribution-based Outlier Detection}
  \item[GPU] \textit{Graphics Processing Unit}
  \item[BCa] \textit{Bias-Corrected and Accelerated}
  \item[FDR-BH] \textit{False Discovery Rate-Benjamini–Hochberg}
  \item[RBC] \textit{Rank-Biserial Correlation}
  \item[IQR] \textit{Interquartile Range}
\end{siglas}

\begin{simbolos}
  \item[$x(t)$] Sinal de entrada, função do tempo.
  \item[$y(t)$] Segundo sinal de entrada, função do tempo.
  \item[$x_{\mathrm{an}}(t)$] Representação analítica do sinal \(x(t)\), obtida pela Transformada de Hilbert.
  \item[$y_{\mathrm{an}}(t)$] Representação analítica do sinal \(y(t)\), obtida pela Transformada de Hilbert.
  \item[$\phi_x(t)$] Fase instantânea do sinal \(x(t)\).
  \item[$\phi_y(t)$] Fase instantânea do sinal \(y(t)\).
  \item[$\Delta \phi(t)$] Diferença de fase instantânea entre os sinais \(x(t)\) e \(y(t)\), definida por \(\Delta \phi(t)=\phi_x(t)-\phi_y(t)\).
  \item[$z(t)$] Sinal interferométrico, definido como 
  \[
  z(t)=\frac{x_{\mathrm{an}}(t)\, y_{\mathrm{an}}^*(t)}{\lvert x_{\mathrm{an}}(t)\rvert\,\lvert y_{\mathrm{an}}(t)\rvert} = e^{i\Delta \phi(t)}.
  \]
  \item[$SZ(f)$] Densidade espectral de potência (PSD) do sinal \(z(t)\), em função da frequência \(f\).
  \item[$f_\Delta$] Diferença entre as frequências centrais dos sinais envolvidos no acoplamento cross-frequency.
  \item[$B$] Largura de banda utilizada para integrar a PSD na definição do índice CF-PLM.
\end{simbolos}

% Sumário
\pdfbookmark[0]{\contentsname}{toc}
\tableofcontents*
\cleardoublepage

% ----------------------------------------------------------
% ELEMENTOS TEXTUAIS
% ----------------------------------------------------------
\textual
\pagenumbering{arabic}
\setcounter{page}{1}

% ----------------------------------------------------------
% Capítulos
\part{Introdução e Fundamentação Teórica}
\chapter{Introdução}
\label{chap:introducao}
A neurociência tem avançado na compreensão da sincronização entre o cérebro e processos fisiológicos, destacando o papel das interações dinâmicas na integração entre sistemas corporais e neurais. Nesse contexto, o conceito de \textit{Body--Brain Dynamic System} (BBDS) tem ganhado espaço como uma abordagem para investigar como oscilações neurais se sincronizam com ritmos fisiológicos – como os da frequência cardíaca, respiratória, entre outros –, modulando a atividade cerebral \cite{cohen2017where,criscuolo2022cognition}. Esse entendimento tem impulsionado o desenvolvimento de intervenções de neuromodulação capazes de modificar padrões rítmicos e, consequentemente, a função cerebral.

O corpo humano possui uma capacidade intrínseca de sincronizar seus ritmos fisiológicos com estímulos ambientais e internos. Por exemplo, o ritmo respiratório pode alinhar-se a padrões de atividade sensorial e cognitiva \cite{haas1985effects}, e estados psicofisiológicos – como ansiedade, depressão e estresse – influenciam tanto a frequência cardíaca quanto a atividade neural \cite{criscuolo2022cognition}. Pesquisas recentes demonstram que a variabilidade dos ritmos cardíacos e respiratórios gera ciclos de alta e baixa excitabilidade, modulando a integração e a regulação neural. Estudos em neurocardiologia evidenciam que a interação entre mecanismos sensoriais (como barorreceptores e quimiorreceptores) e centros neurais superiores é essencial para a regulação dos padrões cardíacos \cite{marcondes2024linguagem}. Além disso, pesquisas com potenciais evocados pelo batimento cardíaco (HEPs) reforçam a importância da integração interoceptiva na formação da consciência corporal \cite{banelli2020skipping, mackinnon2013utilizing, park2018neural}.

Para explorar os efeitos da neuromodulação na conectividade neural, este projeto investiga como a estimulação transcraniana por corrente contínua de alta definição (HD-tDCS), aplicada de forma catódica sobre o córtex pré-frontal dorsolateral (DLPFC) esquerdo, impacta os padrões de sincronização cerebral em atletas de elite de basquetebol feminino em repouso (\textit{resting-state}). A escolha da estimulação catódica, que tipicamente reduz a excitabilidade cortical, fundamenta-se na hipótese de que a diminuição da atividade no DLPFC esquerdo pode modular o equilíbrio entre redes neurais, potencialmente reduzindo padrões de hiperconectividade e promovendo uma reorganização funcional mais eficiente. O DLPFC esquerdo foi selecionado como alvo por seu papel central em redes frontoparietais envolvidas em funções executivas e controle cognitivo-motor, processos fundamentais para o desempenho atlético em esportes coletivos. O delineamento experimental adotado é do tipo cruzado (\textit{cross-over}) e duplo-cego, permitindo que os mesmos participantes sejam submetidos tanto à estimulação ativa quanto à condição controle (\textit{sham}), reduzindo assim a influência de variáveis individuais e possibilitando uma análise mais precisa dos efeitos neuromodulatórios.

Neste estudo, além de analisar a sincronização intrafrequencial entre pares de canais de EEG, investigamos a sincronicidade \textit{cross-frequency} entre sinais de eletroencefalografia (EEG) e eletrocardiograma (ECG). Para isso, o sinal de ECG foi convertido em uma representação senoidal simples – baseada no pico R – que delimita de forma clara o ciclo cardíaco. Ao comparar, por exemplo, o canal Fp1 na banda alpha com esse sinal senoidal, é possível quantificar a sincronização entre a atividade cerebral e o ritmo cardíaco. Essa abordagem detalha o acoplamento entre oscilações cerebrais de diferentes frequências e o sinal cardíaco, contribuindo para a compreensão da interação entre os sistemas neural e cardiovascular, com especial relevância para intervenções neuromodulatórias em atletas. Ademais, modelos recentes exploram a previsibilidade dos sinais EEG alinhados com os batimentos cardíacos, ampliando o entendimento dos mecanismos de acoplamento entre a atividade neural e o ritmo cardíaco \cite{vergara2024exploring}.

\section{Neuromodulação e Modulação da Função Cerebral em Contextos Esportivos e Clínicos}
A neuromodulação não invasiva representa um campo em rápida expansão, oferecendo ferramentas poderosas para investigar e modular a atividade cerebral em diversos contextos. Esta seção apresenta uma revisão sistemática das técnicas de neuromodulação, partindo dos princípios neurofisiológicos fundamentais e avançando para aplicações específicas em ambientes clínicos, esportivos e interpessoais. Inicialmente, exploramos as bases neurofisiológicas da estimulação transcraniana por corrente contínua (tDCS) e suas variantes, como a HD-tDCS, detalhando os mecanismos celulares e de rede pelos quais estas técnicas modulam a excitabilidade cortical e a conectividade funcional. Em seguida, examinamos evidências de sua eficácia em contextos clínicos, destacando como a neuromodulação pode normalizar circuitos neurais disfuncionais em diversas condições neuropsiquiátricas. Posteriormente, abordamos aplicações em contextos esportivos e interpessoais, onde estas técnicas demonstram potencial para otimizar o desempenho e modular interações sociais. Finalmente, discutimos abordagens multidimensionais e técnicas alternativas que expandem o horizonte investigativo e terapêutico da neuromodulação. Esta progressão lógica permite compreender como os princípios básicos da neuromodulação se traduzem em aplicações práticas e como diferentes abordagens podem ser integradas para uma compreensão mais completa da dinâmica cerebral e sua modulação.

\subsection{Técnicas Convencionais de Neuromodulação}
A estimulação transcraniana por corrente contínua (tDCS) emergiu como uma técnica não invasiva de neuromodulação capaz de induzir alterações controladas na excitabilidade cortical. O trabalho pioneiro de \citeonline{nitsche2000excitability} estabeleceu os princípios fundamentais desta técnica, demonstrando que correntes elétricas fracas (1-2 mA) aplicadas através do crânio podem modular a excitabilidade do córtex motor humano de maneira polaridade-dependente. Seus experimentos revelaram que a estimulação anódica aumenta a excitabilidade cortical, enquanto a catódica a reduz, com efeitos que persistem por vários minutos após o término da estimulação. Estas alterações foram quantificadas através de potenciais evocados motores (MEPs) induzidos por estimulação magnética transcraniana (TMS), evidenciando o potencial da tDCS como ferramenta para modular a atividade cerebral de forma seletiva, reversível e indolor.

Os mecanismos neurofisiológicos subjacentes a estes efeitos foram inicialmente investigados por \citeonline{purpura1965intracellular}, que demonstraram, através de registros intracelulares, que a polarização anódica despolariza os corpos celulares de neurônios piramidais, aumentando sua excitabilidade, enquanto a polarização catódica os hiperpolariza, reduzindo sua atividade espontânea. Este trabalho fundamental revelou ainda que os efeitos da estimulação dependem da orientação das células em relação ao fluxo de corrente, explicando a especificidade dos efeitos observados em diferentes populações neuronais e regiões corticais.

Avançando na compreensão destes mecanismos, \citeonline{stagg2011physiological} elucidaram como a tDCS influencia a plasticidade sináptica no neocórtex, estabelecendo paralelos com processos de potencialização de longo prazo (LTP) e depressão de longo prazo (LTD). Sua revisão destacou o papel crucial de neurotransmissores como glutamato, GABA, dopamina e serotonina na mediação dos efeitos da tDCS, fornecendo uma base neuroquímica para compreender como esta técnica modifica a aprendizagem motora e a conectividade funcional cerebral.

No contexto brasileiro, \citeonline{okano2013estimulacao} contribuíram significativamente ao revisar as aplicações da tDCS na promoção da saúde e melhoria do desempenho físico. Seu trabalho sistematizou evidências sobre como a tDCS pode modular funções cardiovasculares, controle de apetite e percepção de esforço, destacando o potencial desta técnica como estratégia complementar para otimização do desempenho esportivo e intervenções terapêuticas.

Para superar limitações de focalidade da tDCS convencional, foi desenvolvida a estimulação transcraniana por corrente contínua de alta definição (HD-tDCS). \citeonline{villamar2013hdtdcs} descreveram a configuração \emph{4$\times$1}, na qual um eletrodo central (anódico ou catódico) é circundado por quatro eletrodos de retorno, permitindo uma estimulação mais focal e reduzindo a dispersão de corrente. Modelagens computacionais e estudos clínicos confirmaram que esta técnica restringe a modulação da excitabilidade cortical à região-alvo, produzindo efeitos mais específicos e potencialmente mais duradouros que a tDCS convencional.

Expandindo o escopo das técnicas de neuromodulação, \citeonline{kunze2014high} investigaram os efeitos da HD-tDCS no cérebro utilizando EEG simultâneo, demonstrando que a estimulação do córtex sensorimotor esquerdo gera mudanças agudas e persistentes na sincronização cortical. Estas alterações incluem modificações globais e locais na sincronização neural, com efeitos distintos para estimulação anódica e catódica, especialmente na sincronização relacionada à imaginação motora.

Complementando esta perspectiva, \citeonline{scheler2019neuromodulation} exploraram como a neuromodulação influencia a sincronização neural e a capacidade intrínseca de leitura em redes neuronais. Seu trabalho demonstrou que a neuromodulação pode transformar as propriedades topológicas de redes neurais, alterando a distribuição de conexões e influenciando propriedades intrínsecas dos neurônios através da modulação de canais iônicos específicos. Esta abordagem teórica fornece um arcabouço para compreender como intervenções neuromodulatórias podem ajustar o equilíbrio entre sincronização e heterogeneidade neural, afetando a capacidade de processamento de informações no córtex cerebral.

Além da tDCS, outras técnicas não invasivas como a estimulação transcraniana por corrente alternada (tACS) e a estimulação magnética transcraniana repetitiva (rTMS) também têm demonstrado capacidade de modificar a sincronização neural e a conectividade funcional, contribuindo para uma compreensão mais abrangente de como a neuromodulação pode ser utilizada para ajustar a dinâmica das redes cerebrais em contextos clínicos e experimentais.

\subsection{Aplicações Clínicas}
A aplicação de técnicas de neuromodulação em contextos clínicos tem revelado seu potencial terapêutico para uma variedade de condições neuropsiquiátricas, com evidências crescentes de sua capacidade de restaurar circuitos neurais disfuncionais e modular padrões de atividade cerebral patológicos. Estas intervenções têm sido particularmente promissoras no tratamento de transtornos resistentes às abordagens farmacológicas convencionais.

No campo dos transtornos de humor, \citeonline{singh2024evaluating} conduziram uma investigação inovadora sobre os efeitos da tDCS em pacientes com transtorno depressivo maior (MDD), utilizando análises avançadas de conectividade funcional em dados de EEG em repouso. Seus resultados revelaram que a tDCS induz modificações significativas na topologia das redes cerebrais, particularmente na banda beta, indicando uma redução na randomização e um aumento na propriedade de "small-worldness" após a intervenção. Esta reorganização da arquitetura funcional cerebral é consistente com a hipótese de que pacientes com depressão apresentam redes neurais excessivamente randomizadas, especialmente durante o processamento emocional. Os autores demonstraram, através de análises de Phase Lag Index (PLI) e medidas de "hubness", que a tDCS pode normalizar estes padrões disfuncionais, reduzindo a hiperconectividade nas bandas theta e alpha – associadas ao estado de relaxamento – e aumentando a conectividade na banda beta – associada ao estado de alerta. Estas alterações correlacionaram-se com a melhora clínica, sugerindo mecanismos neurobiológicos específicos pelos quais a tDCS pode exercer seus efeitos antidepressivos.

Em pacientes com epilepsia refratária, \citeonline{toutant2024hdtdcs} demonstraram que a HD-tDCS catódica pode atuar como uma intervenção anti-epiléptica através da dessincronização de redes neurais hipersincrônicas. Seu estudo revelou uma redução significativa na sincronização de baixa frequência (bandas delta e theta) após a aplicação de HD-tDCS, particularmente nas regiões frontocentrais e parietais, acompanhada por um discreto aumento na atividade de alta frequência (bandas beta e gamma). Esta modulação do perfil espectral do EEG sugere uma alteração na excitabilidade cortical que pode interromper os padrões de sincronização excessiva característicos da atividade epileptiforme. Análises de conectividade revelaram ainda uma diminuição na coerência entre pares de eletrodos sobre o foco epiléptico e regiões adjacentes, indicando uma disrupção das redes hipersincrônicas que contribuem para a atividade convulsiva. Os efeitos mais pronunciados foram observados em pacientes com epilepsia focal no córtex frontal, sugerindo uma especificidade anatômica na resposta à neuromodulação.

Investigando os mecanismos fundamentais pelos quais a tDCS modifica a dinâmica cerebral, \citeonline{cukic2018shift} utilizaram medidas de Recurrence Quantification Analysis para caracterizar as mudanças no estado cerebral induzidas pela estimulação. Seu estudo com 16 indivíduos saudáveis demonstrou que a tDCS exerce efeitos específicos de polaridade sobre a dinâmica cortical: a estimulação catódica resultou em valores significativamente menores de Mean State Shift (MSS) em comparação com a anódica, indicando uma transição do sistema cerebral para diferentes regiões do espaço de estados. Além disso, a estimulação catódica afetou a State Variance (SV), enquanto a anódica não produziu alterações detectáveis neste parâmetro. Os autores propõem um modelo teórico baseado em princípios termodinâmicos, sugerindo que o cérebro em repouso ocupa um estado de energia mínima com alta probabilidade, e a estimulação desloca o sistema para um estado de maior energia e menor probabilidade. Esta perspectiva oferece um arcabouço conceitual para compreender como intervenções neuromodulatórias podem induzir transições entre diferentes estados cerebrais, com implicações para o tratamento de condições caracterizadas por padrões disfuncionais de atividade neural.

Em uma abordagem mais ampla, \citeonline{dong2023efficacy} realizaram uma revisão sistemática e meta-análise sobre a eficácia da estimulação cerebral não invasiva (NIBS) em pacientes com transtornos de consciência (DoC). Analisando 17 estudos randomizados controlados com 377 pacientes, os autores concluíram que a NIBS melhora significativamente o estado de consciência em comparação com estimulação simulada. A análise por subgrupos revelou que a estimulação magnética transcraniana repetitiva (rTMS) aplicada ao córtex pré-frontal dorsolateral esquerdo (DLPFC) foi a modalidade mais eficaz, com efeitos mais pronunciados em pacientes no estado de consciência mínima (MCS) do que naqueles com síndrome de vigília não responsiva (UWS/VS). Além disso, protocolos com múltiplas sessões demonstraram resultados superiores aos tratamentos únicos, sugerindo um efeito dose-dependente e de longa duração. Estes achados destacam o potencial da neuromodulação não invasiva como uma abordagem terapêutica para melhorar a consciência em pacientes com DoC, especialmente quando aplicada de forma sistemática e direcionada a regiões cerebrais específicas.

Coletivamente, estes estudos demonstram que as técnicas de neuromodulação não invasiva podem induzir alterações significativas na atividade cerebral e na conectividade funcional, com potencial para normalizar circuitos neurais disfuncionais em diversas condições neuropsiquiátricas. A especificidade dos efeitos observados – dependentes da polaridade da estimulação, da região cerebral alvo, do estado clínico do paciente e dos parâmetros de estimulação – sugere que estas intervenções podem ser personalizadas para abordar mecanismos patofisiológicos específicos, abrindo caminho para abordagens terapêuticas mais precisas e eficazes.

\subsection{Aplicações em Contextos Esportivos, Emocionais e Interpessoais}
A aplicação de técnicas de neuromodulação tem transcendido o domínio clínico, expandindo-se para contextos esportivos, emocionais e interpessoais, onde demonstra potencial para otimizar o desempenho humano e modular interações sociais. Esta diversificação de aplicações reflete a versatilidade destas técnicas e sua capacidade de influenciar múltiplos aspectos do funcionamento cerebral.

No âmbito esportivo, \citeonline{valenzuela2019enhancement} conduziram um estudo cruzado, duplo-cego e controlado por placebo com triatletas de elite, investigando os efeitos da tDCS anódica sobre o córtex motor. Embora a estimulação não tenha melhorado o desempenho físico durante testes de natação de 800 metros, os autores observaram um aumento significativo no vigor percebido tanto antes quanto após o exercício, com tamanhos de efeito expressivos (1,14 e 0,98, respectivamente). Estes resultados sugerem que a tDCS pode exercer efeitos seletivos sobre aspectos psicológicos do desempenho atlético, potencialmente através da modulação de circuitos neurais envolvidos na regulação do humor e na percepção de esforço, mesmo quando medidas objetivas de desempenho permanecem inalteradas.

Complementando esta perspectiva, estudos com estimulação transcraniana por corrente alternada (tACS) têm revelado efeitos promissores sobre funções cognitivas relevantes para o desempenho esportivo. \citeonline{rostami2020transcranial} demonstraram que a tACS aplicada a 6 Hz sobre o córtex pré-frontal medial (mPFC) melhora significativamente a atenção sustentada, evidenciada por aumentos no número total de acertos e na sensibilidade ao alvo durante tarefas de processamento rápido de informações visuais. Análises de EEG revelaram que estes ganhos comportamentais foram acompanhados por um aumento na densidade espectral de potência na faixa theta nas regiões fronto-centrais e por modulações na sincronização de fase alfa na Rede de Atenção Dorsal (DAN). Estes achados são particularmente relevantes para modalidades esportivas que exigem atenção sustentada e processamento rápido de informações visuais, como esportes coletivos e de precisão.

Avançando na compreensão dos mecanismos subjacentes à modulação da atenção, \citeonline{spooner2020hdtdcs} utilizaram magnetoencefalografia (MEG) para investigar como a HD-tDCS aplicada ao córtex pré-frontal dorsolateral (DLPFC) altera a conectividade funcional dinâmica durante tarefas de atenção visual seletiva. Os autores observaram que a estimulação modifica especificamente a conectividade na banda theta entre regiões frontais e visuais, sugerindo um mecanismo pelo qual a neuromodulação pode influenciar processos atencionais fundamentais para o desempenho cognitivo-motor. Esta modulação da conectividade fronto-visual pode ser particularmente relevante para atletas que dependem de processos atencionais eficientes, como aqueles envolvidos em esportes de alta velocidade ou que exigem tomadas de decisão rápidas.

Em contextos interpessoais, a neuromodulação tem demonstrado capacidade de influenciar processos sociais complexos. \citeonline{long2023transcranial} investigaram como a tDCS aplicada ao lobo temporal anterior direito (rATL) afeta a sincronização neural interpessoal (INS) em casais durante interações sociais. Surpreendentemente, a estimulação reduziu a INS entre o rATL das mulheres e o córtex sensorimotor (SMC) dos homens, acompanhada por uma diminuição nos níveis de empatia emocional. Esta modulação da INS impactou a empatia através de comportamentos não verbais, sem alterar os padrões de interação verbal. Estes resultados sugerem que a INS está associada indiretamente aos processos mentais compartilhados através de inputs sensório-motores, apoiando a teoria de representações compartilhadas durante interações sociais e destacando o potencial da neuromodulação para investigar e potencialmente modular processos sociais fundamentais.

No contexto de reabilitação neurológica, \citeonline{liu2023effects} demonstraram que a tDCS pode induzir alterações significativas na potência do EEG e nas redes funcionais cerebrais em pacientes pós-AVC. Especificamente, a estimulação reduziu a potência das oscilações delta e enfraqueceu a conectividade global da rede delta, enquanto aumentou a potência das oscilações alfa e melhorou a conectividade global e local da rede alfa. Estas alterações nos padrões oscilatórios e na arquitetura de rede sugerem um possível mecanismo neurofisiológico pelo qual a tDCS pode contribuir para a recuperação funcional após lesões cerebrais, com potenciais aplicações em programas de reabilitação neurológica e treinamento cognitivo-motor.

De forma similar, \citeonline{han2022functional} investigaram os efeitos da HD-tDCS no DLPFC esquerdo em pacientes com distúrbios crônicos de consciência (DOC). Após dez sessões de estimulação, os pacientes que responderam ao tratamento apresentaram aumentos significativos na conectividade funcional em regiões frontais e parieto-occipitais nas bandas theta (4-8 Hz) e alfa (8-13 Hz), correlacionados com melhorias nos escores clínicos. Notavelmente, pacientes com aumento de conectividade funcional na banda alfa demonstraram melhor prognóstico a longo prazo, sugerindo que a HD-tDCS pode melhorar a consciência através da modulação da conectividade funcional cerebral e que padrões específicos de resposta neural podem servir como biomarcadores prognósticos.

Expandindo as aplicações para contextos domiciliares, \citeonline{xiao2025enhanced} investigaram o uso de tDCS domiciliar em pacientes com depressão bipolar, analisando alterações na sincronização cortical e desenvolvendo preditores de remissão clínica baseados em aprendizado profundo. Os resultados revelaram que a tDCS aumentou significativamente o Phase Locking Value (PLV) delta nas regiões frontal e temporoparietal, correlacionando-se com a melhora nos sintomas depressivos. Além disso, o PLV beta aumentou em pacientes que atingiram remissão e diminuiu naqueles que não responderam ao tratamento. Notavelmente, os autores conseguiram prever a remissão clínica com 69,45% de acurácia utilizando medidas basais de PLV nas bandas theta, beta e gamma, destacando o potencial de medidas de sincronização neural como biomarcadores preditivos da resposta ao tratamento.

Finalmente, \citeonline{arif2021high} investigaram como a HD-tDCS anódica aplicada ao DLPFC direito e esquerdo afeta a inteligência fluida e a conectividade parieto-frontal. A estimulação do DLPFC direito resultou em tempos de resposta significativamente mais rápidos em tarefas de raciocínio lógico, acompanhados por uma redução na conectividade parieto-frontal esquerda que se correlacionou positivamente com o desempenho. Estes achados corroboram a hipótese de eficiência neural, sugerindo que a estimulação do DLPFC direito pode melhorar a eficiência cognitiva ao reduzir o esforço neural necessário para o processamento de informações. Esta modulação da conectividade funcional entre regiões frontais e parietais pode ter implicações significativas para a otimização do desempenho cognitivo em contextos que exigem raciocínio lógico e tomada de decisão, como certos aspectos do desempenho esportivo e interações sociais complexas.

Coletivamente, estes estudos demonstram o potencial diversificado da neuromodulação para influenciar aspectos psicológicos, cognitivos e sociais do funcionamento humano, transcendendo aplicações puramente clínicas e abrindo novas perspectivas para a otimização do desempenho e bem-estar em múltiplos domínios da experiência humana.

\subsection{Abordagens Multidimensionais, Modelos Matemáticos e Monitoramento Integrado}
À medida que o campo da neuromodulação avança, torna-se evidente que abordagens unidimensionais são insuficientes para capturar a complexidade dos efeitos destas intervenções sobre a dinâmica cerebral. Neste contexto, pesquisadores têm desenvolvido metodologias multidimensionais que integram análises quantitativas de EEG, modelos matemáticos sofisticados e técnicas avançadas de monitoramento, permitindo uma compreensão mais profunda e nuançada dos mecanismos subjacentes à neuromodulação.

\citeonline{zhang2022multidimensional} realizaram uma investigação pioneira sobre os efeitos da HD-tDCS no córtex parietal posterior (Pz) em pacientes com distúrbios de consciência (DOC), empregando uma avaliação multidimensional das métricas de EEG. Após 14 dias de estimulação, os pacientes que responderam ao tratamento apresentaram aumentos significativos na conectividade funcional e nas métricas de rede cerebral, particularmente nas bandas alfa e beta. Estas alterações sugerem uma maior integração e eficiência nas redes neurais, potencialmente refletindo a restauração parcial de circuitos de consciência. Notavelmente, os autores desenvolveram um modelo preditivo baseado em aprendizado de máquina que alcançou 92,9\% de acurácia na identificação de pacientes responsivos, destacando a complexidade espacial normalizada (NSC) como um preditor robusto da resposta ao tratamento. Este estudo não apenas reforça a importância do córtex parietal posterior na manutenção da consciência, mas também demonstra o valor de abordagens multidimensionais na caracterização e previsão dos efeitos da neuromodulação.

Explorando a interface entre neuromodulação e plasticidade cognitiva, \citeonline{jones2017frontoparietal} investigaram como a tDCS anódica combinada com treinamento de memória de trabalho (WM) modula a atividade oscilatória cerebral e o desempenho cognitivo. Utilizando EEG de alta densidade, os autores demonstraram que participantes que receberam tDCS ativa direcionada às redes frontoparietais durante o treinamento apresentaram melhorias significativas no desempenho de WM, enquanto o grupo com estimulação simulada não mostrou mudanças. Estes ganhos comportamentais foram acompanhados por alterações eletrofisiológicas específicas: redução da potência alfa posterior e aumento da sincronia de fase nas bandas alfa e teta. Estas mudanças sugerem uma maior eficiência no processamento neural e um fortalecimento da conectividade funcional nas redes relevantes para a memória de trabalho. O estudo destaca como a combinação de neuromodulação e treinamento cognitivo pode induzir alterações sinérgicas na dinâmica cerebral, potencializando a plasticidade neural e a aprendizagem.

A capacidade da tDCS de remodelar redes cerebrais em repouso foi elegantemente demonstrada por \citeonline{pellegrino2018bilateral}, que utilizaram magnetoencefalografia (MEG) para investigar os efeitos da estimulação bilateral em indivíduos saudáveis. Com o ânodo posicionado sobre o córtex sensório-motor esquerdo e o cátodo no direito, a tDCS real, em comparação com a simulada, induziu uma redução na potência das frequências alfa, beta e gama no córtex frontal esquerdo, acompanhada por um aumento na conectividade global em múltiplas bandas de frequência (delta, alfa, beta e gama). Notavelmente, estes efeitos não se limitaram às regiões diretamente sob os eletrodos, mas se estenderam a áreas distantes, sugerindo uma reorganização ampla das redes cerebrais. Estes resultados destacam o potencial da tDCS para induzir plasticidade cerebral de longo alcance, com implicações significativas para o desenvolvimento de intervenções terapêuticas personalizadas em condições neurológicas.

Para elucidar os mecanismos biofísicos subjacentes aos efeitos da tDCS, \citeonline{riedinger2022model} desenvolveram um modelo matemático sofisticado do circuito córtico-talâmico-cortical (CTC), incorporando o Sistema Reticular Ascendente (ARAS). Este modelo teórico explica como a tDCS pode modular a excitabilidade cerebral em estimulações de curta duração e a potência do EEG em estímulos prolongados, estabelecendo conexões com processos de plasticidade de longo prazo (LTP). Aplicando este modelo a um paradigma de psicose induzida por cetamina, os autores conseguiram reproduzir as alterações de potência no EEG observadas experimentalmente sob tDCS, corroborando a hipótese da disfunção dos receptores NMDA na esquizofrenia. O trabalho destaca o papel crítico do ARAS e da sincronização do ritmo delta no circuito CTC, oferecendo insights valiosos sobre os mecanismos neurobiológicos da psicose precoce e como a neuromodulação pode normalizar circuitos disfuncionais.

Avançando para paradigmas de controle mais sofisticados, \citeonline{zhang2024closed} investigaram abordagens de controle em loop fechado para oscilações gama através de estimulações transcranianas. Utilizando um modelo de rede neural cortical e análises de EEG pré e pós-estimulação, os autores demonstraram que estimulações prolongadas, tanto por tDCS quanto por rTMS, podem aumentar significativamente as oscilações gama, promovendo a liberação de fator neurotrófico derivado do cérebro (BDNF) por astrócitos e, consequentemente, melhorando as conexões neuronais. Este mecanismo oferece uma explicação para os efeitos promotores destas intervenções em lesões traumáticas e doenças neurodegenerativas, estabelecendo uma ligação mecanística entre a neuromodulação, a plasticidade sináptica e a recuperação funcional.

A aplicação de análises de grafos tem proporcionado insights valiosos sobre como a tDCS modula a sincronização cortical e a organização topológica das redes cerebrais. \citeonline{mancini2016assessing} utilizaram EEG de 19 canais e métricas baseadas em Synchronization Likelihood (SL) para avaliar os efeitos imediatos da tDCS em redes funcionais cerebrais durante o repouso. Seus resultados revelaram efeitos específicos de polaridade: a tDCS anodal reduziu a sincronização em áreas frontocentrais na banda teta, enquanto a catodal aumentou a conectividade inter-hemisférica em áreas parieto-occipitais na banda alfa. Estas alterações na sincronização cortical foram acompanhadas por modificações nas propriedades de redes funcionais locais e globais, demonstrando o potencial da tDCS para modular a dinâmica de redes cerebrais de forma dependente da polaridade.

Complementando esta linha de investigação, \citeonline{pellegrino2019transcranial} focaram especificamente nos efeitos da tDCS na sincronização gama cortical, um ritmo neural crucial para diversos processos cognitivos. Utilizando MEG e estimulação auditiva de 40 Hz em um experimento randomizado, controlado por placebo e duplo-cego, os autores observaram que a tDCS bilateral (ânodo na região sensório-motora esquerda, cátodo na direita) reduziu significativamente a sincronização gama em 13 dos 15 participantes. Notavelmente, esta redução foi mais pronunciada em áreas distantes do local de estimulação, como o córtex centro-temporal direito, enquanto a sincronização gama basal e as respostas auditivas iniciais permaneceram inalteradas. Estes resultados sugerem que a tDCS inibe seletivamente a sincronização gama induzida externamente, destacando seu potencial para modular mecanismos de plasticidade cortical.

Em um contexto clínico, \citeonline{schollmann2019anodal} investigaram como a tDCS anódica (atDCS) modula a atividade cortical e a sincronização em pacientes com doença de Parkinson. Em um estudo duplo-cego controlado por placebo com 11 pacientes e 10 controles saudáveis, a atDCS aplicada sobre a área sensório-motora esquerda durante uma tarefa de precisão motora melhorou significativamente os sintomas motores e modulou a atividade e sincronização cortical na faixa beta alta (22-27 Hz). Especificamente, observou-se uma redução da atividade no córtex sensório-motor esquerdo e um aumento da sincronização córtico-cortical durante a execução da tarefa motora. Crucialmente, estes efeitos foram específicos do contexto, ocorrendo apenas durante o processamento motor ativo e não durante o repouso ou na condição placebo. Estes achados sugerem que a atDCS pode ajustar disfunções no circuito motor cortical de forma dependente do estado, oferecendo uma abordagem promissora para a reabilitação motora em condições neurodegenerativas.

Para facilitar a investigação em tempo real dos efeitos da neuromodulação, \citeonline{schesatsky2013simultaneous} desenvolveram um dispositivo inovador que permite o monitoramento simultâneo de EEG durante a aplicação de tDCS. Esta metodologia avançada possibilita a avaliação contínua da excitabilidade cortical durante a estimulação, fornecendo informações valiosas sobre os mecanismos de ação da tDCS e permitindo a otimização em tempo real dos parâmetros de estimulação. Esta abordagem integrada representa um avanço significativo na instrumentação para pesquisa em neuromodulação, facilitando estudos mais precisos e personalizados dos efeitos neurais da estimulação transcraniana.

Coletivamente, estas abordagens multidimensionais, modelos matemáticos e técnicas de monitoramento integrado têm expandido significativamente nossa compreensão dos mecanismos subjacentes à neuromodulação, revelando a complexidade e especificidade dos efeitos destas intervenções sobre a dinâmica cerebral. A integração de múltiplas modalidades de análise e a aplicação de frameworks teóricos sofisticados continuam a impulsionar o desenvolvimento de protocolos de neuromodulação mais eficazes e personalizados, com potencial para transformar o tratamento de diversas condições neurológicas e psiquiátricas.

\subsection{Abordagens Alternativas e Complementares}
Enquanto as técnicas convencionais de neuromodulação continuam a evoluir, o campo tem testemunhado o surgimento de abordagens alternativas e complementares que expandem o horizonte terapêutico e investigativo. Estas novas perspectivas não apenas oferecem caminhos adicionais para modular a atividade cerebral, mas também proporcionam insights únicos sobre os mecanismos neurobiológicos subjacentes à sincronização neural e sua relevância para a cognição, comportamento e estados patológicos.

\subsubsection{Sincronia Neural Interpessoal e Transtornos do Neurodesenvolvimento}
Um domínio particularmente promissor é o estudo da sincronia neural interpessoal (INS) – a coordenação temporal entre os sinais cerebrais de duas pessoas durante interações sociais. \citeonline{boecker2024interpersonal} realizaram uma análise abrangente deste fenômeno e seu potencial terapêutico em transtornos caracterizados por disfunções nas interações sociais. Sua revisão revelou que intervenções combinando estimulação cerebral e neurofeedback podem aprimorar significativamente a coordenação de sinais neurais durante interações sociais, com aplicações particularmente relevantes para transtornos do espectro autista (ASD), transtorno de apego reativo (RAD) e transtorno de ansiedade social (SAD). Os autores destacam que, embora estas abordagens mostrem resultados promissores para aumentar a sincronia comportamental e a conexão social, os protocolos ideais de estimulação e parâmetros de neurofeedback ainda precisam ser refinados através de investigações sistemáticas.

Aprofundando a compreensão da sincronização interpessoal no autismo, \citeonline{mcnaughton2020interpersonal} conduziram uma revisão abrangente abordando quatro domínios críticos: motor, conversacional, fisiológico e neural. Seus achados indicam que, embora a sincronização esteja presente em indivíduos com autismo, ela se manifesta de forma reduzida ou atípica em comparação com controles neurotípicos. Esta redução pode ser atribuída tanto a mecanismos intraindividuais, como controle motor e processamento temporal atípicos, quanto a fatores interindividuais, como diferenças no acoplamento social. Significativamente, os autores propõem que a sincronização interpessoal pode servir como um biomarcador valioso para estratificação e intervenção em autismo, especialmente com o advento de tecnologias como sensores vestíveis e análise automatizada de vídeo, que viabilizam novas abordagens para quantificar a sincronização em contextos naturalísticos.

Em contraste com esta perspectiva focada no neurodesenvolvimento, \citeonline{baldwin2014evidence} oferecem uma visão complementar ao revisar as diretrizes da British Association for Psychopharmacology para o tratamento farmacológico de transtornos de ansiedade, TEPT e TOC. Embora seu foco principal seja a farmacoterapia, os autores enfatizam a importância de abordagens não farmacológicas, como a terapia cognitivo-comportamental, que podem ser tão eficazes quanto medicamentos para muitos transtornos. Esta perspectiva reforça a necessidade de considerar múltiplas modalidades terapêuticas, incluindo técnicas de neuromodulação, dentro de um framework de tratamento integrado que considere a gravidade dos sintomas, comorbidades e preferências individuais dos pacientes.

Avançando metodologicamente na caracterização da sincronização neural interpessoal, \citeonline{gerloff2022autism} exploraram a aplicação de técnicas de aprendizado de máquina para classificar indivíduos com TEA com base em biomarcadores neurais diádicos. Utilizando dados de EEG hyperscanning durante interações sociais, os autores demonstraram que representações gráficas complexas derivadas da sincronização neural, especialmente quando analisadas através de métodos de aprendizado não supervisionado, podem melhorar substancialmente a discriminação entre grupos TEA e controles. Este avanço metodológico sugere que a modelagem baseada em grafos de dados interpessoais oferece uma abordagem promissora para desenvolver biomarcadores mais sensíveis para transtornos do neurodesenvolvimento.

Corroborando a relevância funcional da sincronização neural interpessoal, \citeonline{quinones2021dysfunction} investigaram especificamente se déficits nesta sincronização poderiam explicar dificuldades de comunicação em adultos com autismo. Utilizando espectroscopia funcional no infravermelho próximo (fNIRS) durante conversas naturais, os pesquisadores observaram que indivíduos com autismo apresentaram menor sincronização neural no giro temporoparietal (TPJ) em comparação com controles, e que esta redução estava associada a maiores dificuldades autodeclaradas na comunicação social. Estes achados sugerem que a disfunção na coordenação de respostas cerebrais com um parceiro social constitui um mecanismo biológico subjacente às dificuldades sociais no autismo, destacando a importância de estudar a atividade neural em contextos sociais dinâmicos.

Expandindo esta linha de investigação, \citeonline{key2022greater} exploraram a relação entre competência social e sincronização neural interpessoal em adolescentes com autismo, utilizando EEG hyperscanning durante interações sociais naturais. Seus resultados revelaram que a sincronização neural entre participantes com autismo e confederados neurotípicos foi significativamente maior durante conversas do que em repouso, e que esta sincronização em bandas de frequência theta, alpha e beta correlacionou-se negativamente com a gravidade dos sintomas sociais de autismo e positivamente com indicadores de melhor funcionamento social. Notavelmente, o aumento da sincronização neural foi mais consistente entre participantes do sexo feminino, sugerindo possíveis diferenças relacionadas ao sexo na expressão de habilidades sociais em indivíduos com autismo.

Investigando os mecanismos neurais subjacentes às dificuldades de interação social no autismo, \citeonline{tanabe2012hard} utilizaram fMRI dupla (hiperscanning) para examinar as bases neurais da interação olho no olho em indivíduos com TEA de alto funcionamento. Comparando pares TEA-controle com pares controle-controle, os autores observaram que a coerência inter-brain na região do giro frontal inferior direito (IFG) – associada à intenção compartilhada – estava reduzida nos pares TEA-controle. Além disso, a conectividade funcional intra-cérebro entre IFG e sulco temporal superior (STS) também foi mais fraca nos parceiros neurotípicos de indivíduos TEA, correlacionando-se positivamente com o desempenho em tarefas de atenção conjunta. Estes achados sugerem que a falha na sincronização neural entre parceiros pode constituir um mecanismo fundamental para as dificuldades de interação social no TEA, oferecendo um alvo potencial para intervenções neuromodulatórias.

\subsubsection{Neurofeedback como Abordagem Terapêutica}
Paralelamente aos avanços na compreensão da sincronização neural interpessoal, o neurofeedback tem emergido como uma abordagem terapêutica promissora para diversos transtornos neuropsiquiátricos. \citeonline{hou2021neurofeedback} investigaram a eficácia do treinamento de neurofeedback para aumentar a atividade alfa no lobo parietal em pacientes com transtorno de ansiedade generalizada (TAG). Em um estudo randomizado com 26 mulheres diagnosticadas com TAG, os autores demonstraram reduções significativas nos escores de ansiedade-estado, ansiedade-traço, depressão e insônia após dez sessões de treinamento, com melhorias que continuaram a aumentar quatro semanas após a intervenção. Estes resultados sugerem que o aumento da atividade alfa no córtex parietal pode modular efetivamente a atenção e reduzir sintomas ansiosos, possivelmente através da diminuição do viés atencional para estímulos ameaçadores.

Complementando esta abordagem, \citeonline{zilverstand2015fmri} exploraram o potencial do neurofeedback por fMRI para facilitar a regulação da ansiedade em mulheres com fobia de aranhas. Utilizando uma abordagem baseada em reavaliação cognitiva, os pesquisadores demonstraram que participantes que receberam feedback em tempo real sobre a ativação da ínsula direita e do córtex pré-frontal dorsolateral esquerdo conseguiram reduzir mais efetivamente a ativação da ínsula e relataram menor ansiedade ao final do treinamento, em comparação com o grupo controle. Notavelmente, as mudanças na ativação da ínsula durante o treinamento previram melhorias de longo prazo na redução da fobia, medidas até três meses depois, sugerindo que o neurofeedback pode acelerar e reforçar o aprendizado de estratégias de regulação emocional em transtornos de ansiedade.

Estendendo a aplicação do neurofeedback ao transtorno de ansiedade social, \citeonline{kimmig2019feasibility} avaliaram a viabilidade de um treinamento baseado em espectroscopia funcional no infravermelho próximo (NIRS) para modular a atividade do córtex pré-frontal dorsolateral (dlPFC). Após 15 sessões, os participantes demonstraram melhora significativa no desempenho do neurofeedback, redução dos sintomas de ansiedade social, ansiedade traço e depressão, além de diminuição do viés atencional para estímulos sociais ameaçadores. Particularmente relevante foi a observação de que a presença inicial de dificuldades em regular a atividade do dlPFC na presença de estímulos ameaçadores foi superada ao longo do treinamento, e que melhorias no neurofeedback correlacionaram-se com mudanças em padrões de ativação cerebral. A aceitabilidade desta abordagem foi alta, com 9 dos 12 participantes recomendando o tratamento, sugerindo seu potencial como intervenção terapêutica viável.

No contexto do autismo, \citeonline{direito2021training} investigaram a viabilidade e os efeitos de um treinamento de neurofeedback baseado em fMRI em tempo real, direcionado ao sulco temporal superior posterior (pSTS). Após cinco sessões ao longo de oito semanas, os 15 participantes com autismo demonstraram capacidade de modular a atividade do pSTS, com efeitos positivos imediatos e sustentados após seis meses, incluindo melhorias na identificação de expressões de medo e em medidas de comportamento adaptativo. A análise de neuroimagem revelou recrutamento de redes neurais relacionadas a saliência, controle emocional e aprendizado operante, como a ínsula anterior, o córtex cingulado anterior e o corpo do estriado. A ausência de eventos adversos e a alta taxa de adesão reforçam a viabilidade desta abordagem para populações com autismo.

Abordando desafios metodológicos específicos, \citeonline{steiner2014pilot} avaliaram a viabilidade de um protocolo padronizado de neurofeedback para crianças com autismo de alto funcionamento e dificuldades de atenção. Utilizando o sistema Play Attention®, os autores demonstraram que, após seis semanas de treinamento intensivo, a maioria das crianças conseguiu reduzir comportamentos mal-adaptativos, melhorar a concentração durante tarefas acadêmicas e aumentar o tempo de atenção nas sessões de neurofeedback. Embora a adaptação inicial tenha sido desafiadora, o uso de reforço positivo, pausas e exercícios de respiração auxiliou a manter o engajamento, resultando em melhorias significativas em testes de desempenho acadêmico e controle de resposta.

Para populações com autismo e deficiência intelectual, que tradicionalmente são excluídas de intervenções que exigem alta cooperação metodológica, \citeonline{lamarca2018facilitating} investigaram a viabilidade do uso da metodologia TAGteach para preparar estas crianças para participação em treinamento de neurofeedback. O estudo demonstrou que, em média, após cinco horas de TAGteach, crianças com QI ≤ 80 aprenderam habilidades pré-requisito necessárias, como tolerar a preparação de eletrodos e manter atenção visual. Esta abordagem representa um avanço significativo na inclusão de populações mais severamente afetadas em pesquisas e tratamentos baseados em neurofeedback.

Em uma perspectiva mais ampla, \citeonline{catala2017pharmacological} realizaram uma revisão sistemática com meta-análises em rede para comparar a eficácia e segurança de diversos tratamentos para TDAH. Analisando 190 ensaios randomizados com 26.114 participantes, os autores observaram que, embora intervenções como neurofeedback não tenham mostrado evidência convincente de eficácia em comparação com terapia comportamental e medicamentos, apresentaram taxas de aceitação comparáveis a outras intervenções, sugerindo seu potencial como opção terapêutica complementar. Esta perspectiva é particularmente relevante considerando as análises econômicas apresentadas por \citeonline{arnold2013eeg}, que estimam o custo total de uma intervenção de neurofeedback em aproximadamente 1.500 USD para 24 sessões, tornando-a potencialmente acessível em comparação com outras modalidades terapêuticas.

\subsubsection{Potencialização de Oscilações Neurais}
Uma terceira vertente de abordagens alternativas foca na potencialização direta de oscilações neurais específicas. \citeonline{maiella2022simultaneous} investigaram os efeitos da estimulação simultânea por corrente alternada transcraniana (tACS) e estimulação magnética transcraniana (TMS) na modulação das oscilações gama no córtex pré-frontal dorsolateral (DLPFC). Aplicando tACS em 40 Hz juntamente com TMS em participantes saudáveis, os pesquisadores demonstraram um aumento significativo nas oscilações gama no DLPFC após a estimulação combinada. Estes resultados sugerem que esta abordagem sinérgica pode potencializar a plasticidade neural de forma mais eficaz que cada técnica isoladamente, com implicações terapêuticas para distúrbios que envolvem disfunções nas oscilações gama.

Complementarmente, \citeonline{zrenner2020brain} demonstraram a viabilidade e segurança da rTMS sincronizada com oscilações alfa em tempo real no córtex pré-frontal dorsolateral esquerdo (DLPFC) para pacientes com depressão resistente a antidepressivos. Esta abordagem inovadora, que ajusta a estimulação ao estado oscilatório instantâneo do cérebro, reduziu a atividade alfa em repouso, aumentou oscilações beta induzidas por TMS e mostrou efeitos neuromodulatórios específicos não observados em protocolos tradicionais. Embora os resultados sejam preliminares, baseados em uma única sessão, a técnica destaca o potencial de tratamentos personalizados baseados no estado cerebral, representando um avanço significativo na precisão das intervenções neuromodulatórias.

Integrando estas diversas linhas de pesquisa, \citeonline{konrad2024interpersonal} exploraram como a sincronização neural interpessoal pode ser aplicada no tratamento de transtornos mentais caracterizados por disfunções sociais. Os autores destacam que, apesar dos avanços em tecnologias como neurofeedback e estimulação cerebral para manipular a INS, os estudos ainda estão em estágio inicial e carecem de comprovação robusta sobre eficácia e aplicabilidade clínica. Abordagens indiretas, como intervenções comportamentais e biofeedback, mostram potencial, mas a integração da INS como alvo terapêutico exige investigações adicionais para otimizar protocolos, avaliar efeitos de longo prazo e adaptar tratamentos a contextos específicos. Esta perspectiva integrativa sugere que a INS pode abrir caminhos inovadores para melhorar a coesão social e o tratamento de disfunções sociais em diversos transtornos mentais, representando uma fronteira promissora na interface entre neuromodulação e intervenções psicossociais.

Coletivamente, estas abordagens alternativas e complementares expandem significativamente o arsenal terapêutico e investigativo disponível para modular a atividade cerebral e a sincronização neural. Ao transcender as limitações das técnicas convencionais e abordar aspectos únicos da função cerebral – como a sincronização interpessoal, a autorregulação neural e a potencialização de oscilações específicas – estas abordagens oferecem novas perspectivas sobre os mecanismos neurobiológicos subjacentes a diversos transtornos neuropsiquiátricos e abrem caminhos promissores para intervenções mais personalizadas e eficazes.

\section{Medidas Neurofisiológicas e Análise de Sincronização}
A investigação dos efeitos da HD-tDCS sobre a sincronização cerebral requer metodologias robustas para capturar e quantificar as complexas interações entre sistemas neurais e fisiológicos. Neste contexto, a integração de múltiplas técnicas de registro e análise torna-se fundamental para uma compreensão abrangente dos mecanismos subjacentes à neuromodulação e seus impactos na dinâmica corpo-cérebro.

A eletroencefalografia (EEG) constitui a espinha dorsal metodológica deste estudo, fornecendo dados de alta resolução temporal sobre a atividade neural. Conforme destacado por \citeonline{cohen2017where}, o sinal de EEG reflete principalmente a soma dos potenciais pós-sinápticos de populações de neurônios piramidais, permitindo a extração de oscilações que podem ser decompostas nas bandas clássicas – delta (1-4 Hz), theta (4-8 Hz), alpha (8-13 Hz), beta (13-30 Hz) e gamma (>30 Hz). Cada uma destas bandas está associada a escalas temporais e funções cognitivas ou comportamentais específicas, formando a base para análises de conectividade funcional e sincronização. A abordagem de teoria dos grafos, conforme descrita por \citeonline{bullmore2009complex}, permite caracterizar as propriedades topológicas destas redes neurais, identificando hubs (nós com alta centralidade) e padrões de conectividade que são cruciais para a comunicação neural eficiente.

Complementando o EEG, o eletrocardiograma (ECG) fornece informações precisas sobre os ritmos cardíacos, permitindo investigar a interação entre sistemas neural e cardiovascular. Em nosso protocolo, o sinal de ECG foi convertido em uma representação senoidal baseada no pico R, delimitando claramente o ciclo cardíaco e facilitando análises de sincronização com sinais cerebrais. Esta abordagem alinha-se com estudos recentes que destacam a importância da integração cardio-neural na regulação cognitiva e emocional. Em situações onde o ECG tradicional apresenta limitações, a eletromiografia (EMG) estrategicamente posicionada pode capturar a atividade do músculo peitoral maior, refletindo indiretamente a despolarização ventricular (complexo QRS) e permitindo uma integração mais próxima com outros sinais fisiológicos.

A análise do acoplamento de frequências cruzadas (\textit{cross-frequency coupling}, CFC) entre sinais de EEG e ECG representa uma abordagem particularmente inovadora neste estudo. Este fenômeno, no qual oscilações de diferentes frequências interagem entre si, possibilita compreender processos de integração neural e corporal em múltiplas escalas temporais. \citeonline{criscuolo2022cognition} demonstraram a relevância destas interações entre atividade cerebral e sinais periféricos (\textit{brain-body coupling}) na modulação da cognição, enquanto \citeonline{cohen2017where} destacou o papel das interações entre bandas rápidas (como gamma) e lentas (como theta) em processos cognitivos fundamentais. A investigação de como a HD-tDCS modifica estes padrões de acoplamento pode revelar mecanismos pelos quais a neuromodulação influencia a integração corpo-cérebro.

Para quantificar precisamente a sincronização entre sinais neurais e cardíacos, empregamos o Phase Locking Value (PLV), uma métrica robusta que avalia a consistência da relação de fase entre dois sinais ao longo do tempo. O PLV varia entre 0 (ausência de acoplamento) e 1 (acoplamento perfeito), permitindo identificar padrões de sincronização entre diferentes regiões cerebrais ou entre sinais cerebrais e cardíacos. \citeonline{singh2024evaluating} demonstraram a utilidade do PLV para avaliar alterações na conectividade funcional induzidas por intervenções neuromodulatórias, revelando como a tDCS pode reorganizar redes neurais e modificar padrões de sincronização em pacientes com transtorno depressivo maior. Em nosso estudo com atletas, o PLV permite quantificar como a HD-tDCS catódica sobre o DLPFC esquerdo modifica a sincronização entre oscilações cerebrais e o ritmo cardíaco, potencialmente revelando mecanismos de integração corpo-cérebro específicos desta população.

Complementando o PLV, aplicamos medidas de teoria dos grafos para caracterizar propriedades topológicas das redes funcionais cerebrais e sua modulação pela HD-tDCS. Métricas como centralidade de grau, centralidade de intermediação e centralidade de autovetor permitem identificar nós críticos (hubs) nas redes neurais e avaliar como a neuromodulação altera sua importância funcional. Medidas globais como eficiência de rede, coeficiente de agrupamento e comprimento de caminho característico fornecem insights sobre a organização global da rede e sua capacidade de integração e segregação funcional. Estas análises, fundamentadas no trabalho de \citeonline{bullmore2009complex}, permitem uma caracterização multidimensional dos efeitos da HD-tDCS sobre a arquitetura funcional do cérebro e sua interação com o sistema cardiovascular.

Em conjunto, estas abordagens metodológicas fornecem um arcabouço robusto para investigar como a HD-tDCS catódica sobre o DLPFC esquerdo modifica a sincronização entre sistemas neurais e cardiovasculares em atletas de elite, potencialmente revelando mecanismos neurobiológicos subjacentes à integração corpo-cérebro e suas implicações para o desempenho cognitivo e motor.

\section{Integração Conceitual e Fundamentação do Estudo}
A revisão da literatura apresentada nas seções anteriores estabelece um arcabouço teórico e metodológico robusto para a investigação dos efeitos da HD-tDCS catódica sobre o DLPFC esquerdo na sincronização cerebral em atletas de basquetebol feminino. A escolha da estimulação catódica, em contraste com a anódica frequentemente utilizada em estudos anteriores, fundamenta-se nos achados de \citeonline{purpura1965intracellular} e \citeonline{cukic2018shift}, que demonstraram que a polarização catódica hiperpolariza os corpos celulares de neurônios piramidais e induz alterações específicas na dinâmica cortical, potencialmente reduzindo a hiperconectividade e promovendo uma transição do sistema cerebral para diferentes regiões do espaço de estados.

A seleção do DLPFC esquerdo como alvo da estimulação é respaldada por múltiplos estudos que destacam esta região como um nó crítico em redes frontoparietais envolvidas em funções executivas, atenção e controle cognitivo-motor \cite{dong2023efficacy, arif2021high, jones2017frontoparietal}. Estas funções são particularmente relevantes para o desempenho atlético em esportes coletivos como o basquetebol, que exigem tomada de decisão rápida, atenção seletiva e coordenação sensório-motora refinada. Além disso, a aplicação da HD-tDCS, em vez da tDCS convencional, permite uma estimulação mais focal e precisa, conforme demonstrado por \citeonline{villamar2013hdtdcs}, potencialmente maximizando os efeitos neuromodulatórios sobre circuitos neurais específicos.

A análise da sincronização entre sinais de EEG e ECG representa uma abordagem inovadora para investigar os efeitos da neuromodulação sobre a integração entre sistemas neurais e cardiovasculares. Esta perspectiva alinha-se ao conceito de \textit{Body--Brain Dynamic System} (BBDS) apresentado na introdução e é fundamentada por estudos que demonstram a importância da sincronização entre ritmos cerebrais e fisiológicos para a regulação cognitiva e emocional \cite{criscuolo2022cognition, vergara2024exploring}. A quantificação desta sincronização através de métricas como o Phase Locking Value (PLV) e análises de teoria dos grafos permite caracterizar de forma precisa como a HD-tDCS modifica a arquitetura funcional das redes cerebrais e sua interação com o sistema cardiovascular.

A população de atletas de elite de basquetebol feminino representa um grupo particularmente interessante para esta investigação, dado que atletas de alto rendimento frequentemente apresentam padrões distintos de conectividade funcional e regulação autonômica em comparação com não-atletas. Conforme sugerido por \citeonline{valenzuela2019enhancement}, intervenções neuromodulatórias podem exercer efeitos seletivos sobre aspectos psicológicos do desempenho atlético, mesmo quando medidas objetivas de desempenho permanecem inalteradas. Assim, a análise dos efeitos da HD-tDCS sobre a sincronização EEG-ECG em atletas pode revelar mecanismos neurobiológicos subjacentes à integração corpo-cérebro em indivíduos com alto nível de treinamento físico e cognitivo.

O delineamento experimental cruzado (\textit{cross-over}) e duplo-cego adotado neste estudo alinha-se às melhores práticas metodológicas identificadas na literatura revisada, permitindo controlar variáveis confundidoras e isolar os efeitos específicos da neuromodulação. A análise do estado de repouso (\textit{resting-state}), por sua vez, oferece uma janela para observar a organização intrínseca das redes cerebrais e sua modulação pela HD-tDCS, sem a influência de demandas cognitivas ou motoras específicas.

Em síntese, este estudo integra conceitos e metodologias avançadas de neuromodulação, eletrofisiologia e análise de conectividade para investigar como a HD-tDCS catódica sobre o DLPFC esquerdo modifica a sincronização entre sistemas neurais e cardiovasculares em atletas de elite. Os resultados desta investigação têm o potencial de expandir nossa compreensão sobre os mecanismos pelos quais a neuromodulação influencia a integração corpo-cérebro, com implicações tanto para a ciência básica quanto para aplicações práticas no contexto esportivo e clínico.

\section{Reprodutibilidade e Disponibilidade}
O material desta dissertação, incluindo o texto fonte em \LaTeX, códigos de análise, figuras e tabelas completas dos resultados, está disponível publicamente em \cite{barros2025repository} (\url{https://github.com/dantebarross/efeito-da-neuromodulacao-na-sincronicidade-eeg-ecg}). Esta abordagem visa garantir a transparência e reprodutibilidade do trabalho, permitindo que outros pesquisadores possam verificar, replicar ou expandir os resultados apresentados.

\part{Objetivos e Hipóteses}
\chapter{Objetivos}
\label{chap:objetivos}
Este estudo visa aprofundar a compreensão dos efeitos da neuromodulação catódica, por meio da aplicação de HD-tDCS sobre o DLPFC, na conectividade funcional de atletas de elite de basquetebol feminino durante o repouso. Em particular, o foco recai sobre:
\begin{itemize}
    \item A análise da sincronização de fase entre canais de EEG (mesma banda) e entre EEG e ECG em uma abordagem \textit{cross-frequency};
    \item A utilização dos índices \textit{Phase Lag Index} (PLI) para os casos de mesma banda e \textit{Cross-Frequency Phase Linearity Measurement} (CF-PLM) para os casos de \textit{cross-frequency}, para comparar os períodos pré e pós-estimulação nas condições \textit{catódica} e \textit{sham}.
\end{itemize}

\chapter{Hipóteses}
\label{chap:hipoteses}

A hipótese central do estudo é que a HD-tDCS modula significativamente a sincronicidade entre canais de EEG e ECG em repouso.

\part{Metodologia e Processamento}
\chapter{Metodologia}
\label{chap:metodologia}

Este projeto utiliza dados previamente coletados pelo laboratório ao qual faço parte, envolvendo medidas de eletroencefalograma (EEG) e eletromiografia (EMG) de atletas profissionais de basquetebol feminino. O objetivo central deste estudo é investigar a sincronicidade cérebro-corpo e avaliar o impacto da estimulação transcraniana por corrente contínua de alta definição (HD-tDCS) no desempenho atlético durante arremessos de lance livre.

\section{Participantes e Coleta de Dados}

O estudo foi aprovado pelo Comitê de Ética em Pesquisa da UFABC (protocolo: 08070819.1.0000.5594) e conduzido em conformidade com os princípios éticos estabelecidos pela Declaração de Helsinque para experimentos envolvendo seres humanos. Todas as participantes assinaram o Termo de Consentimento Livre e Esclarecido (TCLE) antes de iniciarem sua participação.

As participantes eram consideradas atletas de elite, com um regime de treinamento superior a 10 horas semanais, e foram selecionadas com base em critérios rigorosos:
\begin{itemize}
    \item Participação regular no programa de treinamento da equipe;
    \item Ausência de doenças ou lesões que pudessem interferir na execução do protocolo;
    \item Assinatura do Termo de Consentimento Livre e Esclarecido (TCLE).
\end{itemize}

A amostra foi caracterizada por meio da mensuração de massa corporal, estatura e coleta de informações detalhadas, como nome, data de nascimento, categoria, experiência esportiva, posição no time, fase da temporada, membro dominante e estilo de arremesso. Contudo, devido a problemas técnicos durante a coleta de dados, apenas os dados de 7 atletas foram utilizados nas análises \cite{moscaleski2022}.

Embora o tamanho reduzido da amostra possa ser considerado uma limitação, ele é uma característica comum em estudos que envolvem populações específicas e de difícil acesso, como atletas de elite. Estudos prévios, como o de Boukrina et al. \cite{boukrina2020}, destacam que, em situações em que o aumento do tamanho da amostra não é viável, estratégias como a homogeneidade da amostra e análises que considerem a variabilidade individual podem proporcionar resultados robustos e significativos.

\section{Delineamento Experimental}

O delineamento experimental seguiu um modelo randomizado, cruzado e duplo-cego, amplamente utilizado em estudos de neurociência aplicada e psicofisiologia para minimizar viés e garantir a validade dos resultados. Esse modelo permitiu que todas as participantes fossem expostas tanto à condição de estimulação catódica (HD-tDCS) quanto à condição sham (simulada), aumentando a robustez das comparações intraindividuais.

\begin{figure}[htb]
    \centering
    \includegraphics[width=0.9\textwidth]{figs/0_intro_e_desenho_experimental/desenho_experimental_drawio.png}
    \caption{Fluxo do protocolo experimental, incluindo a sessão de familiarização, as duas sessões experimentais (com estimulação catódica ou sham), as coletas de EEG/ECG em repouso e a execução de 200 arremessos por sessão.}
    \label{fig:desenho_experimental}
\end{figure}

Antes das sessões experimentais, foi realizada uma sessão de familiarização, na qual as participantes receberam explicações sobre os objetivos, procedimentos, riscos e benefícios do estudo, além de um período de adaptação ao protocolo experimental. Essa familiarização assegurou que todas estivessem confortáveis com os procedimentos e equipamentos utilizados.

As sessões experimentais ocorreram entre janeiro e fevereiro de 2020, com a seguinte estrutura:
\begin{itemize}
    \item \textbf{Sessão 1}: Familiarização com dispositivos e procedimentos do estudo.
    \item \textbf{Sessões 2 e 3}: Protocolo experimental com 200 arremessos por sessão, totalizando 400 arremessos por atleta.
\end{itemize}

As sessões foram realizadas no mesmo local e horário do treinamento habitual das atletas, garantindo a padronização das condições ambientais. A ordem das condições experimentais foi atribuída de maneira aleatória para cada participante, conforme o modelo cruzado.

\section{Questionários e Escalas}
Além das medidas neurofisiológicas, foram aplicados questionários para avaliar aspectos subjetivos relacionados ao desempenho das atletas. Entre os instrumentos utilizados, destacam-se:
\begin{itemize}
    \item \textbf{Escala de Qualidade Total de Recuperação (TQR)}: avalia o estado geral de recuperação física e mental, fornecendo insights sobre como as participantes se sentem prontas para enfrentar novas atividades físicas após as sessões experimentais;
    \item \textbf{Escala de Percepção Subjetiva de Esforço (PSE)}: mede o esforço percebido pelas participantes ao final de cada sessão experimental, permitindo relacionar o desempenho físico ao nível de fadiga subjetiva;
    \item \textbf{Sport Competition Anxiety Test (SCAT)}: identifica os níveis de ansiedade competitiva das participantes, avaliando como essa variável psicológica pode influenciar a execução das tarefas motoras propostas;
    \item \textbf{Questionário de Motivação Relacionado ao Exercício}: investiga os fatores motivacionais das participantes durante o experimento, analisando seu impacto no engajamento e na qualidade do desempenho.
\end{itemize}
Esses instrumentos permitiram uma análise detalhada de variáveis psicológicas complementares às medidas fisiológicas coletadas, contribuindo para uma visão integrada da sincronicidade cérebro-corpo no desempenho esportivo.


\section{Estimulação Transcraniana por Corrente Contínua de Alta Definição (HD-tDCS)}

A HD-tDCS foi realizada com um estimulador digital MxN da Soterix Medical, utilizando eletrodos Ag/AgCl posicionados em uma touca de EEG. O posicionamento dos eletrodos seguiu um protocolo padronizado baseado em modelagem computacional, garantindo precisão e focalidade na estimulação \cite{datta2008}. 

Foram aplicadas duas condições experimentais: estimulação catódica (ativa) e sham (simulada), com as participantes expostas a ambas em sessões diferentes, de forma cruzada e randomizada. A calibração dos equipamentos foi realizada antes de cada sessão para assegurar a qualidade e a confiabilidade dos dados registrados.

\subsection{Processamento e Análise de Dados}

O processamento e análise dos dados seguiram um fluxo estruturado que abrange desde a coleta e organização dos arquivos até a extração de métricas de sincronização e a aplicação de testes estatísticos para avaliar as diferenças entre condições experimentais. A Figura~\ref{fig:fluxo_processamento} apresenta um diagrama geral dessas etapas.

\begin{figure}[htb]
    \centering
    \includegraphics[width=0.9\textwidth]{figs/0_intro_e_desenho_experimental/diagrama_processamento_e_analise_drawio.png}
    \caption{Fluxo geral de processamento e análise de dados, desde a coleta e organização dos arquivos, passando pelas etapas de pré-processamento (EEG e ECG), sincronização temporal, extração de métricas de sincronização (PLI, PLV e CF-PLM) e, por fim, análises estatísticas e de conectividade em rede.}
    \label{fig:fluxo_processamento}
\end{figure}

\subsubsection{Pré-processamento de Dados}
Os sinais de EEG e EMG foram submetidos a etapas de pré-processamento, como filtragem de ruídos e remoção de artefatos por meio de Independent Component Analysis (ICA). Além disso, os sinais de ECG passaram por processos de detecção de picos e extração do ciclo cardíaco. Essas etapas garantiram a qualidade e consistência dos dados analisados.

\subsubsection{Sincronização de Sinais}
Para permitir uma análise integrada, os sinais de EEG e ECG foram alinhados temporalmente, garantindo que as medidas extraídas estivessem sincronizadas e pudessem ser comparadas corretamente. Esse procedimento assegurou a compatibilidade entre os dados das diferentes modalidades.

\subsubsection{Cálculo de Sincronização Funcional}
A sincronização entre os sinais foi avaliada utilizando diferentes métricas. O Phase Locking Value (PLV) e o Phase Lag Index (PLI) foram aplicados para quantificar a conectividade em uma mesma frequência, enquanto a Cross-Frequency Phase Linearity Measurement (CF-PLM) foi utilizada para avaliar acoplamento entre frequências distintas. Antes de sua aplicação nos dados experimentais, esses métodos foram testados e validados por meio de sinais simulados.

\subsubsection{Análise Estatística}
Para investigar diferenças entre condições experimentais, aplicamos métodos estatísticos robustos, incluindo testes de normalidade, testes não-paramétricos e correções para comparações múltiplas. Adicionalmente, utilizamos medidas de centralidade em redes para avaliar a conectividade funcional entre diferentes regiões corticais. Os padrões de conectividade foram representados por meio de gráficos e redes, facilitando a visualização dos resultados.

Esse fluxo sistemático permitiu uma abordagem rigorosa para explorar a sincronicidade cérebro-corpo e avaliar o impacto da estimulação transcraniana no desempenho atlético.

\chapter{Pré-processamento de Dados}
\label{chap:preprocessamento_de_dados}

\section{Preparação dos Dados}
Para este estudo, os dados foram coletados em sessões experimentais com atletas profissionais de basquetebol feminino, submetidas a duas condições: estimulação transcraniana por corrente contínua de alta definição (HD-tDCS) catódica e uma condição de controle (sham). Neste capítulo, descrevemos os procedimentos adotados para organizar, sincronizar e preparar os sinais de eletroencefalografia (EEG) e eletrocardiografia (ECG) para análise.

\subsection{Organização Inicial dos Dados}
Os dados de EEG e ECG foram armazenados em arquivos separados, correspondentes a cada atleta e a cada condição experimental (\textit{pre\_sham}, \textit{post\_sham}, \textit{pre\_cathodic}, \textit{post\_cathodic}). Os sinais de EEG foram originalmente registrados a 1000 Hz, enquanto os sinais de ECG – obtidos a partir do EMG do músculo peitoral maior – foram registrados a 1111,11 Hz. Nesta etapa, os arquivos passaram por:
\begin{itemize}
    \item Identificação e associação a cada atleta e condição;
    \item Renomeação e normalização para garantir consistência;
    \item Verificação de integridade para evitar erros no processamento.
\end{itemize}

\subsection{Sincronização Temporal entre EEG e ECG}

Devido à ausência de sincronização inicial nas gravações, foi necessário alinhar os sinais de EEG e ECG temporalmente. Para tanto, utilizando marcadores do período de repouso, foram realizados os seguintes procedimentos:

\begin{itemize}
    \item Extração dos tempos de início e término do período de repouso para cada modalidade;
    \item Ajuste dos timestamps para que os sinais iniciassem simultaneamente;
    \item Remoção dos primeiros e últimos 15 segundos de cada gravação, a fim de minimizar artefatos de borda.
\end{itemize}

Essa abordagem garante que as análises se baseiem em um estado de repouso sincronizado, eliminando inconsistências temporais entre os registros.

\subsection{Estruturação dos Dados}
Após a sincronização, os dados foram organizados para o processamento subsequente. Os sinais de EEG foram armazenados em formato FIF, preservando as informações dos canais e os metadados, enquanto os dados de ECG foram exportados em formato CSV.
\section{Pré-processamento dos Sinais}

Esta seção descreve os procedimentos aplicados para garantir a qualidade dos sinais, abrangendo etapas de filtragem, remoção de artefatos e segmentação, com foco especial no processamento dos dados de EEG.

\subsection{Pré-processamento do EEG}

O processamento dos dados de EEG compreende duas etapas principais: a preparação dos dados brutos e a limpeza de artefatos utilizando técnicas de decomposição.

\subsubsection{Filtragem, Reamostragem e Preparação dos Dados de EEG}
Para a preparação dos dados de EEG, foram realizadas as seguintes etapas:
\begin{itemize}
    \item \textbf{Aquisição e Carregamento:} Os dados brutos foram baixados do Google Drive e carregados utilizando a biblioteca MNE-Python a partir de arquivos exportados do software BrainVision (ou, alternativamente, em formato EEGLAB). Após o carregamento, os canais foram renomeados conforme uma convenção padronizada (por exemplo, Fp1, Fz, F3, etc.) e canais irrelevantes foram removidos. (Ver Figura~\ref{fig:exemplo_sinais_eeg})
    \item \textbf{Aplicação de Montage:} Aplicou-se o montage padrão \textit{standard\_1005} para assegurar a correta localização dos eletrodos, conforme ilustrado no diagrama do sistema 10-20. (Ver Figura~\ref{fig:sistema_10_20})
    \item \textbf{Definição do Período de Análise:} Com base nas informações extraídas dos arquivos do período de repouso, os dados foram recortados para excluir os primeiros e últimos 15 segundos da gravação, reduzindo artefatos de borda.
    \item \textbf{Filtragem:} Foi aplicado um filtro passa-banda entre 0,5 e 60 Hz para eliminar ruídos fora do intervalo de interesse, seguido de um filtro notch em 60 Hz para remover interferências da rede elétrica (\textit{line noise}).
    \item \textbf{Reamostragem:} Para padronizar a taxa de amostragem e assegurar o alinhamento com outros sinais (como o ECG), bem como o funcionamento das medidas de sincronicidade, os dados de EEG foram reamostrados para 500 Hz.
\end{itemize}

\begin{figure}[htb]
    \centering
    \includegraphics[width=0.8\textwidth]{figs/1_preprocessamento_eeg/2_exemplo_sinais_canais_eeg.png}
    \caption{Exemplo de sinais brutos de EEG, ilustrando a variação natural dos canais ao longo do tempo.}
    \label{fig:exemplo_sinais_eeg}
\end{figure}

\begin{figure}[htb]
    \centering
    \includegraphics[width=0.8\textwidth]{figs/1_preprocessamento_eeg/1_sistema_10_20.png}
    \caption{Ilustração do sistema 10-20, representando a distribuição padronizada dos eletrodos no couro cabeludo. Ressalta-se que o posicionamento apresentado é meramente ilustrativo e não reflete com exatidão a disposição real dos eletrodos.}
    \label{fig:sistema_10_20}
\end{figure}


\subsubsection{Limpeza de Artefatos e Remoção de Componentes de Ruído (ICA)}
Para aprimorar a qualidade dos sinais de EEG, empregou-se a Análise de Componentes Independentes (ICA) seguindo os seguintes passos:
\begin{itemize}
    \item \textbf{Identificação de Canais Afetados por Ruído:} Antes da aplicação do ICA, foram identificados canais cujos sinais estavam significativamente comprometidos. Esses canais foram determinados com base em inspeção visual, considerando sinais com ruído excessivo, desvios severos em relação à morfologia típica do EEG ou padrões incompatíveis com atividade neural esperada. Tais canais foram excluídos da decomposição para evitar que influenciassem negativamente a separação dos componentes, e para que não participassem das análises posteriores.
    \item \textbf{Definição dos Componentes:} O número de componentes foi definido igual ao número de canais restantes após a remoção dos canais afetados por ruído, garantindo uma decomposição eficiente e representativa dos sinais neurais. 
    \item \textbf{Aplicação do ICA:} Utilizou-se o método FastICA para decompor o sinal, preservando as informações essenciais para a análise de sincronicidade. 
    \item \textbf{Identificação de Sinais Afastados da Distribuição Geral:} Utilizou-se a biblioteca \textit{autoreject} do Python para detectar automaticamente sinais que apresentavam desvios significativos em relação aos demais. Esse método não foi empregado para aplicar limiares rígidos de rejeição, mas sim para priorizar a inspeção dos canais e épocas mais discrepantes. Todos os sinais foram inspecionados manualmente, com ênfase nos identificados pelo \textit{autoreject}.
    \item \textbf{Identificação Automática de Artefatos:} Componentes associados a artefatos oculares (utilizando canais como Fp1 e Fp2) e à atividade cardíaca foram identificados automaticamente utilizando métodos da biblioteca MNE-Python. Adicionalmente, uma análise baseada na curtose foi realizada para detectar componentes com alta amplitude, indicativos de artefatos. 
    \item \textbf{Inspeção Visual e Seleção:} Foram gerados gráficos das propriedades dos componentes e visualizações interativas, incluindo animações, para auxiliar na verificação do sinal de todos os canais de EEG com e sem determinados componentes específicos. Por exemplo, foi possível comparar o comportamento do sinal sem os componentes [0, 1, 2] e sem os componentes [0, 1, 2, 3], permitindo avaliar a influência da adição ou remoção do componente 3 no sinal de todos os canais. Além disso, a avaliação dos ICs foi realizada considerando sua distribuição espacial, com auxílio de \textit{topomaps} (\textit{heatmaps} da cabeça), identificando em quais regiões os componentes estavam mais concentrados. Uma forte concentração em áreas comumente associadas à geração de ruídos, como a região próxima aos olhos, pode ser um indicativo de artefato e, portanto, um critério para sua remoção.
    \item \textbf{Aplicação e Salvamento:} Após a definição dos componentes a serem removidos, o ICA foi aplicado para eliminar os artefatos identificados, gerando um sinal de EEG considerado pré-processado, pronto para ser analisado estatisticamente. O sinal resultante foi salvo em formato FIF para futuras análises.
\end{itemize}

\begin{figure}[htb]
    \centering
    \includegraphics[width=0.8\textwidth]{figs/1_preprocessamento_eeg/3_exemplo_compomentes_pos_ICA.png}
    \caption{Exemplo de componentes obtidos após a aplicação do ICA, com mapas topográficos que evidenciam a distribuição espacial de cada componente.}
    \label{fig:componentes_pos_ICA}
\end{figure}

\begin{figure}[htb]
    \centering
    \includegraphics[width=0.8\textwidth]{figs/1_preprocessamento_eeg/4_exemplo_ICA_component_analysis.png}
    \caption{Exemplo de análise detalhada de um componente ICA, apresentando o mapa topográfico, o espectro de frequência e outras características relevantes para a identificação de artefatos.}
    \label{fig:exemplo_ICA_component_analysis}
\end{figure}
\subsection{Pré-processamento do Sinal de ECG}
\label{subsec:preprocess_ecg}

O processamento do sinal de ECG teve como objetivo obter uma versão refinada do sinal, permitindo a detecção precisa dos picos R, a extração de informações de fase e a obtenção do ciclo cardíaco com base nos picos R. Este ciclo foi utilizado no cálculo de sincronicidade de fase. Para isso, os procedimentos foram organizados em três etapas principais: aquisição e segmentação, limpeza do sinal com detecção de picos e aplicação de filtros complementares.

\subsubsection{Aquisição e Segmentação dos Dados de ECG}
\begin{itemize}
    \item \textbf{Aquisição:} Os dados de ECG foram coletados juntamente com informações sobre os tempos de início e fim do período de repouso para cada condição experimental, garantindo a correta identificação dos intervalos de interesse.
    \item \textbf{Segmentação:} Com base nos tempos extraídos dos arquivos do período de repouso, o sinal bruto foi segmentado para selecionar apenas o intervalo correspondente à condição de interesse. Para reduzir artefatos nas bordas, os primeiros e últimos 15 segundos foram removidos.
\end{itemize}

\subsubsection{Limpeza do Sinal e Detecção de Picos}
\begin{itemize}
    \item \textbf{Limpeza:} Utilizou-se a biblioteca \textit{NeuroKit2}, uma biblioteca em Python, para processar o sinal segmentado, removendo ruídos e gerando uma versão refinada do ECG.
    \item \textbf{Detecção Automática de Picos:} Um algoritmo foi aplicado para a detecção automática dos picos R (R-peaks) no sinal limpo, identificando os batimentos cardíacos. A Figura~\ref{fig:ecg_picos_detectados} ilustra um exemplo em que o sinal bruto (cinza) é sobreposto ao sinal limpo (azul), com os picos R destacados em vermelho.
    \item \textbf{Correção Manual dos Picos R:} Após a detecção automática, uma inspeção visual cuidadosa foi realizada utilizando gráficos interativos para:
    \begin{itemize}
        \item Inserir manualmente os picos faltantes, identificados pela ausência de eventos em locais esperados;
        \item Remover manualmente picos falsos, identificados por timestamps incorretos.
    \end{itemize}
    Esse ajuste garante a acurácia na identificação dos batimentos, evitando omissões e inclusões indevidas.
\end{itemize}

\begin{figure}[htb]
    \centering
    \includegraphics[width=0.8\textwidth]{figs/2_preprocessamento_ecg/1_Sinal_de_ECG_-_Picos_Detectados_zoom.png}
    \caption{Exemplo de sinal de ECG com picos detectados. O sinal bruto (cinza) é sobreposto ao sinal limpo (azul), com os picos R marcados em vermelho.}
    \label{fig:ecg_picos_detectados}
\end{figure}

\subsubsection{Aplicação de Filtros Complementares}
Para refinar a definição dos eventos do ECG, foram aplicados filtros adicionais:
\begin{itemize}
    \item \textbf{Filtro de Janela:} Extração de uma janela de \(\pm50\) ms ao redor de cada pico, isolando os segmentos de interesse.
    \item \textbf{Filtro de Cruzamento pelo Zero:} Identificação dos pontos de cruzamento pelo zero nos segmentos próximos aos picos, permitindo um ajuste fino dos limites dos eventos.
\end{itemize}
A combinação desses filtros resultou em um \emph{Sinal Final Filtrado}, que destaca de forma mais clara a morfologia do ECG. A Figura~\ref{fig:ecg_filtros_aplicados} exemplifica o efeito dos filtros, comparando o sinal limpo inicial (linha colorida) com o sinal final filtrado, bem como os picos detectados.

\begin{figure}[htb]
    \centering
    \includegraphics[width=0.8\textwidth]{figs/2_preprocessamento_ecg/2_Sinal_ECG_-_Aplicação_de_Filtros_zoom.png}
    \caption{Exemplo de aplicação de filtros complementares ao sinal de ECG, destacando a morfologia dos picos R (em vermelho).}
    \label{fig:ecg_filtros_aplicados}
\end{figure}
\subsubsection{Geração de Sinais Senoidais e Análise de Fase}

Para aprimorar a análise de sincronização de fase entre os sinais de EEG e ECG, o sinal de ECG foi transformado em uma representação senoidal. Essa transformação apresenta diversos benefícios:
\begin{itemize}
    \item \textbf{Definição Clara do Ciclo Cardíaco:} Ao utilizar os R-peaks para delimitar cada ciclo, a conversão em uma onda senoidal permite definir de forma inequívoca o início e o fim do ciclo cardíaco, fornecendo um marcador preciso para segmentação dos períodos de interesse.
    \item \textbf{Extração Precisa da Fase:} Uma onda senoidal exibe uma variação linear de fase ao longo do tempo, o que facilita a extração da fase instantânea por meio da Transformada de Hilbert, proporcionando uma determinação robusta e consistente.
    \item \textbf{Facilitação da Análise de Sincronização:} Técnicas de sincronização de fase, como o CF-PLM (uma variante do PLV para análise cross-frequency), funcionam melhor quando a fase é clara e bem definida. A representação senoidal torna a fase do ECG mais nítida, permitindo que os algoritmos captem com maior precisão a relação de sincronização entre os ritmos neurais (EEG) e o ritmo cardíaco.
    \item \textbf{Robustez à Variabilidade e Ruído:} A conversão do ECG, que apresenta picos acentuados e variabilidade, para uma forma senoidal suaviza essas irregularidades, melhorando a robustez do método de extração de fase mesmo na presença de ruídos ou artefatos.
    \item \textbf{Integração com a Análise de EEG:} Como os sinais de EEG são frequentemente filtrados para se aproximarem de formas senoidais, padronizar a representação do ECG facilita a integração dos dois tipos de sinal na análise de sincronização, permitindo comparações diretas e métodos cross-frequency mais eficazes.
\end{itemize}

Além desses benefícios, abordagens integradas, como o plugin BrainBeats \cite{cannard2023brainbeats}, facilitam a análise conjunta de sinais de EEG e cardiovasculares, oferecendo ferramentas para a remoção automatizada de artefatos cardíacos e extração de características quantitativas. Estudos como o de Mollakazemi et al. \cite{mollakazemi2021eeg} demonstram que a correta sincronização dos sinais – viabilizada pela transformação do ECG em uma representação senoidal – permite identificar segmentos de EEG com maior relevância funcional, reforçando a importância de uma segmentação temporal precisa na análise de fase.

A Figura~\ref{fig:ecg_comparacao_fase} ilustra a comparação de fase entre o sinal de ECG filtrado (azul), o sinal senoidal gerado (verde) e um sinal simulado (vermelho), evidenciando a coerência de fase entre eles.

\begin{figure}[htb]
    \centering
    \includegraphics[width=0.8\textwidth]{figs/2_preprocessamento_ecg/3_Comparação_de_Fase_entre_Sinais.png}
    \caption{Exemplo de comparação de fase entre o ECG filtrado (azul), o ECG senoidal (verde) e um sinal simulado (vermelho). A boa concordância entre as fases indica a consistência do procedimento de geração do sinal senoidal e da extração de fase.}
    \label{fig:ecg_comparacao_fase}
\end{figure}

Em suma, a transformação do ECG em um sinal senoidal não apenas define claramente o ciclo cardíaco, mas também possibilita a extração de uma fase contínua, essencial para a análise de sincronização de fase entre EEG e ECG utilizando métodos de extração de fase empregados neste estudo.

\subsubsection{Estrutura do Dado Final e Armazenamento}

O conjunto final de dados resultante do processamento do ECG foi estruturado em um DataFrame que integra as seguintes variáveis:
\begin{itemize}
    \item \textbf{Tempo:} Timestamps sincronizados.
    \item \textbf{Sinal Bruto (EMG):} Valor original do sinal.
    \item \textbf{Sinal Limpo (ECG):} Versão filtrada do sinal.
    \item \textbf{Picos:} Indicador binário dos R-peaks detectados.
    \item \textbf{First Filtered:} Sinal obtido após a aplicação do filtro de janela (\(\pm50\) ms).
    \item \textbf{Final Filtered:} Sinal final obtido após a combinação dos filtros aplicados.
    \item \textbf{ECG Senoidal:} Sinal senoidal derivado dos R-peaks.
\end{itemize}

Este DataFrame foi exportado em formato CSV para facilitar o acesso e a análise subsequente dos dados.


\part{Análises e Resultados}
\chapter{Métodos de Análise de Sincronização de Fase}
\label{chap:6_metodos_de_analise_de_sincronizacao_de_fase}

Neste capítulo, apresentamos os fundamentos teóricos e práticos dos métodos utilizados para analisar a sincronização de fase entre sinais fisiológicos. Em particular, discutiremos o Phase Lag Index (PLI) e o Cross-Frequency Phase Linearity Measurement (CF-PLM). Também foi testado o tradicional Phase Locking Value (PLV) para comparação e os resultados obtidos com o PLV estão disponíveis no anexo para referência. Para este estudo, optamos por utilizar o PLI para a sincronização entre canais de EEG (same-frequency) e o CF-PLM para a análise entre EEG e ECG (cross-frequency).

\section{Fundamentos dos Métodos}

A análise de sincronização de fase visa quantificar a consistência da diferença de fase entre dois sinais ao longo do tempo. Os métodos utilizados neste estudo são baseados em abordagens que extraem a fase instantânea dos sinais por meio da Transformada de Hilbert, permitindo a análise de como as fases se relacionam entre si.

\subsection{Phase Lag Index (PLI)}

O PLI é um índice amplamente utilizado para medir a sincronização de fase entre sinais que operam na mesma faixa de frequência, como os canais de EEG dentro de uma mesma banda. Ao contrário do Phase Locking Value (PLV), o PLI é robusto à mistura de sinais e aos efeitos de condução de volume, pois foca na assimetria da distribuição das diferenças de fase. Especificamente, ele desconsidera valores próximos de zero que podem resultar de sincronizações espúrias, muitas vezes induzidas por fontes comuns ou artefatos, concentrando-se apenas em atrasos de fase que são mais informativos sobre a interação funcional.

Conforme discutido por \citeauthor{seraj2018cerebral} (\citeyear{seraj2018cerebral}), o PLI apresenta vantagens metodológicas, fornecendo uma medida mais pura da sincronização de fase, uma vez que elimina a influência de picos que podem ocorrer devido a volume conduction ou outras fontes de ruído. Essa robustez torna o PLI particularmente adequado para a avaliação da sincronização entre canais de EEG, onde a variabilidade e a interferência de sinais de fundo podem comprometer a precisão de medidas tradicionais como o PLV.

Além disso, o PLI é eficaz na detecção de atrasos sistemáticos entre os sinais, contribuindo para uma melhor compreensão da dinâmica funcional cerebral. No entanto, vale ressaltar que, para interações entre sinais que operam em frequências distintas (por exemplo, entre EEG e ECG), métodos como o CF-PLM são mais indicados. Nesta pesquisa, optamos por utilizar o PLI para a análise de sincronização entre canais de EEG em bandas idênticas, enquanto empregamos o CF-PLM para investigar a sincronização cross-frequency entre EEG e ECG. Os resultados comparativos obtidos com o PLV também foram testados e estão disponíveis no anexo para referência.

\subsection{Cross-Frequency Phase Linearity Measurement (CF-PLM)}

O CF-PLM é um método desenvolvido para analisar a sincronização de fase entre sinais que operam em frequências distintas, ou seja, para detectar acoplamento cross-frequency. Essa abordagem é especialmente útil para avaliar a relação entre os ritmos neurais do EEG (tipicamente de alta frequência) e o ritmo cardíaco do ECG (geralmente de baixa frequência). 

De acordo com \citeauthor{sorrentino2022detection} (\citeyear{sorrentino2022detection}), o método estende o conceito de Phase Linearity Measurement (PLM) para a análise de n:m sincronização entre sinais, sem a necessidade de hipóteses a priori sobre as frequências envolvidas. O procedimento baseia-se nos seguintes passos:

\begin{enumerate}
    \item \textbf{Cálculo dos Sinais Analíticos:} Para os sinais de interesse \(x(t)\) e \(y(t)\), obtém-se suas representações analíticas \(x_{\mathrm{an}}(t)\) e \(y_{\mathrm{an}}(t)\) por meio da Transformada de Hilbert, que fornecem, respectivamente, as fases \(\phi_x(t)\) e \(\phi_y(t)\).
    \item \textbf{Construção do Sinal Interferométrico:} Calcula-se o sinal interferométrico \(z(t)\) utilizando a fórmula:
    \[
    z(t) = \frac{x_{\mathrm{an}}(t)\, y_{\mathrm{an}}^*(t)}{\lvert x_{\mathrm{an}}(t)\rvert\, \lvert y_{\mathrm{an}}(t)\rvert} = e^{i\Delta \phi(t)},
    \]
    onde \(\Delta \phi(t) = \phi_x(t) - \phi_y(t)\) é a diferença de fase instantânea entre os sinais. Note que \(z(t)\) possui amplitude unitária, isolando assim a informação de fase.
    \item \textbf{Análise da Densidade Espectral de Potência (PSD):} Através da Transformada de Fourier, calcula-se a PSD de \(z(t)\). Em condições de acoplamento iso-frequencial, o pico na PSD é centralizado em \(f = 0\). Já em condições de acoplamento cross-frequency, a presença de um desvio (i.e., um pico deslocado de zero) indica a diferença entre as frequências centrais dos sinais.
    \item \textbf{Cálculo do CF-PLM:} O índice CF-PLM é obtido integrando a PSD em uma janela estreita \([f_\Delta - B, f_\Delta + B]\) centrada no pico (onde \(f_\Delta\) representa a diferença de frequência entre os sinais) e normalizando pelo poder total da PSD:
    \[
    \text{CF-PLM} = \frac{\displaystyle\int_{f_\Delta - B}^{f_\Delta + B} SZ(f) \, df}{\displaystyle\int_{-\infty}^{+\infty} SZ(f) \, df}.
    \]
\end{enumerate}

A transformação do ECG em uma representação senoidal (conforme descrito no Capítulo~\ref{chap:preprocessamento_de_dados}) é crucial para que a extração da fase seja clara e contínua, o que, por sua vez, melhora a detecção de sincronização cross-frequency entre os sinais de EEG e ECG. Essa abordagem permite que o CF-PLM capture com precisão a intensidade do acoplamento, sem que a amplitude dos sinais influencie o índice, tornando-o robusto à variabilidade dos dados.

Em resumo, o CF-PLM possibilita uma estimativa confiável da sincronização de fase entre sinais de frequências diferentes, sendo particularmente indicado para a análise de interações entre os ritmos neurais e o ritmo cardíaco, conforme validado por \citeauthor{sorrentino2022detection} (\citeyear{sorrentino2022detection}).

\subsection{Comparação com o Phase Locking Value (PLV)}

Para fins de validação e comparação, também testamos o PLV, um método tradicional e amplamente utilizado para a análise de sincronização de fase. Conforme destacado por \citeauthor{seraj2018cerebral} (\citeyear{seraj2018cerebral}), embora o PLV seja intuitivo e eficaz na medição da consistência da diferença de fase entre sinais na mesma faixa de frequência, ele apresenta limitações notáveis, como a sua sensibilidade a ruídos e aos efeitos de volume conduction, que podem levar à detecção de sincronizações espúrias.

Essas limitações se tornam ainda mais evidentes na análise cross-frequency, onde os sinais operam em escalas de frequência distintas. Em nossa investigação, os resultados obtidos com o PLV foram utilizados para contrastar a sensibilidade e a robustez dos métodos alternativos: o PLI, para sincronização iso-frequencial entre canais de EEG, e o CF-PLM, para a análise de acoplamento cross-frequency entre EEG e ECG. Os resultados experimentais confirmaram que, embora o PLV seja adequado para sinais com a mesma faixa de frequência, ele não captura de maneira confiável a sincronização entre sinais de diferentes frequências, corroborando as observações de Seraj (2018).

Portanto, optamos por utilizar o PLI e o CF-PLM como índices principais neste estudo. Detalhes comparativos e resultados do PLV, que evidenciam suas limitações e complementam nossa análise, estão disponíveis no anexo.

\section{Validação Experimental com Injeção de Sinais}

Para validar os métodos utilizados neste estudo, realizamos experimentos com injeção controlada de sinais senoidais sobre dados reais de ECG e EEG, adquiridos durante sessões experimentais reais. O objetivo foi verificar a capacidade dos índices CF-PLM, PLV e PLI em identificar corretamente diferentes tipos e intensidades de sincronização de fase artificialmente introduzidas.

A técnica empregada consistiu nas seguintes etapas:

\begin{enumerate}
    \item Seleção de segmentos representativos dos sinais originais de ECG e EEG.
    \item Geração de sinais senoidais com frequências e fases específicas utilizando o modelo de Kuramoto, permitindo controle preciso das condições experimentais.
    \item Aplicação controlada desses sinais sobre os sinais originais usando máscaras de injeção, variando a porcentagem de contribuição (0\%, 25\%, 50\%, 75\%, e 100\%) dos sinais injetados.
    \item Cálculo dos índices CF-PLM, PLV e PLI sobre esses sinais modificados, para avaliar o desempenho dos métodos de sincronização propostos em diferentes cenários.
\end{enumerate}

Foram conduzidos três cenários principais:

\begin{itemize}
    \item \textbf{Cross-frequency (1~Hz no ECG, 40~Hz no EEG):} Para testar especialmente o índice CF-PLM em situações onde o acoplamento ocorre entre frequências distintas.
    \item \textbf{Same-frequency (10~Hz no ECG e EEG com defasagem):} Para avaliar sensibilidade e desempenho de todos os índices quando os sinais apresentam frequências iguais, porém com defasagem de fase fixa configurada (\(\pi/4\)).
    \item \textbf{Same-frequency com phase lag zero (ambos 10~Hz sem defasagem):} Cenário idealizado para demonstrar a robustez do PLI contra sincronização espúria.
\end{itemize}

Exemplos dos sinais antes e após a injeção no cenário Cross-frequency são mostrados nas Figuras~\ref{fig:ecg_injection} e~\ref{fig:eeg_injection}, ilustrando a adição de sinais artificiais em frequências distintas sobre o sinal original.

\begin{figure}[htb]
    \centering
    \includegraphics[width=0.8\textwidth]{figs/3_2_testing_connectivity_metrics/1_ECG_Original_vs_Injecao_Cross-frequency.png}
    \caption{ECG: comparação entre o sinal original e o sinal senoidal injetado (1~Hz), cenário Cross-frequency.}
    \label{fig:ecg_injection}
\end{figure}

\begin{figure}[htb]
    \centering
    \includegraphics[width=0.8\textwidth]{figs/3_2_testing_connectivity_metrics/3_EEG_Original_vs_Injecao_Cross-frequency.png}
    \caption{EEG: comparação entre o sinal original e o sinal senoidal injetado (40~Hz), cenário Cross-frequency.}
    \label{fig:eeg_injection}
\end{figure}

Já no cenário Same-frequency, onde tanto o ECG quanto o EEG recebem sinais senoidais da mesma frequência (10~Hz) mas com pequena defasagem (\(\pi/4\)), temos os resultados ilustrados pelas Figuras~\ref{fig:eeg_original_vs_injection_samefreq} e~\ref{fig:eeg_injected_samefreq}. Esses gráficos destacam visualmente a interferência resultante da injeção.

\begin{figure}[htb]
    \centering
    \includegraphics[width=0.8\textwidth]{figs/3_2_testing_connectivity_metrics/10_EEG_Original_vs_Injecao_Same-frequency.png}
    \caption{EEG original (azul) e sinal de injeção de 10 Hz (vermelho) com pequena defasagem (\(\pi/4\)).}
    \label{fig:eeg_original_vs_injection_samefreq}
\end{figure}

\begin{figure}[htb]
    \centering
    \includegraphics[width=0.8\textwidth]{figs/3_2_testing_connectivity_metrics/11_EEG_Injetado_Same-frequency.png}
    \caption{Sinal de EEG após a injeção controlada de uma senóide (verde) comparado ao sinal original sem injeção (azul).}
    \label{fig:eeg_injected_samefreq}
\end{figure}

Em seguida, realizamos a extração das fases instantâneas usando a transformada de Hilbert e geramos o sinal interferométrico. Esses passos estão exemplificados visualmente nas Figuras~\ref{fig:fases_instantaneas_samefreq} e~\ref{fig:sinal_interferometrico_samefreq}.

\begin{figure}[htb]
    \centering
    \includegraphics[width=0.8\textwidth]{figs/3_2_testing_connectivity_metrics/12_Passo1_Fases_Same-frequency.png}
    \caption{Fases instantâneas extraídas dos sinais EEG e ECG injetados (ambos a 10 Hz).}
    \label{fig:fases_instantaneas_samefreq}
\end{figure}

\begin{figure}[htb]
    \centering
    \includegraphics[width=0.8\textwidth]{figs/3_2_testing_connectivity_metrics/13_Passo2_Interferometrico_Same-frequency.png}
    \caption{Sinal interferométrico gerado pela diferença de fase instantânea (cenário same-frequency).}
    \label{fig:sinal_interferometrico_samefreq}
\end{figure}

A seguir, calculamos o índice CF-PLM utilizando a FFT sobre o sinal interferométrico, exemplificado pela Figura~\ref{fig:fft_psd_samefreq}.

\begin{figure}[htb]
    \centering
    \includegraphics[width=0.8\textwidth]{figs/3_2_testing_connectivity_metrics/14_Passo3_FFT_PSD_Same-frequency.png}
    \caption{PSD do sinal interferométrico, indicando o pico em 0 Hz devido ao offset constante de fase entre os sinais de mesma frequência.}
    \label{fig:fft_psd_samefreq}
\end{figure}

Finalmente, analisamos o cenário especial "Same-frequency Com Phase Lag Zero", onde as fases são perfeitamente sincronizadas. As Figuras~\ref{fig:zerolag_phases_final} e~\ref{fig:zerolag_difference_final} comprovam que o PLI é efetivamente insensível a tais sincronizações.

\begin{figure}[htb]
    \centering
    \includegraphics[width=0.8\textwidth]{figs/3_2_testing_connectivity_metrics/15_ZeroLag_Fases_Same-frequency Com Phase Lag Zero.png}
    \caption{Fases desenroladas em cenário sem defasagem (10 Hz), com sobreposição quase exata dos sinais.}
    \label{fig:zerolag_phases_final}
\end{figure}

\begin{figure}[htb]
    \centering
    \includegraphics[width=0.8\textwidth]{figs/3_2_testing_connectivity_metrics/16_ZeroLag_Diferenca_Fase_Same-frequency Com Phase Lag Zero.png}
    \caption{Diferença de fase próxima a zero, indicando ausência completa de defasagem no cenário de Phase Lag Zero.}
    \label{fig:zerolag_difference_final}
\end{figure}

O comportamento geral das métricas nos diferentes cenários e níveis de injeção está resumido na Figura~\ref{fig:comparativo_metricas}.

\begin{figure}[htb]
    \centering
    \includegraphics[width=\textwidth]{figs/3_2_testing_connectivity_metrics/17_Comparativo_Subplots_Experimentos.png}
    \caption{Comparação dos índices CF-PLM, PLV e PLI nos três cenários estudados, em função da porcentagem de injeção aplicada.}
    \label{fig:comparativo_metricas}
\end{figure}

Esses resultados justificam a escolha do CF-PLM para análise de sincronização entre ECG e EEG (cross-frequency) e do PLI para sincronização em frequências iguais, evitando detecção de acoplamentos triviais como em condução de volume (phase lag zero). O PLV é reportado apenas para referência complementar devido à sua sensibilidade elevada em condições triviais.

\section{Análise de Conectividade ao Longo do Tempo}
\label{sec:connectivity_over_time}

Para investigar a dinâmica da sincronização ao longo da sessão experimental, o sinal foi dividido em janelas de 10 segundos, ao longo da gravação total de 4 minutos e 30 segundos de cada coleta. Em cada janela, foi calculada uma medida de sincronicidade (PLI, PLV ou CF-PLM) para cada par de canais, para cada banda de frequência, para cada condição (cathodic e sham) e para cada atleta. Em seguida, a mediana desses valores foi extraída, fornecendo uma medida robusta da conectividade ao longo do tempo para cada configuração.

Essa abordagem permite visualizar a evolução temporal da sincronização e comparar a estabilidade dos diferentes índices. Ressalta-se que, em nossas análises, enfatizamos o \emph{Wilcoxon RBC} e o p-valor corrigido por Bonferroni como os principais indicadores de tamanho de efeito e significância estatística, respectivamente, dada sua robustez em contextos não paramétricos e na presença de variabilidade e outliers.

A seguir, são apresentadas as séries temporais obtidas para três métricas principais:
\begin{itemize}
    \item \textbf{CF-PLM (EEG-ECG):} A Figura~\ref{fig:cfplm_time_cat} mostra a mediana do CF-PLM ao longo do tempo para a condição cathodic, refletindo a sincronização cross-frequency entre EEG e ECG.
    \item \textbf{PLI (EEG-EEG):} A Figura~\ref{fig:pli_time_cat} apresenta a mediana do PLI ao longo do tempo para a condição cathodic, indicando a sincronização iso-frequencial entre canais cerebrais.
    \item \textbf{PLV (EEG-EEG):} A Figura~\ref{fig:plv_time_cat} exibe a mediana do PLV ao longo do tempo para a condição cathodic, utilizada aqui para comparação, pois o PLV é mais sensível a ruídos e efeitos de volume conduction.
\end{itemize}

Cada gráfico é construído a partir dos valores calculados em janelas de 10 segundos e a mediana de cada janela foi utilizada para representar a medida de sincronicidade de forma robusta, minimizando o impacto de variações pontuais e outliers.

\begin{figure}[htb]
    \centering
    \includegraphics[width=0.8\textwidth]{figs/4_connectivity_over_time/Mediana_do_CF-PLM_ao_longo_do_tempo_(EEG_ECG)_Catódica.png}
    \caption{Mediana do CF-PLM ao longo do tempo para a condição cathodic (EEG-ECG). Cada ponto representa a mediana da medida de sincronicidade calculada em janelas de 10 segundos, evidenciando a evolução do acoplamento cross-frequency entre EEG e ECG.}
    \label{fig:cfplm_time_cat}
\end{figure}

\begin{figure}[htb]
    \centering
    \includegraphics[width=0.8\textwidth]{figs/4_connectivity_over_time/Mediana_do_PLI_ao_longo_do_tempo_(EEG_EEG)_Catódica.png}
    \caption{Mediana do PLI ao longo do tempo para a condição cathodic (EEG-EEG). O gráfico mostra como a sincronização iso-frequencial entre canais cerebrais varia ao longo da gravação.}
    \label{fig:pli_time_cat}
\end{figure}

\begin{figure}[htb]
    \centering
    \includegraphics[width=0.8\textwidth]{figs/4_connectivity_over_time/Mediana_do_PLV_ao_longo_do_tempo_(EEG_EEG)_Catódica.png}
    \caption{Mediana do PLV ao longo do tempo para a condição cathodic (EEG-EEG). Este índice serve para comparação com o PLI, embora seja mais sensível a ruídos e efeitos de volume conduction.}
    \label{fig:plv_time_cat}
\end{figure}
 
\chapter{Análise de Distribuição e Normalidade}
\label{chap:analise_distribuicao_normalidade}

Nesta seção, investigamos a forma das distribuições das métricas de conectividade, tanto nos seus valores “puros” quanto nas diferenças (median\_diff) entre condições (Pós – Pré). Inicialmente, apresentamos as distribuições das métricas específicas \textbf{PLI no EEG-EEG e CF-PLM no EEG-ECG)} que desejamos manter. Em seguida, explicamos que, para testar o efeito da estimulação, calculamos a diferença entre os valores pós e pré (por exemplo, \emph{pós-sham} menos \emph{pré-sham}), e finalmente discutimos a escolha dos testes estatísticos com base nessas distribuições.

\section{Distribuição das Métricas de Conectividade}

Antes de subtrair os valores pré dos pós, as métricas de conectividade são extraídas diretamente dos sinais, refletindo as medidas originais sem a influência do efeito de estimulação. As figuras a seguir ilustram as distribuições “puras” \textbf{apenas} para:

\begin{itemize}
    \item \textbf{PLI (EEG-EEG)}: Avalia sincronização de fase iso-frequencial entre canais cerebrais de EEG.
    \item \textbf{CF-PLM (EEG-ECG)}: Mede o acoplamento \emph{cross-frequency} entre EEG e ciclo cardíaco obtido através do ECG.
\end{itemize}

\noindent As faixas de frequência investigadas incluem: delta, theta, alpha, beta e gamma.

% -----------------------------------------------------------------------------
% CF-PLM (EEG-ECG)
% -----------------------------------------------------------------------------
\begin{figure}[htb]
    \centering
    \includegraphics[width=0.7\textwidth]{figs/3_1_connectivity_metrics/Distribuição_de_CF-PLM_(EEG-ECG)_por_Banda.png}
    \caption{Distribuição de CF-PLM (EEG-ECG) por banda. Observa-se a concentração dos valores em faixas mais baixas (próximas de 0), com maior densidade para as bandas delta e theta.}
    \label{fig:cfplm_eeg_ecg}
\end{figure}

% -----------------------------------------------------------------------------
% PLI (EEG-EEG)
% -----------------------------------------------------------------------------
\begin{figure}[htb]
    \centering
    \includegraphics[width=0.7\textwidth]{figs/3_1_connectivity_metrics/Distribuição_de_PLI_(EEG-EEG)_por_Banda.png}
    \caption{Distribuição de PLI (EEG-EEG) por banda. Embora a maior parte dos valores se concentre em torno de zero, algumas bandas (alpha e gamma) apresentam caudas mais extensas, indicando pares de canais com defasagem de fase mais consistente.}
    \label{fig:pli_eeg_eeg}
\end{figure}

No geral, observamos que:

\begin{itemize}
    \item \textbf{EEG-EEG (PLI)}: A distribuição pode se estender para valores altos, sugerindo sincronizações mais robustas em algumas bandas.
    \item \textbf{EEG-ECG (CF-PLM)}: Geralmente concentra-se em valores mais próximos de zero, mas com alguma distribuição que indica acoplamento \emph{cross-frequency} pontual.
\end{itemize}

Essas observações fornecem uma visão inicial do comportamento das métricas “puras” de conectividade, servindo de ponto de partida para a comparação entre condições (Pós e Pré), apresentada a seguir.

\section{Distribuição das Diferenças (\texttt{median\_diff})}

Para testar o efeito da estimulação (\emph{cathodic} versus \emph{sham}), os valores medidos após a intervenção (pós) foram comparados com os valores obtidos antes (pré). Assim, definimos a métrica:

\[
\texttt{median\_diff} = (\text{pós}) - (\text{pré}),
\]

o que visa isolar o efeito da intervenção, removendo variações comuns que estariam presentes independentemente da estimulação.

As distribuições das diferenças foram avaliadas por meio de histogramas com \emph{Kernel Density Estimation} (KDE), para as métricas que mantivemos nesta análise (PLI para EEG-EEG e CF-PLM para EEG-ECG).

Abaixo, ilustramos exemplos desses histogramas.

% Exemplo: PLI (EEG-EEG) Diferença (sem os de EEG-ECG removidos, etc)
\begin{figure}[htb]
    \centering
    \includegraphics[width=0.8\textwidth]{figs/6_distribuicao_metricas_conectividade/Distribuição_da_Diferença_da_PLI_(Pós_-_Pré)_por_Faixa_de_Frequência_EEG_EEG.png}
    \caption{Distribuição da diferença da PLI (Pós -- Pré) em EEG-EEG, por faixa de frequência.}
    \label{fig:pli_freq_eeg_eeg}
\end{figure}

% Exemplo: CF-PLM (EEG-ECG) Diferença
\begin{figure}[htb]
    \centering
    \includegraphics[width=0.8\textwidth]{figs/6_distribuicao_metricas_conectividade/Distribuição_da_Diferença_da_CF-PLM_(Pós_-_Pré)_por_Faixa_de_Frequência_EEG_ECG.png}
    \caption{Distribuição da diferença da CF-PLM (Pós -- Pré) em EEG-ECG, por faixa de frequência.}
    \label{fig:cf_plm_freq_eeg_ecg}
\end{figure}

Esses exemplos ajudam a visualizar como cada métrica (PLI ou CF-PLM) varia entre as condições pós e pré, em diferentes faixas de frequência, e fundamentam nossa escolha de testes estatísticos, discutida no próximo capítulo.

\subsubsection{Exemplo Individual por Métrica e Banda}
Para ilustrar de forma mais específica como essas distribuições se comportam em um caso individual, a Figura~\ref{fig:median_cf_plm_diff_ath4_alpha_eeg_ecg} exibe a distribuição da diferença da métrica \texttt{median\_cf\_plm\_diff} (Pós -- Pré) para o atleta 4, na banda \emph{alpha}, em pares EEG-ECG. Já a Figura~\ref{fig:median_pli_diff_ath4_alpha_eeg_eeg} apresenta a distribuição da diferença da \texttt{median\_pli\_diff} para o mesmo atleta 4 e banda alpha, porém em pares EEG-EEG.

\begin{figure}[htb]
    \centering
    \includegraphics[width=0.8\textwidth]{figs/5_connectivity_metrics_individual_distribution/median_cf_plm_diff_athlete_4_alpha_EEG_ECG.png}
    \caption{Distribuição da \texttt{median\_cf\_plm\_diff} (Pós -- Pré) para o atleta 4, banda alpha, em pares EEG-ECG.}
    \label{fig:median_cf_plm_diff_ath4_alpha_eeg_ecg}
\end{figure}

\begin{figure}[htb]
    \centering
    \includegraphics[width=0.8\textwidth]{figs/5_connectivity_metrics_individual_distribution/median_pli_diff_athlete_4_alpha_EEG_EEG.png}
    \caption{Distribuição da \texttt{median\_pli\_diff} (Pós -- Pré) para o atleta 4, banda alpha, em pares EEG-EEG.}
    \label{fig:median_pli_diff_ath4_alpha_eeg_eeg}
\end{figure}

Essas figuras exemplificam como as diferenças entre as condições \emph{cathodic} (azul) e \emph{sham} (vermelho) podem se sobrepor ou divergir. Em alguns casos, a curva KDE de uma condição pode deslocar-se à direita (indicando um aumento na métrica após a estimulação) ou à esquerda (indicando redução), enquanto em outros a sobreposição é substancial, sugerindo pouca mudança entre as condições. Esse tipo de análise individual é útil para verificar a variabilidade intra-sujeito e entender melhor se os efeitos observados são consistentes ou pontuais.

\section{Verificação de Normalidade e Escolha do Teste Estatístico}
Para definir se o teste estatístico a ser empregado é paramétrico ou não-paramétrico, é necessário verificar a normalidade das distribuições de interesse. Nesse contexto, testes como o de \emph{Shapiro-Wilk} ou \emph{Kolmogorov-Smirnov} podem ser aplicados para cada grupo de canais e faixa de frequência, tanto para a PLI quanto para a CF-PLM.

\paragraph{Considerações:}
\begin{itemize}
    \item \textbf{Tamanho amostral}: dada a quantidade relativamente grande de observações (após agregação), mesmo pequenas diferenças em relação à distribuição normal podem resultar em rejeição estatística da hipótese de normalidade.
    \item \textbf{Forma das distribuições}: visualmente, muitas distribuições parecem aproximadamente simétricas e unimodais. Ainda assim, pequenas assimetrias ou caudas mais alongadas podem exigir cautela na adoção de testes paramétricos.
    \item \textbf{Interpretação}: se a maioria das distribuições não satisfizer os critérios de normalidade (p.\,ex.\ p-valor $<$ 0.05), a análise inferencial subsequente pode se apoiar em testes não-paramétricos (por exemplo, \emph{Wilcoxon signed-rank} ou \emph{Mann-Whitney}).
\end{itemize}

\paragraph{Próximos Passos:}
\begin{itemize}
    \item Realizar os testes de normalidade (Shapiro-Wilk e/ou Kolmogorov-Smirnov) para cada combinação relevante de \emph{(faixa de frequência, grupo de canais)}.
    \item Avaliar as medidas de assimetria (\emph{skewness}) e curtose (\emph{kurtosis}) para confirmar a adequação de testes paramétricos ou justificar o uso de métodos não-paramétricos.
\end{itemize}

Com base nessa análise preliminar, será possível conduzir as etapas seguintes de inferência estatística, considerando as particularidades de cada métrica (PLI e CF-PLM) e garantindo uma avaliação mais robusta das diferenças entre condições.

\section{Testes de Normalidade e Decisão sobre o Tipo de Teste Estatístico}
A escolha entre testes paramétricos e não paramétricos depende fundamentalmente da distribuição dos dados. Para as métricas de conectividade (median\_pli\_diff e median\_cf\_plm\_diff) agrupadas nos grupos \texttt{EEG\_EEG} e \texttt{EEG\_ECG}, aplicamos uma série de testes de normalidade, a saber: Shapiro-Wilk, Kolmogorov-Smirnov, Anderson-Darling, D'Agostino's K-squared, Jarque-Bera e Lilliefors. Além disso, para atenuar o efeito de valores extremos, os testes foram realizados tanto com os dados originais quanto após a remoção de outliers utilizando o método do \emph{Interquartile Range} (IQR).

\paragraph{Motivações e Procedimentos:}
\begin{itemize}
    \item \textbf{Objetivo:} Verificar se as distribuições das diferenças (Pós -- Pré) seguem uma forma aproximadamente gaussiana, o que permitiria o uso de testes paramétricos.
    \item \textbf{Procedimento:} 
    \begin{itemize}
        \item Os dados foram agrupados por \texttt{channel\_group} (EEG\_EEG e EEG\_ECG) e para cada métrica.
        \item Foram aplicados os testes de normalidade com e sem outliers, permitindo avaliar o efeito destes na distribuição.
    \end{itemize}
\end{itemize}

\paragraph{Principais Resultados:}
\begin{itemize}
    \item \textbf{Grupo EEG\_EEG (PLI):}
    \begin{itemize}
        \item Para \texttt{median\_pli\_diff}, todos os testes (Shapiro-Wilk, Kolmogorov-Smirnov, Anderson-Darling, D'Agostino, Jarque-Bera e Lilliefors) indicaram desvios significativos da normalidade (p-valores muito baixos), mesmo após a filtragem de outliers (10,66\% dos dados).
    \end{itemize}
    
    \item \textbf{Grupo EEG\_ECG (CF-PLM):}
    \begin{itemize}
        \item Para \texttt{median\_cf\_plm\_diff}, mesmo após a remoção de outliers (10,65\% dos dados), os testes continuaram a rejeitar a normalidade, apresentando p-valores próximos de zero na maioria dos casos.
    \end{itemize}
\end{itemize}

\paragraph{Interpretação e Decisão Metodológica:}  
Os resultados dos testes de normalidade demonstram que, em sua maioria, as distribuições das diferenças nas métricas de conectividade não se comportam de maneira normal, mesmo após a remoção de outliers. Essa violação do pressuposto de normalidade indica que a aplicação de testes paramétricos (que assumem uma distribuição gaussiana dos dados) poderia levar a inferências incorretas. Portanto, optamos por utilizar testes não paramétricos para as análises estatísticas subsequentes, garantindo robustez e validade às conclusões sem a necessidade de assumir normalidade dos dados.

\chapter{Análise Estatística Não Paramétrica}
\label{chap:analise_estatistica_np}
Considerando que as distribuições das métricas de conectividade (diferença entre Pós e Pré) não se comportam de maneira normal (ver Capítulo~\ref{chap:analise_distribuicao_normalidade}), optamos por empregar testes estatísticos não paramétricos para comparar as condições de estimulação \textit{cathodic} versus \textit{sham}. Essa escolha evita pressupostos inadequados sobre a distribuição dos dados. Além da violação de normalidade, a presença de \texit{outliers} reforça a adoção de testes não paramétricos neste cenário de heterogeneidade amostral.

Nesta etapa, foram aplicados os seguintes testes estatísticos não paramétricos:

\begin{itemize}
    \item \textbf{\textit{Mann-Whitney U}:} Teste para amostras independentes, usado para avaliar se a distribuição das diferenças entre as condições difere significativamente entre os grupos. O tamanho do efeito foi estimado com base na estatística U e no número de observações em cada condição.
    
    \item \textbf{\textit{Wilcoxon signed-rank}:} Teste para amostras pareadas, que compara os valores dentro de cada participante, canal e faixa de frequência, permitindo avaliar mudanças intraindividuais entre as condições.

    \item \textbf{\textit{Kruskal-Wallis}:} Teste para múltiplos grupos, empregado para verificar diferenças globais ao longo das faixas de frequência. A significância indica se pelo menos uma das bandas se comporta de forma distinta, sem especificar quais.
\end{itemize}

Os testes foram realizados separadamente para cada uma das métricas de conectividade (\texttt{median\_plv\_diff}, \texttt{median\_pli\_diff} e \texttt{median\_cf\_plm\_diff}) e para os dois grupos de canais: \texttt{EEG\_EEG} e \texttt{EEG\_ECG}. Ressalta-se que, embora o código inclua procedimentos para remoção de dados nulos, na prática esses valores não estão presentes, servindo apenas para tratamento de exceções.

\section{Resultados dos Testes}
Nesta seção, comparamos as diferenças (pós -- pré) das métricas de conectividade entre as condições \textit{cathodic} e \textit{sham} por meio de três testes não-paramétricos: \textit{Mann-Whitney U}, \textit{Wilcoxon signed-rank} e \textit{Kruskal-Wallis}. Cada teste fornece uma perspectiva ligeiramente distinta sobre as diferenças entre os grupos de dados:

\begin{itemize}
  \item \textbf{\textit{Mann-Whitney U} (amostras independentes):} Avalia se as distribuições da métrica em \textit{cathodic} e \textit{sham} diferem de forma consistente em seus valores medianos, quando os grupos não são necessariamente pareados.
  \item \textbf{\textit{Wilcoxon signed-rank} (amostras pareadas):} Verifica se há diferença na mesma variável entre dois estados para o \emph{mesmo} participante ou canal, isto é, um teste de antes e depois (pós -- pré) que controla variações intraindividuais.
  \item \textbf{\textit{Kruskal-Wallis} (comparação de múltiplos grupos):} Geralmente utilizado para verificar se há diferença \emph{global} entre três ou mais grupos, mas aqui também se aplica a subdivisões (como faixas de frequência) para avaliar se alguma banda difere significativamente das demais.
\end{itemize}

\subsection{Métricas para \texttt{median\_pli\_diff}}

\paragraph{Grupo EEG\_EEG:}
\begin{itemize}
    \item \textbf{\textit{Mann-Whitney U}:} As faixas \emph{alpha}, \emph{delta}, \emph{gamma} e \emph{theta} exibiram diferenças estatisticamente significativas (p-valores $< 0.001$) entre \emph{cathodic} e \textit{sham}. Já a faixa \emph{beta} teve um resultado marginal ($p \approx 0.051$), sugerindo uma possível tendência à diferença, mas não alcançando significância pelo critério habitual ($\alpha=0.05$).
    \item \textbf{\textit{Wilcoxon}:} Ao tratar cada participante (ou par de canais) como pareado, observamos que todas as bandas apresentaram p-valores muito baixos, com tamanho de efeito em torno de $0.477$. Isso reforça a ideia de que, na maior parte das faixas de frequência, há uma mudança consistente entre pré e pós, distinta para \textit{cathodic} vs. \textit{sham}.
    \item \textbf{\textit{Kruskal-Wallis}:} Novamente, \emph{alpha}, \emph{delta}, \emph{gamma} e \emph{theta} demonstraram diferenças significativas na comparação global. A \emph{beta} foi marginal, o que sugere que, ao analisar globalmente as bandas, \emph{beta} não se destacou tanto.
\end{itemize}

\paragraph{Interpretação:}
Os resultados indicam que a \texttt{median\_pli\_diff} (EEG-EEG) sofre mudanças significativas na maior parte das faixas de frequência, seja considerando grupos independentes (\textit{Mann-Whitney U}), seja observando pares de medidas para cada sujeito/canal (\textit{Wilcoxon}). Já o \textit{Kruskal-Wallis}, que avalia diferenças globais entre bandas, mostra que \emph{alpha}, \emph{delta}, \emph{gamma} e \emph{theta} se destacam de forma consistente, enquanto \emph{beta} está no limite de significância. 
Na prática, isso sugere que a estimulação catódica influencia de modo mais evidente a sincronização nessas quatro bandas em comparação à \emph{beta}. Ademais, em faixas como \emph{alpha}, \emph{delta} e \emph{theta}, o \emph{effect size} geralmente aponta para valores mais elevados de \emph{PLI} sob \textit{cathodic}, indicando um provável aumento na sincronia de fase frente ao \textit{sham}.

\subsection{Métricas para \texttt{median\_cf\_plm\_diff}}
\paragraph{Grupo EEG\_ECG:}
\begin{itemize}
    \item \textbf{\textit{Mann-Whitney U}:} As faixas \emph{alpha}, \emph{beta} e \emph{delta} apresentaram diferenças significativas (p-valores $< 0.05$), enquanto \emph{gamma} e \emph{theta} não mostraram significância estatística.
    \item \textbf{\textit{Wilcoxon}:} Todos os testes resultaram em p-valores muito baixos, com tamanhos de efeito em torno de 0.358, indicando um efeito consistente de \textit{cathodic} vs. \textit{sham} quando avaliado \emph{intra-sujeito/canal}.
    \item \textbf{\textit{Kruskal-Wallis}:} As bandas \emph{alpha}, \emph{beta} e \emph{gamma} revelaram diferenças na análise global, mas \emph{delta} e \emph{theta} não se destacaram nesse critério.
\end{itemize}

\paragraph{Interpretação:}

No caso da \texttt{median\_cf\_plm\_diff} (EEG-ECG), encontramos resultados que divergem ligeiramente conforme o teste aplicado: enquanto \textit{Mann-Whitney U} e \textit{Wilcoxon} apontam a significância de \textit{alpha}, \textit{beta} e \textit{delta}, o \textit{Kruskal-Wallis} identifica \textit{gamma}, porém não confirma \textit{delta} e \textit{theta}. Isso reflete a menor sensibilidade do teste global (\textit{Kruskal-Wallis}) para certas diferenças pontuais, bem como a possível variabilidade nos pares EEG-ECG. De forma geral, \textit{alpha} e \textit{beta} exibem valores \texttt{cf\_plm\_diff} mais elevados na condição \textit{cathodic}, indicando maior sincronicidade EEG-ECG após a estimulação catódica em comparação à condição \textit{sham}


\subsection{Conclusão Geral dos Testes}
Em síntese, os três testes não-paramétricos evidenciam mudanças estatisticamente significativas nas métricas \texttt{median\_pli\_diff} e \texttt{median\_cf\_plm\_diff}, abrangendo várias faixas de frequência. Cada teste ressalta um aspecto distinto:
\begin{itemize}
    \item \textit{Mann-Whitney U}: Indica diferenças globais entre \textit{cathodic} e \textit{sham}.
    \item \textit{Wilcoxon}: Reforça as mudanças intraindividuais (pós--pré), sinalizando consistência dos efeitos.
    \item \textit{Kruskal-Wallis}: Mostra variações no espectro como um todo, apontando quais faixas são mais afetadas.
\end{itemize}

Esses achados, em conjunto, corroboram a hipótese de que a neuromodulação catódica sobre o DLPFC impacta os padrões de sincronização em \textit{resting-state}, tanto no PPC \textit{same-frequency} EEG-EEG quanto no PPC \textit{cross-frequency} EEG-ECG. A consistência entre testes diferentes aumenta a confiança de que os efeitos observados não são atribuíveis a ruídos ou a variações aleatórias.

\section{Discussão e Justificativa dos Métodos}
Os testes de normalidade indicaram violações significativas dos pressupostos de normalidade para as métricas de conectividade analisadas, o que justificou o uso de testes não paramétricos para comparar as condições de estimulação (\texttt{\textit{cathodic}} versus \texttt{\textit{sham}}). Esses métodos não dependem de pressupostos sobre a distribuição dos dados.

A aplicação conjunta de múltiplos testes (\textit{Mann-Whitney U}, \textit{Wilcoxon} e \textit{Kruskal-Wallis}) permitiu uma avaliação abrangente das diferenças entre as condições, considerando tanto comparações entre grupos independentes quanto análises pareadas intraindividuais. Os resultados indicam que, para a maioria das faixas de frequência e para as métricas \texttt{median\_pli\_diff} e \texttt{median\_cf\_plm\_diff}, existem diferenças estatisticamente significativas entre as condições, reforçando a importância da modulação induzida pela estimulação.

Por outro lado, os achados para as métricas de sincronização, em especial no contexto EEG–ECG, exibiram variações dependendo do teste estatístico e da faixa de frequência considerada. Esse fato indica que cada método (Mann–Whitney, \textit{Wilcoxon} ou Kruskal–Wallis) pode captar aspectos distintos dos dados ou apresentar sensibilidades diferentes à variabilidade intra e intersujeitos.

\section{Detecção de \textit{Outliers}, Análise \textit{Bootstrap} e Correções para Comparações Múltiplas}
Para avaliar o impacto dos pontos atípicos nas análises subsequentes, optamos por realizar uma etapa adicional de detecção e remoção de \texit{outliers} utilizando o método ECOD, especialmente adequado ao nosso conjunto de dados devido à sua natureza não paramétrica, identificando anomalias sem pressupor uma distribuição específica dos dados. Estudos demonstram que o ECOD supera diversas técnicas convencionais de detecção de \texit{outliers} em termos de acurácia e eficiência \cite{li2022ecod}.

Aplicamos o ECOD considerando as métricas \texttt{median\_pli\_diff} e \texttt{median\_cf\_plm\_diff}. Inicialmente, o dataset continha 122915 entradas; após a aplicação do ECOD, aproximadamente 5,00\% dos dados foram identificados como \texit{outliers} e removidos.

Posteriormente, implementamos um pipeline de análise baseado em \textit{Bootstrap} acelerado por placa de vídeo (GPU) para o cálculo de intervalos de confiança \textit{Bias-Corrected and Accelerated} (BCa). Esse método é particularmente robusto, pois ajusta tanto o viés quanto a aceleração da distribuição \textit{Bootstrap}, permitindo capturar assimetrias e a influência residual de \texit{outliers} nos dados. Embora computacionalmente custoso, o método BCa é amplamente reconhecido como uma das abordagens mais precisas para a estimação de intervalos de confiança em situações onde os pressupostos de normalidade não são atendidos. Optamos por esse método em detrimento de outras técnicas devido à sua capacidade de corrigir distorções na distribuição da estatística estimada.

Para avaliar a significância estatística após múltiplas comparações, utilizamos a função \texttt{multipletests} da biblioteca Python \texttt{statsmodels} com o método \textit{Bonferroni}. Este procedimento ajusta os p-valores originais multiplicando-os pelo número total de comparações realizadas, tornando o critério de significância mais conservador e minimizando o risco de falsos positivos. Dessa forma, efeitos são considerados significativos quando o p-valor corrigido por \textit{Bonferroni} for inferior a 0,05.

Além disso, nosso pipeline incluiu o cálculo de tamanhos de efeito utilizando diversas métricas, tais como:
\begin{itemize}
    \item \textbf{\textit{Cohen's d}} e \textbf{\textit{Hedges' g}}: que quantificam a magnitude da diferença entre as condições em termos de desvios-padrão;
    \item \textbf{\textit{Rank-Biserial Correlation} (RBC)}: derivado do teste de \textit{Wilcoxon}, que fornece uma interpretação robusta baseada em postos.
\end{itemize}

Essas métricas complementares permitem uma avaliação abrangente do efeito da estimulação e possibilitam comparar os resultados obtidos com diferentes abordagens, como o método de \textit{Bonferroni}, Holm e \textit{False Discovery Rate-Benjamini–Hochberg} (FDR-BH), fornecendo uma visão mais robusta e diversificada dos achados.

Em resumo, nossa abordagem compreende as seguintes etapas:
\begin{enumerate}
    \item \textbf{Detecção de \textit{Outliers}:} Utilizamos o método ECOD para identificar e remover aproximadamente 5\% dos dados considerados anômalos, garantindo a robustez das análises.
    \item \textbf{Análise \textit{Bootstrap} com GPU:} Implementamos o cálculo de intervalos de confiança BCa, estimando viés, erro padrão e tamanhos de efeito por meio de reamostragem acelerada, assegurando precisão mesmo em distribuições assimétricas.
    \item \textbf{Testes Não Paramétricos e Correção para Comparações Múltiplas:} Aplicamos testes não paramétricos, como o teste de \textit{Wilcoxon} para dados emparelhados, e corrigimos os p-valores utilizando o método \textit{Bonferroni} (além de outras correções complementares), minimizando o risco de erros do tipo I.
\end{enumerate}

Os resultados deste processamento serão apresentados em detalhes nos anexos, permitindo a comparação dos resultados obtidos com e sem a remoção de \texit{outliers}, bem como a análise dos diferentes tamanhos de efeito e métodos de correção.

\section{Distribuição de Tamanhos de Efeito e p-valores}
\label{sec:effect_size_distribution}
Nesta etapa, examinamos a distribuição das estimativas de tamanho de efeito (\textit{Cohen's d}, \textit{Hedges' g} e \textit{Wilcoxon RBC}) e dos valores-p (brutos e corrigidos por \textit{Bonferroni}) para as análises de PLI (EEG-EEG) e CF-PLM (EEG-ECG), considerando cenários com e sem \texit{outliers}. A Figura~\ref{fig:effectsizehist_all} ilustra, em quatro subfiguras, como essas métricas se distribuem ao longo dos pares de canal e faixas de frequência analisados.

\begin{figure}[htb]
    \centering
    % Subfigura 1: PLI (EEG-EEG), Sem Outliers
    \subfloat[Sem Outliers -- PLI (EEG-EEG)]{
        \includegraphics[width=0.45\textwidth]{figs/7_bootstrap_results_analysis/1_effect_size_histograms/Effect_Size_Histograms_PLI_EEGEEG_Sem_Outliers.png}
    }
    \quad
    % Subfigura 2: CF-PLM (EEG-ECG), Sem Outliers
    \subfloat[Sem Outliers -- CF-PLM (EEG-ECG)]{
        \includegraphics[width=0.45\textwidth]{figs/7_bootstrap_results_analysis/1_effect_size_histograms/Effect_Size_Histograms_CFPLM_EEGECG_Sem_Outliers.png}
    }
    \\
    % Subfigura 3: PLI (EEG-EEG), Com Outliers
    \subfloat[Com Outliers -- PLI (EEG-EEG)]{
        \includegraphics[width=0.45\textwidth]{figs/7_bootstrap_results_analysis/1_effect_size_histograms/Effect_Size_Histograms_PLI_EEGEEG_Com_Outliers.png}
    }
    \quad
    % Subfigura 4: CF-PLM (EEG-ECG), Com Outliers
    \subfloat[Com Outliers -- CF-PLM (EEG-ECG)]{
        \includegraphics[width=0.45\textwidth]{figs/7_bootstrap_results_analysis/1_effect_size_histograms/Effect_Size_Histograms_CFPLM_EEGECG_Com_Outliers.png}
    }
    \caption[Distribuições de tamanhos de efeito e valores-p]{Distribuição das métricas de tamanho de efeito (\textit{Cohen's d}, \textit{Hedges' g} e \textit{Wilcoxon RBC}) e dos valores-p (brutos e corrigidos por \textit{Bonferroni}) para PLI (EEG-EEG) e CF-PLM (EEG-ECG), em cenários com e sem \texit{outliers}. O \textit{Wilcoxon RBC} e o p-valor corrigido por \textit{Bonferroni} (vertical tracejada vermelha em $p=0.05$) foram escolhidos como as métricas para representar, respectivamente, o tamanho do efeito e a significância estatística nas análises subsequentes.}
    \label{fig:effectsizehist_all}    
\end{figure}

\subsection{Distribuição dos Tamanhos de Efeito}
\paragraph{\textit{Cohen's d} e \textit{Hedges' g}}
\begin{itemize}
    \item A maior parte dos valores concentra-se em torno de zero, indicando que, para a maioria dos pares, as diferenças entre as condições \textit{cathodic} e \textit{sham} são pequenas ou não significativas.
    \item Valores significativos (representados pelas barras vermelhas nos histogramas) tendem a se afastar de zero, sinalizando diferenças mais acentuadas. Por exemplo, valores de \textit{Cohen's d} ou \textit{Hedges' g} superiores a 0.5 (ou inferiores a -0.5) sugerem um efeito moderado, enquanto valores acima de 0.8 (ou inferiores a -0.8) indicam um efeito alto.
    \item Embora \textit{Hedges' g} difira de \textit{Cohen's d} ao aplicar uma correção para tamanhos amostrais pequenos, ambas as métricas exibem comportamentos semelhantes nos histogramas.
\end{itemize}

\paragraph{\textit{Wilcoxon Rank-Biserial Correlation} (\textit{Wilcoxon RBC})}
\begin{itemize}
    \item O \textit{Wilcoxon RBC} é derivado do teste não paramétrico de \textit{Wilcoxon} e reflete a correlação de postos entre as condições, tipicamente variando de -1 a +1.
    \item Por não exigir pressupostos de normalidade, o RBC se mostra mais robusto no tratamento de dados heterogêneos e na presença de \texit{outliers}.
    \item Valores acima de 0.3 ou abaixo de -0.3 sugerem um efeito moderado; valores acima de 0.5 (ou abaixo de -0.5) indicam um efeito alto, e quando se aproximam de ±1, as condições diferem de forma quase absoluta.
    \item Devido a essa robustez, o RBC foi escolhido como nosso principal indicador de tamanho de efeito nas análises subsequentes.
\end{itemize}

\subsection{Distribuição de p-valores (Brutos e Corrigidos)}
\begin{itemize}
    \item Os histogramas de p-valores brutos mostram uma forte concentração em torno de 1 (indicando resultados não significativos) e uma cauda próxima de 0 (sinalizando potenciais resultados significativos).
    \item Após a correção de \textit{Bonferroni} (indicada pela linha vertical tracejada em \(p=0.05\)), muitos dos valores que eram marginalmente significativos foram deslocados para a região de não significância, evidenciando o caráter conservador deste método de correção.
    \item Devido ao elevado número de comparações, a utilização do método \textit{Bonferroni} minimiza a probabilidade de falsos positivos, sendo adotado como critério principal para a significância estatística.
\end{itemize}

\subsection{Comparação Entre Cenários (Com e Sem Outliers)}
\begin{itemize}
    \item \textbf{Impacto da Remoção de \textit{Outliers}:} De modo geral, a remoção de \texit{outliers} reduz ligeiramente o número de casos significativos em EEG-EEG, mas não altera substancialmente a distribuição dos tamanhos de efeito ou dos p-valores. No caso do EEG-ECG, a diferença entre manter ou remover \texit{outliers} é mínima, indicando que a presença de valores extremos tem pouco impacto na detecção de efeitos significativos nesse grupo.

    \item \textbf{PLI (EEG-EEG):} O número de comparações realizadas no cenário EEG-EEG é consideravelmente maior, pois cada um dos 61 canais de EEG forma pares com os demais canais EEG. Isso resulta em \( 61 \times 60 \) pares únicos por banda de frequência. Com 5 bandas de frequência e 6 atletas, obtemos um total de aproximadamente \( 109.800 \) comparações. Entre essas, foram identificados entre \( 363 \) (com remoção de \texit{outliers}) e \( 378 \) (sem remoção de \texit{outliers}) casos significativos.

    \item \textbf{CF-PLM (EEG-ECG):} Em contraste, no cenário EEG-ECG, cada um dos 61 canais EEG é comparado a um único canal de ECG, resultando em \( 61 \times 1 \) pares por banda de frequência. Assim, considerando as mesmas 5 bandas e 6 atletas, temos um total de \( 3.660 \) comparações, o que representa uma redução drástica na quantidade de testes estatísticos realizados em relação ao EEG-EEG. Como esperado, isso levou à detecção de um número muito menor de casos significativos, totalizando apenas \( 11 \).

    \item \textbf{Robustez do RBC e do \textit{Bonferroni}:} Independentemente da remoção de \texit{outliers}, as comparações que apresentam valores elevados de \textit{Wilcoxon RBC} e p-valores corrigidos abaixo de \( 0.05 \) permanecem confiáveis. Isso reforça a utilidade dessas métricas como principais indicadores da magnitude e significância estatística dos efeitos encontrados, independentemente da heterogeneidade dos dados.
\end{itemize}

Em resumo, os histogramas de \textit{Wilcoxon RBC} (indicador de tamanho de efeito) e os p-valores corrigidos por \textit{Bonferroni} (indicador de significância estatística) evidenciam quais pares de canais apresentam diferenças robustas entre as condições \textit{cathodic} e \textit{sham}. Embora \textit{Cohen's d} e \textit{Hedges' g} também sejam úteis para quantificar a magnitude do efeito, enfatizamos o RBC devido à sua robustez, natureza não paramétrica e resiliência à heterogeneidade dos dados. Esses resultados fornecem uma base sólida para as análises topográficas e de rede apresentadas nas seções seguintes.

\subsection{Conclusões Principais}
\begin{itemize}
    \item A distribuição dos dados mostra que a maioria dos pares de canais apresenta diferenças pequenas entre as condições, com os valores de tamanho de efeito concentrando-se em torno de zero. Esse comportamento é esperado, dado o alto número de comparações e a aplicação de métodos rigorosos de correção múltipla, como \textit{Bonferroni}.
    
    \item Nos casos onde há significância estatística, os tamanhos de efeito se afastam de zero de forma mais pronunciada (conforme evidenciado por \textit{Cohen's d}, \textit{Hedges' g} ou RBC), indicando diferenças que podem ser relevantes do ponto de vista da dinâmica de conectividade funcional.

    \item O \textit{Wilcoxon RBC} se destaca como a métrica escolhida para quantificar tanto a direção quanto a magnitude das diferenças, sem assumir pressupostos de normalidade. Essa característica torna o RBC uma escolha apropriada para as próximas etapas da análise, que incluem a caracterização topográfica e a construção dos grafos de conectividade.

\end{itemize}

Dessa forma, a análise dos histogramas de tamanhos de efeito e dos p-valores fornece um panorama inicial detalhado. Embora a maioria dos pares de canais não apresente diferenças significativas, observa-se um conjunto de casos com efeitos inicialmente moderados ou altos, mas que, após a correção para múltiplas comparações, se restringem predominantemente aos valores de efeito mais elevados. Esses achados servem como base para investigações posteriores, focadas na identificação de padrões espaciais e espectrais na neuromodulação, contribuindo para uma compreensão mais aprofundada das interações entre EEG e ECG nas condições experimentais avaliadas.


\postextual

% Bibliografia
\bibliographystyle{abntex2-alf}
\bibliography{bibliografia}

% Anexos e Apêndices
\begin{apendicesenv}

\partapendices
%%%%%%%%%%%%%%%%%%%%%%%%%%%%%%%%%%%%%%%%%%%%%%%%
\chapter{Análise de Redes de Conectividade para PLI (EEG-EEG)}
\label{apendice:pli_eeg_eeg}

Nesta seção, apresentamos as redes de conexões significativas obtidas pela métrica PLI para pares EEG-EEG, tanto para o cenário sem outliers quanto para o cenário com outliers. Cada figura corresponde a uma banda de frequência e complementa a análise apresentada no corpo do texto.

\section{Cenário Sem Outliers}

\standardfigure{figs/7_bootstrap_results_analysis/2_network_graphs/PLI_EEG-EEG_Sem_Outliers/Banda_Delta_(0.5_a_4_Hz)_-_Análise_de_Rede_-_PLI_EEG-EEG_Sem_Outliers.png}
{Rede de conexões significativas na banda Delta (0.5--4\,Hz) para PLI (EEG-EEG) sem outliers.}
{apend_pli_delta_sem}

\standardfigure{figs/7_bootstrap_results_analysis/2_network_graphs/PLI_EEG-EEG_Sem_Outliers/Banda_Theta_(4_Hz_a_8_Hz)_-_Análise_de_Rede_-_PLI_EEG-EEG_Sem_Outliers.png}
{Rede de conexões significativas na banda Theta (4--8\,Hz) para PLI (EEG-EEG) sem outliers.}
{apend_pli_theta_sem}

\standardfigure{figs/7_bootstrap_results_analysis/2_network_graphs/PLI_EEG-EEG_Sem_Outliers/Banda_Alpha_(8_Hz_a_13_Hz)_-_Análise_de_Rede_-_PLI_EEG-EEG_Sem_Outliers.png}
{Rede de conexões significativas na banda Alpha (8--13\,Hz) para PLI (EEG-EEG) sem outliers.}
{apend_pli_alpha_sem}

\standardfigure{figs/7_bootstrap_results_analysis/2_network_graphs/PLI_EEG-EEG_Sem_Outliers/Banda_Beta_(13_Hz_a_30_Hz)_-_Análise_de_Rede_-_PLI_EEG-EEG_Sem_Outliers.png}
{Rede de conexões significativas na banda Beta (13--30\,Hz) para PLI (EEG-EEG) sem outliers.}
{apend_pli_beta_sem}

\standardfigure{figs/7_bootstrap_results_analysis/2_network_graphs/PLI_EEG-EEG_Sem_Outliers/Banda_Gamma_(30_Hz_a_60_Hz)_-_Análise_de_Rede_-_PLI_EEG-EEG_Sem_Outliers.png}
{Rede de conexões significativas na banda Gamma (30--60\,Hz) para PLI (EEG-EEG) sem outliers.}
{apend_pli_gamma_sem}

\section{Cenário Com Outliers}

\standardfigure{figs/7_bootstrap_results_analysis/2_network_graphs/PLI_EEG-EEG_Com_Outliers/Banda_Delta_(0.5_a_4_Hz)_-_Análise_de_Rede_-_PLI_EEG-EEG_Com_Outliers.png}
{Rede de conexões significativas na banda Delta (0.5--4\,Hz) para PLI (EEG-EEG) com outliers.}
{apend_pli_delta_com}

\standardfigure{figs/7_bootstrap_results_analysis/2_network_graphs/PLI_EEG-EEG_Com_Outliers/Banda_Theta_(4_Hz_a_8_Hz)_-_Análise_de_Rede_-_PLI_EEG-EEG_Com_Outliers.png}
{Rede de conexões significativas na banda Theta (4--8\,Hz) para PLI (EEG-EEG) com outliers.}
{apend_pli_theta_com}

\standardfigure{figs/7_bootstrap_results_analysis/2_network_graphs/PLI_EEG-EEG_Com_Outliers/Banda_Alpha_(8_Hz_a_13_Hz)_-_Análise_de_Rede_-_PLI_EEG-EEG_Com_Outliers.png}
{Rede de conexões significativas na banda Alpha (8--13\,Hz) para PLI (EEG-EEG) com outliers.}
{apend_pli_alpha_com}

\standardfigure{figs/7_bootstrap_results_analysis/2_network_graphs/PLI_EEG-EEG_Com_Outliers/Banda_Beta_(13_Hz_a_30_Hz)_-_Análise_de_Rede_-_PLI_EEG-EEG_Com_Outliers.png}
{Rede de conexões significativas na banda Beta (13--30\,Hz) para PLI (EEG-EEG) com outliers.}
{apend_pli_beta_com}

\standardfigure{figs/7_bootstrap_results_analysis/2_network_graphs/PLI_EEG-EEG_Com_Outliers/Banda_Gamma_(30_Hz_a_60_Hz)_-_Análise_de_Rede_-_PLI_EEG-EEG_Com_Outliers.png}
{Rede de conexões significativas na banda Gamma (30--60\,Hz) para PLI (EEG-EEG) com outliers.}
{apend_pli_gamma_com}


%%%%%%%%%%%%%%%%%%%%%%%%%%%%%%%%%%%%%%%%%%%%%%%%
\chapter{Análise de Redes de Conectividade para CF-PLM (EEG-ECG)}
\label{apendice:cfplm_eeg_ecg}

Nesta seção, apresentamos as redes de conexões significativas obtidas pela métrica CF-PLM para pares EEG-ECG, tanto para o cenário sem outliers quanto para o cenário com outliers. Observa-se que o número total de pares significativos é constante (11 casos) independentemente da remoção de outliers.

\section{Cenário Sem Outliers}

\standardfigure{figs/7_bootstrap_results_analysis/2_network_graphs/CF-PLM_EEG-ECG_Sem_Outliers/Banda_Delta_(0.5_a_4_Hz)_-_Análise_de_Rede_-_CF-PLM_EEG-ECG_Sem_Outliers.png}
{Rede CF-PLM na banda Delta (0.5--4\,Hz) para EEG-ECG sem outliers.}
{apend_cfplm_delta_sem}

\standardfigure{figs/7_bootstrap_results_analysis/2_network_graphs/CF-PLM_EEG-ECG_Sem_Outliers/Banda_Theta_(4_Hz_a_8_Hz)_-_Análise_de_Rede_-_CF-PLM_EEG-ECG_Sem_Outliers.png}
{Rede CF-PLM na banda Theta (4--8\,Hz) para EEG-ECG sem outliers.}
{apend_cfplm_theta_sem}

\standardfigure{figs/7_bootstrap_results_analysis/2_network_graphs/CF-PLM_EEG-ECG_Sem_Outliers/Banda_Alpha_(8_Hz_a_13_Hz)_-_Análise_de_Rede_-_CF-PLM_EEG-ECG_Sem_Outliers.png}
{Rede CF-PLM na banda Alpha (8--13\,Hz) para EEG-ECG sem outliers.}
{apend_cfplm_alpha_sem}

\standardfigure{figs/7_bootstrap_results_analysis/2_network_graphs/CF-PLM_EEG-ECG_Sem_Outliers/Banda_Beta_(13_Hz_a_30_Hz)_-_Análise_de_Rede_-_CF-PLM_EEG-ECG_Sem_Outliers.png}
{Rede CF-PLM na banda Beta (13--30\,Hz) para EEG-ECG sem outliers.}
{apend_cfplm_beta_sem}

\standardfigure{figs/7_bootstrap_results_analysis/2_network_graphs/CF-PLM_EEG-ECG_Sem_Outliers/Banda_Gamma_(30_Hz_a_60_Hz)_-_Análise_de_Rede_-_CF-PLM_EEG-ECG_Sem_Outliers.png}
{Rede CF-PLM na banda Gamma (30--60\,Hz) para EEG-ECG sem outliers.}
{apend_cfplm_gamma_sem}

\section{Cenário Com Outliers}

\standardfigure{figs/7_bootstrap_results_analysis/2_network_graphs/CF-PLM_EEG-ECG_Com_Outliers/Banda_Delta_(0.5_a_4_Hz)_-_Análise_de_Rede_-_CF-PLM_EEG-ECG_Com_Outliers.png}
{Rede CF-PLM na banda Delta (0.5--4\,Hz) para EEG-ECG com outliers.}
{apend_cfplm_delta_com}

\standardfigure{figs/7_bootstrap_results_analysis/2_network_graphs/CF-PLM_EEG-ECG_Com_Outliers/Banda_Theta_(4_Hz_a_8_Hz)_-_Análise_de_Rede_-_CF-PLM_EEG-ECG_Com_Outliers.png}
{Rede CF-PLM na banda Theta (4--8\,Hz) para EEG-ECG com outliers.}
{apend_cfplm_theta_com}

\standardfigure{figs/7_bootstrap_results_analysis/2_network_graphs/CF-PLM_EEG-ECG_Com_Outliers/Banda_Alpha_(8_Hz_a_13_Hz)_-_Análise_de_Rede_-_CF-PLM_EEG-ECG_Com_Outliers.png}
{Rede CF-PLM na banda Alpha (8--13\,Hz) para EEG-ECG com outliers.}
{apend_cfplm_alpha_com}

\standardfigure{figs/7_bootstrap_results_analysis/2_network_graphs/CF-PLM_EEG-ECG_Com_Outliers/Banda_Beta_(13_Hz_a_30_Hz)_-_Análise_de_Rede_-_CF-PLM_EEG-ECG_Com_Outliers.png}
{Rede CF-PLM na banda Beta (13--30\,Hz) para EEG-ECG com outliers.}
{apend_cfplm_beta_com}

\standardfigure{figs/7_bootstrap_results_analysis/2_network_graphs/CF-PLM_EEG-ECG_Com_Outliers/Banda_Gamma_(30_Hz_a_60_Hz)_-_Análise_de_Rede_-_CF-PLM_EEG-ECG_Com_Outliers.png}
{Rede CF-PLM na banda Gamma (30--60\,Hz) para EEG-ECG com outliers.}
{apend_cfplm_gamma_com}

\section{Comparação Geral entre Cenários para CF-PLM}
Ao comparar os resultados dos cenários com e sem outliers para CF-PLM (EEG-ECG), observa-se que:
\begin{itemize}
    \item A estrutura das redes, ou seja, a topologia e a contagem de conexões, permanece praticamente inalterada entre os cenários.
    \item O número total de pares significativos (11 casos) é o mesmo, evidenciando a robustez dos efeitos cross-frequency.
    \item Os pares com \emph{Wilcoxon RBC} de +1 persistem, reforçando a consistência dos resultados.
\end{itemize}
  
\end{apendicesenv}

\input{postextual/anexos}

% Se necessário, índice remissivo
% \printindex

\end{document}